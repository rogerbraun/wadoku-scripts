% Dieser Befehl bestimmt die Klasse unseres Dokuments, sozusagen die Vorlage. Wir nehmen die Article-Klasse aus dem KOMA-Skript-Projekt. http://www.komascript.de/
\documentclass[a4paper,10pt,bibtotocnumbered,smallheadings,pointednumbers]{scrartcl}
% Genzi (http://kuniyoshi.fastmail.fm/xetex/) von Kazuomi Kuniyoshi hilft bei der automatischen Schriftauswahl beim Wechsel zwischen Japanisch und Deutsch. Standardmäßig nimmt unsere Version davon  Minion als westliche, Kozuka als japanische Font.
%Für Mac die nächste Zeile auskommentieren und die darauffolgende löschen.
% \usepackage[hoefler,hiragino]{genzi}
\usepackage{genzi}
\usepackage{xltxtra}
\usepackage{lexikon}
\usepackage{setspace}
\usepackage[paper=a4paper,left=35mm,right=35mm,top=35mm,bottom=43mm]{geometry} 
\defaultfontfeatures{Mapping=tex-text}
\usepackage{hanging}
\usepackage{multicol}
\usepackage{ragged2e}

\newenvironment{entry}{%
\par\leavevmode\hangpara{1.5mm}{1}\ignorespaces}{\RaggedRight\par}
%\setlength{\parindent}{2em}

\newenvironment{haupteintrag}{}{}
\newenvironment{untereintraege}{}{}
\newcommand*{\lemma}[1]{%
{\textbf{#1}}%
\markboth{#1}{#1}}
\newcommand*{\schreibung}[1]{{#1}}
\newcommand*{\pos}[1]{
{\textbf{#1}}}
\newcommand*{\domaene}[1]{
{\textit{#1}}}
\newcommand*{\uebersetzung}[1]{#1}
% Ab hier beginnt der eigentliche Dokumententext.
\begin{document}

\pagestyle{dictheadings} %% use scrpage or fancyhdr for a differnt layout
\setlength\parsep{0pt}	
\setlength\itemsep{0pt}
\begin{multicols}{2}
\begin{haupteintrag}
\begin{entry}
\lemma{hayai [1]}
\schreibung{早い; はやい [1]}  
\pos{A}
\domaene{}
\uebersetzung{(<POS: Adj.>) früh; <GENKIrK3>frühzeitig; frühmorgens; verfrüht; vor der Zeit.}
\end{entry}
\end{haupteintrag}
\begin{untereintraege}
\begin{entry}
\lemma{hayaiden'sha [1]}
\schreibung{早い電車}  
\pos{}
\domaene{}
\uebersetzung{die ersten Züge<Gen.: mpl>; die erste Bahn<Gen.: f>.}
\end{entry}
\begin{entry}
\lemma{hayakuokiru}
\schreibung{早く起きる}  
\pos{}
\domaene{}
\uebersetzung{früh aufstehen.}
\end{entry}
\begin{entry}
\lemma{hayakuneru}
\schreibung{早く寝る}  
\pos{}
\domaene{}
\uebersetzung{früh schlafen gehen; früh zu Bett gehen.}
\end{entry}
\begin{entry}
\lemma{kimihakekkon'surunohamadahayai。}
\schreibung{君は結婚するのはまだ早い。}  
\pos{}
\domaene{Bsp.}
\uebersetzung{{<Dom.: Bsp.>} Du bist noch zu jung, um zu heiraten.}
\end{entry}
\begin{entry}
\lemma{natsuhayorugaakerunogahayai。}
\schreibung{夏は夜が明けるのが早い。}  
\pos{}
\domaene{Bsp.}
\uebersetzung{{<Dom.: Bsp.>} Im Sommer wird es früh hell.}
\end{entry}
\begin{entry}
\lemma{hayarebahayaihodoyoi。}
\schreibung{早ければ早いほどよい。}  
\pos{}
\domaene{Bsp.}
\uebersetzung{{<Dom.: Bsp.>} Je eher umso besser.}
\end{entry}
\begin{entry}
\lemma{hayaihanashiga}
\schreibung{早い話が}  
\pos{}
\domaene{}
\uebersetzung{kurzum; in einem Wort; um mich kurz zu fassen; um gleich zur Sache zu kommen.}
\end{entry}
\begin{entry}
\lemma{hayakutomo}
\schreibung{早くとも}  
\pos{}
\domaene{}
\uebersetzung{frühestens.}
\end{entry}
\begin{entry}
\lemma{…gahayaika}
\schreibung{…が早いか}  
\pos{}
\domaene{}
\uebersetzung{nicht früher als ….}
\end{entry}
\end{untereintraege}

%\include{testdic}

\end{multicols}
\end{document}

