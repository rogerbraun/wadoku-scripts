% Dieser Befehl bestimmt die Klasse unseres Dokuments, sozusagen die Vorlage. Wir nehmen die Article-Klasse aus dem KOMA-Skript-Projekt. http://www.komascript.de/
\documentclass[a4paper,12pt,bibtotocnumbered,smallheadings,pointednumbers]{scrartcl}
% Genzi (http://kuniyoshi.fastmail.fm/xetex/) von Kazuomi Kuniyoshi hilft bei der automatischen Schriftauswahl beim Wechsel zwischen Japanisch und Deutsch. Standardmäßig nimmt unsere Version davon  Minion als westliche, Kozuka als japanische Font.
%Für Mac die nächste Zeile auskommentieren und die darauffolgende löschen.
% \usepackage[hoefler,hiragino]{genzi}
\usepackage{genzi}
\usepackage{xltxtra}
\usepackage{lexikon}
\usepackage{setspace}
\usepackage[paper=a4paper,left=35mm,right=35mm,top=35mm,bottom=43mm]{geometry} 
\defaultfontfeatures{Mapping=tex-text}
\usepackage{hanging}
\usepackage{multicol}
\usepackage{ragged2e}
\newenvironment{entry}{%
\par\leavevmode\hangpara{1.5mm}{1}\ignorespaces}{\RaggedRight\par}
\setlength{\parindent}{2em}
\newcommand*{\mainentry}[1]{%
{\sffamily\bfseries{#1}}%
\markboth{#1}{#1}}
% Ab hier beginnt der eigentliche Dokumententext.
\begin{document}

\pagestyle{dictheadings} %% use scrpage or fancyhdr for a differnt layout
\begin{multicols}{2}


\begin{entry}
\mainentry{aikidou}
{合気道 (合氣道)}
{  \textit{Budō} Aikidō (n) (elegante Form der Selbstverteidigung; betont Ausweichen, Kreisbewegungen, Würfe und Hebel).}\end{entry}
\begin{entry}
\mainentry{aikidouka}
{合気道家}
{Aikidō·ka (m); Aikidō-Sportler (m).}\end{entry}
\begin{entry}
\mainentry{aikyaku}
{相客}
{ Mitgast (m); Zimmergenosse (m); Stubennachbar (m); Reisegefährte (m); Mitreisender (m); weiterer Gast (m) (jmd., der Zimmer od. Tisch mit einem teilt).}\end{entry}
\begin{entry}
\mainentry{aikyaccha-}
{アイ・キャッチャー; アイキャッチャー}
{Blickfang (m) (von engl. eye-catcher).}\end{entry}
\begin{entry}
\mainentry{aikyan}
{ICANN; アイキャン}
{  \textit{Internet, Org.} ICANN (vergibt Domainnamen und IP-Adressen im Internet; von engl. Internet Corporation for Assigned Names and Numbers).}\end{entry}
\begin{entry}
\mainentry{aikyuu}
{哀求}
{ inständige Bitte (f); Flehen (n).}\end{entry}
\begin{entry}
\mainentry{aikyu-}
{IQ; アイ・キュー; アイキュー}
{  \textit{Psych.} IQ (m); Intelligenz-Quotient (m) (Abk.).}\end{entry}
\begin{entry}
\mainentry{aikyu-gahyakuhachiaru}
{IQが108ある}
{einen IQ von 108 haben.}\end{entry}
\begin{entry}
\mainentry{aikyuusuru}
{哀求する}
{inständig bitten; flehen.}\end{entry}
\begin{entry}
\mainentry{aigyou}
{愛楽}
{ Liebe (f).}\end{entry}
\begin{entry}
\mainentry{aigyou}
{愛ぎょう; あいぎょう (愛敬)}
{ [1] Liebe (f) und Respekt (m); Verehrung (f). [2] Attraktivität (f) (z. B. der Figur oder der Stimme einer Person); Liebreiz (m); Charme (m). [3] Sympathie (f). [4] Beziehung (f) eines Ehepaares; eheliche Harmonie (f).}\end{entry}
\begin{entry}
\mainentry{aikyou }
{哀叫}
{ Aufschreien (n) aus Trauer.}\end{entry}
\begin{entry}
\mainentry{aikyou }
{愛きょう; あいきょう (愛敬 [1]; 愛嬌)}
{ [1] Charme (m); Anmut (f); Attraktivität (f); Liebreiz (m) // einnehmendes Wesen (n); Liebenswürdigkeit (f). [2] etw.NArN, das einen anderen erfreut.}\end{entry}
\begin{entry}
\mainentry{aikyou }
{愛郷}
{ Liebe (f) zur Heimat (⇒ aikyō·shin 愛郷心2203944).}\end{entry}
\begin{entry}
\mainentry{aikyouafurerubakaridearu}
{愛嬌溢れるばかりである}
{vor Liebenswürdigkeit überschäumen.}\end{entry}
\begin{entry}
\mainentry{aikyouge}
{愛嬌毛}
{ ins Gesicht gerutschte Haarsträhne (f); Schmachtlocke (f).}\end{entry}
\begin{entry}
\mainentry{aikyougen}
{あい狂言 (間狂言)}
{  \textit{Theat.} Ai·kyōgen (n); heiteres Zwischenspiel (n) im Nō (⇒ kyōgen 狂言5999961).}\end{entry}
\begin{entry}
\mainentry{aikyoushoubai}
{愛敬商売; 愛嬌商売}
{Gewerbe (n), in dem ein gewinnendes Aussehen wichtig ist.}\end{entry}
\begin{entry}
\mainentry{aikyoushin}
{愛郷心}
{Liebe (f) zur Heimat; Lokalpatriotismus (m).}\end{entry}
\begin{entry}
\mainentry{aikyoushingaaru}
{愛郷心がある}
{seine Heimat lieben.}\end{entry}
\begin{entry}
\mainentry{aikyousuru}
{哀叫する}
{vor Traurigkeit aufschreien.}\end{entry}
\begin{entry}
\mainentry{aikyoutappurida}
{愛嬌たっぷりだ}
{strahlen; mit Liebenswürdigkeiten überschütten.}\end{entry}
\begin{entry}
\mainentry{aikyoutappurini}
{愛敬たっぷりに}
{außerordentlich charmant.}\end{entry}
\begin{entry}
\mainentry{aikyounoaru}
{愛敬のある; 愛嬌のある}
{charmant; liebenswürdig; freundlich; lieblich; anmutig; ansprechend; entzückend; einnehmend.}\end{entry}
\begin{entry}
\mainentry{aikyounoaruegao}
{愛敬のある笑顔}
{charmantes Lächeln (n).}\end{entry}
\begin{entry}
\mainentry{aikyounoarushissaku}
{愛敬のある失策; 愛嬌のある失策}
{liebenswürdiger Fehler (m); lustiger Fehler (m).}\end{entry}
\begin{entry}
\mainentry{aikyounoarujosei}
{愛きょうのある女性}
{charmante Frau (f).}\end{entry}
\begin{entry}
\mainentry{aikyounoii}
{愛嬌のいい}
{charmant; liebreizend; anmutig; nett.}\end{entry}
\begin{entry}
\mainentry{aikyounonai}
{愛敬のない; 愛嬌のない}
{unfreundlich; kurzangebunden; trocken; brüsk; zugeknöpft.}\end{entry}
\begin{entry}
\mainentry{aikyounonaion'na}
{愛敬のない女}
{reizlose Frau (f).}\end{entry}
\begin{entry}
\mainentry{aikyounonaihito}
{愛嬌のない人}
{unfreundliche Person (f).}\end{entry}
\begin{entry}
\mainentry{aikyounonaihenji}
{愛敬のない返事}
{unfreundliche Antwort (f); brüske Erwiderung (f); kurzangebundene Entgegnung (f).}\end{entry}
\begin{entry}
\mainentry{aikyoubi}
{愛敬日; 愛嬌日}
{Gnadenfrist (f); Nachfrist (m) (⇒ onkei·bi 恩恵日3510490).}\end{entry}
\begin{entry}
\mainentry{aikyoubokuro}
{あいきょうぼくろ (愛嬌ぼくろ; 愛敬ぼくろ; 愛敬黒子; 愛嬌黒子)}
{charmanter Leberfleck (m).}\end{entry}
\begin{entry}
\mainentry{aikyoumono}
{愛きょう者; あいきょう者 (愛敬者; 愛嬌者)}
{Clown (m); Spaßmacher (m); Spaßvogel (m); gewitzte Person (f).}\end{entry}
\begin{entry}
\mainentry{aikyoumononosaru}
{愛敬者の猿}
{süßer Affe (m); spaßiger Affe (m).}\end{entry}
\begin{entry}
\mainentry{aikyouwarai}
{愛嬌笑い}
{einnehmendes Lachen (n); gewinnendes Lachen (n); leutseliges Lachen}\end{entry}
\begin{entry}
\mainentry{aikyouwourimononishiteiru}
{愛嬌を売りものにしている}
{mit seinen Reizen hausieren gehen.}\end{entry}
\begin{entry}
\mainentry{aikyouwouru}
{愛嬌を売る}
{jmdm. den Hof machen; poussieren; sich anbiedern.}\end{entry}
\begin{entry}
\mainentry{aikyouwofurimaku}
{愛敬を振りまく; 愛嬌を振りまく; 愛嬌を降りまく; 愛嬌をふりまく (愛敬を振り撒く)}
{seinen Charme verstreuen; versuchen, jedem zu gefallen; seine Liebenswürdigkeit verstrahlen.}\end{entry}
\begin{entry}
\mainentry{aigiri}
{間切り; 間切}
{  \textit{Musikinstr.} Aigiri (f) (eine Bambusflöte).}\end{entry}
\begin{entry}
\mainentry{aigin }
{哀吟}
{ [1] trauriges Singen (n) eines Liedes. [2] trauriges Lied (n).}\end{entry}
\begin{entry}
\mainentry{aigin }
{愛吟}
{ (poet.) Lieblingsgedicht (n); Lieblingsgesang (m).}\end{entry}
\begin{entry}
\mainentry{aigin }
{合い銀; 合銀; 間銀}
{ [1] Differenz (f) zwischen dem gegenwärtigen Preis und dem Verkaufspreis. [2] Gebühr (f) für eine Transaktion.}\end{entry}
\begin{entry}
\mainentry{aikinka}
{愛きん家 (愛禽家)}
{ Vogelliebhaber (m) (⇒ aichō·ka 愛鳥家8290530).}\end{entry}
\begin{entry}
\mainentry{aiginshuu}
{愛吟集}
{Sammlung (f) jmds. Lieblingsgedichte.}\end{entry}
\begin{entry}
\mainentry{aiginsuru}
{愛吟する}
{es lieben, zu rezitieren; gerne rezitieren.}\end{entry}
\begin{entry}
\mainentry{aiginsurushi}
{愛吟する詩}
{Lieblingsgedicht (n).}\end{entry}
\begin{entry}
\mainentry{aiku}
{アイク}
{  \textit{Persönlichk.} Hubert van Eyck (niederl. Maler; 1380?–1441).}\end{entry}
\begin{entry}
\mainentry{aikuouzan}
{阿育王山}
{  \textit{Bergn.} Ayuwangshan (m); (japan.) Aikuō-zan (m) (Berg in der Prov. Zhejiang östlich von Ningbo).}\end{entry}
\begin{entry}
\mainentry{aikugi}
{合くぎ (合い釘; 合釘; 間釘)}
{ Dübel (m); Leistenstift (m); Nagel (m) mit zwei Spitzen.}\end{entry}
\begin{entry}
\mainentry{aigusa}
{藍草}
{ \textit{Bot.} Indigopflanze (f).}\end{entry}
\begin{entry}
\mainentry{aigusuri}
{合い薬; 合薬 [1]}
{  \textit{Med.} (ugs.) Spezifikum (n) (für die Konstitution eines bestimmten Menschen angepasste Medizin).}\end{entry}
\begin{entry}
\mainentry{aikuchi }
{あいくち (合い口; 合口 [1]; 匕首 [1])}
{ Dolch (m); Stilett (n) (ohne Stichblatt; ⇒ hishu 匕首4395051).}\end{entry}
\begin{entry}
\mainentry{aikuchi }
{合い口; 合口 [2]; 相口}
{ [1] gutes Zusammenpassen (n) (wie Topf und Deckel). [2] Sympathie (f); Kongenialität (f); Geistesverwandtschaft (f); Wesensähnlichkeit (f); Zuneigung (f). (⇒ aishō 相性5030676).}\end{entry}
\begin{entry}
\mainentry{aikuchidetsukisasu}
{匕首で突き刺す}
{mit dem Dolch erstechen.}\end{entry}
\begin{entry}
\mainentry{aikuchiwonomu}
{匕首をのむ}
{einen Dolch versteckt (unter den Kleidern) haben.}\end{entry}
\begin{entry}
\mainentry{aikuchiwonondeiru}
{合口を呑んでいる; 匕首を呑んでいる}
{einen Dolch versteckt tragen.}\end{entry}
\begin{entry}
\mainentry{aiguma }
{あいぐま (藍隈)}
{  \textit{Kabuki} blaues Schminken (n) des Gesichtes // blau geschminktes Gesicht (n) eines Geistes oder bösen Adeligen.}\end{entry}
\begin{entry}
\mainentry{aiguma }
{藍隈}
{blaue Theaterschminke (f) (beim Kabuki).}\end{entry}
\begin{entry}
\mainentry{aikurushii}
{愛くるしい; あいくるしい}
{ reizend; süß; niedlich; nett; entzückend; lieb; herzig; goldig.}\end{entry}
\begin{entry}
\mainentry{aikurushiihohoemi}
{愛くるしい微笑み}
{reizendes Lächeln (n); gewinnendes Lächeln (n).}\end{entry}
\begin{entry}
\mainentry{aikurushiime}
{愛くるしい目}
{reizende Augenfpl.}\end{entry}
\begin{entry}
\mainentry{aikurushigeda}
{愛くるしげだ}
{süß sein; reizend sein.}\end{entry}
\begin{entry}
\mainentry{aikurushisa}
{愛くるしさ}
{Liebligkeit (f); Schönheit (f); Reiz (m).}\end{entry}
\begin{entry}
\mainentry{aikei }
{愛恵}
{ Liebkosen (n); liebevolle Behandlung (f); Sorge (f).}\end{entry}
\begin{entry}
\mainentry{aikei }
{愛敬 [2]}
{ von Herzen kommender Respekt (m); Liebe (f) und Respekt (m); Verehrung (f) (⇒   愛敬4179744).}\end{entry}
\begin{entry}
\mainentry{aikeisuru}
{愛敬する}
{Liebe und Respekt zeigen; lieben und respektieren.}\end{entry}
\begin{entry}
\mainentry{aike-efu}
{IKF; アイ・ケー・エフ; アイケーエフ}
{  \textit{Sport} Internationaler Kendō-Verband (m).}\end{entry}
\begin{entry}
\mainentry{aiken }
{愛犬}
{ [1] Schoßhund (m); Gesellschaftshund (m) // Hund (m), den man liebt. [2] Liebe (f) zu Hunden; Begeisterung (f) für Hunde.}\end{entry}
\begin{entry}
\mainentry{aiken }
{合い拳; 合拳}
{ Unentschieden bei Schere, Papier und Stein.}\end{entry}
\begin{entry}
\mainentry{aikenka}
{愛犬家}
{Hundefreund (m); Hundenarr (m).}\end{entry}
\begin{entry}
\mainentry{aiko }
{あいこ; 相こ (相子)}
{ [1] (ugs.) Unentschieden (n); Remis (n). [2] (ugs.) Ausgeglichenheit (f); Ebenbürtigkeit (f); Gleichwertigkeit (f). (⇒ hiki·wake 引き分け7600180).}\end{entry}
\begin{entry}
\mainentry{aigo }
{あいご; アイゴ (藍子)}
{  \textit{Fischk.} Aigo (m) (Siganus fuscescens).}\end{entry}
\begin{entry}
\mainentry{aiko }
{愛顧}
{ (schriftspr.) Gunst (f); freundliche Gesinnung (f); Gewogenheit (f); Begünstigung (f); Patronage (f) (⇒ onko 恩顧5542668).}\end{entry}
\begin{entry}
\mainentry{aigo }
{愛護}
{ Schutz (m); Hegen (n) und Pflegen (n).}\end{entry}
\begin{entry}
\mainentry{aiko }
{愛子 [1]}
{  \textit{weibl. Name} Aiko.}\end{entry}
\begin{entry}
\mainentry{aigo }
{相碁}
{  \textit{Go, Shōgi} Spiel (n) zweier gleich starker Spieler (die sich gleich ab dem Eröffnungszug mit den Zügen abwechseln; ⇒ tagai·sen 互い先9758292).}\end{entry}
\begin{entry}
\mainentry{aigou}
{哀号}
{ [1] Weinen (n) aus Trauer. [2] verweinte Stimme (f).}\end{entry}
\begin{entry}
\mainentry{aikou }
{愛好}
{ Liebe (f); Vorliebe (f); Geschmack (m).}\end{entry}
\begin{entry}
\mainentry{aikou }
{愛校}
{ Liebe (f) zur eigenen Schule.}\end{entry}
\begin{entry}
\mainentry{aikou }
{愛甲}
{  \textit{Gebietsn.} AikōnNAr (Gemeinde in der Präf. Kanagawa).}\end{entry}
\begin{entry}
\mainentry{aikouka}
{愛好家}
{Liebhaber (m); Amateur (m); Fan (m).}\end{entry}
\begin{entry}
\mainentry{aikousha}
{愛好者}
{ Liebhaber (m); Amateur (m); Schwärmer (m); Fan (m); Enthusiast (m).}\end{entry}
\begin{entry}
\mainentry{aikoushin}
{愛校心}
{ Liebe (f) zu seiner alten Schule; Verbundenheit (f) zur alten Schule; Liebe (f) zur Alma Mater.}\end{entry}
\begin{entry}
\mainentry{aikousuru}
{愛好する}
{lieben; schätzen.}\end{entry}
\begin{entry}
\mainentry{aigousuru}
{哀号する}
{aus Trauer weinen.}\end{entry}
\begin{entry}
\mainentry{aikouno}
{愛好の}
{geliebt; beliebt.}\end{entry}
\begin{entry}
\mainentry{aikoou}
{相呼応}
{ Sprechen (n) im gegenseitigen Einverständnis; Handeln (n) im gegenseitigen Einverständnis.}\end{entry}
\begin{entry}
\mainentry{aikoousuru}
{相呼応する}
{im gegenseitigen Einverständnis sprechen; im gegenseitigen Einverständnis handeln.}\end{entry}
\begin{entry}
\mainentry{aikoku }
{哀こく (哀哭)}
{ Jammern (n); Klagen (n); Schluchzen (n).}\end{entry}
\begin{entry}
\mainentry{aikoku }
{愛国}
{ Vaterlandsliebe (f); Patriotismus (m) (⇒ aikoku·shin 愛国心6635834; ⇒ yūkoku 憂国1288630).}\end{entry}
\begin{entry}
\mainentry{aikokuundou}
{愛国運動}
{patriotische Bewegung (f).}\end{entry}
\begin{entry}
\mainentry{aikokuka}
{愛国歌}
{patriotisches Lied (n).}\end{entry}
\begin{entry}
\mainentry{aikokukyou}
{愛国狂}
{Chauvinismus (m).}\end{entry}
\begin{entry}
\mainentry{aikokukousai}
{愛国公債}
{patriotische Anleihe (f).}\end{entry}
\begin{entry}
\mainentry{aikokusha}
{愛国者}
{Patriot (m); Chauvinist (m).}\end{entry}
\begin{entry}
\mainentry{aikokushugi}
{愛国主義}
{Patriotismus (m); Nationalismus (m) (vor dem Krieg im Sinne des übersteigerten „Nationalismus“ benutzt und klingt deshalb auch heute noch etwas chauvinistisch).}\end{entry}
\begin{entry}
\mainentry{aikokushin}
{愛国心}
{ Vaterlandsliebe (f); Patriotismus (m); patriotische Gesinnung (f).}\end{entry}
\begin{entry}
\mainentry{aikokushinganiwakanibokkishita。}
{愛国心がにわかに勃起した。}
{ \textit{Bsp.} Der Patriotismus erhebt sich auf einmal.}\end{entry}
\begin{entry}
\mainentry{aikokushin'noaruhito}
{愛国心のある人}
{patriotische Person (f).}\end{entry}
\begin{entry}
\mainentry{aikokushin'nonaihito}
{愛国心の無い人}
{wenig patriotische Person (f).}\end{entry}
\begin{entry}
\mainentry{aikokushinwoaoru}
{愛国心を煽る}
{den Patriotismus in jmdm. anfachen.}\end{entry}
\begin{entry}
\mainentry{aikokushinwokosuisuru}
{愛国心を鼓吹する}
{den Patriotismus anstacheln.}\end{entry}
\begin{entry}
\mainentry{aikokushinwotsuchikau}
{愛国心を培う}
{den Patriotismus fördern.}\end{entry}
\begin{entry}
\mainentry{aikokushinwomotte}
{愛国心をもって}
{patriotisch; mit patriotischer Gesinnung.}\end{entry}
\begin{entry}
\mainentry{aikokusuru}
{哀哭する}
{jammern; klagen; schluchzen.}\end{entry}
\begin{entry}
\mainentry{aikokudantai}
{愛国団体}
{patriotischer Verein (m).}\end{entry}
\begin{entry}
\mainentry{aikokuteki}
{愛国的}
{patriotisch; vaterländisch.}\end{entry}
\begin{entry}
\mainentry{aikokutekiseishin}
{愛国的精神}
{Patriotismus (m).}\end{entry}
\begin{entry}
\mainentry{aikokutekina}
{愛国的な}
{patriotisch; vaterländisch.}\end{entry}
\begin{entry}
\mainentry{aikokuno}
{愛国の}
{patriotisch.}\end{entry}
\begin{entry}
\mainentry{aikokunoshishi}
{愛国の志士}
{Patriot (m).}\end{entry}
\begin{entry}
\mainentry{aikokunojou}
{愛国の情}
{patriotisches Gefühl (n); Vaterlandsliebe (f); Patriotismus (m) (⇒ aikoku·shin 愛国心6635834).}\end{entry}
\begin{entry}
\mainentry{aikokunonetsujou}
{愛国の熱情}
{patriotische Leidenschaft (f).}\end{entry}
\begin{entry}
\mainentry{aikokufujinkai}
{愛国婦人会}
{ \textit{Gesch.} vaterländischer Frauenverein (m).}\end{entry}
\begin{entry}
\mainentry{aigoshi}
{相ごし (相輿)}
{ gemeinsame Benutzung (f) einer Sänfte.}\end{entry}
\begin{entry}
\mainentry{aikosuru}
{愛顧する}
{gewogen sein; protegieren; begünstigen; jmdm. wohl wollen.}\end{entry}
\begin{entry}
\mainentry{aigosuru}
{愛護する}
{hegen; pflegen; hüten; schützen; freundlich behandeln; Sorge tragen.}\end{entry}
\begin{entry}
\mainentry{aikotto}
{ICOT; アイコット}
{  \textit{Org.} ICOT (n); Institute (n) for New Generation Computer Technology.}\end{entry}
\begin{entry}
\mainentry{aikotonaru}
{相異なる}
{ unterschiedlich sein; sich unterscheiden; differieren.}\end{entry}
\begin{entry}
\mainentry{aikotonaruiken}
{相異なる意見}
{unterschiedliche Meinungenfpl.}\end{entry}
\begin{entry}
\mainentry{aikotonarutachiba}
{相異なる立場}
{unterschiedliche Standpunktempl.}\end{entry}
\begin{entry}
\mainentry{aikotoba}
{合い言葉; 合言葉; 合いことば; 合ことば (合い詞; 相言葉)}
{ [1] Passwort (n); Parole (f). [2] Motto (n); Slogan (m).}\end{entry}
\begin{entry}
\mainentry{aikotobadehanasu}
{合い言葉で話す}
{im Jargon sprechen; einen Geheimcode sprechen; in Parolen sprechen.}\end{entry}
\begin{entry}
\mainentry{aikotobanisuru}
{合い言葉にする}
{[1] etwas zur Parole machen; ein Passwort festsetzen. [2] etwas zu seinem Motto machen; versuchen, ein selbstgestecktes Ziel zu erreichen.}\end{entry}
\begin{entry}
\mainentry{aikotobawoiu}
{合い言葉を言う; 合言葉を言う; 合い言葉をいう}
{das Passwort sagen; die Parole sagen.}\end{entry}
\begin{entry}
\mainentry{aikona-ru}
{アイコナール}
{  \textit{Optik} Eikonal (n).}\end{entry}
\begin{entry}
\mainentry{aikona-ruhouteishiki}
{アイコナール方程式}
{ \textit{Optik} Eikonal-Funktion (f).}\end{entry}
\begin{entry}
\mainentry{aikonikotaeru}
{愛顧に応える}
{sich jmds. Patronage würdig erweisen; eine Gefälligkeit erwidern.}\end{entry}
\begin{entry}
\mainentry{aikonisuru}
{相子にする}
{ausgleichen.}\end{entry}
\begin{entry}
\mainentry{aikoninaru}
{あいこになる; 相子になる}
{[1] unentschieden enden; mit einem Remis zu Ende gehen; quitt sein. [2] ebenbürtig sein; gleichwertig sein.}\end{entry}
\begin{entry}
\mainentry{aikonifai}
{アイコニファイ}
{ Ikonifizierung (f) (von engl. iconify).}\end{entry}
\begin{entry}
\mainentry{aikonimukuiru}
{愛顧に報いる}
{sich jmds. Patronage würdig erweisen; eine Gefälligkeit erwidern.}\end{entry}
\begin{entry}
\mainentry{aikonokurasuto}
{アイコノクラスト}
{ Ikonoklast (f); Bilderstürmer (m).}\end{entry}
\begin{entry}
\mainentry{aikonokurazumu}
{アイコノクラズム}
{ Bildersturm (m); Ikonoklasmus (m) (von engl. iconoclasm).}\end{entry}
\begin{entry}
\mainentry{aikonosuko-pu}
{アイコノスコープ}
{  \textit{TV} Ikonoskop (n); Bildspeicherröhre (f) (von engl. iconoscope).}\end{entry}
\begin{entry}
\mainentry{aikonosutashisu}
{アイコノスタシス}
{  \textit{griech. Christent.} Ikonostase (f) (mit Ikonen bedeckte, von Türen durchbrochene Wand, die Altar‑ und Gemeinderaum voneinander trennt).}\end{entry}
\begin{entry}
\mainentry{aigoma}
{合いごま; 合ごま (合い駒; 合駒; 間駒)}
{  \textit{Shōgi} Stein (m), der so gesetzt wird, dass ein Schach des Gegners verhindert wird.}\end{entry}
\begin{entry}
\mainentry{aikowoukeru}
{愛顧を受ける}
{bei jmdm. in Gunst stehen.}\end{entry}
\begin{entry}
\mainentry{aikowoushinau}
{愛顧を失う}
{jmds. Gewogenheit verlieren.}\end{entry}
\begin{entry}
\mainentry{aikowoeru}
{愛顧を得る}
{sich jmdn. gewogen machen; jmds. Gunst erwerben.}\end{entry}
\begin{entry}
\mainentry{aikowokou}
{愛顧を請う (愛顧を乞う)}
{jmdn. um seine Gunst bitten; jmdn. um weitere Gewogenheit bitten.}\end{entry}
\begin{entry}
\mainentry{aikon}
{アイコン}
{ [1]  \textit{EDV} Icon (n) (grafische Sinnbilder einer grafischen Computer-Benutzeroberfläche). [2] Ikone (f) (geweihtes Tafelbild der orthodoxen Kirche).}\end{entry}
\begin{entry}
\mainentry{aikontakuto}
{アイ・コンタクト; アイコンタクト}
{ Blickkontakt (m) (von engl. eye contact).}\end{entry}
\begin{entry}
\mainentry{aikonwoikkaikurikkusuru}
{アイコンを1回クリックする}
{ \textit{EDV} einmal auf ein Icon klicken.}\end{entry}
\begin{entry}
\mainentry{aikonwosonoforuda-madedoraggushitedoroppusuru}
{アイコンをそのフォルダーまでドラッグしてドロップする}
{ \textit{EDV} ein Icon auf einen Ordner ziehen und loslassen.}\end{entry}
\begin{entry}
\mainentry{aisa }
{ISA; アイサ [1]}
{  \textit{EDV} ISA (f) (Abk. für engl. Industry Standard Architecture).}\end{entry}
\begin{entry}
\mainentry{aisa }
{あいさ; アイサ [2] (秋沙)}
{  \textit{Vogelk.} Säger (m) (Mergus; gänsegroße Entenart).}\end{entry}
\begin{entry}
\mainentry{aisai}
{愛妻}
{ [1] jmds. über alles geliebte Frau (f); jmds. bessere Hälfte (f). [2] Liebe zur Ehefrau (f).}\end{entry}
\begin{entry}
\mainentry{aisaika}
{愛妻家}
{zärtlicher Ehemann (m); treu hingebungsvoller Mann (m).}\end{entry}
\begin{entry}
\mainentry{aisaibentou}
{愛妻弁当}
{von der Ehefrau liebevoll vorbereitetes Bentō (n).}\end{entry}
\begin{entry}
\mainentry{aisaibou}
{I細胞}
{ I-Zelle (f); Einschlusskörperchen (n) (engl. inclusion cell).}\end{entry}
\begin{entry}
\mainentry{aisaiboubyou}
{I細胞病}
{ \textit{Med.} I-Zellkrankheitf; Mukolipidose II (eine Stoffwechselstörung).}\end{entry}
\begin{entry}
\mainentry{aizakari}
{愛盛り; 愛盛; 愛ざかり}
{ höchster Liebreiz (m) (insbes. in Bez. auf Kinder).}\end{entry}
\begin{entry}
\mainentry{aisaku}
{間作 [1]; 相作}
{ gemischter Anbau (m) verschiedener Feldfrüchte.}\end{entry}
\begin{entry}
\mainentry{aisakusuru}
{間作する}
{verschiedene Feldfrüchte zwischen einander anbauen.}\end{entry}
\begin{entry}
\mainentry{aisatsu}
{あいさつ (挨拶; あい拶)}
{ [1] Gruß (m); Ansprache (f); Begrüßungsrede (f). [6] Entschuldigung<sp}\end{entry}
\begin{entry}
\mainentry{aisatsugawari}
{挨拶代わり}
{Erwiderung (f) eines Grußes; Geste (f) der Freundschaft.}\end{entry}
\begin{entry}
\mainentry{aisatsugawarini}
{挨拶代わりに}
{als Gruß; als Geste der Freundschaft // (ironisch) um einem Gegner seine Fähigkeiten zu zeigen.}\end{entry}
\begin{entry}
\mainentry{aisatsugiru}
{挨拶切る}
{ alle Kontakte mit jmdm. abbrechen.}\end{entry}
\begin{entry}
\mainentry{aisatsujou}
{挨拶状}
{Grußkarte (f); Kartengruß (m); Gruß (m).}\end{entry}
\begin{entry}
\mainentry{aisatsusuru}
{あいさつする (挨拶する)}
{[1] begrüßen; grüßen; seine Verbeugen machen; salutieren. [2] sich entschuldigen; antworten.}\end{entry}
\begin{entry}
\mainentry{aisatsunashini}
{挨拶なしに}
{[1] ohne Gruß. [2] ohne Nachricht.}\end{entry}
\begin{entry}
\mainentry{aisatsuniiku}
{挨拶に行く}
{einen Besuch abstatten.}\end{entry}
\begin{entry}
\mainentry{aisatsunin}
{挨拶人}
{Vermittler (m); Schlichter (m).}\end{entry}
\begin{entry}
\mainentry{aisatsunokotobawokakeru}
{挨拶の言葉をかける}
{jmdn. grüßen.}\end{entry}
\begin{entry}
\mainentry{aisatsunoshikata}
{挨拶の仕方}
{Etikette (f); gesellschaftliche Umgangsformenfpl.}\end{entry}
\begin{entry}
\mainentry{aisatsumawari}
{挨拶まわり}
{Höflichkeitsbesuch (m) zu Neujahr.}\end{entry}
\begin{entry}
\mainentry{aisatsumawariwosuru}
{挨拶回りをする}
{Höflichkeitsbesuche machen.}\end{entry}
\begin{entry}
\mainentry{aisatsumosezuni}
{挨拶もせずに}
{ohne Gruß.}\end{entry}
\begin{entry}
\mainentry{aisatsumosokosokoni}
{挨拶もそこそこに}
{nach einer hastigen Verabschiedung.}\end{entry}
\begin{entry}
\mainentry{aisatsumono}
{挨拶者}
{Grußredner (m) (z. B. bei einer Hochzeit).}\end{entry}
\begin{entry}
\mainentry{aisatsuwookuru}
{挨拶を送る}
{einen Gruß senden.}\end{entry}
\begin{entry}
\mainentry{aisatsuwokaesu}
{挨拶を返す}
{jmds. Gruß erwidern.}\end{entry}
\begin{entry}
\mainentry{aisatsuwokawasu}
{あいさつを交わす (挨拶を交わす; 挨拶をかわす)}
{Grüße austauschen; sich vor einander verbeugen.}\end{entry}
\begin{entry}
\mainentry{aisatsuwonukinishite}
{挨拶を抜きにして}
{den Gruß auslassend.}\end{entry}
\begin{entry}
\mainentry{aisabasu}
{ISAバス}
{ \textit{EDV} ISA-Bus (m).}\end{entry}
\begin{entry}
\mainentry{aisabi}
{あいさび (藍錆)}
{ [1] rostiges Blau (n). [2] rostigblau gefärbter Stoff (m).}\end{entry}
\begin{entry}
\mainentry{aizaburou }
{愛三郎}
{  \textit{männl. Name} Aizaburō.}\end{entry}
\begin{entry}
\mainentry{aizaburou }
{相三郎}
{  \textit{männl. Name} Aizaburō.}\end{entry}
\begin{entry}
\mainentry{aisamu}
{ISAM; アイサム}
{  \textit{EDV} ISAM (f) (Abk. für engl. indexed sequential access method).}\end{entry}
\begin{entry}
\mainentry{aisaresouninai}
{愛されそうにない}
{nicht liebenswürdig.}\end{entry}
\begin{entry}
\mainentry{aisareteinai}
{愛されていない}
{ungeliebt.}\end{entry}
\begin{entry}
\mainentry{aisareteiru}
{愛されている}
{geliebt werden.}\end{entry}
\begin{entry}
\mainentry{aisareyoutosuru}
{愛されようとする}
{versuchen, jmds. Liebe zu gewinnen.}\end{entry}
\begin{entry}
\mainentry{aizawa }
{会沢}
{  \textit{Familienn.} Aizawa.}\end{entry}
\begin{entry}
\mainentry{aizawa }
{相沢}
{  \textit{Familienn.} AizawaNAr.}\end{entry}
\begin{entry}
\mainentry{aizawa }
{相澤}
{  \textit{Familienn.} Aizawa.}\end{entry}
\begin{entry}
\mainentry{aisan}
{愛さん (愛餐)}
{  \textit{Christent.} Agape (f); abendliches Mahl (n) der frühen Christen (⇒ a’gapē アガペー1540200).}\end{entry}
\begin{entry}
\mainentry{aisankai}
{愛餐会}
{ \textit{Christent.} Agape (f); abendliches Mahl (n) der frühen Christen.}\end{entry}
\begin{entry}
\mainentry{aishi }
{哀史}
{ (schriftspr.) traurige Geschichte (f); Geschichte (f), in der von traurigen Vorfällen berichtet wird (⇒ hiwa 悲話3361123).}\end{entry}
\begin{entry}
\mainentry{aiji }
{愛児 [1]}
{ jmds. geliebtes Kind (n); Lieblingskind (n) (⇒ chōji 寵児5612268).}\end{entry}
\begin{entry}
\mainentry{aishi }
{哀思}
{ trauriges Gefühl (n).}\end{entry}
\begin{entry}
\mainentry{aiji }
{愛次}
{  \textit{männl. Name} Aiji.}\end{entry}
\begin{entry}
\mainentry{aishi }
{哀詞}
{ Gefühl (n) der Trauer (in Poesie verwendet).}\end{entry}
\begin{entry}
\mainentry{aiji }
{愛治}
{  \textit{männl. Name} Aiji.}\end{entry}
\begin{entry}
\mainentry{aishi }
{哀詩}
{ Elegie (f); Trauergedicht (n).}\end{entry}
\begin{entry}
\mainentry{aiji }
{愛二}
{  \textit{männl. Name} Aiji.}\end{entry}
\begin{entry}
\mainentry{aishi }
{愛子 [2]}
{ [1] geliebtes Kind (n). [2] Liebe zu einem Kind (n).}\end{entry}
\begin{entry}
\mainentry{aiji }
{愛冶}
{  \textit{männl. Name} Aiji.}\end{entry}
\begin{entry}
\mainentry{aishi }
{間紙 [b]}
{ Zwischenlegpapier (n) (um zu verhindern, dass Dinge sich gegenseitig beschädigen; auch ai·gami 間紙7381571).}\end{entry}
\begin{entry}
\mainentry{aishiaisaretaitoiukimochi}
{愛し愛されたいという気持}
{Gefühl (n), zu lieben und geliebt zu werden.}\end{entry}
\begin{entry}
\mainentry{aishiau}
{愛し合う; 愛しあう}
{ jmdn. den Hof machen; um die Liebe werben.}\end{entry}
\begin{entry}
\mainentry{aiji-}
{IG; アイ・ジー; アイジー}
{  \textit{Biochem.} Immunglobulin (n); (Abk.) Ig (n) (Protein, mit den Eigenschaften eines Antikörpers).}\end{entry}
\begin{entry}
\mainentry{aishi- }
{IC; アイ・シー [1]; アイシー [1]}
{  \textit{EDV} IC (m); integrierter Schaltkreis (m) (Akronym für engl. integrated circuit; ⇒ shūseki·kairo 集積回路3273087).}\end{entry}
\begin{entry}
\mainentry{aishi- }
{アイ・シー [2]; アイシー [2]}
{ Ich verstehe (von engl. I see).}\end{entry}
\begin{entry}
\mainentry{aishi-a-rushi-}
{ICRC; アイ・シー・アール・シー; アイシーアールシー}
{  \textit{Org.} Internationale Komitee (n) vom Roten Kreuz; (Abk.) IKRK (n) (Abk. für engl. International Committee of the Red Cross).}\end{entry}
\begin{entry}
\mainentry{aishi-a-rupi-}
{ICRP; アイ・シー・アール・ピー; アイシーアールピー}
{  \textit{Org.} Internationalen Strahlenschutzkommission (f) (von engl. International Commission on Radiological Protection).}\end{entry}
\begin{entry}
\mainentry{aiji-efu}
{IGF; アイ・ジー・エフ; アイジーエフ}
{  \textit{Biol.} insulinartiger Wachstumsfaktor (m) (Abk. für engl. insulin-like growth factor).}\end{entry}
\begin{entry}
\mainentry{aishi-efutexi-yu-}
{ICFTU; アイ・シー・エフ・ティー・ユー; アイシーエフティーユー}
{  \textit{Org.} Internationaler Bund (m) Freier Gewerkschaften; IBFG (f) (Abk. für engl. International Confederation of Free Trade Unions; ICFTU).}\end{entry}
\begin{entry}
\mainentry{aishi-emu}
{ICM; アイ・シー・エム; アイシーエム}
{  \textit{Org.} Internationaler Mathematikerkongress (m) (Abk. für engl. International Congress of Mathematicians).}\end{entry}
\begin{entry}
\mainentry{aiji-o-}
{IGO; アイ・ジー・オー; アイジーオー}
{ zwischenstaatliche Organisation (f) (Abk. für engl. Intergovernmental Organization).}\end{entry}
\begin{entry}
\mainentry{aishi-ka-do}
{ICカード; アイ・シー・カード; アイシーカード}
{  \textit{EDV} IC-Karte (f) (Karte mit einem integrierten Schaltkreis zum Beispiel als Speichermedium, Schlüssel etc.).}\end{entry}
\begin{entry}
\mainentry{aishi-shi-}
{ICC; アイ・シー・シー; アイシーシー}
{  \textit{Org.} Internationale Handelskammer (f); (Abk.) IHK (f) (Abk. für engl. International Chamber of Commerce).}\end{entry}
\begin{entry}
\mainentry{aishi-shi-pi-}
{ICCP; アイ・シー・シー・ピー; アイシーシーピー}
{  \textit{Org.} Komitee (n) für Informations‑, Computer‑ und Kommunikationstechnologiepolitik; Information, Computer and Communications Policy Committee (der OECD).}\end{entry}
\begin{entry}
\mainentry{aishi-je-}
{ICJ; アイ・シー・ジェー; アイシージェー}
{  \textit{Org.} Internationale Juristenkommission (f) (Abk. für engl. International Commission of Jurists).}\end{entry}
\begin{entry}
\mainentry{aishi-tagu}
{ICタグ; アイ・シー・タグ; アイシータグ}
{  \textit{Elektrot.} RFID-Chip (m); RFID-Transponder (m) (Transponder, der z. B. einen Nummerncode gespeichert hat und zur Identifikation von Gegenständen, Personen oder Tieren dient).}\end{entry}
\begin{entry}
\mainentry{aishi-texi-esu}
{ICTS}
{  \textit{Verkehr} Transitsystem (n) mittlerer Größe (Abk. für engl. intermediate capacity transit system). }\end{entry}
\begin{entry}
\mainentry{aishi-bi-emu}
{ICBM; アイ・シー・ビー・エム; アイシービーエム}
{  \textit{Milit.} Interkontinentalrakete (f); ICBM (f) (Akronym für engl. intercontinental ballistic missile).}\end{entry}
\begin{entry}
\mainentry{aishi-pi-o-}
{ICPO; アイ・シー・ピー・オー; アイシーピーオー}
{  \textit{Org.} Interpol (f); Internationale Kriminalpolizeiliche Organisation (f) (Abk. für engl. International Criminal Police Organization).}\end{entry}
\begin{entry}
\mainentry{aishi-bi-pi-}
{ICBP; アイ・シー・ビー・ピー; アイシービーピー}
{  \textit{Org.} Internationaler Rat (m) für den Vogelschutz (Abk. für engl. International Council for Bird Preservation).}\end{entry}
\begin{entry}
\mainentry{aishi-pi-yu-e-i-}
{ICPUAE; アイ・シー・ピー・ユー・エー・イー; アイシーピーユーエーイー}
{  \textit{Org.} Internationale Konferenz (f) für die friedliche Nutzung der Kernenergie (Abk. für engl. International Conference on the Peaceful Uses of Atomic Energy).}\end{entry}
\begin{entry}
\mainentry{aishi-yu- }
{ICU [1]; アイ・シー・ユー [1]; アイシーユー [1]}
{  \textit{Med.} Intensivstation (f) (Abk. für engl. intensive care unit; Abk.).}\end{entry}
\begin{entry}
\mainentry{aishi-yu- }
{ICU [2]; アイ・シー・ユー [2]; アイシーユー [2]}
{  \textit{Ortsn.} International Christian University (f); ICU (f) (eine der wichtigeren nicht-staatlichen Unis in Tōkyō, Mitaka; Abk.; ⇒ Kokusai·kirisutokyō·daigaku 国際基督教大学0642687).}\end{entry}
\begin{entry}
\mainentry{aishi-yu-range-jisaienshizusama-insutexichu-to}
{ICU Language Sciences Summer Institute; アイ・シー・ユー・ランゲージ・サイエンシズ・サマー・インスティチュート; アイシーユーランゲージサイエンシズサマーインスティチュート}
{  \textit{Verlagsn.} ICU Language Sciences Summer Institute (Tōkyō, Mitaka).}\end{entry}
\begin{entry}
\mainentry{aishi-ryoken}
{IC旅券}
{Reisepass (m) mit integriertem Chip.}\end{entry}
\begin{entry}
\mainentry{aishi-ro-do}
{アイシー・ロード; アイシーロード}
{ vereiste Straße (von engl. icy road).}\end{entry}
\begin{entry}
\mainentry{aishixeido}
{アイ・シェイド; アイシェイド}
{ Visier (n); Schirm (m) einer Mütze (von engl. eye-shade).}\end{entry}
\begin{entry}
\mainentry{aije-efu}
{IJF; あい・ジェー・エフ; あいジェーエフ}
{  \textit{Org.} Internationale Judo-Föderation (f) (von engl. International Judo Federation).}\end{entry}
\begin{entry}
\mainentry{aishixe-do}
{アイ・シェード; アイシェード}
{ Visier (n); Schirm (m) einer Mütze (von engl. eye-shade).}\end{entry}
\begin{entry}
\mainentry{aishiki}
{亜意識}
{  \textit{Psych.} Unterbewusstsein (n) (sehr selten; ⇒ senzai·ishiki 潜在意識5760357).}\end{entry}
\begin{entry}
\mainentry{aishisugiru}
{愛し過ぎる; 愛しすぎる}
{zu sehr lieben.}\end{entry}
\begin{entry}
\mainentry{aishitsu}
{合い室; 合室; 相い室; 相室}
{ [1] gemeinsames Benutzen (n) eines Zimmers. [2] gemeinsam benutztes Zimmer (n).}\end{entry}
\begin{entry}
\mainentry{aijitsu}
{愛日}
{ [1] Sonnenschein (m) im Winter. [2] Geizen (n) mit der Zeit. [3] Geizen (n) mit der Zeit, um seine Kindspflicht zu erfüllen // Erfüllung (f) der Kindespflichten.}\end{entry}
\begin{entry}
\mainentry{aishiteiru}
{愛している}
{verliebt sein; lieben.}\end{entry}
\begin{entry}
\mainentry{aishitene。}
{愛してね。}
{ \textit{Bsp.} Hab mich lieb!}\end{entry}
\begin{entry}
\mainentry{aishiteru。}
{愛してる。}
{ \textit{Bsp.} Ich liebe Dich!}\end{entry}
\begin{entry}
\mainentry{aisha }
{愛社}
{ Liebe (f) zur eigenen Firma; Loyalität (f) gegenüber der eigenen Firma.}\end{entry}
\begin{entry}
\mainentry{aisha }
{愛車}
{ jmds. eigenes Auto (n); jmds. geliebtes Auto (n); Lieblingsauto (n).}\end{entry}
\begin{entry}
\mainentry{aisha }
{間遮}
{  \textit{Shōgi} Stein (m), der so gesetzt wird, dass ein Schach des Gegners verhindert wird (⇒ ai·goma 合駒0145871).}\end{entry}
\begin{entry}
\mainentry{aishaina-}
{アイ・シャイナー; アイシャイナー}
{  \textit{Kosmetik} Eyeliner (m) (von japan.-engl. eye shiner).}\end{entry}
\begin{entry}
\mainentry{aijaku }
{愛着 [a]; 愛著}
{ [1]  \textit{Buddh.} Begierde (f); Sehnsucht (f); Verlangen (n) // fleischliche Begierde (f); Liebe (f). [2] Zuneigung (f); Liebe (f); Anhänglichkeit (f); Vorliebe (f) (→ aichaku 愛着1226801).}\end{entry}
\begin{entry}
\mainentry{aijaku }
{相酌}
{ gegenseitiges Einschenken (n) und Trinken (n).}\end{entry}
\begin{entry}
\mainentry{aijakugaaru}
{愛着がある}
{anhänglich sein; Anhänglichkeit zeigen; zugetan sein.}\end{entry}
\begin{entry}
\mainentry{aijakusuru}
{愛着する [a]}
{an jmdm. hängen; sein Herz an jmdn. hängen.}\end{entry}
\begin{entry}
\mainentry{aijakudeyaru}
{相酌でやる}
{sich gegenseitig einschenkend trinken.}\end{entry}
\begin{entry}
\mainentry{aijakuhanai}
{愛着はない}
{nicht zu getan sein; nicht mögen.}\end{entry}
\begin{entry}
\mainentry{aijakuya}
{相借家}
{[1] gemeinsames Mieten (n) eines Hauses. [2] gemeinsam gemietetes Haus (n).}\end{entry}
\begin{entry}
\mainentry{aijakuri}
{合じゃくり (合い決り; 合決り; 合決; 合い抉り; 合抉り; 合抉)}
{  \textit{Archit.} Ai·jakuri (Verbindung zwischen Holzbrettern, die sich um die Hälfte der Dicke überlagern). (Ai·jakuriaijakuri; ⇒ ai·gaki 相欠き; 合欠き5800064).}\end{entry}
\begin{entry}
\mainentry{aishaseishin}
{愛社精神}
{ Loyalität (f) gegenüber der eigenen Firma; Firmenloyalität (f).}\end{entry}
\begin{entry}
\mainentry{aishadou}
{アイ・シャドウ; アイシャドウ}
{  \textit{Kosmetik} Lidschatten (m) (von engl. eye-shadow; ⇒ ai·shadō アイ・シャドー0938002).}\end{entry}
\begin{entry}
\mainentry{aishado-}
{アイ・シャドー; アイシャドー}
{  \textit{Kosmetik} Lidschatten (m) (von engl. eye-shadow).}\end{entry}
\begin{entry}
\mainentry{aishado-wokokunuttaon'na}
{アイシャドーを濃く塗った女}
{ \textit{Kosmetik} Frau (f), die Lidschatten trägt.}\end{entry}
\begin{entry}
\mainentry{aishado-woshiteiru}
{アイシャドーをしている}
{ \textit{Kosmetik} Lidschatten tragen.}\end{entry}
\begin{entry}
\mainentry{aishado-wotsukeru}
{アイシャドーをつける}
{ \textit{Kosmetik} Lidschatten auflegen.}\end{entry}
\begin{entry}
\mainentry{aijamisen}
{相三味線; 合三味線}
{ [1] Shamisen-Spieler (m), der immer einen bestimmten Sänger oder eine Kurtisanin begleitet. [2]  \textit{Mus.} Shamisen-Begleitstimme (f).}\end{entry}
\begin{entry}
\mainentry{aishawokattenaraheiku}
{愛車を駆って奈良へ行く}
{mit seinem Auto nach Nara fahren.}\end{entry}
\begin{entry}
\mainentry{aishuu }
{哀愁}
{ Wehmut (f); Trauer (f); Kummer (m); Betrübnis (f) (⇒ yūshū 憂愁2443564).}\end{entry}
\begin{entry}
\mainentry{aishuu }
{愛執}
{ Anhänglichkeit (f); Liebe (f) (urspr. aus dem buddh. Sprachgebrauch; ⇒ aijaku 1 愛着; 愛著5273520).}\end{entry}
\begin{entry}
\mainentry{aishuuniosowareru}
{哀愁におそわれる}
{Wehmut beschleicht jmdn.}\end{entry}
\begin{entry}
\mainentry{aishuuwoobiru}
{哀愁を帯びた}
{traurig; bekümmert.}\end{entry}
\begin{entry}
\mainentry{aishuuwokanjiru}
{哀愁を感じる}
{sich trübselig fühlen.}\end{entry}
\begin{entry}
\mainentry{aishuuwoshizumu}
{哀愁を沈む}
{in Trauer versinken.}\end{entry}
\begin{entry}
\mainentry{aishuuwososoru}
{哀愁をそそる}
{jmdn. mit Wehmut erfüllen.}\end{entry}
\begin{entry}
\mainentry{aishuuwowoobiru}
{哀愁をを帯びる}
{traurig sein; bekümmert sein.}\end{entry}
\begin{entry}
\mainentry{aisho}
{愛書}
{ [1] Liebe (f) zu Büchern. [2] Lieblingsbuch (n).}\end{entry}
\begin{entry}
\mainentry{aishou }
{哀傷}
{ Kummer (m); Trauer (f).}\end{entry}
\begin{entry}
\mainentry{aijou }
{哀情}
{ Traurigkeit (f).}\end{entry}
\begin{entry}
\mainentry{aishou }
{愛しょう [1]; 愛唱 [1] (愛誦 [1])}
{ Spaß (m) am Rezitieren; Spaß (m) am Lesen (⇒ aigin 愛吟4912007).}\end{entry}
\begin{entry}
\mainentry{aijou }
{愛嬢}
{ jmds. geliebte Tochter (f); Lieblingstochter (f) // (häufig für jmds. anderen Tochter verwendet) Ihr Fräulein (n) Tochter.}\end{entry}
\begin{entry}
\mainentry{aishou }
{愛しょう [2] (愛妾)}
{ (schriftspr.) Geliebte (f); geliebte Mätresse (f); geliebte Nebenfrau (f); Lieblingskonkubine (f).}\end{entry}
\begin{entry}
\mainentry{aijou }
{愛情}
{ [1] Zuneigung (f); Herzlichkeit (f). [2] geschlechtliche Liebe (f); Liebe (f) zum anderen Geschlecht.}\end{entry}
\begin{entry}
\mainentry{aishou }
{愛しょう [3] (愛誦 [2])}
{ Liebe (f) zum Lesen.}\end{entry}
\begin{entry}
\mainentry{aijou }
{逢い状; 逢状}
{ Notizzettel (m), um eine Geisha zu seinem Platz zu bitten.}\end{entry}
\begin{entry}
\mainentry{aishou }
{愛唱 [2]}
{ Spaß (m) am Singen; Lieblingslied (n).}\end{entry}
\begin{entry}
\mainentry{aishou }
{愛称}
{ Kosename (m) (⇒ adana あだな6258886; ⇒ nikku·nēmu ニック・ネーム3147017).}\end{entry}
\begin{entry}
\mainentry{aishou }
{相性 [1]; 合い性; 合性 [1]}
{ [1] Sympathie (f); Kongenialität (f); Geistesverwandtschaft (f); Wesensähnlichkeit (f); Affinität (f); Zuneigung (f). [2] eheliches Glück (n) (nach dem Horoskop).}\end{entry}
\begin{entry}
\mainentry{aijouaruhito}
{愛情ある人}
{liebevolle Person (f); herzliche Person (f).}\end{entry}
\begin{entry}
\mainentry{aishouka }
{愛唱歌}
{Lieblingslied (n).}\end{entry}
\begin{entry}
\mainentry{aishouka }
{愛誦歌}
{Lieblingsgedicht (n); Gedicht (n), das jmd. am liebsten rezitiert.}\end{entry}
\begin{entry}
\mainentry{aijouganakunaru}
{愛情がなくなる; 愛情が無くなる}
{die Zuneigung zu jmdm. verlieren.}\end{entry}
\begin{entry}
\mainentry{aijougafukai}
{愛情が深い}
{voller Liebe sein.}\end{entry}
\begin{entry}
\mainentry{aishougayoi}
{相性がよい; 相性が良い}
{gut zueinander passen; das gleiche Temperament haben.}\end{entry}
\begin{entry}
\mainentry{aishougawarui}
{相性が悪い}
{schlecht zusammenpassen.}\end{entry}
\begin{entry}
\mainentry{aishoukyoku}
{愛唱曲}
{Lieblingslied (n).}\end{entry}
\begin{entry}
\mainentry{aishougo}
{愛称語}
{Kosewort (n).}\end{entry}
\begin{entry}
\mainentry{aishousuru }
{哀傷する}
{betrauern.}\end{entry}
\begin{entry}
\mainentry{aishousuru }
{愛唱する}
{gern singen.}\end{entry}
\begin{entry}
\mainentry{aishousuru }
{愛誦する}
{gern rezitieren.}\end{entry}
\begin{entry}
\mainentry{aijouseikatsu}
{愛情生活}
{Liebesleben (n).}\end{entry}
\begin{entry}
\mainentry{aishoudeyobitsuketeiru}
{愛称で呼びつけている}
{jmdn. immer mit seinem Kosenamen nennen.}\end{entry}
\begin{entry}
\mainentry{aishoudeyobu}
{愛称で呼ぶ}
{jmdn. beim Kosenamen rufen.}\end{entry}
\begin{entry}
\mainentry{aijouniueta}
{愛情に飢えた}
{nach Zuneigung hungernd.}\end{entry}
\begin{entry}
\mainentry{aijouniueteiru}
{愛情に飢えている}
{nach Zuneigung dürsten.}\end{entry}
\begin{entry}
\mainentry{aijouniueteirukodomo}
{愛情に飢えている子供}
{ein nach Liebe hungerndes Kind (n).}\end{entry}
\begin{entry}
\mainentry{aijounimukuiru}
{愛情に報いる}
{jmds. Liebe erwidern.}\end{entry}
\begin{entry}
\mainentry{aijounoaru}
{愛情のある; 愛情の有る}
{liebevoll; warmherzig; zärtlich.}\end{entry}
\begin{entry}
\mainentry{aishounoii}
{相性のいい}
{zusammenpassend; von gleichem Temperament; kompatibel.}\end{entry}
\begin{entry}
\mainentry{aishounouta}
{哀傷の歌}
{Elegie (f); Klagelied (n).}\end{entry}
\begin{entry}
\mainentry{aijounokomotta}
{愛情のこもった (愛情の籠もった; 愛情の篭った)}
{liebevoll; warmherzig; zärtlich.}\end{entry}
\begin{entry}
\mainentry{aijounokomottakotoba}
{愛情のこもった言葉}
{liebevolle Wortenpl; warmherzig Wortenpl; zärtliche Wortenpl.}\end{entry}
\begin{entry}
\mainentry{aijounosukoshimonaihito}
{愛情の少しもない人}
{Person (f) ohne jedes Gefühl.}\end{entry}
\begin{entry}
\mainentry{aijounonai}
{愛情のない; 愛情の無い}
{lieblos; kaltherzig.}\end{entry}
\begin{entry}
\mainentry{aijounofukaihito}
{愛情の深い人}
{Person (f) mit viel Gefühl.}\end{entry}
\begin{entry}
\mainentry{aishounoyoi}
{相性のよい}
{kongenial; gut zusammen passend.}\end{entry}
\begin{entry}
\mainentry{aishounowarui}
{相性の悪い}
{schlecht zueinander passend.}\end{entry}
\begin{entry}
\mainentry{aishoumei}
{愛称名}
{Kosename (m).}\end{entry}
\begin{entry}
\mainentry{aijouwoidaku}
{愛情を抱く}
{jmdn. mögen; jmdn. gern haben.}\end{entry}
\begin{entry}
\mainentry{aijouwoushinau}
{愛情を失う}
{die Zuneigung verlieren.}\end{entry}
\begin{entry}
\mainentry{aijouwoeru}
{愛情を得る}
{jmds. Liebe gewinnen.}\end{entry}
\begin{entry}
\mainentry{aijouwokatamukeru}
{愛情を傾ける}
{sein Herz an etw. hängen.}\end{entry}
\begin{entry}
\mainentry{aijouwokomete}
{愛情をこめて}
{mit Liebe; liebevoll.}\end{entry}
\begin{entry}
\mainentry{aijouwosasageru}
{愛情を捧げる}
{jmdm. seine Liebe schenken.}\end{entry}
\begin{entry}
\mainentry{aijouwoshimesu}
{愛情を示す}
{Liebe zeigen.}\end{entry}
\begin{entry}
\mainentry{aijouwososogu}
{愛情を注ぐ}
{seine Liebe verschenken.}\end{entry}
\begin{entry}
\mainentry{aishouwomiru}
{相性を見る}
{ein Horoskop erstellen, um zu sehen, ob ein Paar zusammenpasst.}\end{entry}
\begin{entry}
\mainentry{aijouwomotsu}
{愛情を持つ}
{Liebe empfinden.}\end{entry}
\begin{entry}
\mainentry{aijouwomoyoosu}
{哀情を催す}
{Traurigkeit empfinden.}\end{entry}
\begin{entry}
\mainentry{aishoka}
{愛書家}
{Bücherfreund (m); Bibliophiler (m).}\end{entry}
\begin{entry}
\mainentry{aishoku}
{哀色}
{ trauriger Gesichtsausdruck (m).}\end{entry}
\begin{entry}
\mainentry{aishoheki}
{愛書癖}
{Bibliophilie (f).}\end{entry}
\begin{entry}
\mainentry{aishirai}
{あいしらい}
{ [1] Empfangen (n); Gesellschaft. [2] Zusammensetzung (f); Zusammenstellung (f). [3]  \textit{Nō, Kyōgen} Nebenrolle (f).}\end{entry}
\begin{entry}
\mainentry{aishiraidokoro}
{あいしらい所 (あひしらひ所)}
{Platz (m) für die Bewirtung bzw. fürs Übernachten der Gäste.}\end{entry}
\begin{entry}
\mainentry{aijiru}
{愛汁}
{ (Slang) Scheidensekret (n); Vaginalsekret (n).}\end{entry}
\begin{entry}
\mainentry{aishiru }
{相知る}
{ [1] sich gegenseitig gut kennen. [2] einander lieben; einander die Liebe schwören.}\end{entry}
\begin{entry}
\mainentry{aishiru }
{藍汁; 藍澱}
{Indigoküpe (n); Indigolösung (f).}\end{entry}
\begin{entry}
\mainentry{aijirushi}
{合印 [2]; 合い印 [2]; 合いじるし (合標; 合い標)}
{ [1] Erkennungszeichen (n); Merkmal (n). [2] Stempel (m) für den erfolgten Vergleich zweier Dokumente o. Ä.}\end{entry}
\begin{entry}
\mainentry{aijirushigaau。}
{合印が合う。}
{ \textit{Bsp.} Die Abgleichstempel stimmen überein.}\end{entry}
\begin{entry}
\mainentry{aijirushiwotsukeru}
{合印をつける}
{mit einem Abgleichzeichen versehen.}\end{entry}
\begin{entry}
\mainentry{aijirou}
{愛次郎}
{  \textit{männl. Name} Aijirō.}\end{entry}
\begin{entry}
\mainentry{aishin}
{愛心}
{ Liebe (f); Zuneigung (f).}\end{entry}
\begin{entry}
\mainentry{aijin}
{愛人}
{ [1] Geliebte (f); Freundin (f) (seit Ende der Tokugawa-Zeit als Übersetzungswort für sweet heart, lover etc. verwendet) // (nach dem Zweiten Weltkrieg leicht verhüllend) Liebhaber (m); Hausfreund (m) // Mätresse (f)}\end{entry}
\begin{entry}
\mainentry{aijingadekiru}
{愛人ができる}
{einen Freund finden; eine Freundin finden.}\end{entry}
\begin{entry}
\mainentry{aishingyoro}
{アイシンギョロ (愛新覚羅)}
{  \textit{chin. Gesch.} Aixin-jueluomNAr (Familienname der Könige der chin. Qing-Dynastie).}\end{entry}
\begin{entry}
\mainentry{aishingu}
{アイシング}
{ [1] Glasieren (n); Überziehen (n) mit Zucker. [2]  \textit{Eishockey} Icing (n) (eine Art Regelverstoß).}\end{entry}
\begin{entry}
\mainentry{aishinguzapakku}
{アイシング・ザ・パック; アイシングザパック}
{ \textit{Sport} Icing (n) (eine Art Regelverstoß im Eishockey; von engl. icing the puck).}\end{entry}
\begin{entry}
\mainentry{aishingurasu}
{アイシングラス}
{ Hausenblase (f); Fischleim (m) (von engl. isinglass).}\end{entry}
\begin{entry}
\mainentry{aishinzuru}
{相信ずる}
{sich gegenseitig glauben.}\end{entry}
\begin{entry}
\mainentry{aijindoushi}
{愛人同士}
{Verliebter (m); Liebespaar (n).}\end{entry}
\begin{entry}
\mainentry{aijinbanku}
{愛人バンク}
{Callgirl-Agentur (f).}\end{entry}
\begin{entry}
\mainentry{aizu}
{合図; 相図}
{ Signal (n); Zeichen (n); Wink (m).}\end{entry}
\begin{entry}
\mainentry{aisu }
{アイス}
{ [1] Eis (n). [2] Eiscreme (f) (Abk.; ⇒ aisu·kurīmu アイスクリーム6667544) // Eiskaffee (m) (Abk.; ⇒ aisu·kōhī アイスコーヒー9601412). [3]  \textit{Sport} Eishockey (n)}\end{entry}
\begin{entry}
\mainentry{aisu }
{愛す}
{ lieben; gern haben; geneigt sein; Gefallen finden (⇒ aisuru 愛する3803790).}\end{entry}
\begin{entry}
\mainentry{aisu }
{愛洲}
{  \textit{Familienn.} AisuNAr.}\end{entry}
\begin{entry}
\mainentry{aisuakkusu}
{アイス・アックス; アイスアックス}
{ \textit{Bergsteigen} Pickel (m); Eispickel (m) (von engl. ice ax).}\end{entry}
\begin{entry}
\mainentry{aisuari-na}
{アイス・アリーナ; アイスアリーナ}
{Eisstadion (n) (von engl. ice arena).}\end{entry}
\begin{entry}
\mainentry{aisuaruji-}
{アイス・アルジー; アイスアルジー}
{ \textit{Bot.} Eisalge (f) (von engl. ice algae).}\end{entry}
\begin{entry}
\mainentry{aisuu-ru}
{アイス・ウール; アイスウール}
{ glänzender Wollfaden (von engl. ice wool).}\end{entry}
\begin{entry}
\mainentry{aisuuo-ta-}
{アイス・ウオーター; アイスウオーター}
{Eiswasser (n) (von engl. ice water).}\end{entry}
\begin{entry}
\mainentry{aisuuxo-ru}
{アイス・ウォール; アイスウォール}
{ \textit{Bergsteigen} Eiswand (f) (von engl. icewall).}\end{entry}
\begin{entry}
\mainentry{aisukinesu}
{アイスキネス}
{  \textit{Persönlichk.} Aischines; Äschines (griech. Redner; 389–315 v. Chr.).}\end{entry}
\begin{entry}
\mainentry{aisukyappu}
{アイス・キャップ; アイスキャップ}
{ Eiskappe (f) (von engl. ice cap).}\end{entry}
\begin{entry}
\mainentry{aisukyandyi}
{アイス・キャンディ; アイスキャンディ}
{ \textit{Kochk.} Eis (n) am Stiel (von engl. ice candy).}\end{entry}
\begin{entry}
\mainentry{aisukyandyi-}
{アイス・キャンディー; アイスキャンディー}
{ \textit{Kochk.} Eis (n) am Stiel (von engl. ice candy).}\end{entry}
\begin{entry}
\mainentry{aisukyande-}
{アイス・キャンデー; アイスキャンデー}
{ Eis (n) am Stiel (von japan.-engl. ice candy).}\end{entry}
\begin{entry}
\mainentry{aisukyande-nobou}
{アイスキャンデーの棒}
{Eisstiel (m).}\end{entry}
\begin{entry}
\mainentry{aisukyu-bu}
{アイス・キューブ; アイスキューブ}
{Eiswürfel (m) (von engl. ice cube).}\end{entry}
\begin{entry}
\mainentry{aisukyurosu}
{アイスキュロス}
{  \textit{Persönlichk.} Aischylos; Äschylus (griech. Dichter; 525–456 v. Chr.).}\end{entry}
\begin{entry}
\mainentry{aisukuri-mu}
{アイス・クリーム; アイスクリーム}
{  \textit{Kochk.} Eiscreme (f); Sahneeis (n); Speiseeis (n); Eis (n) (mit mehr als 15 % Milchanteil und davon mehr als 8 % Milchfett; von engl. ice cream).}\end{entry}
\begin{entry}
\mainentry{aisukuri-muko-n}
{アイスクリーム・コーン; アイスクリームコーン}
{Eistüte (f) (von engl. ice-cream cone).}\end{entry}
\begin{entry}
\mainentry{aisukuri-musande-}
{アイス・クリーム・サンデー; アイスクリーム・サンデー; アイスクリームサンデー}
{Eis-Sundae (n) (Eis mit Garnierung wie Sirup, Früchten, Sahne o. Ä.).}\end{entry}
\begin{entry}
\mainentry{aisukuri-museizouki}
{アイスクリーム製造機}
{Eismaschine (f).}\end{entry}
\begin{entry}
\mainentry{aisukuri-museizoukikai}
{アイスクリーム製造機械}
{Eismaschine (f).}\end{entry}
\begin{entry}
\mainentry{aisukuri-muso-da}
{アイス・クリーム・ソーダ; アイスクリーム・ソーダ; アイスクリームソーダ}
{Fruchtsaft-Cocktail (m) mit Speiseeis (von engl. ice cream soda).}\end{entry}
\begin{entry}
\mainentry{aisukuri-mufuri-za-}
{アイスクリーム・フリーザー; アイスクリームフリーザー}
{Eismaschine (f) (von engl. ice-cream freezer).}\end{entry}
\begin{entry}
\mainentry{aisuko-hi-}
{アイス・コーヒー; アイスコーヒー}
{ eisgekühlter Kaffee (m); Kaffee (m) mit Eiswürfeln; Eiskaffee (m) (von engl. iced coffee; Sommer).}\end{entry}
\begin{entry}
\mainentry{aisushou}
{アイス・ショウ; アイスショウ}
{ \textit{Eislaufen} Eisrevue (f) (von engl. ice show).}\end{entry}
\begin{entry}
\mainentry{aisusho-}
{アイス・ショー; アイスショー}
{Eisrevue (f) (von engl. ice show).}\end{entry}
\begin{entry}
\mainentry{aizushori}
{合図処理}
{Signalverarbeitung (f).}\end{entry}
\begin{entry}
\mainentry{aisusuke-ta-}
{アイス・スケーター; アイススケーター}
{ \textit{Eislaufen} Eisläufer (m) (von engl. ice skater).}\end{entry}
\begin{entry}
\mainentry{aisusuke-to}
{アイス・スケート; アイススケート}
{ Eislaufen (n); Eislauf (m); Schlittschuhlaufen (n) (von engl. ice-skate).}\end{entry}
\begin{entry}
\mainentry{aisusuke-togutsu}
{アイススケート靴}
{Schlittschuhempl.}\end{entry}
\begin{entry}
\mainentry{aisusuke-tojou}
{アイススケート場}
{Eisbahn (f); Schlittschuhbahn (f).}\end{entry}
\begin{entry}
\mainentry{aisusuke-toniyuku}
{アイススケートに行く}
{Eislaufen gehen.}\end{entry}
\begin{entry}
\mainentry{aisusuke-towosuru}
{アイススケートをする}
{Eis laufen; Schlittschuh laufen.}\end{entry}
\begin{entry}
\mainentry{aisusumakku}
{アイス・スマック; アイススマック}
{ mit Schokolade überzogene Eiskreme (f) (von engl. ice smack).}\end{entry}
\begin{entry}
\mainentry{aizusuru}
{合図する; 相図する}
{ein Zeichen geben; Signal geben; winken.}\end{entry}
\begin{entry}
\mainentry{aisudanshingu}
{アイス・ダンシング; アイスダンシング}
{ \textit{Eislaufen} Eistanz (m) (von engl. ice dancing).}\end{entry}
\begin{entry}
\mainentry{aisudansu}
{アイス・ダンス; アイスダンス}
{ \textit{Sport} Eistanz (m) (von engl. ice dance).}\end{entry}
\begin{entry}
\mainentry{aisutexi}
{アイス・ティ; アイスティ}
{Eistee (m) (von engl. ice tea).}\end{entry}
\begin{entry}
\mainentry{aisutexi-}
{アイス・ティー; アイスティー}
{Eistee (m) (von engl. ice tea).}\end{entry}
\begin{entry}
\mainentry{aisutongu}
{アイス・トング; アイストング}
{Zange (f) für Eiswürfel (von engl. ice tongs).}\end{entry}
\begin{entry}
\mainentry{aizunichuuisuru}
{合図に注意する}
{auf Signale achten.}\end{entry}
\begin{entry}
\mainentry{aizunitewoageru}
{合図に手を上げる}
{die Hand zum Signal heben.}\end{entry}
\begin{entry}
\mainentry{aizunokane}
{合図の鐘}
{Signalglocke (f).}\end{entry}
\begin{entry}
\mainentry{aizunohata}
{合図の旗}
{Signalflagge (f).}\end{entry}
\begin{entry}
\mainentry{aisuba-gu}
{アイス・バーグ; アイスバーグ}
{Eisberg (m) (von engl. iceberg).}\end{entry}
\begin{entry}
\mainentry{aisuha-ken}
{アイス・ハーケン; アイスハーケン}
{ \textit{Bergsteigen} Eishaken (m) (aus d. Dtsch.von dtsch. Eishaken).}\end{entry}
\begin{entry}
\mainentry{aisuba-n}
{アイス・バーン; アイスバーン}
{ \textit{Meteor.} Glatteis (n); vereiste Straße (f) // vereiste Schneeoberfläche (f); vereiste Skipiste (f) (von dtsch. Eisbahn).}\end{entry}
\begin{entry}
\mainentry{aisubain}
{アイス・バイン; アイスバイン}
{  \textit{Kochk.} Eisbein (n) (gepökeltes u. gekochtes Schweinebein).}\end{entry}
\begin{entry}
\mainentry{aisubaggu}
{アイス・バッグ; アイスバッグ}
{Eistasche (f) (Segeltuchtasche zum Transport von Eis; von engl. ice-bag).}\end{entry}
\begin{entry}
\mainentry{aisuhanma-}
{アイス・ハンマー; アイスハンマー}
{ \textit{Bergsteigen} Eishammer (m) (aus d. Dtsch.von dtsch. Eishammer).}\end{entry}
\begin{entry}
\mainentry{aisubi-ru}
{アイス・ビール; アイスビール}
{Eisbier (n) (bei –3°C gelagertes Bier).}\end{entry}
\begin{entry}
\mainentry{aisupikku}
{アイス・ピック; アイスピック}
{Eispickel (m); Bar-Eispickel (zum Zerkleinern von Eis; von engl. ice pick).}\end{entry}
\begin{entry}
\mainentry{aisupikkeru}
{アイス・ピッケル; アイスピッケル}
{ \textit{Bergsteigen} Eispickel (m) (aus d. Dtsch.von dtsch. Eispickel bzw. von engl. ice und dtsch. Pickel).}\end{entry}
\begin{entry}
\mainentry{aisufo-ru}
{アイス・フォール; アイスフォール}
{ \textit{Bergsteigen} gefrorener Wasserfall (m) // Eisbruch (m) (wenn ein Gletscher über eine höhere Geländestufe abbricht). (von engl. icefall).}\end{entry}
\begin{entry}
\mainentry{aisupuranto}
{アイス・プラント; アイスプラント}
{ \textit{Bot.} Eiskraut (n) (Mesembryanthemum crystallinum; von engl. ice plant).}\end{entry}
\begin{entry}
\mainentry{aisubureika-}
{アイス・ブレイカー; アイスブレイカー}
{[1]  \textit{Seef.} Eisbrecher (m). [2] erster Witz (m) um das „Eis“ beim Publikum zu brechen. (von engl. icebreaker).}\end{entry}
\begin{entry}
\mainentry{aisubure-ka-}
{アイス・ブレーカー; アイスブレーカー}
{[1]  \textit{Seef.} Eisbrecher (m). [2] erster Witz (m) um das „Eis“ beim Publikum zu brechen. (von engl. icebreaker).}\end{entry}
\begin{entry}
\mainentry{aisupe-ru }
{アイス・ぺール; アイスぺール}
{Eiskübel (m) (von engl. ice pail).}\end{entry}
\begin{entry}
\mainentry{aisupe-ru }
{アイス・ペール; アイスペール}
{Sektkühler (m); Eiseimer (m) (von engl. ice pail).}\end{entry}
\begin{entry}
\mainentry{aisubeki}
{愛すべき (愛す可き)}
{liebeswert; lieb; liebeswürdig.}\end{entry}
\begin{entry}
\mainentry{aisubo-to}
{アイス・ボート; アイスボート}
{[1] Eisjacht (f); Eissegler (m). [2] Eisbrecher (m). (von engl. iceboat).}\end{entry}
\begin{entry}
\mainentry{aisubokkusu}
{アイス・ボックス; アイスボックス}
{Eisschrank (m); Kühlschrank (m) (von engl. icebox).}\end{entry}
\begin{entry}
\mainentry{aisuhokke-}
{アイス・ホッケー; アイスホッケー}
{  \textit{Sport} Eishockey (n) (von engl. ice hockey).}\end{entry}
\begin{entry}
\mainentry{aisuhokke-nosenshu}
{アイスホッケーの選手}
{ \textit{Sport} Eishockeyspieler (m).}\end{entry}
\begin{entry}
\mainentry{aisuhokke-rinku}
{アイスホッケー・リンク; アイスホッケーリンク}
{Eishockeysspielfeld (n).}\end{entry}
\begin{entry}
\mainentry{aizumai}
{相住まい; 相住い; 相住 [a]}
{ Zusammenleben (n).}\end{entry}
\begin{entry}
\mainentry{aisumanai}
{相済まない; 相すまない; あいすまない}
{ [1] etw. nicht so lassen können. [2] meinen, sich entschuldigen zu müssen.}\end{entry}
\begin{entry}
\mainentry{aisumanu}
{あいすまぬ}
{ [1] etw. nicht so lassen können. [2] meinen, sich entschuldigen zu müssen.}\end{entry}
\begin{entry}
\mainentry{aizumi }
{相住み; 相住 [b]}
{ Zusammenleben (n).}\end{entry}
\begin{entry}
\mainentry{aizumi }
{藍住}
{  \textit{Stadtn.} Aizumi (Stadt im Osten der Präf. Tokushima).}\end{entry}
\begin{entry}
\mainentry{aizumi }
{藍墨}
{Indigofarbe (f) in der Form eines Tuschsteines.}\end{entry}
\begin{entry}
\mainentry{aisumimasen。}
{相済みません。}
{ \textit{Bsp.} Es tut mir außerordentlich leid.}\end{entry}
\begin{entry}
\mainentry{aisumu}
{相済む}
{ [1] zum Ende kommen. [2] sich rechtfertigen; sich entschuldigen.}\end{entry}
\begin{entry}
\mainentry{aisumonaka}
{アイスもなか}
{mit Eis gefüllte Waffel (f); mit Eis gefüllte Monaka (n).}\end{entry}
\begin{entry}
\mainentry{aisuyotto}
{アイス・ヨット; アイスヨット}
{Eisjacht (f); Eissegler (m).}\end{entry}
\begin{entry}
\mainentry{aisurando}
{アイスランド}
{  \textit{Ländern.} IslandnNAr; Republik (f) Island (Inselstaat im Europäischen Nordmeer).}\end{entry}
\begin{entry}
\mainentry{aisurandokyouwakoku}
{アイスランド共和国}
{ \textit{Ländern.} Republik (f) Island (Inselstaat im Europäischen Nordmeer).}\end{entry}
\begin{entry}
\mainentry{aisurandogo}
{アイスランド語}
{ \textit{Sprache} Isländisch (n).}\end{entry}
\begin{entry}
\mainentry{aisurandojin}
{アイスランド人}
{Isländer (m); Isländerin (f).}\end{entry}
\begin{entry}
\mainentry{aisurandono}
{アイスランドの}
{isländisch.}\end{entry}
\begin{entry}
\mainentry{aisurandopopi-}
{アイスランド・ポピー; アイスランドポピー}
{ \textit{Bot.} Island-Mohn (m) (von engl. Iceland poppy).}\end{entry}
\begin{entry}
\mainentry{aizuri}
{相掏り (相掏摸)}
{ Komplize (m).}\end{entry}
\begin{entry}
\mainentry{aizurie}
{藍摺り絵; 藍摺絵}
{Holzdruck (m) mit Indigo.}\end{entry}
\begin{entry}
\mainentry{aisurinku}
{アイス・リンク; アイスリンク}
{Eislauffläche (f); Eisbahn (f) (von engl. ice rink).}\end{entry}
\begin{entry}
\mainentry{aisuru}
{愛する}
{ jmdn. lieben; jmdn. gern haben; jmdm. geneigt sein; an jmdm. Gefallen finden; jmdn. für liebenswert halten.}\end{entry}
\begin{entry}
\mainentry{aisuruotto}
{愛する夫}
{geliebter Ehemann (m).}\end{entry}
\begin{entry}
\mainentry{aisuruko}
{愛する子}
{geliebtes Kind (n).}\end{entry}
\begin{entry}
\mainentry{aisurukodomotachi}
{愛する子供たち}
{die geliebten Kindernpl.}\end{entry}
\begin{entry}
\mainentry{aisuruhito}
{愛する人}
{Geliebter (m); Geliebte (f).}\end{entry}
\begin{entry}
\mainentry{aisurumono}
{愛する者; 愛するもの}
{jmds. Liebling (m); Geliebter (m); Objekt (n) der Liebe.}\end{entry}
\begin{entry}
\mainentry{aisuruyouninaru}
{愛するようになる}
{sich verlieben.}\end{entry}
\begin{entry}
\mainentry{aisuruwagako}
{愛するわが子}
{jmds. Liebling (m); jmds. Kind (n).}\end{entry}
\begin{entry}
\mainentry{「aizuwaidoshatto」}
{「アイズ・ワイド・シャット」; 「アイズワイドシャット」}
{  \textit{Filmtitel}  (Film von Stanley Kubrick; 1999).}\end{entry}
\begin{entry}
\mainentry{aizuwoshitedenshawotomeru}
{合図をして電車を止める}
{einen Zug mit Zeichen anhalten.}\end{entry}
\begin{entry}
\mainentry{aizuwosuru}
{合図をする}
{ein Zeichen geben.}\end{entry}
\begin{entry}
\mainentry{aizuwomitetoru}
{合図を見て取る}
{ein Zeichen empfangen.}\end{entry}
\begin{entry}
\mainentry{aisei}
{愛婿 (愛壻)}
{ jmds. Lieblingsschwiegersohn (m).}\end{entry}
\begin{entry}
\mainentry{aiseki }
{哀惜}
{ (schriftspr.) Trauer (f) (⇒ aitō 哀悼7766748).}\end{entry}
\begin{entry}
\mainentry{aiseki }
{愛惜}
{ [1] Liebe zu etw. od. jmdm., dass man sich nicht davon trennen möchte oder es betrauert, wenn ihm etw. zustößt. [2] Trauer (f) über die Trennung von etw.}\end{entry}
\begin{entry}
\mainentry{aiseki }
{相席; 合い席; 合席}
{ Sitzen (n) am selben Tisch.}\end{entry}
\begin{entry}
\mainentry{aisekisuru }
{哀惜する}
{trauern; etw. beklagen; innig bedauern; von Herzen bedauern.}\end{entry}
\begin{entry}
\mainentry{aisekisuru }
{愛惜する}
{ungern hergeben; sich ungern trennen von etw.; jmdn. vermissen; sehr lieb haben.}\end{entry}
\begin{entry}
\mainentry{aisekisuru }
{相席する; 合い席する; 合席する}
{den Tisch miteinander teilen; am selben Tisch sitzen.}\end{entry}
\begin{entry}
\mainentry{aisekinojou }
{哀惜の情}
{Kondolenz (f).}\end{entry}
\begin{entry}
\mainentry{aisekinojou }
{愛惜の情}
{Abschiedsschmerz (m); Trennungsschmerz (m).}\end{entry}
\begin{entry}
\mainentry{aisetsu}
{哀切; あいせつ}
{ (schriftspr.) Wehmut (f); Traurigkeit (f); Trauer (f).}\end{entry}
\begin{entry}
\mainentry{aizetsu}
{哀絶}
{ außergewöhnlich große Traurigkeit (f).}\end{entry}
\begin{entry}
\mainentry{aisetsusa}
{哀切さ}
{(schriftspr.) Wehmut (f); Traurigkeit (f); Trauer (f).}\end{entry}
\begin{entry}
\mainentry{aisessuru}
{相接する}
{ sich treffen; in Kontakt kommen; angrenzen.}\end{entry}
\begin{entry}
\mainentry{aisetsuda}
{哀切だ}
{wehmütig sein; trauen; klagen}\end{entry}
\begin{entry}
\mainentry{aisetsuna}
{哀切な}
{wehmütig; traurig und verloren.}\end{entry}
\begin{entry}
\mainentry{aisetsuni}
{哀切に}
{wehmütig; traurig; klagend.}\end{entry}
\begin{entry}
\mainentry{aisetsumi}
{哀切み}
{(schriftspr.) Wehmut (f); Traurigkeit (f); Trauer (f).}\end{entry}
\begin{entry}
\mainentry{aizenaha}
{アイゼナハ}
{  \textit{Stadtn.} EisenachnNAr (Stadt im Thüringer Wald).}\end{entry}
\begin{entry}
\mainentry{aiseru}
{愛せる}
{ lieben können.}\end{entry}
\begin{entry}
\mainentry{aiseruko}
{アイセル湖}
{  \textit{Seen.} IJsselmeer (n) (durch Abschlussdeich gebildeter See in den Niederlanden; wird mit großem I und großem J geschrieben, weil es sich dabei eigentlich um eine holländische Ligaur handelt).}\end{entry}
\begin{entry}
\mainentry{aisen}
{相先}
{  \textit{Go, Shōgi} Spiel (n) zweier gleich starke Spieler, die sich gleich ab dem Eröffnungszug mit den Zügen abwechseln.}\end{entry}
\begin{entry}
\mainentry{aizen }
{アイゼン}
{  \textit{Bergsteigen} Steigeisen (n) (Abk. für dtsch. Steigeisen).}\end{entry}
\begin{entry}
\mainentry{aizen }
{あい然 (靄然; 藹然)}
{ [1] Ziehen (n) von Wolken und Dunst. [2] Stille (f); Ruhe (f); Frieden (m).}\end{entry}
\begin{entry}
\mainentry{aizen }
{愛染}
{ [1] weltliche Begierdenfpl. [2]  \textit{Buddh.} RāgarājafNAr (buddh. Göttin der Liebe; Abk. für Aizen·myōō 愛染明王8005422).}\end{entry}
\begin{entry}
\mainentry{aizengoshite}
{相前後して}
{ [1] um diese Zeit. [2] nacheinander; hintereinander. (⇒ ai·tsuide 相次いで6016702).}\end{entry}
\begin{entry}
\mainentry{aizengosuru}
{相前後する}
{ [1] direkt aufeinanander folgen. [2] fast gleichzeitig passieren.}\end{entry}
\begin{entry}
\mainentry{aizentaru}
{靄然たる; 藹然たる}
{still; ruhig; friedlich.}\end{entry}
\begin{entry}
\mainentry{aizenhawa-}
{アイゼンハワー}
{  \textit{Persönlichk.} Dwight David Eisenhower (34. Präsident der USA; 1953–1961).}\end{entry}
\begin{entry}
\mainentry{aizenmyouou}
{愛染明王; 愛染妙王}
{  \textit{Buddh.} RāgarājafNAr (buddh. Göttin der Liebe).}\end{entry}
\begin{entry}
\mainentry{aizenmyououhou}
{愛染明王法}
{ \textit{Buddh.} Aizenmyōō-Methode (f) (im Shingon-Buddhismus).}\end{entry}
\begin{entry}
\mainentry{aizenmenga-shoukougun}
{アイゼンメンガー症候群}
{  \textit{Med.} Eisenmenger-Syndrom (n).}\end{entry}
\begin{entry}
\mainentry{aizenmengerushoukougun}
{アイゼンメンゲル症候群}
{  \textit{Med.} Eisenmenger-Syndrom (n).}\end{entry}
\begin{entry}
\mainentry{aisenryou}
{藍染料}
{Indigofarbe (n).}\end{entry}
\begin{entry}
\mainentry{aiso }
{哀訴}
{ (schriftspr.) [1] flehentliche Bitte (f). [2] Beschwerde (f). (⇒ aigan 哀願0975250).}\end{entry}
\begin{entry}
\mainentry{aiso }
{愛想 [a]; あいそ}
{ [1] Liebenswürdigkeit (f); angenehme Ausstrahlung (f); Freundlichkeit (f); Artigkeit (f); Geselligkeit (f); Leutseligkeit (f); Umgänglichkeit (f); Gastfreundlichkeit (f). [2] Rechnung (f). (→ aichaku 愛着1226801).}\end{entry}
\begin{entry}
\mainentry{aisou}
{愛想 [b]; あいそう}
{ [1] Liebenswürdigkeit (f); Artigkeit (f); Geselligkeit (f); Leutseligkeit (f); Umgänglichkeit (f); Gastfreundlichkeit (f). [2] Rechnung (f). (→ aiso 愛想5298885).}\end{entry}
\begin{entry}
\mainentry{aizou }
{愛憎}
{ (schriftspr.) Liebe (f) und Hass (m); Zuneigung (f) und Abneigung (f).}\end{entry}
\begin{entry}
\mainentry{aizou }
{愛蔵}
{ (schriftspr.) kostbar; Lieblings… (⇒ hizō 秘蔵2525265).}\end{entry}
\begin{entry}
\mainentry{aizougaaru}
{愛憎がある}
{parteiisch sein.}\end{entry}
\begin{entry}
\mainentry{aizousuru}
{愛蔵する}
{sorgfältig aufbewahren.}\end{entry}
\begin{entry}
\mainentry{aizounaku}
{愛憎なく}
{unparteiisch; ohne Zu‑ oder Abneigung.}\end{entry}
\begin{entry}
\mainentry{aizounokaramiattakankei}
{愛憎のからみ合った関係}
{eine auf Hass-Liebe beruhende Beziehung (f).}\end{entry}
\begin{entry}
\mainentry{aizounosho}
{愛蔵の書}
{wie ein Schatz gehütetes Buch (n).}\end{entry}
\begin{entry}
\mainentry{aizounoshomotsu}
{愛蔵の書物}
{wie ein Schatz gehütetes Schriftstück (n).}\end{entry}
\begin{entry}
\mainentry{aizounonengatsuyoi}
{愛憎の念が強い}
{sehr parteiisch sein.}\end{entry}
\begin{entry}
\mainentry{aizounonengafukai}
{愛憎の念が深い}
{sehr parteiisch sein.}\end{entry}
\begin{entry}
\mainentry{aisounoyoi}
{愛想のよい}
{entgegenkommend; gesellig; gefällig; freundlich; gastfreundlich; zuvorkommend; aufmerksam; umgänglich.}\end{entry}
\begin{entry}
\mainentry{aizouban}
{愛蔵版}
{bibliophile Ausgabe (f); Liebhaberausgabe (f).}\end{entry}
\begin{entry}
\mainentry{aizouheison}
{愛憎併存}
{Ambivalenz (f).}\end{entry}
\begin{entry}
\mainentry{aisouwoyokusuru}
{愛想をよくする}
{entgegenkommend sein; sich jm warm halten.}\end{entry}
\begin{entry}
\mainentry{aisogatsukiru}
{愛想が尽きる; 愛想がつきる}
{hoffnungslos sein; überdrüssig werden; nichts mehr zu tun haben wollen; sich schämen.}\end{entry}
\begin{entry}
\mainentry{aisokinechikkusu}
{アイソキネチックス}
{ Isokinetik (f) (von engl. isokinetics).}\end{entry}
\begin{entry}
\mainentry{aisokinetexikusu}
{アイソキネティクス}
{ Isokinetik (f) (von engl. isokinetics).}\end{entry}
\begin{entry}
\mainentry{aisokinetexikkusu}
{アイソキネティックス}
{ Isokinetik (f) (von engl. isokinetics).}\end{entry}
\begin{entry}
\mainentry{aisokinetexikkutore-ningu}
{アイソキネティック・トレーニング; アイソキネティックトレーニング}
{ isokinetisches Training (n).}\end{entry}
\begin{entry}
\mainentry{aisoku}
{愛息}
{ jmds. geliebter Sohn (m).}\end{entry}
\begin{entry}
\mainentry{aisokonau}
{会い損なう; 会い損う; 会いそこなう}
{ jmdn. verpassen; jmdn. treffen wollen, aber ihn nicht treffen.}\end{entry}
\begin{entry}
\mainentry{aisozaimu}
{アイソザイム}
{  \textit{Chem.} Isozym (n); Isoenzym (n).}\end{entry}
\begin{entry}
\mainentry{aisoshianin}
{アイソシアニン}
{  \textit{Chem., Fotog.} Isozyanin (n) (⇒ isoshianin イソシアニン5937261).}\end{entry}
\begin{entry}
\mainentry{aisosutashi-}
{アイソスタシー}
{ Isostasie (f).}\end{entry}
\begin{entry}
\mainentry{aisosuru}
{哀訴する}
{[1] flehentlich bitten; inständig bitten; ein Bittgesuch einreichen. [2] jmdm. seine Not klagen; sich beschweren.}\end{entry}
\begin{entry}
\mainentry{aisotaipu}
{アイソタイプ; ISOTYPE}
{  \textit{Chem.} Isotyp (m) (z. B. eines Kristalls).}\end{entry}
\begin{entry}
\mainentry{aisodukashi}
{愛想尽かし; 愛想尽し; 愛想尽; 愛想づかし}
{ [1] Handlungenfpl und Wortenpl, die dazu führen, dass man abgestoßen wird. [2] Aufhören (n) der Freundschaft. [3] Berechnung (f); Abrechnung (f).}\end{entry}
\begin{entry}
\mainentry{aisodukashiwoiu }
{愛想づかしを言う}
{sich von jmdm. lossagen; mit jdmm. brechen.}\end{entry}
\begin{entry}
\mainentry{aisodukashiwoiu }
{愛想尽かしを言う}
{Gehässigkeiten sagen; mit dem Bruch drohen.}\end{entry}
\begin{entry}
\mainentry{aisodukashiwosuru}
{愛想尽かしをする}
{etw. tun, um sich von jmdm. zu entfremden.}\end{entry}
\begin{entry}
\mainentry{aisoto-pu}
{アイソトープ; Isotope; ISOTOPE}
{  \textit{Phys.} Isotop (n) (Atom, das sich von einem andern des gleichen chem. Elements nur in seiner Masse unterscheidet; ⇒ dōi·tai 同位体8061425).}\end{entry}
\begin{entry}
\mainentry{aisoto-pukamera}
{アイソトープ・カメラ; アイソトープカメラ}
{Isotop-Kamera (f).}\end{entry}
\begin{entry}
\mainentry{aisoto-pushousha}
{アイソトープ照射}
{ \textit{Phys., Med.} Isotopenbestrahlung (f).}\end{entry}
\begin{entry}
\mainentry{aisoto-pudeshoushasuru}
{アイソトープで照射する}
{mit Isotopen bestrahlen.}\end{entry}
\begin{entry}
\mainentry{aisoto-pudenchi}
{アイソトープ電池}
{Isotopenbatterie (f); Radionuklidbatterie (f).}\end{entry}
\begin{entry}
\mainentry{aisoto-puryouhou}
{アイソトープ療法}
{ \textit{Med.} Isotop-Therapie (f).}\end{entry}
\begin{entry}
\mainentry{aisoto-n}
{アイソトーン}
{  \textit{Kernphys.} Isotonenpl (Atomkerne mit gleicher Neutronen‑, aber unterschiedlicher Protonenzahl).}\end{entry}
\begin{entry}
\mainentry{aisotonikku}
{アイソトニック}
{  \textit{Phys.} isotonisch; den gleichen osmotischen Druck habend (Lösungen).}\end{entry}
\begin{entry}
\mainentry{aisotonikkuinryou}
{アイソトニック飲料}
{isotonisches Getränk (n) (Getränk, das für einen ausgeglichenen Mineralstoffhaushalt im Körper sorgen soll).}\end{entry}
\begin{entry}
\mainentry{aisotonikkusu}
{アイソトニックス}
{ Isotonik (f) (von engl. isotonics).}\end{entry}
\begin{entry}
\mainentry{aisotoron}
{アイソトロン}
{  \textit{Phys.} Isotron (n) (Gerät zur Isotopentrennung).}\end{entry}
\begin{entry}
\mainentry{aisonakukotowaru}
{愛想無く断る}
{glatt abschlagen.}\end{entry}
\begin{entry}
\mainentry{aisonikakeru}
{愛想に欠ける}
{es an gutem Willen fehlen lassen, nicht freundlich genug sein.}\end{entry}
\begin{entry}
\mainentry{aisonoii}
{愛想のいい}
{entgegenkommend; gesellig; gefällig; gastfreundlich; zuvorkommend; aufmerksam; umgänglich.}\end{entry}
\begin{entry}
\mainentry{aisonotsukiruhanashi}
{愛想の尽きる話}
{widerwärtige Angelegenheit (f).}\end{entry}
\begin{entry}
\mainentry{aisononai}
{愛想のない}
{unfreundlich; barsch; grob; rücksichtslos.}\end{entry}
\begin{entry}
\mainentry{aisononaikeshiki}
{愛想の無い景色}
{wenig gastfreundliches Land (n).}\end{entry}
\begin{entry}
\mainentry{aisononaihenji}
{愛想の無い返事}
{kurzangebundene Antwort (f).}\end{entry}
\begin{entry}
\mainentry{aisonometorikkuzuhou}
{アイソノメトリック図法}
{ isometrische Darstellung (f); isometrische Projektion (f).}\end{entry}
\begin{entry}
\mainentry{aisonoyoi}
{愛想のよい}
{entgegenkommend; gesellig; gefällig; gastfreundlich; zuvorkommend; aufmerksam; umgänglich.}\end{entry}
\begin{entry}
\mainentry{aisoba-}
{アイソバー}
{  \textit{Meteor.} Isobare (f).}\end{entry}
\begin{entry}
\mainentry{aisoparametorikku}
{アイソパラメトリック}
{  \textit{Math.} isoperimetrisch.}\end{entry}
\begin{entry}
\mainentry{aisofo-mu}
{アイソフォーム}
{  \textit{Chem.} Isoform (n) (Molekül von identischer Zusammensetzung aber unterschiedlichem Aufbau wie ein anderes) //  \textit{Molekularbiol.} Isoform (n) (Version z. B. eines Proteins mit leichten oder größeren Unterschieden zur sonstigen Form).}\end{entry}
\begin{entry}
\mainentry{aisoposu}
{アイソポス}
{  \textit{Persönlichk.} Äsop (Held einer frühgriech. volkstüml. Erzählung; ⇒ Isoppu イソップ7924624).}\end{entry}
\begin{entry}
\mainentry{aisoma}
{アイソマ}
{ Isomer (m).}\end{entry}
\begin{entry}
\mainentry{aisoma-}
{アイソマー}
{ Isomer (m).}\end{entry}
\begin{entry}
\mainentry{aisoma-shifuto}
{アイソマー・シフト; アイソマーシフト}
{ \textit{Chem.} Isomerieverschiebung (f) (von engl. isomer shift).}\end{entry}
\begin{entry}
\mainentry{aisome}
{アイソメ}
{ [1] Isometrie (f). [2] isometrische Darstellung (f); isometrische Projektion (f).}\end{entry}
\begin{entry}
\mainentry{aizome}
{藍染め; 藍染}
{Indigofärberei (f).}\end{entry}
\begin{entry}
\mainentry{aisometsuke}
{藍染め付け; 藍染付け; 藍染付; 青華瓷}
{ [1] blau gefärbter Stoff (m). [2] Keramik (f) mit kobaltblauer Unterglasurbemalung (f).}\end{entry}
\begin{entry}
\mainentry{aisometorikusu}
{アイソメトリクス}
{  \textit{Geom., Physiol.} Isometrie (f) (von engl. isometrics).}\end{entry}
\begin{entry}
\mainentry{aisometorikku}
{アイソメトリック}
{  \textit{Geom., Physiol.} Isometrie (f); isometrisch.}\end{entry}
\begin{entry}
\mainentry{aisometorikkusu}
{アイソメトリックス}
{  \textit{Geom., Physiol.} Isometrie (f) (von engl. isometrics).}\end{entry}
\begin{entry}
\mainentry{aisometorikkuzuhou}
{アイソメトリック図法}
{isometrische Darstellung (f); isometrische Projektion (f).}\end{entry}
\begin{entry}
\mainentry{aizomenogijutsu}
{藍染めの技術}
{Indigofärberei (f); Technik (f), mit Indigo zu färben.}\end{entry}
\begin{entry}
\mainentry{aizomeya}
{藍染屋}
{Färber (m); Indigofärber (m).}\end{entry}
\begin{entry}
\mainentry{aisomokosomotsukihateru}
{愛想も小想も尽き果てる; 愛想もこそも尽き果てる}
{die Nase von jmdm. voll haben.}\end{entry}
\begin{entry}
\mainentry{aisomonaku}
{愛想もなく}
{in unfreundlicher Weise; harsch; grob; rücksichtslos.}\end{entry}
\begin{entry}
\mainentry{aisoyoihohoemiwoukabete}
{愛想よい微笑みを浮かべて}
{mit einem liebenswürdigen Lächeln.}\end{entry}
\begin{entry}
\mainentry{aisoyoku}
{愛想よく}
{[1] liebenswürdig. [2] auf freundliche Weise.}\end{entry}
\begin{entry}
\mainentry{aisoyokusuru}
{愛想よくする}
{liebenswürdig zu jmdm. sein; freundlich sein; sich umgänglich zeigen.}\end{entry}
\begin{entry}
\mainentry{aisoyokufurumau}
{愛想よく振る舞う}
{sich liebenswürdig verhalten.}\end{entry}
\begin{entry}
\mainentry{aisoyokumotenasu}
{愛想よくもてなす}
{jmdn. freundlich empfangen.}\end{entry}
\begin{entry}
\mainentry{aisore-shon}
{アイソレーション}
{ Isolation (f).}\end{entry}
\begin{entry}
\mainentry{aisore-shonshindoro-mu}
{アイソレーション・シンドローム; アイソレーションシンドローム}
{Isolationssyndrom (n) (von engl. isolation syndrome).}\end{entry}
\begin{entry}
\mainentry{aisore-shonbu-su}
{アイソレーション・ブース; アイソレーションブース}
{schalldichter Raum (m) im Fernsehstudio (von engl. isolation booth).}\end{entry}
\begin{entry}
\mainentry{aisore-ta}
{アイソレータ}
{ Isolator (m).}\end{entry}
\begin{entry}
\mainentry{aisore-ta-}
{アイソレーター}
{ Isolator (m).}\end{entry}
\begin{entry}
\mainentry{aisoretto}
{アイソレット}
{ [1]  \textit{Wz.} Isolette (f) (eine Art Wärmeisolierglas). [2]  \textit{Med.} (Bez. f.) Brutkasten (m); Inkubator (n).}\end{entry}
\begin{entry}
\mainentry{aisowarai}
{愛想笑い; 愛想笑}
{süßes Lachen (n); versöhnliches Lachen (n) // höflich oberflächliches Lachen (n), wenn man etwas nicht wirklich lustig findet.}\end{entry}
\begin{entry}
\mainentry{aisowaraisuru}
{愛想笑いする; 愛想笑する}
{jmdn. schmeichlerisch anlächeln.}\end{entry}
\begin{entry}
\mainentry{aisowoiu}
{愛想を言う}
{Komplimente machen; jmdm. schmeicheln.}\end{entry}
\begin{entry}
\mainentry{aisowotsukasareru}
{愛想をつかされる}
{jmds. Gunst verlieren; aufgegeben werden; entfremdet werden.}\end{entry}
\begin{entry}
\mainentry{aisowotsukasu}
{愛想を尽かす; 愛想をつかす}
{genug haben von …; etw. satt haben; angewidert sein; nichts mehr wissen wollen von …; jmdn. aufgeben; sich jmdm. entfremden; jmdn. nicht mehr lieben; das Vertrauen verlieren.}\end{entry}
\begin{entry}
\mainentry{aison}
{愛孫}
{ geliebter Enkel (m).}\end{entry}
\begin{entry}
\mainentry{aita }
{あ痛; あいた [1]}
{ Aua!; Autsch!}\end{entry}
\begin{entry}
\mainentry{aida }
{あいだ; 間 [2]}
{ [1] Raum (m); Zeitraum (m); Intervall (n). [3] Beziehung (f).}\end{entry}
\begin{entry}
\mainentry{aita }
{愛他}
{ Altruismus (m); Nächstenliebe (f).}\end{entry}
\begin{entry}
\mainentry{aida }
{英田}
{  \textit{Ortsn.} Aida (Ortschaft im Osten der Präf. Okayama).}\end{entry}
\begin{entry}
\mainentry{aita }
{開いた; 空いた; 明いた; あいた [2]}
{ [1] offen (z. B. Fenster, Tür); auf. [2] leer; frei; unbesetzt; unbenutzt.}\end{entry}
\begin{entry}
\mainentry{aida }
{会田}
{  \textit{Familienn.} Aida.}\end{entry}
\begin{entry}
\mainentry{aida }
{簑田}
{  \textit{Familienn.} Aida.}\end{entry}
\begin{entry}
\mainentry{aida }
{藍田 [1]}
{Indigofeld (n).}\end{entry}
\begin{entry}
\mainentry{aita-n}
{Iターン; アイ・ターン; アイターン}
{  \textit{Soziol.} Phänomen (n), dass sich Uni-Absolventen, die in der Stadt studiert haben, eine Stelle in der Provinz suchen und sich dort niederlassen (von japan.-engl. I turn).}\end{entry}
\begin{entry}
\mainentry{aitai}
{I帯}
{  \textit{Anat.} I‑Band (n); isotropes Band (n) (helles Band über quergestreifte Muskulaturfasern).}\end{entry}
\begin{entry}
\mainentry{aitai }
{あいたい (靉靆)}
{ [1] Dahinziehen (n) der Wolken // Dunkelheit (f) der Wolken; Dicke (f) der Wolken. [2] Brille (f).}\end{entry}
\begin{entry}
\mainentry{aitai }
{相対 [1]}
{ Tête-à-tête (n); Direktheit (f); Unvermitteltheit (f) // nur zwischen zwei Personen; ohne dritte Partei.}\end{entry}
\begin{entry}
\mainentry{aitaiadoresu}
{相対アドレス}
{ \textit{EDV} relative Adresse (f).}\end{entry}
\begin{entry}
\mainentry{aitaiadoresuko-dyingu}
{相対アドレスコーディング}
{ \textit{EDV} relative Codierung (f).}\end{entry}
\begin{entry}
\mainentry{aitaikanjou}
{相対勘定}
{ \textit{Buchhaltung} Gegenkonto (n); Wertberichtigungskonto (n).}\end{entry}
\begin{entry}
\mainentry{aitaikenka}
{相対けんか (相対喧嘩)}
{Streit (m).}\end{entry}
\begin{entry}
\mainentry{aitaikenkawosuru}
{相対けんかをする}
{einen Streit beginnen.}\end{entry}
\begin{entry}
\mainentry{aitaishitesuwaru }
{相対して座る [1]}
{sich gegenüber sitzen.}\end{entry}
\begin{entry}
\mainentry{aitaishitesuwaru }
{相対して座る [2] (相対して坐る)}
{sich gegenüber sitzen.}\end{entry}
\begin{entry}
\mainentry{aitaijini}
{相対死; 相対死に}
{Doppelselbstmord (m).}\end{entry}
\begin{entry}
\mainentry{aitaizuku}
{相対ずく (相対尽く; 相対尽)}
{ [1] direkt; unter vier Augen. [2] laut gegenseitiger Übereinkunft; im gegenseitigen Einverständnis.}\end{entry}
\begin{entry}
\mainentry{aitaizukuda}
{相対尽くだ; 相対尽だ; 相対ずくだ}
{im gegenseitigen Einverständnis sein.}\end{entry}
\begin{entry}
\mainentry{aitaizukude}
{相対ずくで (相対尽で)}
{im gegenseitigen Einverständnis.}\end{entry}
\begin{entry}
\mainentry{aitaisuru}
{相対する}
{ (sich) gegenüberstehen; gegenüber sein; gegenüberliegen.}\end{entry}
\begin{entry}
\mainentry{aitaisurukaku}
{相対する角}
{gegenüberliegende Winkel (m).}\end{entry}
\begin{entry}
\mainentry{aitaisurugunzei}
{相対する軍勢}
{sich gegenüberstehende Heerenpl.}\end{entry}
\begin{entry}
\mainentry{aitaisuruhito}
{相対する人}
{jmds. Gegenüber (n); Visavis (n).}\end{entry}
\begin{entry}
\mainentry{aitaiserusanshou}
{相対セル参照}
{ \textit{EDV} relative Zellreferenz (f).}\end{entry}
\begin{entry}
\mainentry{aitaisouba}
{相対相場}
{ausgehandelter Marktpreis (m).}\end{entry}
\begin{entry}
\mainentry{aitaitaru}
{あいたいたる (靉靆たる)}
{[1] dick; dunkel (Wolken). [2] (übertr.) dunkel; düster.}\end{entry}
\begin{entry}
\mainentry{aitaide}
{相対で}
{persönlich; direkt; unmittelbar; unter sich; von Angesicht zu Angesicht; unter vier Augen.}\end{entry}
\begin{entry}
\mainentry{aitaitorihiki}
{相対取り引き; 相対取引き; 相対取引}
{ausgehandelter Geschäftsverkehr (m).}\end{entry}
\begin{entry}
\mainentry{aitaibaibai}
{相対売買}
{Geschäft (n), bei dem Preise und Mengen ausgehandelt sind.}\end{entry}
\begin{entry}
\mainentry{aitaimitainoichinenkara}
{会いたい見たいの一念から}
{aufgrund des brennenden Wunsches, jemanden zu sehen.}\end{entry}
\begin{entry}
\mainentry{aidaokuridenpou}
{間送電報}
{zurückgestelltes Telegramm (n).}\end{entry}
\begin{entry}
\mainentry{aitagaini}
{相互いに; 相互に [a]}
{ auf Gegenseitigkeit; gegenseitig; einander.}\end{entry}
\begin{entry}
\mainentry{aitagainiyuzuriau}
{相互に譲り合う}
{einander nachgeben; auf einander zugehen.}\end{entry}
\begin{entry}
\mainentry{aidagaumakuiku}
{間がうまくいく}
{mit etw. klar kommen.}\end{entry}
\begin{entry}
\mainentry{aidagasuiteiru}
{間が透いている}
{es ist eine Lücke zwischen ….}\end{entry}
\begin{entry}
\mainentry{aidagara}
{間柄; 間がら}
{ Verhältnis (n); Beziehung (f) (⇒ tsuzuki·gara 続柄7070185).}\end{entry}
\begin{entry}
\mainentry{aidagui}
{間食い; 間食 [a]}
{ Imbiss (m); Zwischenmahlzeit (f) (⇒ kanshoku 間食7439046).}\end{entry}
\begin{entry}
\mainentry{aidaguisuru}
{間食いする; 間食する}
{zwischen den Mahlzeiten essen.}\end{entry}
\begin{entry}
\mainentry{aitakuchigafusagaranai}
{開いた口がふさがらない (開いた口が塞がらない)}
{sprachlos sein; vollkommen verblüfft sein.}\end{entry}
\begin{entry}
\mainentry{aitakutetamaranai}
{会いたくてたまらない}
{jmdn. unbedingt treffen wollen; jmdn. sehr vermissen.}\end{entry}
\begin{entry}
\mainentry{aidajuu}
{間中 [1]}
{während.}\end{entry}
\begin{entry}
\mainentry{aitashugi}
{愛他主義}
{Altruismus (m) (⇒ hakuai·shugi 博愛主義0407970).}\end{entry}
\begin{entry}
\mainentry{aitashugisha}
{愛他主義者}
{Altruist (m).}\end{entry}
\begin{entry}
\mainentry{aitashin}
{愛他心}
{Altruismus (m).}\end{entry}
\begin{entry}
\mainentry{aitazusaete}
{相携えて; 相たずさえて}
{(schriftspr.) Hand in Hand; zusammen; Schulter an Schulter.}\end{entry}
\begin{entry}
\mainentry{aitazusaeru}
{相携える; 相たずさえる}
{ (schriftspr.) zusammenarbeiten; kooperieren; Hand in Hand arbeiten.}\end{entry}
\begin{entry}
\mainentry{aitaseki}
{空いた席}
{freier Sitzplatz (m); unbesetzter Stuhl (m).}\end{entry}
\begin{entry}
\mainentry{aitasetsu}
{愛他説}
{Altruismus (m).}\end{entry}
\begin{entry}
\mainentry{aitaxtsu}
{あ痛っ}
{ Aua!; Autsch!}\end{entry}
\begin{entry}
\mainentry{aidate}
{藍建て; 藍建}
{Indigo-Verküpung (f); Verwandlung (f) von unlöslichem Indigo in lösliches Indigoweiß durch Reduktion.}\end{entry}
\begin{entry}
\mainentry{aitateki}
{愛他的}
{altruistisch.}\end{entry}
\begin{entry}
\mainentry{aitatekikoudou}
{愛他的行動}
{altruistisches Verhalten (n).}\end{entry}
\begin{entry}
\mainentry{aitatokoronikakiireru}
{あいたところに書き入れる}
{die leeren Stellen ausfüllen; die Lücken ausfüllen.}\end{entry}
\begin{entry}
\mainentry{aidana}
{相店}
{ [1] gemeinsames Mieten (n) eines Hauses. [2] gemeinsam gemietetes Haus (n).}\end{entry}
\begin{entry}
\mainentry{aidani}
{間に}
{[1] unter.}\end{entry}
\begin{entry}
\mainentry{aidaniaru}
{間にある}
{eingreifen.}\end{entry}
\begin{entry}
\mainentry{aidanihaittahito}
{間に入った人}
{Mittelsmann (m); Vermittler (m).}\end{entry}
\begin{entry}
\mainentry{aidanihairu}
{間に入る}
{unterbrechen.}\end{entry}
\begin{entry}
\mainentry{aidaburyu-shi-}
{IWC; アイ・ダブリュー・シー; アイダブリューシー}
{  \textit{Org.} Internationale Walfangkommission (f) (Abk. für engl. International Whaling Commission; Abk.: IWC).}\end{entry}
\begin{entry}
\mainentry{aidaburyu-wai}
{IWY; アイ・ダブリュー・ワイ; アイダブリューワイ}
{ Internationales Jahr (n) der Frauen (1975).}\end{entry}
\begin{entry}
\mainentry{aidaho}
{アイダホ}
{  \textit{Gebietsn.} IdahonNAr (Bundesstaat der USA; Abk.: Id., Ida.).}\end{entry}
\begin{entry}
\mainentry{aidahoshuu}
{アイダホ州}
{ \textit{Gebietsn.} Bundesstaat (m) Idaho.}\end{entry}
\begin{entry}
\mainentry{aidahoshuuno}
{アイダホ州の}
{von Idaho.}\end{entry}
\begin{entry}
\mainentry{aidahoshuunohito}
{アイダホ州の人}
{Einwohner (m) des amerik. Bundesstaates Idaho.}\end{entry}
\begin{entry}
\mainentry{aidama}
{あい玉 (藍玉 [1])}
{Indigomn in Kugelform; Indigo-Farbstoff (m) in Kugelform.}\end{entry}
\begin{entry}
\mainentry{aidayasuaki}
{会田安明}
{  \textit{Persönlichk.} Aida Yasuaki (japan. Mathematiker; 1747–1817).}\end{entry}
\begin{entry}
\mainentry{aidawoakeru}
{間を空ける; 間を開ける; 間をあける [1]}
{Platz lassen zwischen etw.; Abstand lassen.}\end{entry}
\begin{entry}
\mainentry{aidawooite}
{間を置いて}
{in Abständen.}\end{entry}
\begin{entry}
\mainentry{aidawooitehanasu}
{間を置いて話す}
{Pause in seiner Rede machen; mit Pausen sprechen.}\end{entry}
\begin{entry}
\mainentry{aidawookazunihanasu}
{間を置かずに話す}
{ohne Pause reden.}\end{entry}
\begin{entry}
\mainentry{aidawosaku}
{間を裂く}
{sich zwischen etw. schieben.}\end{entry}
\begin{entry}
\mainentry{aidawosukashite}
{間を透かして; 間を空かして}
{in Abständen; spärlich.}\end{entry}
\begin{entry}
\mainentry{aidawosukasu}
{間を透かす; 間を空かす}
{Abstände zwischen den Pfosten.}\end{entry}
\begin{entry}
\mainentry{aidawotsumetekaku}
{間を詰めて書く; 間をつめて書く}
{eng schreiben; ohne Abstand schreiben.}\end{entry}
\begin{entry}
\mainentry{aidawotsumeru}
{間をつめる}
{keinen Abstand lassen.}\end{entry}
\begin{entry}
\mainentry{aidawotoru}
{間を取る}
{eine Pause machen.}\end{entry}
\begin{entry}
\mainentry{aitandokoro}
{あいたん所 (朝所)}
{ Aitan·dokoro (m); Palast (m) im Nordosten des Kabinettsamtsgebäudes für Diners und Staatsangelegenheiten.}\end{entry}
\begin{entry}
\mainentry{aichi }
{愛知}
{  \textit{Gebietsn.} AichinNAr (Name einer Präf. in der Chūbu-Region; Präfekturhauptstadt ist Nagoya).}\end{entry}
\begin{entry}
\mainentry{aichi }
{愛智}
{  \textit{Familienn.} Aichi.}\end{entry}
\begin{entry}
\mainentry{aichiikadaigaku}
{愛知医科大学}
{ \textit{Univ.-N.} Medizinische Universität (f) Aichi.}\end{entry}
\begin{entry}
\mainentry{aichiidai}
{愛知医大}
{ \textit{Univ.-N.} Medizinische Universität (f) Aichi (Abk.).}\end{entry}
\begin{entry}
\mainentry{aichikyouikudaigaku}
{愛知教育大学}
{ \textit{Univ.-N.} Nationale Universität (f) für Erziehungswesen Aichi (staatliche Universität in Kariya, Aichi; 1949 gegründet).}\end{entry}
\begin{entry}
\mainentry{aichiken}
{愛知県}
{ \textit{Gebietsn.} Präfektur (f) Aichi (in der Chūbu-Region).}\end{entry}
\begin{entry}
\mainentry{aichikenritsugeijutsudaigaku}
{愛知県立芸術大学}
{ \textit{Univ.-N.} Kunsthochschule (f) der Präfektur Aichi (öffentliche Universität in Nagakute, Aichi; 1966 gegründet).}\end{entry}
\begin{entry}
\mainentry{aichikougenkokuteikouen}
{愛知高原国定公園}
{ \textit{Gebietsn.} Aichi-Hochland-Quasinationalpark (m).}\end{entry}
\begin{entry}
\mainentry{aichibanpaku}
{愛知万博}
{Weltausstellung (f) in Aichi (2005).}\end{entry}
\begin{entry}
\mainentry{aichaku}
{愛着 [b]}
{ Zuneigung (f); Liebe (f); Anhänglichkeit (f); Vorliebe (f) (→ aijaku 愛着5273520; ⇒ shūchaku 執着7693181).}\end{entry}
\begin{entry}
\mainentry{aichakugatashiyoushouhin}
{愛着型仕様商品}
{No-Name-Produkt (n); Nicht-Marken-Produkt (n) (sehr seltener Ausdruck).}\end{entry}
\begin{entry}
\mainentry{aichakusuru}
{愛着する [b]}
{an jmdm. hängen; sein Herz an jmdn. hängen.}\end{entry}
\begin{entry}
\mainentry{aichakuwooboeru}
{愛着を覚える}
{jmdn. ins Herz schließen; jmdm. innig zugetan sein.}\end{entry}
\begin{entry}
\mainentry{aichakuwokanjiru}
{愛着を感じる}
{jmdn. ins Herz schließen; jmdm. innig zugetan sein; große Anhänglichkeit für jmdn. zeigen.}\end{entry}
\begin{entry}
\mainentry{aichakuwomotsu}
{愛着を持つ}
{sich an jmdn. hängen; Zuneigung zu jmdm. haben}\end{entry}
\begin{entry}
\mainentry{aichuu }
{間中 [2]; 相中 [1]}
{  \textit{Kabuki} Aichū (Schauspieler-Rang bzw. ein Schauspieler dieses Ranges).}\end{entry}
\begin{entry}
\mainentry{aichou }
{哀弔}
{ Betrauern (n) (den Tod eines Menschen).}\end{entry}
\begin{entry}
\mainentry{aichou }
{愛ちょう (愛寵)}
{ [1] Liebe (f); Zuneigung (f); Wohlwollen (n) (insbes. von einem edlen Menschen). [2] geliebte Person (f). (⇒ chōai 寵愛7687726).}\end{entry}
\begin{entry}
\mainentry{aichou }
{哀調}
{ traurige Melodie (f); wehmütiger Ton (m).}\end{entry}
\begin{entry}
\mainentry{aichou }
{愛重}
{ Lieben (n) und Ernst-Nehmen (n).}\end{entry}
\begin{entry}
\mainentry{aichou }
{愛鳥}
{ [1] Vogel (m) (als Haustier). [2] Zuneigung (f) zu Vögeln.}\end{entry}
\begin{entry}
\mainentry{aichouka}
{愛鳥家}
{Vogelliebhaber (m).}\end{entry}
\begin{entry}
\mainentry{aichoukahai}
{愛鳥家肺}
{  \textit{Med.} Vogelliebhaber’sche Lunge (f).}\end{entry}
\begin{entry}
\mainentry{aichoushuukan}
{愛鳥週間}
{Vogelwoche (f) (Woche vom 10. bis 16. Mai zum Schutz der Vögel; ⇒ bādo·wīku バードウィーク6982037).}\end{entry}
\begin{entry}
\mainentry{aichiyousui}
{愛知用水}
{ \textit{Ortsn.} Aichi-Kanal (m) (Bewässerungskanal der Wasser des Kiso-gawa durch die Nōbi-Ebene bis zur Chita-Halbinsel transportiert; 1961 fertiggestellt).}\end{entry}
\begin{entry}
\mainentry{aichousuru }
{愛重する}
{lieben und für wichtig halten.}\end{entry}
\begin{entry}
\mainentry{aichousuru }
{愛寵する}
{lieben; in Ehren halten.}\end{entry}
\begin{entry}
\mainentry{aichouwoobita}
{哀調をおびた; 哀調を帯びた}
{wehmutsvoll.}\end{entry}
\begin{entry}
\mainentry{aichouwoobitakoede}
{哀調を帯びた声で}
{mit wehmütiger Stimme.}\end{entry}
\begin{entry}
\mainentry{aitsu}
{あいつ; アイツ (彼奴 [1])}
{ (ugs.) er; sie; dieser Dingsda (m) (Ableitung von ayatsu あやつ; 彼奴5604793).}\end{entry}
\begin{entry}
\mainentry{aidu}
{会津}
{  \textit{Gebietsn.} AizunNAr (Gebiet im Westen der Präf. Fukushima mit dem Aizu-Becken in der Mitte).}\end{entry}
\begin{entry}
\mainentry{aitsuide}
{相次いで; 相継いで}
{ hintereinander; auf einander folgend; einer nach dem anderen; in rascher Folge.}\end{entry}
\begin{entry}
\mainentry{aitsuu}
{哀痛}
{ Betrauern.}\end{entry}
\begin{entry}
\mainentry{aitsuujiru}
{相通じる}
{ sich gegenseitig gefühlsmäßig verstehen.}\end{entry}
\begin{entry}
\mainentry{aitsuusuru}
{哀痛する}
{betrauern.}\end{entry}
\begin{entry}
\mainentry{aitsuuzuru}
{相通ずる}
{ [1] gemeinsam sein; auf alle zutreffen. [2] gut zusammen passen; harmonieren.}\end{entry}
\begin{entry}
\mainentry{aitsugu}
{相次ぐ; 相継ぐ; 相つぐ; あいつぐ}
{ (schriftspr.) aufeinander folgen; hintereinander geschehen.}\end{entry}
\begin{entry}
\mainentry{aitsugunau}
{相償う}
{ einen Schaden kompensieren.}\end{entry}
\begin{entry}
\mainentry{aitsugufukou}
{相次ぐ不幸}
{aufeinander folgende Unglückenpl.}\end{entry}
\begin{entry}
\mainentry{aidujou}
{会津城}
{ \textit{Ortsn.} Aizu-Schloss (n); Wakamatsu-Schloss (n); Kranich-Schloss (n) (teilweise wiedererrichtetes Schloss in Aizu-Wakamatsu).}\end{entry}
\begin{entry}
\mainentry{aidutakada}
{会津高田}
{  \textit{Stadtn.} Aizu·Takada (Stadt im Westen der Präf. Fukushima; ehemalige Poststation).}\end{entry}
\begin{entry}
\mainentry{aiduchi}
{相づち (相槌; 相鎚; 合槌)}
{ [1] abwechselnder Hammerschlag (m) beim Schmieden. [2] (übertr.) kurzer zustimmender Kommentar (m) zu dem was der Gesprächspartner sagt (wie „aha!“ oder „so so!“) // zustimmendes Zuhören (n).}\end{entry}
\begin{entry}
\mainentry{aiduchiwoutsu}
{相槌を打つ; 合槌を打つ}
{jmdm. zustimmen; jmdm. beistimmen; zustimmend nicken; zustimmen; beipflichten; sagen, dass man zustimmt.}\end{entry}
\begin{entry}
\mainentry{aitsutoshachouhatsuutsuuda。}
{あいつと社長はつうつうだ。}
{ \textit{Bsp.} Der steckt doch mit dem Boss unter einer Decke.}\end{entry}
\begin{entry}
\mainentry{aitsutomeru}
{相勤める; 相務める}
{ eine Rolle spielen; arbeiten.}\end{entry}
\begin{entry}
\mainentry{aitsunihakan'ninbukuronochogakiretayo。}
{あいつには堪忍袋の緒が切れたよ。}
{ \textit{Bsp.} Ich habe mit ihm die Geduld verloren. // Das ging über meine Hutschnur.}\end{entry}
\begin{entry}
\mainentry{aidunuri}
{会津塗り; 会津塗}
{Aizu-Lackware (f).}\end{entry}
\begin{entry}
\mainentry{aitsunoatsukamashisanihaakireru。}
{あいつの厚かましさにはあきれる。}
{ \textit{Bsp.} Ich bin sprachlos über sein Unverschämtheit.}\end{entry}
\begin{entry}
\mainentry{aitsuhaonbusuryadakkoda。}
{あいつはおんぶすりゃ抱っこだ。}
{ \textit{Bsp.} Wenn du ihm den Finger hinstreckst, nimmt er die ganze Hand.}\end{entry}
\begin{entry}
\mainentry{aitsuhanomuyorihokaninounonaiotokoda。}
{あいつは飲むよりほかに能のない男だ。}
{ \textit{Bsp.} Er kann nichts anderes als trinken.}\end{entry}
\begin{entry}
\mainentry{aidubange}
{会津坂下}
{  \textit{Stadtn.} Aizu·Bange (Stadt im Westen der Präf. Fukushima; ehemalige Poststation).}\end{entry}
\begin{entry}
\mainentry{aitsubo}
{あいつぼ (藍壺)}
{Indigomn.}\end{entry}
\begin{entry}
\mainentry{aiduhongou}
{会津本郷}
{  \textit{Stadtn.} Aizu·Hongō (Stadt im Westen der Präf. Fukushima).}\end{entry}
\begin{entry}
\mainentry{aiduma}
{合い褄; 合褄; 相褄}
{  \textit{Kleidung} Aizuma (n) (Teil des Kimonos unter dem Kragen) // am Aizuma gemessene Kimono-Breite.}\end{entry}
\begin{entry}
\mainentry{aitsura}
{あいつら}
{(ugs.) sie.}\end{entry}
\begin{entry}
\mainentry{aiduwakamatsu}
{会津若松}
{  \textit{Stadtn.} Aizu-WakamatsunNAr (Stadt im Westen der Präf. Fukushima; bekannt für Herstellung von Lacken; Tsuruga-jō, Burg des Fürsten Aizu aus dem 16. Jhd.).}\end{entry}
\begin{entry}
\mainentry{aiduwakamatsushikankouka}
{会津若松市観光課}
{  \textit{Verlagsn.} Aizu-Wakamatsushi Kankōka (Aizu-Wakamatsu).}\end{entry}
\begin{entry}
\mainentry{aitsuwomakasunohawakenaikotosa。}
{あいつを負かすのは訳ないことさ。}
{ \textit{Bsp.} Ich kann ihn ganz einfach schlagen.}\end{entry}
\begin{entry}
\mainentry{aite}
{相手}
{ [1] Partner (m); Kamerad (m); Kollege (m); Freund (m). [2] der andere; Gegner (m); Gegenspieler (m); Antagonist (m); Konkurrent (m). [3] Objekt (n). [4] anderer Teil (m) eines Paares; Pendant (n). (⇒ aibō 相棒</sp}\end{entry}
\begin{entry}
\mainentry{aidea}
{アイデア}
{ Idee (f) (von engl. idea; ⇒ aidia アイディア2247421).}\end{entry}
\begin{entry}
\mainentry{aideagaukabu}
{アイデアが浮かぶ}
{eine gute Idee haben; es kommt einem eine gute Idee.}\end{entry}
\begin{entry}
\mainentry{aideashouhin}
{アイデア商品}
{originelles Produkt (n).}\end{entry}
\begin{entry}
\mainentry{aideaman}
{アイデア・マン; アイデアマン}
{Mann (m) mit neuen Ideen (von engl. idea man).}\end{entry}
\begin{entry}
\mainentry{aidearisuto}
{アイデアリスト}
{ Idealist (m).}\end{entry}
\begin{entry}
\mainentry{aidearizumu}
{アイデアリズム}
{ Idealismus (m) (⇒ aidiarizumu アイディアリズム0973602).}\end{entry}
\begin{entry}
\mainentry{aidearu}
{アイデアル}
{ [1] ideal. [2] idealistisch.}\end{entry}
\begin{entry}
\mainentry{aidearuda}
{アイデアルだ}
{[1] ideal sein. [2] idealistisch sein.}\end{entry}
\begin{entry}
\mainentry{aideawodasu}
{アイデアを出す}
{mit einer guten Idee herausrücken.}\end{entry}
\begin{entry}
\mainentry{aideawotsunoru}
{アイデアを募る}
{Ideen sammeln; zu Ideen auffordern.}\end{entry}
\begin{entry}
\mainentry{aitei}
{愛弟}
{ [1] Liebe zum jüngeren Bruder // geliebter jüngerer Bruder (m). [2] Liebe (f) zu seinen Anhängern // geliebter Anhänger (m); geliebter Jünger (m).}\end{entry}
\begin{entry}
\mainentry{aidyia}
{アイディア}
{ Idee (f) (von engl. idea).}\end{entry}
\begin{entry}
\mainentry{aidyiashouhin}
{アイディア商品}
{neues Produkt (n) mit einer zündenden Idee.}\end{entry}
\begin{entry}
\mainentry{aidyiaman}
{アイディア・マン; アイディアマン}
{Mann (m) mit neuen Ideen; jmd.NArN, der immer gute Ideen hat (von engl. idea man).}\end{entry}
\begin{entry}
\mainentry{aidyiarisuto}
{アイディアリスト}
{ Idealist (m).}\end{entry}
\begin{entry}
\mainentry{aidyiarizumu}
{アイディアリズム}
{ Idealismus (m).}\end{entry}
\begin{entry}
\mainentry{aidyiaru}
{アイディアル}
{ ideal.}\end{entry}
\begin{entry}
\mainentry{aidyiaruda}
{アイディアルだ}
{ideal sein.}\end{entry}
\begin{entry}
\mainentry{aitexi-}
{IT; アイ・ティー; アイティー}
{ Informationstechnologie (f) (Akronym für engl. Information Technology).}\end{entry}
\begin{entry}
\mainentry{aidyi-}
{ID; アイ・ディー; アイディー}
{ Identifikation (f); Personalausweis (m).}\end{entry}
\begin{entry}
\mainentry{aidyi-i-}
{IDE; アイ・ディー・イー; アイディーイー}
{  \textit{EDV} IDE (f) (Abk. für engl. Integrated Drive Electronics).}\end{entry}
\begin{entry}
\mainentry{aidyi-i-inta-fe-su}
{IDEインターフェース}
{ \textit{EDV} IDE-Interface (n).}\end{entry}
\begin{entry}
\mainentry{aidyi-i-noha-dodyisukudoraibu}
{IDEのハードディスク・ドライブ; IDEのハードディスクドライブ}
{ \textit{EDV} IDE-Festplatte (f).}\end{entry}
\begin{entry}
\mainentry{aitexi-esu}
{ITS; アイ・ティー・エス; アイティーエス}
{  \textit{Kfz-W.} Intelligente (f) Transport-Systemenpl (von engl. Intelligent Transportation System; Stauvermeidung, elektronische Gebührenerhebung, Unfallwarnung etc.).}\end{entry}
\begin{entry}
\mainentry{aitexi-ka}
{IT化}
{Informatisierung (f) (etwa seit dem Jahr 2000 verwendet).}\end{entry}
\begin{entry}
\mainentry{aidyi-ka-do}
{IDカード; アイ・ディー・カード; アイディーカード}
{ Identitätskarte (f); Personalausweis (m); Dienstmarke (f) (von engl. ID card = Abk. für identification card).}\end{entry}
\begin{entry}
\mainentry{aitexi-shi-}
{ITC; アイ・ティ・ーシー; アイティーシー}
{  \textit{Org.} Internationale Handelskommission (f) (ITC; Abk. für engl. International Trade Commission).}\end{entry}
\begin{entry}
\mainentry{aidyi-shi-}
{IDC}
{  \textit{Firmenn.} International Data CorporationNAr.}\end{entry}
\begin{entry}
\mainentry{aitexi-zenekon}
{ITゼネコン}
{große IT-Firma (f); große Firma (f) der Informationstechnologie.}\end{entry}
\begin{entry}
\mainentry{aitexi-texi-}
{ITT; アイ・ティー・ティー; アイティーティー}
{  \textit{Firmenn.} ITTNAr; International Telephone and Telegraph Corporation (f) (amerik. Mischkonzern; gegründet 1920; Sitz: New York).}\end{entry}
\begin{entry}
\mainentry{aitexi-texi-efu}
{ITTF; アイ・ティー・ティー・エフ; アイティーティーエフ}
{  \textit{Sport} Internationale Tischtennis-Föderation (f).}\end{entry}
\begin{entry}
\mainentry{aidyi-texi-bui}
{IDTV; アイ・ディー・ティー・ブイ; アイディーティーブイ}
{  \textit{TV} höher auflösendes Fernsehen (Abk. für engl. improved definition TV).}\end{entry}
\begin{entry}
\mainentry{aidyi-bangou}
{ID番号; アイ・ディー番号; アイディー番号}
{ Identifikationsnummer (f) (von engl. identification und japan. „Nummer“; Abk.).}\end{entry}
\begin{entry}
\mainentry{aitexi-bijon}
{ITビジョン}
{  \textit{TV} IT-Vision (f); Intertext-Vision (f) (Fernsehen mit Feedback-Möglichkeit für die Zuseher).}\end{entry}
\begin{entry}
\mainentry{aitexi-bui}
{ITV; アイ・ティー・ブイ; アイティーブイ}
{ industrielles Fernsehen (n); Industriefernsehen (n) (Abk. für engl. industrial television).}\end{entry}
\begin{entry}
\mainentry{aitexi-yu-}
{ITU; アイ・ティー・ユー; アイティーユー}
{  \textit{Org.} Internationaler Fernmeldeverein (m) (Abk. für engl. International Telecommunication Union).}\end{entry}
\begin{entry}
\mainentry{aitexi-yu-texi-esuesu}
{ITU-TSS; アイ・ティー・ユー・ティー・エス・エス; アイティーユーティーエスエス}
{ \textit{Org.} Sektor (m) für Standardisierung im Fernmeldewesen; ITU-TSS (m) (von engl. ITU Telecommunication Standardization Sector).}\end{entry}
\begin{entry}
\mainentry{aitexi-yu-texi-kankoku}
{ITU-T勧告}
{ \textit{Org.} Empfehlung (f) des Sektors für Standardisierung im Fernmeldewesen; ITU-T-Empfehlung (f).}\end{entry}
\begin{entry}
\mainentry{aitexiesujapan}
{ITS-Japan; アイ・ティ・エス・ジャパン; アイティエスジャパン}
{  \textit{Firmenn.} ITS-JapannNAr.}\end{entry}
\begin{entry}
\mainentry{aidyirisuto}
{アイディリスト}
{ Idealist (m).}\end{entry}
\begin{entry}
\mainentry{aiteiru }
{開いている; あいている}
{[1] offen stehen. [2] leer sein; leer stehen.}\end{entry}
\begin{entry}
\mainentry{aiteiru }
{空いている}
{frei sein; nicht besetzt sein; nicht in Gebrauch sein.}\end{entry}
\begin{entry}
\mainentry{aiteiruie}
{あいている家}
{leer stehendes Haus (n).}\end{entry}
\begin{entry}
\mainentry{aiteiruisu}
{あいている椅子}
{leerer Stuhl (m).}\end{entry}
\begin{entry}
\mainentry{aiteirushinbun}
{あいている新聞}
{gerade nicht gelesene Zeitung (f).}\end{entry}
\begin{entry}
\mainentry{aiteirutsurikawa}
{あいている吊革}
{unbenutzter Halteriemen (m); freier Halteriemen (m).}\end{entry}
\begin{entry}
\mainentry{aiteirubin}
{あいているびん}
{leere Flasche (f).}\end{entry}
\begin{entry}
\mainentry{aiteiruheya }
{あいている部屋}
{unbenutztes Zimmer (m).}\end{entry}
\begin{entry}
\mainentry{aiteiruheya }
{空いている部屋}
{freies Zimmer (m).}\end{entry}
\begin{entry}
\mainentry{aiteiruhounote}
{空いている方の手; あいている方の手}
{jmds. freie Hand (f).}\end{entry}
\begin{entry}
\mainentry{aitegainai}
{相手がいない}
{keinen Partner haben.}\end{entry}
\begin{entry}
\mainentry{aitekata}
{相手方 [a]}
{ [1] die andere Seite (f); die gegnerische Seite (f). [2] Widersacher (m); Gegner (m). [3] (obsol.) Beklagter (m); Angeklagter (m) (Begriff aus der Edo-Zeit).}\end{entry}
\begin{entry}
\mainentry{aitegata}
{相手方 [b]}
{ [1] die andere Seite (f); die gegnerische Seite (f). [2] Widersacher (m); Gegner (m). [3] (obsol.) Beklagter (m); Angeklagter (m) (Begriff aus der Edo-Zeit). (→ aite·kata 相手方8369743).}\end{entry}
\begin{entry}
\mainentry{aiteganaitame}
{相手がないため}
{aus Mangel eines Partners.}\end{entry}
\begin{entry}
\mainentry{aitegawarui}
{相手が悪い}
{einen schwierigen Gegner haben.}\end{entry}
\begin{entry}
\mainentry{aitekawaredonushikawarazu}
{相手変われど主変わらず}
{sich nicht ändern, auch wenn die anderen sich ändern.}\end{entry}
\begin{entry}
\mainentry{aitekoku}
{相手国}
{Partnerland (n); Handelspartner (m); Partnerstaat (m) (Land, das das Ziel von Handel oder Diplomatie ist).}\end{entry}
\begin{entry}
\mainentry{aitesakishouhyouseihin}
{相手先商標製品}
{ \textit{Wirtsch.} Verkauf (m) der Waren eines Herstellers unter einem anderen Markennamen; OEM (n) (engl. original equipment manufacturing; ⇒ ō·ī·emu OEM0093890).}\end{entry}
\begin{entry}
\mainentry{aideshi}
{相弟子}
{ Mitschüler (m); Lehrlingskollege (m).}\end{entry}
\begin{entry}
\mainentry{aiteshidai}
{相手次第}
{Abhängigkeit (f) vom Charakter oder Verhalten des anderen.}\end{entry}
\begin{entry}
\mainentry{aiteshidaide}
{相手次第で}
{je nach Partner.}\end{entry}
\begin{entry}
\mainentry{aitetsu}
{あい鉄 (藍鉄)}
{ bläulich-violettes Schwarz (n).}\end{entry}
\begin{entry}
\mainentry{aitetsuiro}
{藍鉄色}
{bläulich-violoettes Schwarz (n).}\end{entry}
\begin{entry}
\mainentry{aitedoru}
{相手取る; 相手どる}
{ [1] sich jmdn. zum Gegner machen; jmdn. herausfordern. [2] gegen jmdn. klagen.}\end{entry}
\begin{entry}
\mainentry{aitenashini}
{相手なしに}
{ohne Partner.}\end{entry}
\begin{entry}
\mainentry{aitenishinai}
{相手にしない}
{gar nicht ernst nehmen; sich mit jmdm. nicht abgeben wollen; jmdn. ignorieren.}\end{entry}
\begin{entry}
\mainentry{aitenisuru}
{相手にする}
{es mit jmdm. zu tun haben; von jmdm. Notiz nehmen.}\end{entry}
\begin{entry}
\mainentry{aiteninaru}
{相手になる (相手に成る)}
{Gesellschaft leisten; mitspielen; zusammenspielen; Partner werden.}\end{entry}
\begin{entry}
\mainentry{aitenoatamaniichigekiwokuwaeyoutonerau}
{相手の頭に一撃を加えようとねらう}
{mit einem Schlag auf den Kopf des Gegners zielen.}\end{entry}
\begin{entry}
\mainentry{aitenoatamaniichigekiwokuwaeyokunerau}
{相手の頭に一撃を加えよくねらう}
{mit einem Schlag auf den Kopf des Gegners zielen.}\end{entry}
\begin{entry}
\mainentry{aitenoitogatenitoruyouniwakaru}
{相手の意図が手に取るように分かる}
{jmds. Absicht gut verstehen.}\end{entry}
\begin{entry}
\mainentry{aitenogironwofunsaisuru}
{相手の議論を粉砕する}
{jmds. Argumente vernichten.}\end{entry}
\begin{entry}
\mainentry{aitenotaidonimukatsuku。}
{相手の態度にむかつく。}
{ \textit{Bsp.} Sein Benehmen macht mich ganz krank.}\end{entry}
\begin{entry}
\mainentry{aitenotakuraramiwofunsaisuru}
{相手の企らみを粉砕する}
{die Pläne des Gegners vereiteln.}\end{entry}
\begin{entry}
\mainentry{aitenodagekiwohazusu}
{相手の打撃を外す}
{den Schlägen des Gegners ausweichen.}\end{entry}
\begin{entry}
\mainentry{aitenotewoyomu}
{相手の手を読む}
{den nächsten Zug des Gegners erahnen.}\end{entry}
\begin{entry}
\mainentry{aitenonai}
{相手のない}
{ohne Partner; partnerlos.}\end{entry}
\begin{entry}
\mainentry{aitenopanchiwokaihisuru}
{相手のパンチを回避する}
{dem Schlag eines Gegners ausweichen.}\end{entry}
\begin{entry}
\mainentry{aitenope-suwomidasu}
{相手のペースを乱す}
{das Tempo des Gegners durcheinander bringen.}\end{entry}
\begin{entry}
\mainentry{aitenohougashii}
{相手の方が強い}
{ \textit{Bsp.} Der Gegner ist stärker.}\end{entry}
\begin{entry}
\mainentry{aitenohoutsuyoi。}
{相手の方強い。}
{ \textit{Bsp.} Der Gegner ist stärker.}\end{entry}
\begin{entry}
\mainentry{aitenome}
{相手の目}
{gegnerisches Gebiet (n).}\end{entry}
\begin{entry}
\mainentry{aitenorikiryouwomisadameru}
{相手の力量を見定める}
{die Kraft des Gegners abschätzen.}\end{entry}
\begin{entry}
\mainentry{aitebo-ruwointa-seputosuru}
{相手ボールをインターセプトする}
{den Pass des Gegners abfangen.}\end{entry}
\begin{entry}
\mainentry{aitemu}
{アイテム}
{ [1] Element (n); Eintrag (m); Einzelangabe (f); Posten (m); Bestandteil (m); Einheit (f). [2] (übertr.) erforderlicher Artikel (m); unbedingt benötigter Gegenstand (m). (von engl. item).}\end{entry}
\begin{entry}
\mainentry{aiteyaku}
{相手役}
{Partner (m); Filmpartner (m); Gegenspieler (m) (im Theat. od. Film).}\end{entry}
\begin{entry}
\mainentry{aiteyakuwotsutomeru}
{相手役を勤める}
{den Partner von jmdm. spielen.}\end{entry}
\begin{entry}
\mainentry{aiterasu}
{相照らす}
{ [1] sich gegenseitig anstrahlen. [2] etw. mit einander vergleichen.}\end{entry}
\begin{entry}
\mainentry{aiterikishinitsupparimakeru}
{相手力士に突っ張り負ける}
{ \textit{Sumō} den Gegner mit den Händen stoßen und verlieren.}\end{entry}
\begin{entry}
\mainentry{aitewouchigakedetaosu}
{相手を内掛けで倒す}
{ \textit{Sumō} den Gegner mit einem Uchigake werfen.}\end{entry}
\begin{entry}
\mainentry{aitewokekaesu}
{相手を蹴返す}
{einen Gegner zu Boden treten.}\end{entry}
\begin{entry}
\mainentry{aitewosuru}
{相手をする}
{jmdm. Gesellschaft leisten.}\end{entry}
\begin{entry}
\mainentry{aitewosotogakedetaosu}
{相手を外掛けで倒す}
{ \textit{Sumō} den Gegner mit einem äußerer Fußkantenwurf zu Boden werfen.}\end{entry}
\begin{entry}
\mainentry{aitewodohyougiwaheoshitsumeru}
{相手を土俵ぎわへ押し詰める}
{den Gegner an den Rand des Ringes treiben.}\end{entry}
\begin{entry}
\mainentry{aitewonondekakaru}
{相手を呑んでかかる}
{seinen Gegner nicht ernst nehmen.}\end{entry}
\begin{entry}
\mainentry{aitewofurikittego-rusuru}
{相手を振り切ってゴールする}
{seinen Gegner abschütteln und ein Tor machen.}\end{entry}
\begin{entry}
\mainentry{aiden}
{相殿 [a]}
{ [1] Verehrung (f) von mehr als einer Gottheit in einer Schreinhalle. [2] Schreinhalle (f), in der mehr als eine Gottheit verehrt wird. (→ ai·dono 相殿9530619).}\end{entry}
\begin{entry}
\mainentry{aidentexitexi}
{アイデンティティ}
{ Identität (f) (von engl. identity).}\end{entry}
\begin{entry}
\mainentry{aidentexitexi-}
{アイデンティティー}
{ Identität (f) (von engl. identity).}\end{entry}
\begin{entry}
\mainentry{aidentexitexi-ka-do}
{アイデンティティー・カード; アイデンティティーカード}
{Identitätskarte (f) (von engl. identity card).}\end{entry}
\begin{entry}
\mainentry{aidentexitexi-kuraishisu}
{アイデンティティー・クライシス; アイデンティティークライシス}
{Identitätskrise (f).}\end{entry}
\begin{entry}
\mainentry{aidentexitexi-nokiki}
{アイデンティティーの危機}
{Identitätskrise (f).}\end{entry}
\begin{entry}
\mainentry{aidentexitexi-wosagasu}
{アイデンティティーを探す; アイデンティティーをさがす}
{seine Identität suchen.}\end{entry}
\begin{entry}
\mainentry{aidentexitexi-woshuchousuru}
{アイデンティティーを主張する}
{seine Identität geltend machen.}\end{entry}
\begin{entry}
\mainentry{aidentexitexi-womiushinau}
{アイデンティティーを見失う}
{seine Identität verlieren.}\end{entry}
\begin{entry}
\mainentry{aidentexitexikuraishisu}
{アイデンティティ・クライシス; アイデンティティクライシス}
{Identitätskrise (f) (von engl. identity crisis).}\end{entry}
\begin{entry}
\mainentry{aidentexifai}
{アイデンティファイ}
{ Identifikation (f).}\end{entry}
\begin{entry}
\mainentry{aidentexifike-shon}
{アイデンティフィケーション}
{ Identifikation (f).}\end{entry}
\begin{entry}
\mainentry{aidentexifike-shonka-do}
{アイデンティフィケーション・カード; アイデンティフィケーションカード}
{Identifikationskarte (f).}\end{entry}
\begin{entry}
\mainentry{aitou }
{哀悼}
{ Trauer (f) (über den Tod eines Menschen); Beileid (n); Kondolenz (f); Mitgefühl (n) mit Trauernden (⇒ aiseki 哀惜3509631).}\end{entry}
\begin{entry}
\mainentry{aitou }
{愛東}
{  \textit{Ortsn.} Aitō (Ortschaft im östlichen Zentrum der Präf. Shiga).}\end{entry}
\begin{entry}
\mainentry{aitouka}
{哀悼歌}
{Elegie (f); Klagelied (n).}\end{entry}
\begin{entry}
\mainentry{aidousa}
{I動作}
{ \textit{Regelungstechnik} Integralverhalten (n); I-Verhalten (n); integrierendes Verhalten (n); Verhalten (n) eines I-Gliedes.}\end{entry}
\begin{entry}
\mainentry{aitousha}
{哀悼者}
{Trauernder (m); Trauergast (m).}\end{entry}
\begin{entry}
\mainentry{aitoujiru}
{相投じる}
{ [1] miteinander übereinstimmen; harmonieren. [2] in derselben Unterkunft übernachten.}\end{entry}
\begin{entry}
\mainentry{aitoushin}
{愛党心}
{ Parteiloyalität (f); Parteiverbundenheit (f).}\end{entry}
\begin{entry}
\mainentry{aitousuru}
{哀悼する}
{um jmds. Tod trauern.}\end{entry}
\begin{entry}
\mainentry{aitouzuru}
{相投ずる}
{ miteinander übereinstimmen; harmonieren.}\end{entry}
\begin{entry}
\mainentry{aitounoiwoarawashite}
{哀悼の意を表して}
{um die Trauer zu zeigen.}\end{entry}
\begin{entry}
\mainentry{aitounoiwoarawasu}
{哀悼の意を表す}
{sein Beileid ausdrücken.}\end{entry}
\begin{entry}
\mainentry{aitounoiwohyousuru}
{哀悼の意を表する}
{sein Beileid ausdrücken; seiner Trauer Ausdruck verleihen.}\end{entry}
\begin{entry}
\mainentry{aitounoji}
{哀悼の辞}
{Trauerrede (f); Grabrede (f).}\end{entry}
\begin{entry}
\mainentry{aitounojiwonoberu}
{哀悼の辞を述べる}
{eine Trauerrede halten.}\end{entry}
\begin{entry}
\mainentry{aidoka}
{AIDCA; アイドカ}
{  \textit{Marketing} Aufmerksamkeit (f), Interesse (n), Wunsch (m), Überzeugung (f), Aktion (f) (fünf Schritte bis zum Kauf einer Ware; Abk. für engl. attention, interest, desire, conviction, action</sp}\end{entry}
\begin{entry}
\mainentry{aidoku}
{愛読}
{ Spaß (m) am Lesen; Liebe (f) zum Lesen.}\end{entry}
\begin{entry}
\mainentry{aitoguuzentonotawamure}
{愛と偶然との戯れ; 愛と偶然のたわむれ}
{ \textit{Werktitel} NAr (Lustspiel von Pierre Carlet Chamblain de Marivaux; 1730).}\end{entry}
\begin{entry}
\mainentry{aidokusha}
{愛読者}
{Leser (m); Leserkreis (m).}\end{entry}
\begin{entry}
\mainentry{aidokushagaooi}
{愛読者が多い}
{einen großen Leserkreis haben.}\end{entry}
\begin{entry}
\mainentry{aidokusho}
{愛読書}
{Lieblingsbuch (n).}\end{entry}
\begin{entry}
\mainentry{aidokusuru}
{愛読する}
{gern lesen; mit wachsendem Interesse lesen // regelmäßig lesen.}\end{entry}
\begin{entry}
\mainentry{aidokunohon}
{愛読の本}
{jmds. Lieblingsbuch (n).}\end{entry}
\begin{entry}
\mainentry{aidogurafu}
{アイドグラフ}
{ Eidograph (m) (eine Art Pantograph).}\end{entry}
\begin{entry}
\mainentry{aidoko}
{合い床; 合床; 相床}
{ Nebeneinanderlegen (n) der Betten // nebeneinander liegende Bettennpl.}\end{entry}
\begin{entry}
\mainentry{aidoshi}
{相年}
{ Gleichaltrigkeit (f).}\end{entry}
\begin{entry}
\mainentry{aitotonou}
{相整う; 相調う}
{ [1] abgeschlossen sein (Vorbereitungen; das Heiratsarrangement). [2] in Ordnung sein; geordnet sein. [3] harmonisieren.}\end{entry}
\begin{entry}
\mainentry{aidona-}
{アイ・ドナー; アイドナー}
{ \textit{Med.} Hornhaut-Spender (m); Augenspender (m) (von engl. eye donor).}\end{entry}
\begin{entry}
\mainentry{aidono}
{相殿 [b]}
{ [1] Verehrung (f) von mehr als einer Gottheit in einer Schreinhalle. [2] Schreinhalle (f), in der mehr als eine Gottheit verehrt wird.}\end{entry}
\begin{entry}
\mainentry{aidoforu}
{EIDOPHOR [a]; アイドフォル}
{ Eidophor (n) (eine Fernsehgroßbild-Projektionsanlage).}\end{entry}
\begin{entry}
\mainentry{aidoho-ru}
{EIDOPHOR [b]; アイドホール}
{ Eidophor (n) (eine Fernsehgroßbild-Projektionsanlage).}\end{entry}
\begin{entry}
\mainentry{aitoma-tofu}
{アイトマートフ}
{  \textit{Persönlichk.} Tschingis Aitmatow (kirgisischer Schriftsteller; 1928–).}\end{entry}
\begin{entry}
\mainentry{aidomanohousoku}
{アイドマの法則}
{ \textit{Werbung} AIDAMAs Regel (f) (Abk. für engl. attention, interest, desire, memory und action).}\end{entry}
\begin{entry}
\mainentry{aitomonau}
{相伴う; 相伴なう}
{ [1] begleiten; mit jmdm. gehen. [2] mit sich bringen; begleitet sein von etw // zusammengehen; übereinstimmen. (verstärkte Version von tomonau). (⇒ tsure·datsu 連れ立つ0350784).}\end{entry}
\begin{entry}
\mainentry{aitomoni}
{相共に; 相ともに}
{ miteinander; gemeinsam; zusammen.}\end{entry}
\begin{entry}
\mainentry{aidori}
{合い取り; 合取り; 合取; 相取り; 相取}
{ [1] gemeinsames Tun (n) //  jmd., mit dem man etw. gemeinsam tut. [2] gemeinsames Kneten (n) von Teig für Mochi // jmd. mit dem man Mochi-Teig knetet (Mochi-Kneten ist ein recht komplexer Vorgang; eine Person stampft den Teig mit einer Art großen Holzhammer in einem Mörser; die andere Person wendet den Teig; wenn man nicht gut zusammenarbeitet, wird der Teigwender verletzt).}\end{entry}
\begin{entry}
\mainentry{aidorisuru}
{合い取りする; 合取りする; 合取する; 相取りする; 相取する}
{[1] etw. gemeinsam tun. [2] gemeinsam Teig für Mochi kneten.}\end{entry}
\begin{entry}
\mainentry{aidoringu}
{アイドリング}
{  \textit{Kfz-W.} Leerlauf (m) (von engl. idling).}\end{entry}
\begin{entry}
\mainentry{aidoringusuru}
{アイドリングする}
{im Leerlauf laufen; leerlaufen.}\end{entry}
\begin{entry}
\mainentry{aidoru}
{アイドル [1]}
{ [1] Idol (n). [2] Star (m); Starlet (n).}\end{entry}
\begin{entry}
\mainentry{aidoru…}
{アイドル… [2]}
{ Leerlauf (m) // leer laufend; untätig; unproduktiv (von engl. idle).}\end{entry}
\begin{entry}
\mainentry{aidorukashu}
{アイドル歌手}
{Popsänger (m) (der vor allem durch sein Aussehen und weniger durch Talent zum Star wird); (entsprechende) Popsängerin (f).}\end{entry}
\begin{entry}
\mainentry{aidorukayou}
{アイドル歌謡}
{von einem Star gesungenes Lied (n).}\end{entry}
\begin{entry}
\mainentry{aidorugia}
{アイドル・ギア; アイドルギア}
{ \textit{Kfz-W.} Zwischenrad (n) (von engl. idler gear).}\end{entry}
\begin{entry}
\mainentry{aidorugiya}
{アイドル・ギヤ; アイドルギヤ}
{ \textit{Kfz-W.} Zwischenrad (n) (von engl. idler gear).}\end{entry}
\begin{entry}
\mainentry{aidorukyapitaru}
{アイドル・キャピタル; アイドルキャピタル}
{brachliegendes Kapital (n); totes Kapital (n); unproduktives Kapital (n) (von engl. idle capital).}\end{entry}
\begin{entry}
\mainentry{aidorukosuto}
{アイドル・コスト; アイドルコスト}
{unproduktive Kostenpl (von engl. idle cost).}\end{entry}
\begin{entry}
\mainentry{aidorushisutemu}
{アイドル・システム; アイドルシステム}
{unproduktives System (n) (von engl. idle system).}\end{entry}
\begin{entry}
\mainentry{aidorutaimu}
{アイドル・タイム; アイドルタイム}
{unproduktive Zeit (f) (von engl. idle time).}\end{entry}
\begin{entry}
\mainentry{aidoruro-ru}
{アイドル・ロール; アイドルロール}
{Stützwalze (f) (von engl. idle roll).}\end{entry}
\begin{entry}
\mainentry{ainaka}
{相中 [2]; 相仲}
{ [1] Zwischenraum (m) // zwischen. [2] Verhältnis (n); Beziehung (f) // (insbes.) vertrautes Verhältnis (n). (vor allem in der Meiji-Zeit verwendetes Wort).}\end{entry}
\begin{entry}
\mainentry{ainakabasuru}
{相半ばする; 相半する; 相なかばする}
{ (schriftspr.) [1] sich gegenseitig aufheben; sich ausgleichen. [2] Halbe-Halbe sein.}\end{entry}
\begin{entry}
\mainentry{ainakikekkon}
{愛なき結婚}
{Heirat (f) ohne Liebe.}\end{entry}
\begin{entry}
\mainentry{ainadanomi}
{あいな頼み; あいな頼}
{ Bitte (f), deren Erfüllung unwahrscheinlich ist // übertriebene Erwartung (f); unangemessene Erwartung (f).}\end{entry}
\begin{entry}
\mainentry{ainame}
{アイナメ; あいなめ (鮎魚女; 鮎並; 鮎並女; 愛女)}
{  \textit{Fischk.} Ainame (m) (ca. 40 cm langer essbarer Seefisch; Hexagrammos otakii).}\end{entry}
\begin{entry}
\mainentry{ainarandeiku}
{相並んで行く}
{neben einander gehen.}\end{entry}
\begin{entry}
\mainentry{ainarandetatsu}
{相並んで立つ}
{Schulter an Schulter stehen; neben einander stehen.}\end{entry}
\begin{entry}
\mainentry{ainaru}
{相成る; 相なる; あいなる}
{ (förml.) werden.}\end{entry}
\begin{entry}
\mainentry{ainiueru}
{愛に飢える}
{nach Liebe hungern.}\end{entry}
\begin{entry}
\mainentry{ainioboreru}
{愛におぼれる (愛に溺れる)}
{abgöttisch lieben; in jmdn. vernarrt sein.}\end{entry}
\begin{entry}
\mainentry{ainiku}
{あいにく (生憎 [a]; 合憎)}
{ leider; unglücklicherweise; ungelegen; zu ungelegener Zeit.}\end{entry}
\begin{entry}
\mainentry{ainikuonozominohinhagozaimasen。}
{あいにくお望みの品はございません。}
{ \textit{Bsp.} Ich fürchte, wir haben nicht, was Sie wünschen.}\end{entry}
\begin{entry}
\mainentry{ainikusonohihaamedatta。}
{生憎その日は雨だった。}
{ \textit{Bsp.} Leider hat es an diesem Tag geregnet.}\end{entry}
\begin{entry}
\mainentry{ainikuda}
{あいにくだ (生憎だ)}
{unglücklich sein; ungelegen sein; unpassend sein.}\end{entry}
\begin{entry}
\mainentry{ainikuto}
{生憎と}
{unglücklicherweise; leider.}\end{entry}
\begin{entry}
\mainentry{ainikunagara}
{あいにくながら}
{Es tut mir leid, aber ….}\end{entry}
\begin{entry}
\mainentry{ainikunakotoniminarusudatta。}
{あいにくな事にみな留守だった。}
{ \textit{Bsp.} Leider war niemand zu Hause.}\end{entry}
\begin{entry}
\mainentry{ainikuno}
{あいにくの (生憎の)}
{unglücklich; unpassend; ungelegen; unzeitig.}\end{entry}
\begin{entry}
\mainentry{ainikunotenki}
{あいにくの天気}
{schlechtes Wetter (n).}\end{entry}
\begin{entry}
\mainentry{ainikunotenkida。}
{あいにくの天気だ。}
{ \textit{Bsp.} Es ist unpassendes Wetter.}\end{entry}
\begin{entry}
\mainentry{ainisomeru}
{藍に染める}
{indigoblau färben; tiefblau färben.}\end{entry}
\begin{entry}
\mainentry{ainihodasareru}
{愛にほだされる}
{von seinen Gefühlen gefesselt sein (üblicher ist jō ni hoda·sareru 情にほだされる0746055).}\end{entry}
\begin{entry}
\mainentry{ainimukuiru}
{愛に報いる}
{jmds. Liebe erwidern.}\end{entry}
\begin{entry}
\mainentry{ainiyuku}
{会いに行く}
{gehen, um jmdn. zu treffen.}\end{entry}
\begin{entry}
\mainentry{ainu}
{アイヌ; Ainu; AINU}
{ Ainu (m).}\end{entry}
\begin{entry}
\mainentry{ainuken}
{アイヌ犬}
{ \textit{Zool.} Ainu-Hund (m) (japanische Hunderasse).}\end{entry}
\begin{entry}
\mainentry{ainugo}
{アイヌ語}
{ \textit{Sprache} Ainu-Sprache (f); Ainu (m).}\end{entry}
\begin{entry}
\mainentry{ainujin}
{アイヌ人}
{ \textit{Anthropol.} Ainu (m).}\end{entry}
\begin{entry}
\mainentry{ainushinpou}
{アイヌ新法}
{ \textit{Rechtsw.} Neues Ainu-Gesetz (n) (Kurzbezeichnung für ein Gesetz zur Bewahrung und Förderung der Ainukultur von 1997).}\end{entry}
\begin{entry}
\mainentry{ainusetsu}
{アイヌ説}
{ \textit{Anthropol.} Ainu-Theorie (f) (Theorie, dass die Ainu die Ureinwohner Japans waren).}\end{entry}
\begin{entry}
\mainentry{ainuzoku}
{アイヌ族}
{Ainu-Volk (n).}\end{entry}
\begin{entry}
\mainentry{ainuno}
{アイヌの}
{ainuisch.}\end{entry}
\begin{entry}
\mainentry{ainuminzoku}
{アイヌ民族}
{Ainu-Volk (n).}\end{entry}
\begin{entry}
\mainentry{aineiasu}
{アイネイアス}
{  \textit{Persönlichk.} ÄneasmNAr (Held der griech.-röm. Sage).}\end{entry}
\begin{entry}
\mainentry{aineku}
{アイネク}
{  \textit{Werktitel}  (von W. A. Mozart; Abk.).}\end{entry}
\begin{entry}
\mainentry{ainezu}
{あいねず (藍鼠 [a])}
{ bläuliches Grau (n) (→ ai·nezumi 藍鼠1141950).}\end{entry}
\begin{entry}
\mainentry{ainezuno}
{藍ねずの}
{blaugrau.}\end{entry}
\begin{entry}
\mainentry{ainezumi}
{あいねずみ (藍鼠 [b])}
{ bläuliches Grau (n).}\end{entry}
\begin{entry}
\mainentry{ainen}
{愛念}
{ [1] starke Liebe. [2] Liebe (f); Zuneigung (f) // Leidenschaft (f); Lust (f).}\end{entry}
\begin{entry}
\mainentry{aino }
{アイノ}
{ Ainu (m) (⇒ Ainu アイヌ9872951).}\end{entry}
\begin{entry}
\mainentry{aino }
{愛野}
{  \textit{Ortsn.} Aino (Ortschaft im Südosten der Präf. Nagasaki).}\end{entry}
\begin{entry}
\mainentry{aino }
{あいの; アイの (藍の)}
{indigoblau; indigo.}\end{entry}
\begin{entry}
\mainentry{ainoakariwotomosu}
{愛の灯をともす}
{(poet.) jmdn. verehren; in jmdn. verliebt sein.}\end{entry}
\begin{entry}
\mainentry{ainogakkou}
{愛の学校}
{  \textit{Werktitel} NAr (Roman von Edmondo De Amicis; 1886).}\end{entry}
\begin{entry}
\mainentry{ainokami}
{愛の神}
{Liebesgott (m); Amor (m); Cupido (m); Eros (m) // Liebesgöttin (f); Venus (f); Aphrodite (f).}\end{entry}
\begin{entry}
\mainentry{ainokawaki}
{愛の渇き}
{Durst (m) nach Liebe.}\end{entry}
\begin{entry}
\mainentry{ainokizuna}
{愛の絆}
{Band (n) der Liebe; Liebesbeziehung (f); Fesselnfpl der Liebe.}\end{entry}
\begin{entry}
\mainentry{ainokisuwosuru}
{愛のキスをする}
{aus Liebe küssen.}\end{entry}
\begin{entry}
\mainentry{ainokyougen}
{間の狂言}
{ \textit{Theat.} Zwischenspiel (n).}\end{entry}
\begin{entry}
\mainentry{ainokusabi}
{合いのくすび (合いの楔; 合の楔; 間の楔)}
{ [1] Verbindendungskeil (m). [2] Nebentätigkeit (f). [3] Einschub (m) (zur Unterstützung).}\end{entry}
\begin{entry}
\mainentry{ainokekkon}
{愛の結婚}
{Liebesheirat (f) (⇒ ren’ai·kekkon 恋愛結婚8983529).}\end{entry}
\begin{entry}
\mainentry{ainokesshou}
{愛の結晶}
{Frucht (f) der Liebe // (insbes.) Nachwuchs (m); Kind (n).}\end{entry}
\begin{entry}
\mainentry{ainoko}
{合いの子; 合の子; あいの子; あいのこ (間の子)}
{ [1] Mischling (m); Bastard (m); Mischrasse (f). [2] Hybrid; Kreuzung (f). [3] Zwischending (n). (⇒ konketsu·ji 混血児1015978).}\end{entry}
\begin{entry}
\mainentry{ainokokuhaku}
{愛の告白}
{Liebeserklärung (f).}\end{entry}
\begin{entry}
\mainentry{ainokokuhakuwosuru}
{愛の告白をする}
{jmdm. seine Liebe erklären; eine Liebeserklärung machen.}\end{entry}
\begin{entry}
\mainentry{ainokono}
{合の子の (間の子の)}
{Mischling…; Hybrid…; gemischt.}\end{entry}
\begin{entry}
\mainentry{ainokobentou}
{合の子弁当 (間の子弁当)}
{ \textit{Kochk.} Bentō (n) mit Reis und westlichen Beilagen.}\end{entry}
\begin{entry}
\mainentry{ainokomottategami}
{愛のこもった手紙}
{liebevoller Brief (m).}\end{entry}
\begin{entry}
\mainentry{ainokowotsukuru}
{合の子を作る; 合の子をつくる}
{kreuzen.}\end{entry}
\begin{entry}
\mainentry{ainogonge}
{愛の権化}
{Inkarnation (f) der Liebe.}\end{entry}
\begin{entry}
\mainentry{ainosasayaki}
{愛のささやき}
{Liebesgeflüster (n); Liebesgespräch (n).}\end{entry}
\begin{entry}
\mainentry{ainosabaku}
{愛の砂漠}
{ \textit{Werktitel} NAr (Roman von François Mauriac; 1925).}\end{entry}
\begin{entry}
\mainentry{ainoshuku}
{合いの宿; 合の宿 (間の宿)}
{ [1] Herberge (f) zwischen den Herbergsstationen. [2] behelfsmäßige Benutzung (f).}\end{entry}
\begin{entry}
\mainentry{ainoshirushi }
{愛のしるし}
{Zeichen (n) der Liebe; Liebesgabe (f); Liebespfand (m).}\end{entry}
\begin{entry}
\mainentry{ainoshirushi }
{愛の印}
{Zeichen (n) der Liebe.}\end{entry}
\begin{entry}
\mainentry{ainosu}
{愛の巣}
{Liebesnest (n).}\end{entry}
\begin{entry}
\mainentry{ainosuke}
{愛之助}
{  \textit{männl. Name} Ainosuke.}\end{entry}
\begin{entry}
\mainentry{ainosuwoitonamu}
{愛の巣を営む}
{sich ein Liebesnest einrichten // einen Haushalt zu zweien günden.}\end{entry}
\begin{entry}
\mainentry{ainotaishou}
{愛の対象}
{Objekt (n) der Liebe.}\end{entry}
\begin{entry}
\mainentry{ainotaishouhasorezoretengokunochuushinda。}
{愛の対象はそれぞれ天国の中心だ。}
{ \textit{Bsp.} Jeder geliebte Gegenstand ist der Mittelpunkt eines Paradieses. ().}\end{entry}
\begin{entry}
\mainentry{ainote }
{合いの手; 合の手; 相の手 (間の手)}
{ [1]  \textit{Mus.} Zwischenspiel (n) (bei Shamisen-Musik zwischen Vokalpartien). [2]  \textit{Mus.} eingeworfener Ruf zur Belebung (f) traditioneller japanischer Musik. [3] Zwischenruf (m); Zwischenbemerkung (f).}\end{entry}
\begin{entry}
\mainentry{ainote }
{愛の手}
{helfende Hand (f).}\end{entry}
\begin{entry}
\mainentry{ainotewoireru}
{合の手を入れる}
{[1] ein Zwischenspiel einfügen. [2] Rufe einfügen.}\end{entry}
\begin{entry}
\mainentry{ainotewonukasu}
{合の手を抜かす}
{das Zwischenspiel weglassen.}\end{entry}
\begin{entry}
\mainentry{ainoto}
{合いの戸}
{Tür (f) zwischen Räumen.}\end{entry}
\begin{entry}
\mainentry{ainonai}
{愛のない}
{lieblos; gefühlskalt; unfreundlich.}\end{entry}
\begin{entry}
\mainentry{ainonaikatei}
{愛のない家庭}
{lieblose Familie (f).}\end{entry}
\begin{entry}
\mainentry{ainohi}
{愛の日}
{Valentinstag (m).}\end{entry}
\begin{entry}
\mainentry{ainoma}
{相の間; 合の間}
{ Zwischenzimmer (m); Verbindungszimmer (m).}\end{entry}
\begin{entry}
\mainentry{ainomyouyaku}
{愛の妙薬}
{ [1]  \textit{Werktitel} NAr (komische Oper von Gaetano Donizetti; 1832). [2] (poet.; übertr.) erfolgreiche Liebesstrategie (f).}\end{entry}
\begin{entry}
\mainentry{ainomuchi}
{愛の鞭}
{aus Liebe zugefügte Bestrafung (f) // Prügelstrafe (f).}\end{entry}
\begin{entry}
\mainentry{ainomebae}
{愛の芽生え}
{aufkeimende Liebe (f).}\end{entry}
\begin{entry}
\mainentry{ainori}
{相乗り; 相乗 [1]; あい乗り; あいのり (合い乗り)}
{ [1] Fahren (n) zu zweit; Zusammenfahren (n). [2] gemeinsame Unternehmung (f).}\end{entry}
\begin{entry}
\mainentry{ainorikouho}
{相乗り候補}
{Kandidat (m), der von mehr als einer Partei unterstützt wird.}\end{entry}
\begin{entry}
\mainentry{ainorisuru}
{相乗りする; 相乗する; あい乗りする}
{zusammen fahren; etw. zusammen unternehmen.}\end{entry}
\begin{entry}
\mainentry{ainoribangumi}
{相乗り番組; 相乗番組}
{ \textit{TV, Radio} Programm (n), das von verschiedenen Firmen gesponsert wird.}\end{entry}
\begin{entry}
\mainentry{aiba }
{愛馬}
{ [1] Lieblingspferd (n). [2] Zuneigung (f) zu einem Pferd.}\end{entry}
\begin{entry}
\mainentry{aiba }
{合端; 合歯}
{  \textit{Archit.} zusammengesetzte Oberfläche (f).}\end{entry}
\begin{entry}
\mainentry{aiba }
{相葉}
{  \textit{Familienn.} Aiba.}\end{entry}
\begin{entry}
\mainentry{aiba-}
{アイ・バー; アイバー}
{ Ösenstange (f); Ösenbolzen (m) (von engl. eyebar).}\end{entry}
\begin{entry}
\mainentry{aibaku}
{アイバク}
{  \textit{Persönlichk.} Qutb-ud-din Aibak (Begründer des Sultanat von Delhi; ?–1210).}\end{entry}
\begin{entry}
\mainentry{aibagokoro}
{愛馬心}
{Zuneigung (f) zu seinem Pferd.}\end{entry}
\begin{entry}
\mainentry{aihateru}
{相果てる}
{ sterben; vergehen; enden (verstärkte Form zu hateru 果てる7712064).}\end{entry}
\begin{entry}
\mainentry{aihamazumiuchikara。}
{愛はまず身内から。}
{ \textit{Sprichw.} Nächstenliebe beginnt daheim.}\end{entry}
\begin{entry}
\mainentry{aihamu}
{相はむ (相食む)}
{ sich gegenseitig fressen (⇒ kotsuniku·aihamu 骨肉相食む3183662).}\end{entry}
\begin{entry}
\mainentry{aibayashi}
{間ばやし (間囃子)}
{  \textit{Theat.} zwischen den Sketchen des Yose-Theaters aufgeführte Musikstücke.}\end{entry}
\begin{entry}
\mainentry{aihara}
{相原}
{  \textit{Familienn.} Aihara.}\end{entry}
\begin{entry}
\mainentry{aihan }
{合判 [1]; 合い判 [1]; 相判 [1]}
{ [1] gemeinsamer Stempel (m). [2] Identifizierungsstempel (m). (⇒ ai·in 合印5310706).}\end{entry}
\begin{entry}
\mainentry{aiban }
{相判 [2]; 合い判 [2]; 合判 [2] (間判)}
{ Aiban (n); mittlere Größe (f) (von Heften, Fotos, Holzdrucken etc.; ⇒ ai·in 合い印5310706).}\end{entry}
\begin{entry}
\mainentry{aihan }
{相反 [1]}
{ Widerspruch (m); Gegensatz (m) (⇒ ai·hansuru 相反する0736417).}\end{entry}
\begin{entry}
\mainentry{aiban }
{相番}
{ gemeinsamer Dienst (m) mit jmdm. anderen // Person (f), mit der man Dienst hat; Person (f) mit demselben Dienst.}\end{entry}
\begin{entry}
\mainentry{aihan }
{相版}
{ gemeinsame Veröffentlichung (f).}\end{entry}
\begin{entry}
\mainentry{aibanku}
{アイ・バンク; アイバンク}
{ \textit{Med.} Augenbank (f); Hornhautbank (f) (hält menschliche Augenhornhäute für die Transplantation bereit; von engl. eye bank).}\end{entry}
\begin{entry}
\mainentry{aihansuru}
{相反する}
{ widersprechen; im Gegensatz zu einander stehen.}\end{entry}
\begin{entry}
\mainentry{aihansurugainen}
{相反する概念}
{widersprüchliche Konzeptionenfpl; gegensätzliche Ideenfpl.}\end{entry}
\begin{entry}
\mainentry{aihansurukanjounikurushimu}
{相反する感情に苦しむ}
{sich mit widersprechenden Gefühlen quälen.}\end{entry}
\begin{entry}
\mainentry{aihansuruchikara}
{相反する力}
{entgegengesetzte Kräftefpl.}\end{entry}
\begin{entry}
\mainentry{aibanho-}
{アイバンホー}
{  \textit{Persönlichk.} Ivanhoe (Held der gleichnamigen Erzählung von Sir Walter Scott, die im englische Mittelalter spielt).}\end{entry}
\begin{entry}
\mainentry{aihi}
{愛妃}
{ geliebte Königin (f); geliebte Prinzessin (f).}\end{entry}
\begin{entry}
\mainentry{aibi-}
{アイビー}
{ [1] Efeu (m). (von engl. ivy). [2] Ivy League (f) (acht exklusive und angesehene Universitäten im Nordosten der USA; Abk.). [3]  \textit{Mode} Ivy Style (m) (Abk.).}\end{entry}
\begin{entry}
\mainentry{aipi-}
{IP; アイ・ピー; アイピー}
{ [1]  \textit{EDV} Information-Provider (m). [2]  \textit{EDV} IP (n) ().}\end{entry}
\begin{entry}
\mainentry{aipi-a-ru}
{IPR; アイ・ピー・アール; アイピーアール}
{  \textit{Org.} Institut (n) für Beziehungen um den Pazifik (Abk. für engl. Institute of Pacific Relations).}\end{entry}
\begin{entry}
\mainentry{aibi-a-rudyi-}
{IBRD; アイ・ビー・アール・ディー; アイビーアールディー}
{  \textit{Org.} Internationale Bank für Wiederaufbau und Entwicklung; Weltbank (Abk. für engl. International Bank for Reconstruction and Development).}\end{entry}
\begin{entry}
\mainentry{aipi-adoresu}
{IPアドレス}
{ \textit{EDV} IP-Adresse (f).}\end{entry}
\begin{entry}
\mainentry{aibi-e-}
{IBA; アイ・ビー・エー; アイビーエー}
{  \textit{Sport} Internationaler Basketball-Verband (für Amatuere); IBA (Abk. für engl. International Baseball Association).}\end{entry}
\begin{entry}
\mainentry{aibi-emu}
{IBM; アイ・ビー・エム; アイビーエム}
{ [1]  \textit{Firmenn.} IBMNAr; International Business MachinesNAr (amerik. Computerfirma; http://www.ibm.com/ bzw. http://www.ibm.co.jp/). [2] (übertr.) Computer (von IBM).}\end{entry}
\begin{entry}
\mainentry{aibi-emugokanki}
{IBM互換機}
{ \textit{EDV} IBM-Kompatibilität (f).}\end{entry}
\begin{entry}
\mainentry{aibi-emuchu-rihikenkyuujo}
{IBMチューリヒ研究所}
{ \textit{Ortsn.} IBM Forschungslabor Zürich; (engl.) IBM Zurich Research Laboratory.}\end{entry}
\begin{entry}
\mainentry{aibi-emupi-shi-}
{IBM PC; アイ・ビー・エム・ピー・シー; アイビーエムピーシー}
{ \textit{Wz.} IBM PC (m).}\end{entry}
\begin{entry}
\mainentry{aipi-eru}
{IPL; アイ・ピー・エル; アイピーエル}
{  \textit{EDV} IPL (m) (Abk. für engl. initial program loader).}\end{entry}
\begin{entry}
\mainentry{aipi-o-dyi-}
{IPOD [a]; アイ・ピー・オー・ディー; アイピーオーディー}
{ Internationales Tiefseebohrprojekt (n); IPOD (n) (Abk. für engl. International Project of Ocean Drilling).}\end{entry}
\begin{entry}
\mainentry{aibi-katto}
{アイビー・カット; アイビーカット}
{College-Haarschnitt (m) (von engl. ivy cut).}\end{entry}
\begin{entry}
\mainentry{aibi-karejji}
{アイビー・カレッジ; アイビーカレッジ}
{College (n) der Ivy League (von engl. Ivy college).}\end{entry}
\begin{entry}
\mainentry{aibi-je-}
{IBJ; アイ・ビー・ジェー; アイビージェー}
{  \textit{Firmenn.} Industrial Bank of JapanNAr.}\end{entry}
\begin{entry}
\mainentry{aipi-su}
{アイ・ピース; アイピース}
{ \textit{Optik} Okular (n) (von engl. eyepiece).}\end{entry}
\begin{entry}
\mainentry{aibi-sutairu}
{アイビー・スタイル; アイビースタイル}
{College-Stil (m); Kleidung (f), wie sie an den amerikanischen Ostküsten-Unis der Ivy League getragen wird (von japan.-engl. ivy style).}\end{entry}
\begin{entry}
\mainentry{aipi-texi-ji-}
{IPTG; アイ・ピー・ティー・ジー; IPTG; アイピーティージー}
{  \textit{Biochem.} Isopropyl-β-D-thiogalactopyranosid (n); IPTG (n).}\end{entry}
\begin{entry}
\mainentry{aipi-paketto}
{IPパケット}
{ \textit{EDV} IP-Paket (n).}\end{entry}
\begin{entry}
\mainentry{aibi-mu}
{Iビーム; アイ・ビーム; アイビーム}
{ [1]  \textit{Archit., Maschinenb.} Doppel-T-Träger (m). [2]  \textit{EDV} I-Beam (m); I-Balken (m); I-Balken-Zeiger (m). (von engl. I-beam).}\end{entry}
\begin{entry}
\mainentry{aipi-yu-}
{IPU; アイ・ピー・ユー; アイピーユー}
{  \textit{Org.} interparlamentarische Union (f); (engl.) Inter-Parliamentary Union (f).}\end{entry}
\begin{entry}
\mainentry{aibi-ri-gu}
{アイビー・リーグ; アイビーリーグ}
{ Ivy League (f) (Zusammenschluss von acht berühmten Universitäten der amerikanischen Ostküste; Harvard, Yale, Columbia, Princeton etc.).}\end{entry}
\begin{entry}
\mainentry{aibi-ri-gunogakusei}
{アイビーリーグの学生}
{Student (m) der Ivy League.}\end{entry}
\begin{entry}
\mainentry{aibi-ri-gunoshusshinsha}
{アイビーリーグの出身者}
{Absolvent (m) der Ivy League.}\end{entry}
\begin{entry}
\mainentry{aibi-ri-gumoderu}
{アイビー・リーグ・モデル; アイビーリーグモデル}
{ \textit{Mode} College-Look (von engl. Ivy League model).}\end{entry}
\begin{entry}
\mainentry{aibi-rukku}
{アイビー・ルック; アイビールック}
{ \textit{Kleidung} [1] College-Stil (m) (Look, wie er an den amerik. Ostküsten-Unis der Ivy Leage getragen wird). [2] Kleidung im College-Stil.}\end{entry}
\begin{entry}
\mainentry{aihixendorufu}
{アイヒェンドルフ}
{  \textit{Persönlichk.} Joseph Freiherr von Eichendorff (dtsch. romantischer Dichter; 1788–1857).}\end{entry}
\begin{entry}
\mainentry{aibiki }
{あいびき (逢い引き; 逢い引; 逢引き; 逢引; 媾い曳き; 媾い曳; 媾曳き; 媾曳)}
{ Rendezvous (n); Stelldichein (n) (⇒ mikkai 密会9525751).}\end{entry}
\begin{entry}
\mainentry{aibiki }
{合いびき [1]; 合びき (合い挽き; 合い挽; 合挽き; 合挽)}
{ Hackfleisch (n) von Rind und Schwein (das gemeinsam gehackt wurde).}\end{entry}
\begin{entry}
\mainentry{aibiki }
{相引き; 相引; 合い引き; 合引き; 合引; 合いびき [2]}
{ [1] geheime Absprache (f); Konspiration (f). [2] Ziehen (n) in verschiedene Richtungen. [3] gleichzeitiger Rückzug (m). [4] Spannen (n) der Bögen gegeneinander. (⇒ nare·ai 馴れ合い9355137). [5]  \textit{Kabuki} Kordell (f) an der Perücke // Schmuckkordell}\end{entry}
\begin{entry}
\mainentry{aibikisuru}
{逢い引きする; 逢引きする; 逢引する; 媾曵きする; 媾曵する; あいびきする}
{mit jmdm. ein Stelldichein haben; sich mit jmdm. ein Rendezvous geben; ein Rendezvous haben.}\end{entry}
\begin{entry}
\mainentry{aibikisuru }
{合い引きする; 合引きする; 合引する; 相引きする; 相引する}
{[1] sich geheim absprechen; konspierieren. [2] in verschiedene Richtungen ziehen. [3] sich gleichzeitig zurückziehen. [4] gegeneinander spannen.}\end{entry}
\begin{entry}
\mainentry{aihitoshii}
{相等しい; 相ひとしい; あい等しい}
{ gleich sein.}\end{entry}
\begin{entry}
\mainentry{aihiman}
{アイヒマン}
{  \textit{Persönlichk.} Karl Adolf Eichmann (Organisator der Transporte jüdischer Menschen in die dtsch. Vernichtungslager; 1906–1962).}\end{entry}
\begin{entry}
\mainentry{aibyou}
{愛猫}
{ Schmusekatze (f); Lieblingskatze (f).}\end{entry}
\begin{entry}
\mainentry{aibyouka}
{愛猫家}
{Katzenliebhaber (m).}\end{entry}
\begin{entry}
\mainentry{aihira-}
{アイヒラー}
{  \textit{Persönlichk.} August Wilhelm Eichler (deutscher Botaniker; 1839–1887).}\end{entry}
\begin{entry}
\mainentry{aibin}
{哀びん (哀愍 [a]; 哀憫 [a])}
{ Mitleid (n); Mitgefühl (n).}\end{entry}
\begin{entry}
\mainentry{aifu}
{合い符; 合符}
{ Gepäckschein (m).}\end{entry}
\begin{entry}
\mainentry{aibu}
{愛ぶ (愛撫)}
{ Liebkosung (f); Zärtlichkeit (f).}\end{entry}
\begin{entry}
\mainentry{aibuibi-efu}
{IVBF; アイ・ブイ・ビー・エフ; アイブイビーエフ}
{  \textit{Sport} Internationale Volleyball Föderation (f).}\end{entry}
\begin{entry}
\mainentry{aibu-mu}
{iブーム}
{ Boom (m) an Neologismen, die ein kleines „i“ für Internet enthalten (wie iMac, i-mode; iPod etc.; begann 1998 mit Apples iMac).}\end{entry}
\begin{entry}
\mainentry{aiferu}
{アイフェル}
{  \textit{Bergn.} Eifel (f) (westl. Teil Rheinischen Schiefergebirges).}\end{entry}
\begin{entry}
\mainentry{aifo-}
{IFO; アイフォー}
{ identifiziertes fliegendes Objekt (n) (⇔ yūfō UFO8322270).}\end{entry}
\begin{entry}
\mainentry{aifo-me-shon}
{Iフォーメーション; アイ・フォーメーション; アイフォーメーション}
{  \textit{American Football} I-Formation (f) (eine Angriffsformation).}\end{entry}
\begin{entry}
\mainentry{aifo-ru}
{IFOR; アイフォール}
{  \textit{Org.} IFOR (f); UN-Friedenstruppenfpl in Bosnia-Herzegovina (von engl. Implementation Force; Abk.).}\end{entry}
\begin{entry}
\mainentry{aifuku}
{合服; 合い服 (間服)}
{ Übergangskleidung (f); Übergangsanzug (m) (⇒ ai·gi 合い着9540765).}\end{entry}
\begin{entry}
\mainentry{aibusuru}
{愛撫する}
{liebkosen; streicheln.}\end{entry}
\begin{entry}
\mainentry{aibusuruyouna}
{愛撫するような}
{liebkosend; zärtlich.}\end{entry}
\begin{entry}
\mainentry{aibusuruyouni}
{愛撫するように}
{liebkosend; zärtlich.}\end{entry}
\begin{entry}
\mainentry{aifuda}
{合い札; 合札}
{ Garderobennummer (f); Garderobenmarke (m); Gepäckschein (m).}\end{entry}
\begin{entry}
\mainentry{aifudawotsukeru}
{合札をつける}
{etw. mit einem Anhänger versehen.}\end{entry}
\begin{entry}
\mainentry{aifudawotoru}
{合札を取る}
{die Garderobenmarke nehmen; den Gepäckschein nehmen.}\end{entry}
\begin{entry}
\mainentry{aiburau}
{アイブラウ}
{  \textit{Anat.} Augenbraue (f) (von engl. eyebrow; ⇒ ai·burō アイブロー6620275).}\end{entry}
\begin{entry}
\mainentry{aiburaupenshiru}
{アイブラウ・ペンシル; アイブラウペンシル}
{ \textit{Kosmetik} Augenbrauenstift (m) (von engl. eyebrow pencil; ⇒ aiburō·penshiru アイブロー・ペンシル6870275).}\end{entry}
\begin{entry}
\mainentry{aifuru}
{アイフル}
{  \textit{Firmenn.} Aiful (große Verbraucherkreditgesellschaft).}\end{entry}
\begin{entry}
\mainentry{aiburou}
{アイ・ブロウ; アイブロウ}
{  \textit{Anat.} Augenbraue (f) (von engl. eyebrow; ⇒ ai·burō アイブロー6620275).}\end{entry}
\begin{entry}
\mainentry{aiburoupenshiru}
{アイブロウ・ペンシル; アイブロウペンシル}
{ \textit{Kosmetik} Augenbrauenstift (m) (von engl. eyebrow pencil; ⇒ aiburō penshiru アイブロー・ペンシル6870275).}\end{entry}
\begin{entry}
\mainentry{aiburo-}
{アイ・ブロー; アイブロー}
{  \textit{Anat.} Augenbraue (f) (von engl. eyebrow; ⇒ mayu まゆ6124904).}\end{entry}
\begin{entry}
\mainentry{aiburo-shixeipu}
{アイブロー・シェイプ; アイブローシェイプ}
{Form (f) der Augenbrauen (von engl. eyebrow shape).}\end{entry}
\begin{entry}
\mainentry{aiburo-shixe-pu}
{アイブロー・シェープ; アイブローシェープ}
{Augenbrauenform (f); Form (f) der Augenbrauen (von engl. eyebrow shape).}\end{entry}
\begin{entry}
\mainentry{aiburo-burashi}
{アイブロー・ブラシ; アイブローブラシ}
{ \textit{Kosmetik} Augenbrauenbürstchen (n) (von engl. eyebrow brush).}\end{entry}
\begin{entry}
\mainentry{aiburo-penshiru}
{アイブロー・ペンシル; アイブローペンシル}
{ \textit{Kosmetik} Augenbrauenstift (m) (von engl. eyebrow pencil).}\end{entry}
\begin{entry}
\mainentry{aibetsu }
{哀別}
{ trauriger Abschied (m); Abschiedstrauer (f).}\end{entry}
\begin{entry}
\mainentry{aibetsu }
{愛別 [2]}
{ Trennung von einem geliebten Menschen.}\end{entry}
\begin{entry}
\mainentry{aibetsu }
{愛別 [1]}
{  \textit{Ortsn.} Aibetsu (Ortschaft im Zentrum Hokkaidōs).}\end{entry}
\begin{entry}
\mainentry{aibekkusu}
{アイベックス}
{  \textit{Zool.} Steinbock (m) (Ziegenarten die auf Englisch als ibex bezeichnet werden; nämlich Iberiensteinbock, Alpensteinbock, Syrischer Steinbock, Sibirischer Steinbock und Äthiopische Steinbock; von engl. ibex).}\end{entry}
\begin{entry}
\mainentry{aibekkusuyagi}
{アイベックス山羊}
{ \textit{Zool.} Steinbock (m) (⇒ aibekkusu アイベックス6796024).}\end{entry}
\begin{entry}
\mainentry{aibetsusuru}
{哀別する}
{traurig Abschied nehmen.}\end{entry}
\begin{entry}
\mainentry{aibetsuriku}
{愛別離苦 (哀別離苦)}
{  \textit{Buddh.} Schmerz (m), geliebte Menschen zu verlassen.}\end{entry}
\begin{entry}
\mainentry{aibetsurikuhayononarai。}
{愛別離苦は世の習い。}
{ \textit{Bsp.} In dieser Welt muss man von geliebten Menschen Abschied nehmen.}\end{entry}
\begin{entry}
\mainentry{aibeya}
{相部屋}
{ Teilen (n) des Zimmers; gemeinsames Übernachten (n) in einem Zimmer (⇒ ai·yado 相宿2366518).}\end{entry}
\begin{entry}
\mainentry{aihendorufu}
{アイヘンドルフ}
{  \textit{Persönlichk.} Joseph Freiherr von Eichendorff (dtsch. romantischer Dichter; 1788–1857).}\end{entry}
\begin{entry}
\mainentry{aibo }
{AIBO; アイボ}
{  \textit{Wz.} AiboNAr (programmierbarer Roboterhund von Sony).}\end{entry}
\begin{entry}
\mainentry{aibo }
{哀慕}
{ traurige Sehnsucht (f).}\end{entry}
\begin{entry}
\mainentry{aibo }
{愛慕}
{ Verehrung (f); Zuneigung (f) (⇒ renbo 恋慕0950481).}\end{entry}
\begin{entry}
\mainentry{aibou}
{相棒 (合い棒)}
{ (ugs.) Kumpel (m); Kumpan (m); Partner (m); Komplize (m); Geselle (m); Genosse (m); Helfershelfer (m).}\end{entry}
\begin{entry}
\mainentry{aibouninaru}
{相棒になる}
{sich verbünden; Partner werden.}\end{entry}
\begin{entry}
\mainentry{aiboshi}
{相星}
{ Punktegleichstand (m) mit einem Gegner.}\end{entry}
\begin{entry}
\mainentry{aiboshininaru}
{相星になる}
{die gleich Anzahl an Gewinnen (bzw. Punkten) wie ein Gegner haben.}\end{entry}
\begin{entry}
\mainentry{aibosuru}
{愛慕する}
{verehren; lieben; sich sehnen; zugetan sein.}\end{entry}
\begin{entry}
\mainentry{aipoddo}
{IPOD [b]; アイポッド}
{ Internationales Tiefseebohrprojekt (n); IPOD (n) (Abk. für engl. International Project of Ocean Drilling).}\end{entry}
\begin{entry}
\mainentry{aibori-}
{アイボリー}
{ [1] Elfenbein (n). [2] Elfenbeinweiß (n). [3] elfenbeinfarbenes dickes Papier (n). (von engl. ivory).}\end{entry}
\begin{entry}
\mainentry{aibori-iro}
{アイボリー色}
{Elfenbeinfarbe (f).}\end{entry}
\begin{entry}
\mainentry{aibori-ko-suto}
{アイボリー・コースト; アイボリーコースト}
{  \textit{Ländern.} Elfenbeinküste (f); Republik (f) Elfenbeinküste; Côte (f) d’Ivoire (Staat in Westafrika; ⇒ Kōto·divoāru コートディヴォアール7626713).}\end{entry}
\begin{entry}
\mainentry{aibori-natto}
{アイボリー・ナット; アイボリーナット}
{ \textit{Bot.} Steinnuss (f); Elfenbeinnuss (f) (Frucht der Elfenbeinpalme; von engl. ivory nut).}\end{entry}
\begin{entry}
\mainentry{aibori-burakku}
{アイボリー・ブラック; アイボリーブラック}
{Elfenbeinschwarz (n) (Pigment von verbranntem Elfenbein; von engl. ivory black).}\end{entry}
\begin{entry}
\mainentry{aibori-pe-pa-}
{アイボリー・ペーパー; アイボリーペーパー}
{Elfenbeinpapier (n) (von engl. ivory paper).}\end{entry}
\begin{entry}
\mainentry{aibori-howaito}
{アイボリー・ホワイト; アイボリーホワイト}
{Elfenbeinweiß (n) (von engl. ivory white).}\end{entry}
\begin{entry}
\mainentry{aibori-howaitono}
{アイボリーホワイトの}
{elfenbeinfarben; elfenbeinweiß.}\end{entry}
\begin{entry}
\mainentry{aiboruto}
{アイ・ボルト; アイボルト}
{ Ösenbolzen (m); Augenschraube (f) (von engl. eyebolt).}\end{entry}
\begin{entry}
\mainentry{aibore}
{相惚れ; 相惚; 相ぼれ}
{ gegenseitige Liebe (f).}\end{entry}
\begin{entry}
\mainentry{aiboresuru}
{相惚れする; 相惚する; 相ぼれする}
{sich gegenseitig lieben.}\end{entry}
\begin{entry}
\mainentry{aima }
{合間; 合い間; あいま}
{ Pause (f); Zwischenzeit (f) (zwischen zwei Ereignissen).}\end{entry}
\begin{entry}
\mainentry{aima }
{合馬; 間馬}
{  \textit{Shōgi} Figur (f), die dem Gegner Schach bietet // Aktion (f), dem Gegner Schach zu bieten.}\end{entry}
\begin{entry}
\mainentry{aima-}
{アイマー}
{  \textit{Persönlichk.} Theodor Eimer (deutscher Zoologe; 1843–1898).}\end{entry}
\begin{entry}
\mainentry{aimaaimani}
{合間合間に}
{gelegentlich; mit Abständen; in Intervallen.}\end{entry}
\begin{entry}
\mainentry{aimai }
{あいまい (曖昧; 曖眛; あい昧)}
{ [1] Unbestimmtheit (f); Undeutlichkeit (f); Unklarheit (f); Zweideutigkeit (f); Doppelsinn (m). [2] Verruchtheit (f); Dubiosität (f); Anstößigkeit (f); Unanständigkeit (f).}\end{entry}
\begin{entry}
\mainentry{aimai }
{相舞い; 相舞; 合い舞い; 合舞い; 合舞}
{ gemeinsamer Tanz (m) von zwei, drei oder mehr Personen.}\end{entry}
\begin{entry}
\mainentry{aimaiayamari}
{あいまい誤り}
{unklarer Fehler (m) (recht seltene Wendung).}\end{entry}
\begin{entry}
\mainentry{aimaikougaku}
{あいまい工学}
{Fuzzy-Technik (f).}\end{entry}
\begin{entry}
\mainentry{aimaisa}
{あいまいさ (曖昧さ; 瞹昧さ)}
{Unbestimmtheit (f); Undeutlichkeit (f); Unklarheit (f); Zweideutigkeit (f); Doppelsinn (m).}\end{entry}
\begin{entry}
\mainentry{aimaijo}
{曖昧女}
{Frau (f) mit zweifelhaftem Ruf; Gelegenheitsprostituierte (f) (Ausdruck aus der Meiji-Ära).}\end{entry}
\begin{entry}
\mainentry{aimaizukei}
{あいまい図形}
{unklares Diagramm (n).}\end{entry}
\begin{entry}
\mainentry{aimaisei}
{あいまい性}
{Unbestimmtheit (f); Undeutlichkeit (f); Unklarheit (f); Zweideutigkeit (f); Doppelsinn (m).}\end{entry}
\begin{entry}
\mainentry{aimaida}
{あいまいだ; 曖昧だ; 曖眛だ}
{[1] wage sein; undeutlich sein; zweideutig sein. [2] anrüchig sein; schlüpfrig sein; anstößig sein; unanständig sein.}\end{entry}
\begin{entry}
\mainentry{aimaichaya}
{曖昧茶屋 [a]}
{Haus (n) mit zweifelhaftem Ruf; Spelunke (f); Bordell (n) // Person (f), die in einer Spelunke arbeitet. (⇒ aimai·yado 曖昧宿5548514; → aimai·jaya 曖昧茶屋8857519).}\end{entry}
\begin{entry}
\mainentry{aimaidyaya}
{曖昧茶屋 [b]}
{Haus (n) mit zweifelhaftem Ruf; Spelunke (f); Bordell (n) // Person (f), die in einer Spelunke arbeitet. (⇒ aimai·yado 曖昧宿5548514).}\end{entry}
\begin{entry}
\mainentry{aimaidure}
{曖昧連}
{Halbwelt (f) (Eintrag ist nicht zu bestätigen).}\end{entry}
\begin{entry}
\mainentry{aimaidenai}
{あいまいでない}
{bestimmt; deutlich; sicher; klar; zuverlässig.}\end{entry}
\begin{entry}
\mainentry{aimaina}
{あいまいな (曖昧な; あい昧な)}
{vage; mehrdeutig; undeutlich; unsicher; unklar; dunkel; doppelsinnig; unzuverlässig.}\end{entry}
\begin{entry}
\mainentry{aimainaiikatawosuru}
{あいまいな言い方をする}
{in unklarer Weise sprechen; zweideutig reden.}\end{entry}
\begin{entry}
\mainentry{aimainaiikatawosuruhito}
{あいまいな言い方をする人}
{Verschleierer (m).}\end{entry}
\begin{entry}
\mainentry{aimainaimi }
{あいまいな意味}
{unklare Bedeutung (f); zweideutiger Sinn (m).}\end{entry}
\begin{entry}
\mainentry{aimainaimi }
{曖昧な意味}
{doppelsinnige Bedeutung (f); mehrdeutige Bedeutung (f).}\end{entry}
\begin{entry}
\mainentry{aimainakiji}
{曖昧な記事}
{Nachricht (f) von zweifelhafter Herkunft.}\end{entry}
\begin{entry}
\mainentry{aimainakotoba}
{あいまいな言葉 (曖昧な言葉)}
{Doppeldeutigkeitenfpl; zweideutige Formulierungenfpl.}\end{entry}
\begin{entry}
\mainentry{aimainakotobade}
{あいまいな言葉で}
{ganz allgemein; ungefähr.}\end{entry}
\begin{entry}
\mainentry{aimainakotobawotsukau}
{あいまいな言葉を使う}
{zweideutige Aussagen machen; ausweichen.}\end{entry}
\begin{entry}
\mainentry{aimainakotobawotsukauhito}
{あいまいな言葉を使う人}
{jmd.NArN, der ausweichend redet.}\end{entry}
\begin{entry}
\mainentry{aimainakotowoiu}
{あいまいなことを言う; あいまいなことをいう (曖昧なことを言う)}
{eine zweideutige Bemerkung machen; zweideutige Aussagen machen; unbestimmt reden.}\end{entry}
\begin{entry}
\mainentry{aimainagohou}
{あいまいな語法}
{Doppelsinn (m); Zweideutigkeit (f); Mehrdeutigkeit (f); Amphibolie (f).}\end{entry}
\begin{entry}
\mainentry{aimainasanshou}
{あいまいな参照}
{nicht eindeutiger Verweis (m).}\end{entry}
\begin{entry}
\mainentry{aimainataido}
{あいまいな態度}
{unklare Einstellung (f).}\end{entry}
\begin{entry}
\mainentry{aimainataidowoshimesu}
{あいまいな態度を示す}
{eine unklare Haltung einnehmen.}\end{entry}
\begin{entry}
\mainentry{aimainataidowotoru }
{あいまいな態度をとる}
{keine klare Stellung nehmen; sich nicht festlegen.}\end{entry}
\begin{entry}
\mainentry{aimainataidowotoru }
{曖昧な態度をとる}
{eine vage Einstellung einnehmen.}\end{entry}
\begin{entry}
\mainentry{aimainachinjutsuwosuru}
{曖昧な陳述をする}
{eine vage Erklärung abgeben.}\end{entry}
\begin{entry}
\mainentry{aimainatouben'nishuushisuru}
{あいまいな答弁に終始する}
{laufend unklare Antworten geben; das unklare Antworten fortsetzen.}\end{entry}
\begin{entry}
\mainentry{aimainahanashi}
{あいまいな話}
{Doppeldeutigkeitenfpl.}\end{entry}
\begin{entry}
\mainentry{aimainahyougen}
{あいまいな表現}
{unklare Formulierung (f); doppelsinniger Ausdruck (m).}\end{entry}
\begin{entry}
\mainentry{aimainafuusetsu}
{曖昧な風説}
{zweideutiges Gerücht (n).}\end{entry}
\begin{entry}
\mainentry{aimainabubun}
{あいまいな部分}
{Niemandsland (n); unsicheres Gebiet (n).}\end{entry}
\begin{entry}
\mainentry{aimainabunsho}
{あいまいな文書}
{unklare Aufzeichnung (f).}\end{entry}
\begin{entry}
\mainentry{aimainahenji}
{あいまいな返事 (曖昧な返事)}
{unklare Antwort (f); unbestimmte Antwort (f); ausweichende Antwort (f).}\end{entry}
\begin{entry}
\mainentry{aimainahenjiwosuru}
{あいまいな返事をする (曖昧な返事をする)}
{eine unklare Antwort geben.}\end{entry}
\begin{entry}
\mainentry{aimainahentou}
{あいまいな返答}
{unklare Antwort (f).}\end{entry}
\begin{entry}
\mainentry{aimainahentouwosuru}
{あいまいな返答をする}
{eine unklare Antwort geben.}\end{entry}
\begin{entry}
\mainentry{aimainaboin}
{あいまいな母音}
{ \textit{Phon.} unklarer Vokal (m).}\end{entry}
\begin{entry}
\mainentry{aimaini}
{あいまいに (曖昧に)}
{unbestimmt; undeutlich; unsicher; unklar; dunkel; doppelsinnig; doppeldeutig; unzuverlässig.}\end{entry}
\begin{entry}
\mainentry{aimainishiteoku}
{あいまいにしておく (曖昧にしておく)}
{sich nicht festlegen.}\end{entry}
\begin{entry}
\mainentry{aimainisuru }
{あいまいにする (曖昧にする)}
{verschleiern; verdunkeln; verhüllen // unbestimmt sein; undeutlich sein; verschwommen sein.}\end{entry}
\begin{entry}
\mainentry{aimainisurukoto}
{あいまいにすること}
{Verschleierung (f); Verdunkelung (f).}\end{entry}
\begin{entry}
\mainentry{aimaibunpou}
{あいまい文法}
{unklare Grammatik (f).}\end{entry}
\begin{entry}
\mainentry{aimaiboin}
{あいまい母音}
{ \textit{Phon.} unklarer Vokal (m).}\end{entry}
\begin{entry}
\mainentry{aimaimoko}
{あいまいもこ (曖昧模糊)}
{unbestimmt; undeutlich; vage; mehrdeutig.}\end{entry}
\begin{entry}
\mainentry{aimaimokotaru}
{曖昧模糊たる}
{vage; zweideutig.}\end{entry}
\begin{entry}
\mainentry{aimaimokotoshita}
{曖昧模糊とした}
{(schriftspr.) zweideutig und verschwommen; undeutlich; vage; mehrdeutig.}\end{entry}
\begin{entry}
\mainentry{aimaiya}
{あいまい屋 (曖昧屋)}
{Haus (n) mit zweifelhaftem Ruf; Spelunke (f); Bordell (n).}\end{entry}
\begin{entry}
\mainentry{aimaiyado}
{曖昧宿}
{Haus (n) mit zweifelhaftem Ruf; Spelunke (f); Bordell (n).}\end{entry}
\begin{entry}
\mainentry{aimairiron}
{あいまい理論}
{ \textit{Philos.} Fuzzy-Logik (f); Fuzzy-Theorie (f) (bei künstlicher Intelligenz angewandte Nachahmung menschlichen Denkens).}\end{entry}
\begin{entry}
\mainentry{aimaironri}
{曖昧論理}
{ Fuzzy-Logic (f).}\end{entry}
\begin{entry}
\mainentry{aimaironriaishi-}
{あいまい論理IC}
{ \textit{EDV} integrierter Schaltkreis (m), der Fuzzy-Logik benutzt.}\end{entry}
\begin{entry}
\mainentry{aimago}
{相孫}
{ Verwandtschaft, gemeinsame Großeltern zu haben; gemeinsame Verwandtschaft (f) zweiten Grades in aufsteigender Linie.}\end{entry}
\begin{entry}
\mainentry{aimashigoto}
{合間仕事}
{Nebentätigkeit (f).}\end{entry}
\begin{entry}
\mainentry{aimajiwaru}
{相交わる}
{ [1] miteinander Umgang pflegen. [2] sich kreuzen; sich schneiden.}\end{entry}
\begin{entry}
\mainentry{aimasu}
{相摩す}
{ aneinander reiben.}\end{entry}
\begin{entry}
\mainentry{aimatsu}
{相まつ; あいまつ (相俟つ)}
{ (schriftspr.) in Wechselbeziehung stehen; Hand in Hand gehen; korrelieren.}\end{entry}
\begin{entry}
\mainentry{aimakku}
{iMac}
{  \textit{EDV} iMac (m) (Abk. für Internet-Macintosh; preiswerter Computer von Apple).}\end{entry}
\begin{entry}
\mainentry{aimatte}
{相まって; あいまって (相俟って)}
{ (schriftspr.) in Verbindung mit; in Zusammenarbeit mit.}\end{entry}
\begin{entry}
\mainentry{aimani}
{合間に}
{in der Zwischenzeit; in der freien Zeit.}\end{entry}
\begin{entry}
\mainentry{aimarazoku}
{アイマラ族}
{  \textit{Völkerk.} Aimara (span.) Aymará (Indianervolk im Gebiet des Titicacasees, Peru und Bolivien).}\end{entry}
\begin{entry}
\mainentry{aimawonuu}
{合間を縫う}
{etw. in einer Pause tun.}\end{entry}
\begin{entry}
\mainentry{aimawomiteyatteokou。}
{合間を見てやっておこう。}
{ \textit{Bsp.} Das werde ich zwischendurch erledigen.}\end{entry}
\begin{entry}
\mainentry{aimi}
{会見 [1]}
{  \textit{Stadtn.} Aimi (Stadt im Westen der Präf. Tottori).}\end{entry}
\begin{entry}
\mainentry{aimijin}
{藍微塵}
{Ai·mijin (n); gestreifter Stoff (m) aus Fäden in zwei Indigo-Schattierungen.}\end{entry}
\begin{entry}
\mainentry{aimitagai}
{相身互い; 相身互; 相見互い; 相見互; 相見たがい}
{ gegenseitiger Beistand (m); wechselseitige Hilfe (f).}\end{entry}
\begin{entry}
\mainentry{aimitsu}
{愛蜜}
{ (Slang) Scheidensekret (n); Vaginalsekret (n).}\end{entry}
\begin{entry}
\mainentry{aimitsumori}
{相見積り}
{ Kostenvoranschlag (m) für eine Ausschreibung.}\end{entry}
\begin{entry}
\mainentry{aimiru}
{相見る}
{ sich gegenseitig ansehen.}\end{entry}
\begin{entry}
\mainentry{aimirucha}
{あいみるちゃ (藍海松茶; 藍水松茶)}
{ bläuliches Braun (n).}\end{entry}
\begin{entry}
\mainentry{aimin}
{哀みん (哀愍 [b]; 哀憫 [b])}
{ Mitleid (n); Mitgefühl (n).}\end{entry}
\begin{entry}
\mainentry{aimukattesuwaru}
{相向かって座る (相向かって坐る)}
{sich gegenüber sitzen.}\end{entry}
\begin{entry}
\mainentry{aimuko}
{相婿 (相聟)}
{ Ehemann (m) der Schwester; Schwager (m).}\end{entry}
\begin{entry}
\mainentry{aime}
{合い目; 合目 [1]}
{ Schnittpunkt (m); Kreuzungspunkt (m); Treffpunkt (m).}\end{entry}
\begin{entry}
\mainentry{aime-ku}
{アイメーク}
{  \textit{Kosmetik} Augen-Makeup (n).}\end{entry}
\begin{entry}
\mainentry{aime-to}
{アイ・メート; アイメート}
{ Blindenhund (m) (von japan.-engl. eye mate).}\end{entry}
\begin{entry}
\mainentry{aimeru}
{アイメル}
{  \textit{Persönlichk.} Theodor Eimer (deutscher Zoologe; 1843–1898; ☞ Aimā アイマー7587023).}\end{entry}
\begin{entry}
\mainentry{aimo}
{アイモ}
{  \textit{Wz.} Eyemo (f) (Name einer tragbaren 35 mm-Filmkamera).}\end{entry}
\begin{entry}
\mainentry{aimo-do}
{iモード; i-mode}
{  \textit{Wz.} i-modeNAr; i-ModeNAr (Mobiltelefon-Service von NTT Do Co Mo).}\end{entry}
\begin{entry}
\mainentry{aimokamera}
{アイモ・カメラ; アイモカメラ}
{ \textit{Wz.} Eyemo-Kamera (f).}\end{entry}
\begin{entry}
\mainentry{aimokawarazu}
{相も変わらず; 相も変らず}
{ nach wie vor; wie immer; üblich.}\end{entry}
\begin{entry}
\mainentry{aimokawaranu}
{相も変わらぬ; 相も変らぬ; あいも変わらぬ; あいもかわらぬ}
{ nach wie vor; wie immer; üblich.}\end{entry}
\begin{entry}
\mainentry{aimochi}
{相持ち; 相持; 相もち}
{ [1] gemeinsamer Besitz (m); geteilter Besitz (m). [2] Aufteilung (f) einer Rechnung oder von Kosten zu gleichen Teilen auf die Beteiligten. [3] Unentschiedenheit (f). [4] gegenseitige Unterstützung (f).}\end{entry}
\begin{entry}
\mainentry{aimodori}
{逢い戻り; 逢い戻; 逢戻り; 逢戻}
{ Wiedervereinigung (f) eines getrennten Paares.}\end{entry}
\begin{entry}
\mainentry{aimono}
{相物; 間物; 合物 [1]; 四十物}
{ halbgetrockneter Fisch (m).}\end{entry}
\begin{entry}
\mainentry{aimon}
{合い文; 合文; 合い紋; 合紋}
{ [1] gleiche Familienwappennpl. [2] Übereinstimmung (f). [3] Losung (f); Parole (f); nur Insidern bekanntes Zeichen (n).}\end{entry}
\begin{entry}
\mainentry{aiya }
{あいや}
{ [1] Nein. [2] heh!; hallo! (Ausruf, um jmdn. anzuhalten).}\end{entry}
\begin{entry}
\mainentry{aiya }
{藍屋}
{Färber (m); Indigofärber (m).}\end{entry}
\begin{entry}
\mainentry{aiyaku }
{合い役; 合役}
{ derselbe Posten (m) // Person (f) mit demselben Posten.}\end{entry}
\begin{entry}
\mainentry{aiyaku }
{相役}
{ Kollege (m); Mitarbeiter (m).}\end{entry}
\begin{entry}
\mainentry{aiyake}
{相親家 (相舅)}
{ Elternmpl und Schwiegerelternmpl.}\end{entry}
\begin{entry}
\mainentry{aiyado}
{相宿}
{ Teilen (n) des Zimmers; gemeinsames Übernachten (n) im selben Zimmer oder Hotel (⇒ ai·beya 相部屋9554143).}\end{entry}
\begin{entry}
\mainentry{aiyadosuru}
{相宿する}
{mit jmdm. das Zimmer teilen; im selben Hotel wohnen.}\end{entry}
\begin{entry}
\mainentry{aiyadoninaru}
{相宿になる}
{im selben Zimmer übernachten; im selben Hotel übernachten.}\end{entry}
\begin{entry}
\mainentry{aiyadonokyaku}
{相宿の客}
{Zimmerkollege (m).}\end{entry}
\begin{entry}
\mainentry{aiyari}
{相槍}
{ gemeinsamer Kampf (m) mehrerer Personen mit dem Speer gegen eine Person.}\end{entry}
\begin{entry}
\mainentry{aiyuu}
{隘勇}
{ taiwanesischer Soldat (m).}\end{entry}
\begin{entry}
\mainentry{aiyu-shi-enu}
{IUCN; アイ・ユー・シー・エヌ; アイユーシーエヌ}
{  \textit{Org.} Internationale Union (f) für Naturschutz; IUCN (f) (engl. International Union for Conservation of Nature and Natural Resources).}\end{entry}
\begin{entry}
\mainentry{aiyu-shi-daburyu-}
{IUCW; アイ・ユー・シー・ダブリュー; アイユーシーダブリュー}
{  \textit{Org.} Internationale Union (f) für die Wohlfahrt von Kindern; (engl.) International Union (f) for Child Welfare.}\end{entry}
\begin{entry}
\mainentry{aiyu-dyi-}
{IUD; アイ・ユー・ディー; アイユーディー}
{  \textit{Med.} Intrauterinpessar (n); IUP (n); IUD (n); Spirale (f).}\end{entry}
\begin{entry}
\mainentry{aiyu-pi-e-shi-}
{IUPAC; アイ・ユー・ピー・エー・シー; アイユーピーエーシー}
{  \textit{Org.} IUPAC (f); (engl.) International Union (f) of Pure and Applied Chemistry (internationaler Verband chemischer Gesellschaften).}\end{entry}
\begin{entry}
\mainentry{aiyu-buchou}
{アイユーブ朝}
{  \textit{Gesch.} Aijubidenmpl (von Sultan Salladin gegründetes ägyptisch-syrisches Herrschergeschlecht).}\end{entry}
\begin{entry}
\mainentry{aiyo}
{あいよ}
{ (ugs.) O.K.!; Klar!}\end{entry}
\begin{entry}
\mainentry{aiyou}
{愛用}
{ bevorzugtes Benutzen (m).}\end{entry}
\begin{entry}
\mainentry{aiyousha}
{愛用者}
{Freund (m) von …; jmd.NArN, der etw. gern benutzt.}\end{entry}
\begin{entry}
\mainentry{aiyousuru }
{相擁する; 相ようする}
{ sich gegenseitig umarmen.}\end{entry}
\begin{entry}
\mainentry{aiyousuru }
{愛用する}
{gern benutzen; mit Vorliebe gebrauchen.}\end{entry}
\begin{entry}
\mainentry{aiyouno}
{愛用の}
{Lieblings….}\end{entry}
\begin{entry}
\mainentry{aiyounokamera}
{愛用のカメラ}
{jmds. bevorzugte Kamera (f).}\end{entry}
\begin{entry}
\mainentry{aiyounosake}
{愛用の酒}
{Lieblingsschnaps (m).}\end{entry}
\begin{entry}
\mainentry{aiyounosutekki}
{愛用のステッキ}
{Lieblingswanderstock (m).}\end{entry}
\begin{entry}
\mainentry{aiyoku}
{愛欲 (愛慾)}
{ sinnliche Liebe (f); Leidenschaft (f); Lust (f); Sexualität (f).}\end{entry}
\begin{entry}
\mainentry{aiyokunioboreru}
{愛欲におぼれる}
{sich der Leidenschaft hingeben; seinen Leidenschaften freien Lauf lassen.}\end{entry}
\begin{entry}
\mainentry{aiyokunimiwoyudaneru}
{愛欲に身を委ねる}
{sich der Leidenschaft hingeben; seinen Leidenschaften freien Lauf lassen.}\end{entry}
\begin{entry}
\mainentry{aiyokunotoriko}
{愛欲の虜}
{Sklave (m) der Leidenschaft.}\end{entry}
\begin{entry}
\mainentry{aiyokunodoreitonaru}
{愛欲の奴隷となる}
{zum Sklaven seiner Leidenschaften werden.}\end{entry}
\begin{entry}
\mainentry{aiyotsu}
{相四つ}
{  \textit{Sumō} dasselbe Preisgeld (n) für beide Gegner (⇒ kenka·yotsu 喧嘩四つ6164980).}\end{entry}
\begin{entry}
\mainentry{aiyomi}
{相読み; 相読}
{ [1] gemeinsames Lesen (n). [2] Zeuge (m); jmd., der etw. bestätigt.}\end{entry}
\begin{entry}
\mainentry{aiyomisuru}
{相読みする; 相読する}
{gemeinsam lesen.}\end{entry}
\begin{entry}
\mainentry{aiyome}
{相嫁}
{ Schwägerin (f); Frau (f) des Bruders.}\end{entry}
\begin{entry}
\mainentry{aiyoru}
{相寄る}
{ einander treffen; sich gegenseitig nähern.}\end{entry}
\begin{entry}
\mainentry{aiyorokobu}
{相喜ぶ}
{sich gegenseitig erfreuen; Freude teilen.}\end{entry}
\begin{entry}
\mainentry{aira }
{姶良}
{  \textit{Stadtn.} AiranNAr (Stadt im Zentrum der Präf. Kagoshima an der Kagoshima-Bucht).}\end{entry}
\begin{entry}
\mainentry{aira }
{吾平}
{  \textit{Ortsn.} Aira (Ortschaft im südosten der Präf. Kagoshima).}\end{entry}
\begin{entry}
\mainentry{airaina-}
{アイ・ライナー; アイライナー}
{ \textit{Kosmetik} Eyeliner (zum Ziehen des Lidstriches; aus d. Engl.von engl. eyeliner).}\end{entry}
\begin{entry}
\mainentry{airain}
{アイ・ライン; アイライン}
{  \textit{Kosmetik} Lidstrich (m) (von japan.-engl. eye line).}\end{entry}
\begin{entry}
\mainentry{airainwoireteiru}
{アイラインを入れている}
{ \textit{Kosmetik} den Lidstrich nachgezogen tragen.}\end{entry}
\begin{entry}
\mainentry{airainwoireru}
{アイラインを入れる}
{ \textit{Kosmetik} den Lidstrich nachziehen.}\end{entry}
\begin{entry}
\mainentry{airainwohiku}
{アイラインを引く}
{ \textit{Kosmetik} den Lidstrich nachziehen.}\end{entry}
\begin{entry}
\mainentry{airaku}
{哀楽}
{ (schriftspr.) Trauer (f) und Freude (f).}\end{entry}
\begin{entry}
\mainentry{airashii}
{愛らしい; あいらしい}
{ süß; nett; entzückend; lieblich.}\end{entry}
\begin{entry}
\mainentry{airashiihito}
{愛らしい人}
{goldige Person (f).}\end{entry}
\begin{entry}
\mainentry{airashiku}
{愛らしく}
{bezaubernd; charmant.}\end{entry}
\begin{entry}
\mainentry{airashisa}
{愛らしさ}
{Schönheit (f); Lieblichkeit (f).}\end{entry}
\begin{entry}
\mainentry{airasshuka-ra-}
{アイラッシュカーラー}
{  \textit{Kosmetik} Lockenstab (m) für Wimpern (von engl. eyelash curler).}\end{entry}
\begin{entry}
\mainentry{airando}
{アイランド}
{ Insel (f) (von engl. island).}\end{entry}
\begin{entry}
\mainentry{airandoha-fukoukoku}
{アイランド・ハーフ広告; アイランドハーフ広告}
{ \textit{Zeitungsw.} halbseitige Werbung ohne weitere Werbung auf derselben Seite (von engl. island-half advertisement).}\end{entry}
\begin{entry}
\mainentry{airisu}
{アイリス}
{ [1]  \textit{Bot.} Iris (f). [2]  \textit{Optik} Irisblende (f). [3]  \textit{griech. Mythol.} IrisfNAr (Göttin des Regenbogens und Götterbotin).}\end{entry}
\begin{entry}
\mainentry{airisuauto}
{アイリス・アウト; アイリスアウト}
{ \textit{Film} Abblende (f) (von engl. iris-out).}\end{entry}
\begin{entry}
\mainentry{airisuin}
{アイリス・イン; アイリスイン}
{ \textit{Film} Aufblende (f) (von engl. iris-in).}\end{entry}
\begin{entry}
\mainentry{airisushibori}
{アイリス絞り}
{ \textit{Optik} Irisblende (f).}\end{entry}
\begin{entry}
\mainentry{airisshu}
{アイリッシュ}
{ irisch // Ire (m) (von engl. Irish).}\end{entry}
\begin{entry}
\mainentry{airisshuurufuhaundo}
{アイリッシュ・ウルフハウンド; アイリッシュウルフハウンド}
{ \textit{Zool., Hunderasse} Irischer Wolfshund (m) (von engl. Irish wolfhound).}\end{entry}
\begin{entry}
\mainentry{airisshuko-hi-}
{アイリッシュ・コーヒー; アイリッシュコーヒー}
{Irishcoffee (m); Irisch Coffee (m) (Kaffee mit einem Schuss Whiskey u. Schlagsahne).}\end{entry}
\begin{entry}
\mainentry{airisshushichu-}
{アイリッシュ・シチュー; アイリッシュシチュー}
{ \textit{Kochk.} Irishstew (n); Irish Stew (Eintopf mit Hammelfleisch, Kartoffeln, Weißkraut u. a.).}\end{entry}
\begin{entry}
\mainentry{airisshusetta-}
{アイリッシュ・セッター; アイリッシュセッター}
{ \textit{Zool., Hunderasse} Irish Setter (m) (langhaariger, kastanienroter britischer Jagdhund).}\end{entry}
\begin{entry}
\mainentry{airisshuha-pu}
{アイリッシュ・ハープ; アイリッシュハープ}
{ \textit{Musikinstr.} irische Harfe (f) (von engl. Irish harp).}\end{entry}
\begin{entry}
\mainentry{airin}
{あいりん; 愛隣}
{  \textit{Ortsn.} AirinnNAr (Gebiet im Nordosten des Bezirkes Nishinari in Ōsaka).}\end{entry}
\begin{entry}
\mainentry{airinchiku}
{愛隣地区}
{ \textit{Ortsn.} AirinnNAr (Gebiet im Nordosten des Bezirkes Nishinari in Ōsaka).}\end{entry}
\begin{entry}
\mainentry{airushi-to}
{アイル・シート; アイルシート}
{ Sitz am Gang (von engl. aisle seat).}\end{entry}
\begin{entry}
\mainentry{airurando}
{アイルランド (愛蘭)}
{  \textit{Ländern., Inseln.} IrlandnNAr (westlichste der Britischen Inseln bzw. Staat auf der Insel Irland).}\end{entry}
\begin{entry}
\mainentry{airurandokyouwagun}
{アイルランド共和軍}
{ \textit{Org.} IRA (f); Irisch-Republikanische Armee (f) (⇒ ai·āru·ē IRA1576686).}\end{entry}
\begin{entry}
\mainentry{airurandokyouwakoku}
{アイルランド共和国}
{  \textit{Ländern.} Republik (f) Irland.}\end{entry}
\begin{entry}
\mainentry{airurandokyouwakokugun}
{アイルランド共和国軍}
{ \textit{Org.} IRA (f); Irisch-Republikanische Armee (f).}\end{entry}
\begin{entry}
\mainentry{airurandogo}
{アイルランド語}
{ \textit{Sprache} Irisch (n).}\end{entry}
\begin{entry}
\mainentry{airurandojiyuukoku}
{アイルランド自由国}
{ \textit{Gebietsn.} Irischer Freistaat (m); Irish Free StatenNAr (von britischer Regierung unabhängige Grafschaften in Irland).}\end{entry}
\begin{entry}
\mainentry{airurandojin}
{アイルランド人}
{Ire (m); Irin (f).}\end{entry}
\begin{entry}
\mainentry{airurandono}
{アイルランドの}
{irisch.}\end{entry}
\begin{entry}
\mainentry{airurandohitono}
{アイルランド人の}
{irisch.}\end{entry}
\begin{entry}
\mainentry{aire}
{アイレ}
{  \textit{Ländern.} ÉirenNAr (irischer Name von Irland).}\end{entry}
\begin{entry}
\mainentry{airesutohowaito}
{アイ・レスト・ホワイト; アイレストホワイト}
{ gebrochenes Weiß (n); nicht grelles Weiß (n) (von japan.-engl. eye rest white).}\end{entry}
\begin{entry}
\mainentry{airetto}
{アイレット}
{  \textit{Schneiderei} Öse (f) (von engl. eyelet).}\end{entry}
\begin{entry}
\mainentry{airettowa-ku}
{アイレット・ワーク; アイレットワーク}
{Einarbeiten (n) einer Öse (von engl. eyelet work).}\end{entry}
\begin{entry}
\mainentry{airen }
{哀れん (哀憐)}
{ Mitleid (n); Mitgefühl (n).}\end{entry}
\begin{entry}
\mainentry{airen }
{愛れん (愛憐)}
{ Sympathie (f).}\end{entry}
\begin{entry}
\mainentry{airensuru}
{愛憐する}
{Mitgefühl haben.}\end{entry}
\begin{entry}
\mainentry{airen'nojounitaezu}
{哀憐の情に堪えず}
{von Mitgefühl überwältigt werden.}\end{entry}
\begin{entry}
\mainentry{airen'nojounitaenai}
{哀憐の情に堪えない}
{von Mitgefühl überwältigt werden.}\end{entry}
\begin{entry}
\mainentry{airo }
{あいろ [1] (文色)}
{ [1] Muster (n); Markierungenfpl. [2] unterschiedlicher Punkt (m).}\end{entry}
\begin{entry}
\mainentry{airo }
{あい路; あいろ [2] (隘路)}
{ [1] Engpass (m); Hohlweg (m). [2] Hindernis (n); Verengung (f); Flaschenhals (m).}\end{entry}
\begin{entry}
\mainentry{airou}
{あいろう (藍蝋)}
{ Airō (n); in Stangenform verarbeitete Indigofarbe (f).}\end{entry}
\begin{entry}
\mainentry{airotonaru}
{あいろとなる (隘路となる)}
{einen Engpass verursachen.}\end{entry}
\begin{entry}
\mainentry{aironi-}
{アイロニー}
{ Ironie (f).}\end{entry}
\begin{entry}
\mainentry{aironikaru}
{アイロニカル}
{ ironisch.}\end{entry}
\begin{entry}
\mainentry{aironikaruda}
{アイロニカルだ}
{ironisch sein; sarkastisch sein.}\end{entry}
\begin{entry}
\mainentry{aironikku}
{アイロニック}
{ Ironie (f); Sarkasmus (m) // ironisch; sakastisch (von engl. ironic).}\end{entry}
\begin{entry}
\mainentry{aironikkuda}
{アイロニックだ}
{ironisch sein; sakastisch sein.}\end{entry}
\begin{entry}
\mainentry{aironinaru}
{隘路になる}
{zu einem Engpass werden.}\end{entry}
\begin{entry}
\mainentry{airowodakaisuru}
{隘路を打開する}
{einen Engpass überwinden.}\end{entry}
\begin{entry}
\mainentry{airowonasu}
{隘路を成す}
{zu einem Engpass werden.}\end{entry}
\begin{entry}
\mainentry{airon}
{アイロン}
{ Bügeleisen (n) (von engl. iron).}\end{entry}
\begin{entry}
\mainentry{airongakakatteiru}
{アイロンがかかっている}
{gebügelt sein.}\end{entry}
\begin{entry}
\mainentry{airondai}
{アイロン台}
{Bügelbrett (n).}\end{entry}
\begin{entry}
\mainentry{airondekogasu}
{アイロンで焦がす}
{mit dem Bügeleisen versengen.}\end{entry}
\begin{entry}
\mainentry{airon'noyonetsuworiyousuru}
{アイロンの余熱を利用する}
{die Restwärme des Bügeleisens ausnutzen.}\end{entry}
\begin{entry}
\mainentry{aironpa-ma}
{アイロン・パーマ; アイロンパーマ}
{Dauerwelle (f) mit dem Brennstab (von japan.-engl. iron perm).}\end{entry}
\begin{entry}
\mainentry{aironbo-do}
{アイロン・ボード; アイロンボード}
{Bügelbrett (n) (von engl. ironing board).}\end{entry}
\begin{entry}
\mainentry{aironwokakaeru}
{アイロンをかかえる}
{bügeln; plätten.}\end{entry}
\begin{entry}
\mainentry{aironwokaketenaiwaishatsu}
{アイロンをかけてないワイシャツ}
{ungebügeltes Hemd (n).}\end{entry}
\begin{entry}
\mainentry{aironwokakeru}
{アイロンをかける}
{bügeln.}\end{entry}
\begin{entry}
\mainentry{aironwotsukeppanashinishiteoku}
{アイロンをつけっ放しにしておく}
{das Bügeleisen eingeschaltet lassen.}\end{entry}
\begin{entry}
\mainentry{aironwotsukeru}
{アイロンをつける}
{das Bügeleisen einschalten.}\end{entry}
\begin{entry}
\mainentry{aironwotomeru}
{アイロンをとめる}
{das Bügeleisen ausschalten.}\end{entry}
\begin{entry}
\mainentry{aiwa}
{哀話}
{ traurige Geschichte (f); tragische Erzählung (f) (⇒ hiwa 悲話3361123).}\end{entry}
\begin{entry}
\mainentry{aiwasu}
{相和す}
{ zusammenhalten; zusammenstehen; glücklich zusammen leben; ein Herz und eine Seele sein; harmonieren (⇒ naka·yoku 仲良く6148108).}\end{entry}
\begin{entry}
\mainentry{aiwoukeireru}
{愛を受け入れる; 愛を受けいれる}
{jmds. Liebe annehmen; jmds. Liebe erwidern.}\end{entry}
\begin{entry}
\mainentry{aiwoukeru}
{愛を受ける}
{jmds. Liebe empfangen.}\end{entry}
\begin{entry}
\mainentry{aiwoushinau}
{愛を失う}
{jmds. Zuneigung verlieren.}\end{entry}
\begin{entry}
\mainentry{aiwoeru}
{愛を得る}
{jmds. Liebe gewinnen.}\end{entry}
\begin{entry}
\mainentry{aiwooku}
{間を置く}
{eine Pause lassen.}\end{entry}
\begin{entry}
\mainentry{aiwooboeru}
{愛を覚える}
{sich in jmdn. verlieben.}\end{entry}
\begin{entry}
\mainentry{aiwokachieru}
{愛を勝ち得る; 愛をかち得る}
{jmds. Liebe gewinnen.}\end{entry}
\begin{entry}
\mainentry{aiwokou}
{哀を乞う}
{um Gnade bitten.}\end{entry}
\begin{entry}
\mainentry{aiwokokuhakusuru}
{愛を告白する}
{jmdm. seine Liebe erklären; eine Liebeserklärung machen.}\end{entry}
\begin{entry}
\mainentry{aiwokomete}
{愛をこめて}
{in Liebe; mit Liebe; liebevoll.}\end{entry}
\begin{entry}
\mainentry{aiwosasayaku}
{愛をささやく}
{liebevoll flüstern.}\end{entry}
\begin{entry}
\mainentry{aiwochikaiau}
{愛を誓い合う}
{sich gegenseitig die Liebe schwören.}\end{entry}
\begin{entry}
\mainentry{ainshutainiumu}
{アインシュタイニウム; Es [a]}
{  \textit{Chem.} Einsteinium (n) (metallisches Transuran; Zeichen: Es).}\end{entry}
\begin{entry}
\mainentry{ainshutainyu-mu}
{アインシュタイニューム; Es [b]}
{  \textit{Chem.} Einsteinium (n) (metallisches Transuran; Zeichen: Es).}\end{entry}
\begin{entry}
\mainentry{ainshutain}
{アインシュタイン}
{  \textit{Persönlichk.} Albert Einstein (dtsch.-amerik. Physiker; 1879–1955).}\end{entry}
\begin{entry}
\mainentry{ainshutain'uchuu}
{アインシュタイン宇宙}
{ \textit{Phys.} Einstein-Universum (n).}\end{entry}
\begin{entry}
\mainentry{ainshutain'eisei}
{アインシュタイン衛星}
{ \textit{Astron.} Einstein-Satellit (m) (Röntgensatellit zur Überprüfung der Relativitätstheorie; gestartet April 2004).}\end{entry}
\begin{entry}
\mainentry{ainshutainkoutoushiki}
{アインシュタイン恒等式}
{ \textit{Phys.} Einstein’sche Gleichung (f); Einstein’sches Gesetz (n); Äquivalenz (f) von Masse und Energie.}\end{entry}
\begin{entry}
\mainentry{ainshutaintou}
{アインシュタイン塔}
{ \textit{Ortsn.} Einsteinturm (m) (Turmteleskop zur Sonnenbeobachtung in Potsdam).}\end{entry}
\begin{entry}
\mainentry{ainshutain'no}
{アインシュタインの}
{Einsteins; Einstein’sch.}\end{entry}
\begin{entry}
\mainentry{ainsutainiumu}
{アインスタイニウム; Es [c]}
{  \textit{Chem.} Einsteinium (n) (radioaktives metallisches Transuran; benannt nach dem Physiker Albert Einstein; Zeichen: Es).}\end{entry}
\begin{entry}
\mainentry{aintoho-fen}
{アイントホーフェン}
{  \textit{Persönlichk.} Willem Einthoven (niederl. Physiologe; 1860–1927).}\end{entry}
\begin{entry}
\mainentry{ainfuxyu-runku}
{アインフュールンク}
{  \textit{bild. Kunst} Einfühlung (f); Empathie (f) (aus d. Dtsch.; mit ungeschickter Transkription).}\end{entry}
\begin{entry}
\mainentry{ainfuxyu-rungu}
{アインフュールング}
{  \textit{bild. Kunst} Einfühlung (f); Empathie (f) (aus d. Dtsch.von dtsch. Einfühlung).}\end{entry}
\begin{entry}
\mainentry{au }
{会う; あう [1] (逢う; 會う)}
{ [1] sich versammeln; sich treffen. [2] sich treffen, um etw. gemeinsam zu machen. [3] passen; harmonieren (diese Bed. wird insbes. 合う geschrieben); // übereinstimmen // einträglich sein; sich lohnen. [4] sich begegnen. [5] zufällig treffen. [6] geraten in …; stoßen auf …; erfahren; erleiden. (⇔ wakareru 別れる4788259).}\end{entry}
\begin{entry}
\mainentry{au }
{合う; あう [2]}
{ [1] passen; angemessen sein. [2] übereinstimmen; übereinkommen; im Einklang stehen; entsprechen; gut stehen. [3] zusammenpassen. [3] korrekt sein.}\end{entry}
\begin{entry}
\mainentry{au }
{遭う; あう [3] (遇う; 邂う; 遘う)}
{ [1] haben (z. B. einen Unfall); treffen auf …; stoßen auf ….}\end{entry}
\begin{entry}
\mainentry{ava-ru}
{アヴァール}
{  \textit{Gesch.} Aware (m); Avare (m) (asiat. Reitervolk).}\end{entry}
\begin{entry}
\mainentry{avangarudo}
{アヴァン・ガルド; アヴァンガルド}
{ Avantgarde (f) (von franz. avant garde; ☞ aban ’gyarudo アバン・ギャルド5112518).}\end{entry}
\begin{entry}
\mainentry{avangyarudo}
{アヴァン・ギャルド; アヴァンギャルド}
{ Avantgarde (f) (von franz. avant garde; ☞ aban ’gyarudo アバン・ギャルド5112518).}\end{entry}
\begin{entry}
\mainentry{avangyarudojazu}
{アヴァンギャルド・ジャズ; アヴァンギャルドジャズ}
{ \textit{Mus.} Avantgarde-Jazz (m).}\end{entry}
\begin{entry}
\mainentry{avange-ru}
{アヴァン・ゲール; アヴァンゲール}
{ [1] Vorkriegszeit (f). [2] Vorkriegsgeneration (f). (von franz. avant-guerre). (⇒ aban·gēru アバン・ゲール7322683).}\end{entry}
\begin{entry}
\mainentry{avantaitoru}
{アヴァン・タイトル; アヴァンタイトル}
{  \textit{Film, Theat.} Sequenz (f) vor dem Titel(von japan.-franz.-engl. avant-title).}\end{entry}
\begin{entry}
\mainentry{avanchu-ru}
{アヴァンチュール}
{ Liebesabenteuer (n); Affäre (f) (von franz. aventure; ⇒ abanchūru アバンチュール2047714).}\end{entry}
\begin{entry}
\mainentry{avisen'na}
{アヴィセンナ}
{  \textit{Persönlichk.} Avicenna (persischer Philosoph und Arzt; 980–1037).}\end{entry}
\begin{entry}
\mainentry{aviniyon}
{アヴィニヨン}
{  \textit{Stadtn.} AvignonnNAr (Hptst. des französischen Départements Vaucluse in der Provence, Frankreich; ⇒ Abinyon アビニョン7392194).}\end{entry}
\begin{entry}
\mainentry{avinyon}
{アヴィニョン}
{  \textit{Stadtn.} AvignonnNAr (Hptst. des französischen Départements Vaucluse in der Provence, Frankreich; ⇒ Abinyon アビニョン7392194).}\end{entry}
\begin{entry}
\mainentry{avesuta}
{アヴェスタ}
{  \textit{Werktitel} AwestaNAr; AvestaNAr (heilige Schriften der Parsen).}\end{entry}
\begin{entry}
\mainentry{avekku}
{アヴェック}
{ (ugs.) Pärchen (n); Liebespaar (n) (von franz. avec; franz. avec bedeutet eigentl. „mit, samt, nebst“; ⇒ abekku アベック1940974).}\end{entry}
\begin{entry}
\mainentry{avenyu}
{アヴェニュ}
{ Avenue (f); Boulevard (m) (aus d. Franz.; ⇒ abenyū アベニュー7277836).}\end{entry}
\begin{entry}
\mainentry{avenyu-}
{アヴェニュー}
{ Avenue (f); Boulevard (m) (aus d. Franz.; ⇒ abenyū アベニュー7277836).}\end{entry}
\begin{entry}
\mainentry{avemaria}
{アヴェ・マリア; アヴェマリア}
{  \textit{Christent.} Ave Maria (n).}\end{entry}
\begin{entry}
\mainentry{avemariya}
{アヴェ・マリヤ; アヴェマリヤ}
{  \textit{Christent.} Ave Maria (n).}\end{entry}
\begin{entry}
\mainentry{avere-ji}
{アヴェレージ}
{ Durchschnitt (m) (von engl. average).}\end{entry}
\begin{entry}
\mainentry{averoesu}
{アヴェロエス}
{  \textit{Persönlichk.} Abul Walid Mohammed Ibn Ahmad Ibn Mohammed Ibn Ruschd (latinisiert) Averroes (arab. Philosoph, Theologe und Mediziner; 1126–1198).}\end{entry}
\begin{entry}
\mainentry{avoka-do}
{アヴォカード}
{  \textit{Bot.} Avocado (f); Avocadobirne (f) (⇒ abokado アボカド3602452).}\end{entry}
\begin{entry}
\mainentry{avokado}
{アヴォカド}
{  \textit{Bot.} Avocado (f); Avocadobirne (f) (⇒ abokado アボカド3602452).}\end{entry}
\begin{entry}
\mainentry{avogadoro}
{アヴォガドロ}
{  \textit{Persönlichk.} Amedeo Avogadro (ital. Physiker u. Chemiker; 1776–1856).}\end{entry}
\begin{entry}
\mainentry{aua-}
{アウアー}
{  \textit{Persönlichk.} Leopold Auer (ungar. Violinist; 1845–1930).}\end{entry}
\begin{entry}
\mainentry{aua-goukin}
{アウアー合金}
{ \textit{Chem.} Auermetall (n) (von Carl Auer von Welsbach erfundene Zündsteinlegierung).}\end{entry}
\begin{entry}
\mainentry{aua-tou}
{アウアー灯 (アウアー燈)}
{ \textit{Elektrot.} Auerlampe (eine Gasglühlampe).}\end{entry}
\begin{entry}
\mainentry{auxe-ge-mu}
{アウェー・ゲーム; アウェーゲーム}
{  \textit{Sport} Auswärtsspiel (n) (von engl. away game).}\end{entry}
\begin{entry}
\mainentry{auerubahhashinkeisou}
{アウエルバッハ神経叢}
{  \textit{Anat.} Auerbach’scher Plexus (m); Auerbach Plexus (m); Plexus (m) myentericus (⇒ kinsōkan·shinkeisō 筋層間神経叢2355252).}\end{entry}
\begin{entry}
\mainentry{aukishin}
{アウキシン}
{  \textit{Biochem.} Auxin (n) (ein Pflanzenwuchsstoff).}\end{entry}
\begin{entry}
\mainentry{augusuchinusu}
{アウグスチヌス}
{  \textit{Persönlichk.} Aurelius Augustinus (lat. Kirchenlehrer; 354–430).}\end{entry}
\begin{entry}
\mainentry{augusutsusu}
{アウグスツス}
{  \textit{Persönlichk.} Augustus (Beiname des röm. Kaisers Oktavian; 63 v. Chr.–14 n. Chr.).}\end{entry}
\begin{entry}
\mainentry{augusutexinusu}
{アウグスティヌス}
{  \textit{Persönlichk.} Aurelius Augustinus (lat. Kirchenlehrer; 354–430).}\end{entry}
\begin{entry}
\mainentry{augusuto}
{アウグスト}
{  \textit{männl. Vorn.} August (m).}\end{entry}
\begin{entry}
\mainentry{augusutoxusu}
{アウグストゥス}
{  \textit{Persönlichk.} Augustus (erste röm. Kaiser; 63 v. Chr.–14 n. Chr.).}\end{entry}
\begin{entry}
\mainentry{aukusuburuku}
{アウクスブルク}
{  \textit{Stadtn.} AugsburgnNAr (Stadt in Bayern).}\end{entry}
\begin{entry}
\mainentry{augusuburuku}
{アウグスブルク}
{  \textit{Stadtn.} AugsburgnNAr (Stadt in Bayern).}\end{entry}
\begin{entry}
\mainentry{auge}
{アウゲ}
{  \textit{Med.} (medizinischer Jargon) Augenheilkunde (f) (japan. Abk. für dtsch. Augenheilkunde).}\end{entry}
\begin{entry}
\mainentry{aushuvittsu}
{アウシュヴィッツ}
{  \textit{Stadtn.} AuschwitznNAr (dtsch. Name der Stadt Oswiecim in Polen; durch das dortige Konzentrationslager wurde der Name zum Synonym für Vernichtungspolitik der Nazis).}\end{entry}
\begin{entry}
\mainentry{aushuuxittsu}
{アウシュウィッツ}
{ [1]  \textit{Stadtn.} AuschwitznNAr (dtsch. Name der Stadt Oswiecim in Polen). [2]  \textit{Gesch.} Konzentrationslager (n) Auschwitz // (Synonym für) die Vernichtungspolitik der Nazis.}\end{entry}
\begin{entry}
\mainentry{aushubittsu}
{アウシュビッツ}
{ [1]  \textit{Stadtn.} AuschwitznNAr (dtsch. Name der Stadt „Oswiecim“ in Polen). [2] Konzentrationslager (n) Auschwitz // (Synonym für) die Vernichtungspolitik (f) der Nazis.}\end{entry}
\begin{entry}
\mainentry{ausu}
{アウス}
{  \textit{Med.} Kürettage (f); Ausschabung (f) der Gebärmutter (Abk. für dtsch. auskratzen).}\end{entry}
\begin{entry}
\mainentry{ausuterurittsu}
{アウステルリッツ}
{  \textit{Stadtn.} AusterlitznNAr (Schlachtort bei Brünn; tschechisch „Slavkov u Brna“).}\end{entry}
\begin{entry}
\mainentry{ausuterurittsunotatakai}
{アウステルリッツの戦い; アウステルリッツの戦}
{ \textit{Gesch.} Dreikaiserschlacht (f); Schlacht (f) von Austerlitz (Sieg Napoleons über den österr. Kaisers Franz II. und den russ. Kaisers Alexander am I2.12. 1805).}\end{entry}
\begin{entry}
\mainentry{ausutoraru}
{アウストラル}
{ Austral (m) (argentinische Währungseinheit).}\end{entry}
\begin{entry}
\mainentry{ausutoraropitekusu}
{アウストラロピテクス}
{  \textit{Zool.} Australopithecus (m) (Übergangsform zwischen Tier u. Mensch; Vormensch, Halbmensch).}\end{entry}
\begin{entry}
\mainentry{ausutoraropitekusurui}
{アウストラロピテクス類}
{ \textit{Zool.} Australopithecinaempl.}\end{entry}
\begin{entry}
\mainentry{ausutorianokyokugaichuuritsu}
{アウストリアの局外中立}
{die österreichische Neutralität (f).}\end{entry}
\begin{entry}
\mainentry{ausutoroajiago}
{アウストロアジア語}
{  \textit{Sprache} austroasiatische Sprachenfpl.}\end{entry}
\begin{entry}
\mainentry{ausutoroneshiago}
{アウストロネシア語}
{  \textit{Sprache} austronesische Sprachenfpl.}\end{entry}
\begin{entry}
\mainentry{ausutoroneshiagozoku}
{アウストロネシア語族}
{ \textit{Sprachw.} austronesische Sprachenfpl.}\end{entry}
\begin{entry}
\mainentry{ausure-ze}
{アウスレーゼ}
{  \textit{Wein} Auslese (f) (aus d. Dtsch.von dtsch. Auslese).}\end{entry}
\begin{entry}
\mainentry{ausenra-ge}
{アウセン・ラーゲ; アウセンラーゲ}
{  \textit{Ski} Haltung (f) mit zum Tal gebeugten Oberkörper (von dtsch. Außenlage).}\end{entry}
\begin{entry}
\mainentry{auta-}
{アウター}
{ Außen… (von engl. outer).}\end{entry}
\begin{entry}
\mainentry{auta-uea}
{アウター・ウエア; アウターウエア}
{Kleidung (f) fürs Freie (von engl. outerwear).}\end{entry}
\begin{entry}
\mainentry{auta-supe-su}
{アウター・スペース; アウタースペース}
{Weltall (n); Weltraum (m) (von engl. outer space).}\end{entry}
\begin{entry}
\mainentry{auta-wa-rudo}
{アウター・ワールド; アウターワールド}
{Außenwelt (f) (von engl. outer world).}\end{entry}
\begin{entry}
\mainentry{autaruki}
{アウタルキ}
{ Autarkie (f) (aus d. Dtsch.von dtsch. Autarkie; ursprüngl. aus d. Griech.).}\end{entry}
\begin{entry}
\mainentry{autaruki-}
{アウタルキー}
{ Autarkie (f) (aus d. Dtsch.von dtsch. Autarkie).}\end{entry}
\begin{entry}
\mainentry{autsu}
{アウツ}
{ [1]  \textit{Ballsport} Aus (n); Out (n) (östereichisch) // aus; out; außerhalb des Spielfeldes. [2]  \textit{Ballsport} ins Aus gegangener Ball (m). [3]  \textit{Golf} die ersten neun Löcher (eines Golfkurtses mit 18 Löchern). [4] Außen (n). [5] Ausfallen (n); Misserfolg (m). (}\end{entry}
\begin{entry}
\mainentry{audyi}
{アウディ}
{  \textit{Firmenn.} AudinNAr (deutscher Automobilhersteller).}\end{entry}
\begin{entry}
\mainentry{autexingu}
{アウティング}
{ Outing (n); Coming-out (n) (aus d. Engl.).}\end{entry}
\begin{entry}
\mainentry{autexinguuxea}
{アウティング・ウェア; アウティングウェア}
{  \textit{Kleidung} leichte Straßenkleidung (f) (von engl. outing wear).}\end{entry}
\begin{entry}
\mainentry{autexingukosuchu-mu}
{アウティング・コスチューム; アウティングコスチューム}
{  \textit{Kleidung} leichte Straßenkleidung (f) (von engl. outing costume).}\end{entry}
\begin{entry}
\mainentry{auteureshiya。}
{逢うて嬉しや。}
{ \textit{Bsp.} Wie schön, dass ich dich getroffen habe.}\end{entry}
\begin{entry}
\mainentry{auteria}
{アウテリア}
{ Exterieur (m) (von japan.-engl. outerior; ⇔ interia インテリア9191151).}\end{entry}
\begin{entry}
\mainentry{auteriadezain}
{アウテリア・デザイン; アウテリアデザイン}
{Exterieur-Design (n) (von japan.-engl. outerior design).}\end{entry}
\begin{entry}
\mainentry{auteriya}
{アウテリヤ}
{ Exterieur (m) (von japan.-engl. outerior).}\end{entry}
\begin{entry}
\mainentry{auteriyadezain}
{アウテリヤ・デザイン; アウテリヤデザイン}
{Exterieur-Design (n) (von japan.-engl. outerior design; ☞ auteria·dezain アウテリア・デザイン0677246).}\end{entry}
\begin{entry}
\mainentry{auto}
{アウト}
{ [1]  \textit{Ballsport} Aus (n); Out (n) () // aus; out; außerhalb des Spielfeldes. [2]  \textit{Ballsport} ins Aus gegangener Ball (m). [3]  \textit{Golf} die ersten neun Löcher (eines Golfplatzes mit 18 Löchern). [4] Außen (n). [5] Ausfallen (n); Misserfolg (m). (vo}\end{entry}
\begin{entry}
\mainentry{autouea}
{アウト・ウエア; アウトウエア}
{Kleidung (f) fürs Freie (von japan.-engl. out wear; ☞ auto  wea アウト・ウェア1213984).}\end{entry}
\begin{entry}
\mainentry{autouxea}
{アウト・ウェア; アウトウェア}
{Kleidung (f) fürs Freie (von japan.-engl. out wear).}\end{entry}
\begin{entry}
\mainentry{autouxea-}
{アウト・ウェアー; アウトウェアー}
{Kleidung (f) fürs Freie (von japan.-engl. out wear; ☞ auto  wea アウト・ウェア1213984).}\end{entry}
\begin{entry}
\mainentry{autoejji}
{アウト・エッジ; アウトエッジ}
{ \textit{Ski} Außenkante (f) (eines Skis; von japan.-engl. out edge).}\end{entry}
\begin{entry}
\mainentry{autoobude-to}
{アウト・オブ・デート; アウトオブデート}
{ veraltet; anachronistisch; obsolet; unmodern; out (von engl. out-of-date; ⇔ appu tsū dēto アップ・ツー・デート4423836).}\end{entry}
\begin{entry}
\mainentry{autoobude-toda}
{アウトオブデートだ}
{veraltet sein; anachronistisch sein; obsolet sein; unmodern sein; out sein.}\end{entry}
\begin{entry}
\mainentry{autoobubaunzu}
{アウト・オブ・バウンズ; アウトオブバウンズ}
{ außerhalb der Spielbahn (von engl. out of bounds).}\end{entry}
\begin{entry}
\mainentry{autoobufasshon}
{アウト・オブ・ファッション; アウトオブファッション}
{unmodern; veraltet (von engl. out of fashion).}\end{entry}
\begin{entry}
\mainentry{autoobupurei}
{アウト・オブ・プレイ; アウトオブプレイ}
{ \textit{Fußball etc.} Spielunterbrechung (f) (von engl. out of play; ☞ auto obu purē アウト・オブ・プレー6550846).}\end{entry}
\begin{entry}
\mainentry{autoobupure-}
{アウト・オブ・プレー; アウトオブプレー}
{ \textit{Fußball etc.} Spielunterbrechung (f) (von engl. out of play).}\end{entry}
\begin{entry}
\mainentry{autoobupojishon}
{アウト・オブ・ポジション; アウトオブポジション}
{ \textit{Tennis, Volleyball} Stellungsfehler (m) (von engl. out of position).}\end{entry}
\begin{entry}
\mainentry{autoka-bu}
{アウト・カーブ; アウトカーブ}
{ \textit{Baseb.} geworfener Ball (m), der eine Kurve weg vom Schlagmann beschreibt (von engl. outcurve).}\end{entry}
\begin{entry}
\mainentry{autokaunto}
{アウトカウント}
{ \textit{Baseb.} Anzahl (f) der ausgeschiedenen Schlagmänner (von engl. out count).}\end{entry}
\begin{entry}
\mainentry{autoko-su}
{アウトコース}
{ [1]  \textit{Rennsport} Außenbahn (f). [2]  \textit{Baseb.} die vom Schlagmann entfernte Seite (f). (von engl. out course). (⇔ in·kōsu インコース5842812).}\end{entry}
\begin{entry}
\mainentry{autoko-suhechokkyuuwonageru}
{アウトコースへ直球を投げる}
{ \textit{Baseb.} einen geraden Ball über die vom Schlagmann entfernte Seite der Schlagplatte werfen.}\end{entry}
\begin{entry}
\mainentry{autoko-na-}
{アウト・コーナー; アウトコーナー}
{  \textit{Baseb.} Outcorner (f) (die von der Homeplate des Schlagmannes entfernte Ecke des Spielfeldes; ⇔ in·kōnā インコーナー0304970).}\end{entry}
\begin{entry}
\mainentry{autosa-to}
{アウト・サート; アウトサート}
{ \textit{Druckw.} außenbeigeheftete Werbung (von engl. outsert; ⇔ insāto インサート7365025).}\end{entry}
\begin{entry}
\mainentry{autosa-tokoukoku}
{アウト・サート広告; アウトサート広告}
{außen beigeheftete Werbung (f).}\end{entry}
\begin{entry}
\mainentry{autosaizu}
{アウト・サイズ; アウトサイズ}
{Übergröße (f) // Warenfpl in Übergröße (von engl. outsize).}\end{entry}
\begin{entry}
\mainentry{autosaida}
{アウト・サイダ; アウトサイダ}
{[1] Außenseiter (m); Outsider (m). [2] Außenstehender. (von engl. outsider).}\end{entry}
\begin{entry}
\mainentry{autosaida-}
{アウトサイダー}
{ Außenseiter (m); Outsider (m).}\end{entry}
\begin{entry}
\mainentry{autosaido}
{アウト・サイド; アウトサイド}
{ [1] Außenseite. [2]  \textit{Baseb.} Außenfeld (n). [3]  \textit{Badminton} nicht aufschlagende Seite (f). (von engl. outside).}\end{entry}
\begin{entry}
\mainentry{autosaidoin}
{アウト・サイド・イン; アウトサイド・イン; アウトサイドイン}
{ \textit{Golf} Schlagtechnik (f), bei der der Schläger zunächst nach außen geführt wird (von engl. outside in).}\end{entry}
\begin{entry}
\mainentry{autosaidoejji}
{アウトサイド・エッジ; アウトサイドエッジ}
{ \textit{Ski} Außenkante (f) (von engl. outside edge).}\end{entry}
\begin{entry}
\mainentry{autosaidokikku}
{アウトサイド・キック; アウトサイドキック}
{ \textit{Fußball} Spielen (n) des Balles mit der Außenseite des Fußes (von engl. outside kick).}\end{entry}
\begin{entry}
\mainentry{autosaidopoketto}
{アウトサイド・ポケット; アウトサイドポケット}
{ \textit{Kleidung} Außentasche (f) (von engl. outside pocket).}\end{entry}
\begin{entry}
\mainentry{autoshu-to}
{アウト・シュート; アウトシュート}
{  \textit{Baseb.} Outshoot (n) (ein Ballwurf, der im Flug die Richtung nach außen wechselt; aus d. Engl.von engl. outshoot; ⇔ in·shūto アウト・シュート3330795).}\end{entry}
\begin{entry}
\mainentry{autosutandyinguakaunto}
{アウトスタンディング・アカウント; アウトスタンディングアカウント}
{ unbezahlte Rechnung (f); ausstehende Rechnung (f) (von engl. outstanding account›).}\end{entry}
\begin{entry}
\mainentry{autosuteppu}
{アウト・ステップ; アウトステップ}
{  \textit{Baseb.} Schritt (m) nach außen, um einen nach innen geworfenen Ball besser treffen zu können (von engl. out step).}\end{entry}
\begin{entry}
\mainentry{autosuteppusuru}
{アウトステップする}
{einen Schritt nach außen machen, um einen nach innen geworfenen Ball besser treffen zu können.}\end{entry}
\begin{entry}
\mainentry{autoso-shingu}
{アウト・ソーシング; アウトソーシング}
{  \textit{Wirtsch.} Outsourcing (n) (Übergabe von Firmenbereichen, die nicht zum Kernbereich gehören, an andere Unternehmen).}\end{entry}
\begin{entry}
\mainentry{autodoa}
{アウト・ドア; アウトドア}
{ Draußen…; draußen (⇔ indoa インドア9820394).}\end{entry}
\begin{entry}
\mainentry{autodoa-}
{アウト・ドアー; アウトドアー}
{ Draußen…; draußen (von engl. outdoors).}\end{entry}
\begin{entry}
\mainentry{autodoa-o-e-}
{アウトドアーOA}
{Büroautomatisierungf vor Ort; Büroautomatisierungf unterwegs.}\end{entry}
\begin{entry}
\mainentry{autodoa-supo-tsu}
{アウトドアー・スポーツ; アウトドアースポーツ}
{Sport (m) im Freien (von engl. outdoor sports).}\end{entry}
\begin{entry}
\mainentry{autodoa-raifu}
{アウトドアー・ライフ; アウトドアーライフ}
{Leben (n) in der Wildnis (von engl. outdoor life).}\end{entry}
\begin{entry}
\mainentry{autodoa-ribingu}
{アウトドアー・リビング; アウトドアーリビング}
{Leben (n) in der Wildnis (von engl. outdoor living).}\end{entry}
\begin{entry}
\mainentry{autodoa-reja-}
{アウトドアー・レジャー; アウトドアーレジャー}
{Freizeitaktivität (f) im Freien (von engl. outdoor leisure).}\end{entry}
\begin{entry}
\mainentry{autodoao-e-}
{アウトドアOA; アウトドア・オー・エー; アウトドアオーエー}
{Büroautomatisierung (f) für die Außenarbeit.}\end{entry}
\begin{entry}
\mainentry{autodoage-mu}
{アウトドア・ゲーム; アウトドアゲーム}
{Freilandspiel (n).}\end{entry}
\begin{entry}
\mainentry{autodoako-to}
{アウトドア・コート; アウトドアコート}
{ \textit{Sport} Spielfeld (n) im Freien (von engl. outdoor court).}\end{entry}
\begin{entry}
\mainentry{autodoashoppu}
{アウトドア・ショップ; アウトドアショップ}
{Laden (n) unter freiem Himmel.}\end{entry}
\begin{entry}
\mainentry{autodoasupo-tsu}
{アウトドア・スポーツ; アウトドアスポーツ}
{Sport (m) im Freien.}\end{entry}
\begin{entry}
\mainentry{autodoasetto}
{アウトドア・セット; アウトドアセット}
{ \textit{Film} Drehort (m) für Außenaufnahmen.}\end{entry}
\begin{entry}
\mainentry{autodoaraifu}
{アウトドア・ライフ; アウトドアライフ}
{Leben (n) in der Wildnis.}\end{entry}
\begin{entry}
\mainentry{autodoaribingu}
{アウトドア・リビング; アウトドアリビング}
{Leben (n) in der Wildnis (von engl. outdoor living).}\end{entry}
\begin{entry}
\mainentry{autodoareja-}
{アウトドア・レジャー; アウトドアレジャー}
{Freizeitaktivität (f) im Freien (von engl. outdoor leisure).}\end{entry}
\begin{entry}
\mainentry{autodoraibu}
{アウトドライブ}
{ \textit{Golf} Outdrive (m) (erster Schlag).}\end{entry}
\begin{entry}
\mainentry{autodoroppu}
{アウト・ドロップ; アウトドロップ}
{  \textit{Baseb.} Ball (m), der mit einem Bogen vom Schlagmann weg fällt (von engl. out drop).}\end{entry}
\begin{entry}
\mainentry{autonisuru}
{アウトにする}
{den Schlagmann ins Out schicken.}\end{entry}
\begin{entry}
\mainentry{autoninaru}
{アウトになる}
{aus dem Spiel sein; nicht mehr im Spiel sein.}\end{entry}
\begin{entry}
\mainentry{autoba-n}
{アウトバーン}
{  \textit{Verkehrsw.} Autobahn (f) (aus d. Dtsch.von dtsch. Autobahn).}\end{entry}
\begin{entry}
\mainentry{autoba-nriyouryoukinchoushuuseido}
{アウトバーン利用料金徴収制度}
{ \textit{Verkehrsw.} Autobahn-Maut (f).}\end{entry}
\begin{entry}
\mainentry{autohai}
{アウト・ハイ; アウトハイ}
{ \textit{Baseb.} geworfener Ball, der einen Bogen nach oben macht (von japan.-engl. outhigh).}\end{entry}
\begin{entry}
\mainentry{autobakku}
{アウトバック}
{ \textit{Gebietsn.} Outback (n) (Landesinneres Australiens).}\end{entry}
\begin{entry}
\mainentry{autofaitexingu}
{アウト・ファイティング; アウトファイティング}
{ \textit{Boxen} Kampf (m) auf Distanz (von engl. outfighting).}\end{entry}
\begin{entry}
\mainentry{autofaito}
{アウト・ファイト; アウトファイト}
{ \textit{Boxen} Kampf (m) auf Distanz (von engl. outfight).}\end{entry}
\begin{entry}
\mainentry{autofi-ruda-}
{アウト・フィールダー; アウトフィールダー}
{ \textit{Baseb.} Außenfeldspieler (m) (von engl. outfielder).}\end{entry}
\begin{entry}
\mainentry{autofi-rudo}
{アウト・フィールド; アウトフィールド}
{ \textit{Baseb.} Außenfeld (von engl. outfield).}\end{entry}
\begin{entry}
\mainentry{autofitto}
{アウト・フィット; アウトフィット}
{[1] Ausstattung (f). [2] Ausstatten (n). (von engl. outfitting).}\end{entry}
\begin{entry}
\mainentry{autofo-kasu}
{アウト・フォーカス; アウトフォーカス}
{ \textit{Fotog., Film} (absichtlich) unscharfe Aufnahme (f) (von engl. out focus).}\end{entry}
\begin{entry}
\mainentry{autoputto}
{アウト・プット; アウトプット}
{  \textit{EDV} Output (m); Leistung (f); Produktion (f); Ertrag (m); Ausstoß (m); Ausgabe (f) (⇔ in·putto インプット7412675).}\end{entry}
\begin{entry}
\mainentry{autoputtosuru}
{アウトプットする}
{ausgeben; Output liefern.}\end{entry}
\begin{entry}
\mainentry{autopureisumento}
{アウト・プレイスメント; アウトプレイスメント}
{  \textit{Wirtsch.} Outplacement (n) (Entlassung einer Führungskraft unter gleichzeitiger Vermittlung an ein anderes Unternehmen; aus d. Engl.von engl. outplacement).}\end{entry}
\begin{entry}
\mainentry{autopureiya-}
{アウト・プレイヤー; アウトプレイヤー}
{ \textit{Tennis} nicht aufschlagender Spieler (m) (von engl. out player; ☞ auto purēyā アウト・プレーヤー0523957).}\end{entry}
\begin{entry}
\mainentry{autopure-sumento}
{アウトプレースメント}
{  \textit{Wirtsch.} Outplacement (n) (Entlassung einer Führungskraft unter gleichzeitiger Vermittlung an ein anderes Unternehmen; aus d. Engl.von engl. outplacement).}\end{entry}
\begin{entry}
\mainentry{autopure-ya-}
{アウト・プレーヤー; アウトプレーヤー}
{ \textit{Tennis} nicht aufschlagender Spieler (m) (von engl. out player).}\end{entry}
\begin{entry}
\mainentry{autopointo}
{アウト・ポイント; アウトポイント}
{ \textit{Sport} Auspunkten (n) (von engl. outpoint).}\end{entry}
\begin{entry}
\mainentry{autobokushingu}
{アウト・ボクシング; アウトボクシング}
{  \textit{Boxen} Outfight (m); Kampf (m) auf Distanz (von japan.-engl. outboxing).}\end{entry}
\begin{entry}
\mainentry{autopoketto}
{アウト・ポケット; アウトポケット}
{ \textit{Kleidung} aufgesetzte Tasche (f) (von japan.-engl. für out pocket; engl. eigentl. outside pocket).}\end{entry}
\begin{entry}
\mainentry{automo-do}
{アウト・モード; アウトモード}
{Antiquiertheit (f) (von engl. outmoded).}\end{entry}
\begin{entry}
\mainentry{autoraitotorihiki}
{アウトライト取り引き; アウトライト取引き; アウトライト取引}
{ \textit{Wirtsch.} Outright-Termingeschäft (n) (⇔ suwappu·torihiki スワップ取引8408796).}\end{entry}
\begin{entry}
\mainentry{autorain}
{アウト・ライン; アウトライン}
{ [1] Umriss (m); Silhouette (f). [2] Umrisslinie (f). [3] Abriss (m). (von engl. outline).}\end{entry}
\begin{entry}
\mainentry{autorainsutecchi}
{アウトライン・ステッチ; アウトラインステッチ}
{Konturstich (beim Sticken; von engl. outline stitching).}\end{entry}
\begin{entry}
\mainentry{autorainfonto}
{アウトライン・フォント; アウトラインフォント}
{ \textit{EDV} Outline-Font (m) (frei skalierbarer Schriftfont, bei dem die Außenlinie der Zeichen gespeichert werden und ihre ausgefüllte Form am Bildschirm oder im Druck erst berechnet wird; im Gegensatz zu Bitmap-Font und Vektor-Font).}\end{entry}
\begin{entry}
\mainentry{autoran}
{アウト・ラン; アウトラン}
{ \textit{Skispringen} Auslauf (m) der Landefläche (von engl. outrun).}\end{entry}
\begin{entry}
\mainentry{autoran'na-}
{アウト・ランナー; アウトランナー}
{[1] Läufer (m) an der Spitze; Inhaber (m) der Spitzenposition. [2] Leithund (m) (eines Hundeschlittengespannes). (von engl. outrunner).}\end{entry}
\begin{entry}
\mainentry{autoriga}
{アウトリガ}
{ Ausleger (m) (von engl. out-rigger).}\end{entry}
\begin{entry}
\mainentry{autoriga-}
{アウトリガー}
{ Ausleger (m); Auslegerkran (m) (von engl. outrigger).}\end{entry}
\begin{entry}
\mainentry{autorukku}
{アウトルック}
{ [1] Aussicht (f); Ausblick (m). [2] Meinung (f); Ansicht (f). (von engl. outlook).}\end{entry}
\begin{entry}
\mainentry{autorei}
{アウト・レイ; アウトレイ}
{Auslagenfpl; Unkostenpl; Ausgabenfpl; Spesenfpl (von engl. outlay).}\end{entry}
\begin{entry}
\mainentry{autoretto}
{アウト・レット; アウトレット}
{ [1] Verkaufsstelle (f). [2] Abzug (m); Auslass (m); Durchlass (m). [3] Steckdose (f). (von engl. outlet).}\end{entry}
\begin{entry}
\mainentry{autorettosutoa}
{アウトレット・ストア; アウトレットストア}
{Verkaufsstelle (f) (von engl. outlet store).}\end{entry}
\begin{entry}
\mainentry{autoro-}
{アウトロー}
{Outlaw (m); Geächteter (m); Verfemter (m); jmd.NArN, der sich nicht an die bestehende Rechtsordnung (f) hält; jmd.NArN, der sich nicht in die Gesellschaftsordnung (f) einfügt // Verbrecher (m).}\end{entry}
\begin{entry}
\mainentry{auhitogotoni}
{会う人毎に}
{jeder, der einem begegnet; wen man auch trifft.}\end{entry}
\begin{entry}
\mainentry{aufushunitto}
{アウフシュニット}
{ Aufschnitt (m) (aus d. Dtsch.von dtsch. Aufschnitt).}\end{entry}
\begin{entry}
\mainentry{aufutakuto}
{アウフタクト}
{  \textit{Mus.} Auftakt (m) (aus d. Dtsch.von dtsch. Auftakt).}\end{entry}
\begin{entry}
\mainentry{aufuhe-ben}
{アウフヘーベン}
{  \textit{Philos.} [1] Aufhebung (f) (eines Widerspruches); Aufheben (n). [2] Heben (n) auf ein höheres Niveau. (aus d. Dtsch.von dtsch. aufheben).}\end{entry}
\begin{entry}
\mainentry{aufuhe-bensuru}
{アウフヘーベンする}
{ \textit{Philos.} aufheben (aus d. Dtsch.von dtsch. aufheben).}\end{entry}
\begin{entry}
\mainentry{auyakusoku}
{会う約束}
{ Vereinbarung (f) eines Treffens.}\end{entry}
\begin{entry}
\mainentry{auyakusokuwosuru}
{会う約束をする}
{ein Treffen mit jmdm. vereinbaren.}\end{entry}
\begin{entry}
\mainentry{aura}
{アウラ}
{ Aura (f); besondere, geheimnisvolle Ausstrahlung (f) (aus d. Dtsch.von dtsch. Aura; ⇒ ōra オーラ9342569).}\end{entry}
\begin{entry}
\mainentry{aurosu}
{アウロス}
{  \textit{Musikinstr.} Aulos (m) (antikes griech. Rohrblattinstrument).}\end{entry}
\begin{entry}
\mainentry{aurora}
{アウロラ}
{  \textit{röm. Mythol.} AurorafNAr (Göttin der Morgenröte).}\end{entry}
\begin{entry}
\mainentry{aun}
{あうん; アウン (阿呍; 阿吽)}
{ [1] Om (n) (magische Silbe des Brahmanismus, die als Hilfe zur Befreiung in der Meditation gesprochen wird; aus d. Sanskr.). [2] das A (n) und das O (n). [3] Ein‑ und Ausatmung (f).}\end{entry}
\begin{entry}
\mainentry{aunsan}
{アウン・サン; アウンサン}
{  \textit{Persönlichk.} Aung San (birmanischer General und Politiker; 1915–1947).}\end{entry}
\begin{entry}
\mainentry{aunsansu-chi-}
{アウン・サン・スー・チー; アウンサン・スーチー; アウンサンスーチー}
{  \textit{Persönlichk.} Aung San Suu Kyi (birmanische Politikerin; 1945–; erhielt 1991 den Friedensnobelpreis).}\end{entry}
\begin{entry}
\mainentry{aunsha}
{阿吽社}
{  \textit{Verlagsn.} Aunsha (Kyōto).}\end{entry}
\begin{entry}
\mainentry{aun'nokokyuu}
{あうんの呼吸 (阿吽の呼吸)}
{die geistige und körperliche Harmonisierung (f) bei einer gemeinsamen Aktivität.}\end{entry}
\begin{entry}
\mainentry{ae }
{あえ [1] (饗)}
{ Bewirtung (f); Empfang (m); Festmahl (n).}\end{entry}
\begin{entry}
\mainentry{ae }
{あえ [2] (和え; 韲え)}
{ Anmachen (n) (von Salat).}\end{entry}
\begin{entry}
\mainentry{aeka}
{あえか}
{ zierlich; dünn und zart; schwach.}\end{entry}
\begin{entry}
\mainentry{aekasa}
{あえかさ}
{Zierlichkeit (f); Schwäche (f).}\end{entry}
\begin{entry}
\mainentry{aekada}
{あえかだ}
{zierlich sein; dünn und zart sein; schwach sein.}\end{entry}
\begin{entry}
\mainentry{aekana}
{あえかな}
{zierlich; dünn und zart; schwach.}\end{entry}
\begin{entry}
\mainentry{aegi}
{あえぎ (喘ぎ)}
{ Keuchen (n); Asthma (n).}\end{entry}
\begin{entry}
\mainentry{aegiaegi}
{喘ぎ喘ぎ}
{keuchend; außer Atem.}\end{entry}
\begin{entry}
\mainentry{aegikokyuu}
{あえぎ呼吸}
{  \textit{Zool.} Hecheln (n) (eines Hundes).}\end{entry}
\begin{entry}
\mainentry{aeginagara}
{あえぎながら (喘ぎながら)}
{keuchend; außer Atem.}\end{entry}
\begin{entry}
\mainentry{aeginagaraiu}
{あえぎながら言う (喘ぎながら言う)}
{ausstoßen; keuchend sagen.}\end{entry}
\begin{entry}
\mainentry{aegu}
{あえぐ (喘ぐ)}
{ [1Gb] keuchen; japsen; schnaufen. [2] (übertr.) sich abplagen; sich quälen.}\end{entry}
\begin{entry}
\mainentry{aegoromo}
{あえ衣 (和え衣; 和衣)}
{  \textit{Kochk.} Ae·goromo (n) (japanisches Dressing).}\end{entry}
\begin{entry}
\mainentry{aezu}
{あえず (敢えず; 敢ず)}
{ etw. nicht tun können; etw. ist zu Ende, bevor man etw. tun kann.}\end{entry}
\begin{entry}
\mainentry{aedukuri}
{あえ作り; あえづくり (和え作り; 和作り; 和作; 韲え作り; 韲作り; 韲作)}
{  \textit{Kochk.} Ae·zukuri (n) (eine Art Salat mit Meeresfrucht‑ oder Geflügelstücken).}\end{entry}
\begin{entry}
\mainentry{aete}
{あえて (敢えて; 敢て)}
{ (schriftspr.) [1] gewagt; kühn; wagemutig; beherzt; unverzagt; entschlossen. [2] auf jeden Fall (mit Verneinung:) nicht davor zurückschrecken; bereit sein; … zu tun.}\end{entry}
\begin{entry}
\mainentry{aete…suru}
{あえて…する (敢えて…する; 敢て…する)}
{… wagen; wagen, … zu tun.}\end{entry}
\begin{entry}
\mainentry{aete…toiu}
{敢て…と言う}
{wagen, … zu sagen.}\end{entry}
\begin{entry}
\mainentry{aeteiu}
{敢えて言う}
{ich traue mich zu sagen ….}\end{entry}
\begin{entry}
\mainentry{aeteieba}
{あえて言えば}
{offen gesagt; überspitz formuliert.}\end{entry}
\begin{entry}
\mainentry{aeteokusokuwokokoromiru}
{あえて憶測を試みる}
{Mutmaßungen anstellen; es mit Raten probieren.}\end{entry}
\begin{entry}
\mainentry{aetekikenwookasu}
{敢て危険を冒す}
{sich kühn einer Gefahr stellen.}\end{entry}
\begin{entry}
\mainentry{aetejisanai}
{あえて辞さない}
{nicht davor zurückschrecken.}\end{entry}
\begin{entry}
\mainentry{aetesuru}
{あえてする (敢えてする)}
{ wagen, zu tun; sich erlauben, zu tun.}\end{entry}
\begin{entry}
\mainentry{aetesuru}
{敢てする}
{wagen, etw. zu tun; sich getrauen, etw. zu tun.}\end{entry}
\begin{entry}
\mainentry{aetetou}
{敢えて問う}
{sich erlauben, zu fragen ….}\end{entry}
\begin{entry}
\mainentry{aetehantaisurutoiuwakedehanai。}
{敢て反対するというわけではない。}
{ \textit{Bsp.} Es ist nicht so, dass ich dagegen bin.}\end{entry}
\begin{entry}
\mainentry{aenai}
{あえない (敢えない; 敢え無い; 敢無い)}
{ [1] traurig; betrübt; tragisch. [2] flüchtig; vergänglich. [3] elend. [4] schwach.}\end{entry}
\begin{entry}
\mainentry{aenaisaigo}
{敢えない最期}
{tragisches Lebensende (n); trauriger Tod (m).}\end{entry}
\begin{entry}
\mainentry{aenaisaigowotogeru}
{あえない最期を遂げる; あえない最期をとげる (敢無い最期を遂げる)}
{ein tragisches Ende finden; tragisch umkommen; elend ins Gras beißen.}\end{entry}
\begin{entry}
\mainentry{aenaku}
{あえなく (敢なく; 敢え無く; 敢無く)}
{trauriger Weise (f); tragischer Weise (f).}\end{entry}
\begin{entry}
\mainentry{aenakusaigowotogeru}
{あえなく最後を遂げる}
{ein tragisches Ende finden; tragisch umkommen; elend ins Gras beißen.}\end{entry}
\begin{entry}
\mainentry{aenakutezan'nendeshita。}
{会えなくて残念でした。}
{ \textit{Bsp.} Schade, dass wir uns nicht treffen konnten!}\end{entry}
\begin{entry}
\mainentry{aenakunaru}
{あえなくなる}
{sterben; verscheiden.}\end{entry}
\begin{entry}
\mainentry{aeneasu}
{アエネアス}
{  \textit{griech. Mythol.} ÄneasmNAr (trojanischer Held und römischer Reichsgründer).}\end{entry}
\begin{entry}
\mainentry{aeneisu}
{アエネイス}
{  \textit{Werktitel} ÄneisNAr (eine Dichtung Vergils).}\end{entry}
\begin{entry}
\mainentry{aene-isu}
{アエネーイス}
{  \textit{Persönlichk.} ÄneasmNAr (trojanischer Held).}\end{entry}
\begin{entry}
\mainentry{aeba}
{饗庭}
{  \textit{Familienn.} Aeba.}\end{entry}
\begin{entry}
\mainentry{aemaze}
{和え交ぜ; 和交ぜ; 和交 (韲え交ぜ; 韲交ぜ; 韲交)}
{  \textit{Kochk.} Ae·maze (n) (in Bonito-Flocken, Essig und Sake eingelegtes Fischfleisch o. Ä.).}\end{entry}
\begin{entry}
\mainentry{aemono}
{あえ物 (和え物; 和物; 韲え物; 韲物; 齏え物; 齏物)}
{  \textit{Kochk.} Aemono (n); (eine Art) gemischter Salat (m) aus Fisch, Muscheln und/oder Gemüse mit Sauce aus Essig oder Miso etc..}\end{entry}
\begin{entry}
\mainentry{aeru }
{あえる [1] (合える; 和える; 韲える; 齏える)}
{ anmachen (Salat).}\end{entry}
\begin{entry}
\mainentry{aeru }
{会える; あえる [2] (逢える)}
{treffen können; begegnen können.}\end{entry}
\begin{entry}
\mainentry{aerozoru}
{アエロゾル}
{ Aerosol (n) (feinste Verteilung fester oder flüssiger Stoffe in Gas).}\end{entry}
\begin{entry}
\mainentry{aerofuro-to}
{アエロフロート}
{  \textit{Firmenn.} Aeroflot (f) (russ. Luftfahrtgesellschaft).}\end{entry}
\begin{entry}
\mainentry{aeromoberu}
{アエロモベル}
{  \textit{Eisenb.} Aeromobil (n); Luftkissenfahrzeug (n).}\end{entry}
\begin{entry}
\mainentry{aeroraito}
{アエロライト}
{  \textit{Astron.} Steinmeteorit (m); Aerolite (m) (⇒ sekishitsu·inseki 石質隕石0879118).}\end{entry}
\begin{entry}
\mainentry{aen}
{亜鉛; Zn}
{  \textit{Chem.} Zink (n) (bläulich weiß glänzendes Metall; Zeichen: Zn).}\end{entry}
\begin{entry}
\mainentry{aenka}
{亜鉛華}
{  \textit{Mineral.} Zinkblume (f); Zinkweiß (n); Zinkoxid (n).}\end{entry}
\begin{entry}
\mainentry{aenkadenpun}
{亜鉛華澱粉}
{ \textit{Chem.} Zink-Pulver (n).}\end{entry}
\begin{entry}
\mainentry{aenkanankou}
{亜鉛華軟膏}
{ \textit{Pharm.} Zinksalbe (f).}\end{entry}
\begin{entry}
\mainentry{aenkabansoukou}
{亜鉛華絆創膏}
{ \textit{Med.} Zink-Pflaster (n).}\end{entry}
\begin{entry}
\mainentry{aenkabutsu}
{亜塩化物}
{  \textit{Chem.} Subchlorid (n).}\end{entry}
\begin{entry}
\mainentry{aenkayu}
{亜鉛華油}
{ \textit{Pharm.} Zinköl (n).}\end{entry}
\begin{entry}
\mainentry{aenkaraeta}
{亜鉛から得た}
{Zink….}\end{entry}
\begin{entry}
\mainentry{aengantai}
{亜沿岸帯}
{  \textit{Geogr.} Sublittoral-Zone (f).}\end{entry}
\begin{entry}
\mainentry{aengantaino}
{亜沿岸帯の}
{ \textit{Geogr.} sublittoral.}\end{entry}
\begin{entry}
\mainentry{aenkou}
{亜鉛鉱}
{Zinkerz (n).}\end{entry}
\begin{entry}
\mainentry{aenkoushou}
{亜鉛鉱床}
{ \textit{Bergbauw.} Zinklagerstätte (f).}\end{entry}
\begin{entry}
\mainentry{aenkouso}
{亜鉛酵素}
{ \textit{Biochem.} Zinkenzym (n); zinkhaltiges Enzym (n).}\end{entry}
\begin{entry}
\mainentry{aensan'en}
{亜鉛酸塩}
{ \textit{Chem.} Zinkat (n).}\end{entry}
\begin{entry}
\mainentry{aenzoku}
{亜鉛族}
{ \textit{Chem.} Zinkgruppe (f); Zink-Elementfamilie (f).}\end{entry}
\begin{entry}
\mainentry{aensosan}
{亜塩素酸}
{  \textit{Chem.} chlorige Säure (f).}\end{entry}
\begin{entry}
\mainentry{aensosan'en}
{亜塩素酸塩}
{ \textit{Chem.} Chlorit (n).}\end{entry}
\begin{entry}
\mainentry{aensosan'natoriumu}
{亜塩素酸ナトリウム}
{ \textit{Chem.} Natriumchlorit (n).}\end{entry}
\begin{entry}
\mainentry{aendaikasutexingu}
{亜鉛ダイカスティング}
{Zinkdruckguss (m).}\end{entry}
\begin{entry}
\mainentry{aenchuukai}
{亜鉛鋳塊}
{Hüttenzink (n); Rohzink (n).}\end{entry}
\begin{entry}
\mainentry{aenchuudoku}
{亜鉛中毒}
{ \textit{Med.} Zinkvergiftung (f).}\end{entry}
\begin{entry}
\mainentry{aentetsu}
{亜鉛鉄}
{verzinktes Eisen (n).}\end{entry}
\begin{entry}
\mainentry{aentekkou}
{亜鉛鉄鉱}
{Zinkerz (n).}\end{entry}
\begin{entry}
\mainentry{aendetsutsumu}
{亜鉛で包む}
{verzinken.}\end{entry}
\begin{entry}
\mainentry{aenteppan}
{亜鉛鉄板}
{Zinkblech (n).}\end{entry}
\begin{entry}
\mainentry{aentoppan}
{亜鉛凸版}
{  \textit{Druckw.} Zinkotypie (f); Photozinkographie (f).}\end{entry}
\begin{entry}
\mainentry{aentoppan'nisuru}
{亜鉛凸版にする}
{zu einer Zinkotypie verarbeiten.}\end{entry}
\begin{entry}
\mainentry{aen'ninita}
{亜鉛に似た}
{zinkartig.}\end{entry}
\begin{entry}
\mainentry{aen'netsu}
{亜鉛熱}
{ \textit{Med.} Zinkfieber (n).}\end{entry}
\begin{entry}
\mainentry{aen'no}
{亜鉛の}
{Zink….}\end{entry}
\begin{entry}
\mainentry{aenhaku}
{亜鉛白 [a]}
{ \textit{Chem.} (reines) Zinkoxid (n); Zinkweiß (n); Chinaweiß (n) (→ aen·baku 亜鉛白0596393).}\end{entry}
\begin{entry}
\mainentry{aenbaku}
{亜鉛白 [b]}
{ \textit{Chem.} (reines) Zinkoxid (n); Zinkweiß (n); Chinaweiß (n) (→ aen·haku 亜鉛白9560930).}\end{entry}
\begin{entry}
\mainentry{aenban }
{亜鉛版}
{  \textit{Druckw.} Zinkotypie (f); Photozinkographie (f).}\end{entry}
\begin{entry}
\mainentry{aenban }
{亜鉛板}
{Zinkblech (n); Zinkplatte (f).}\end{entry}
\begin{entry}
\mainentry{aenbantsuchiishi}
{亜鉛バン土石}
{ \textit{Mineral.} Zinkaluminit (m).}\end{entry}
\begin{entry}
\mainentry{aenban'nisuru}
{亜鉛版にする}
{in Zink ätzen; zu einer Zinkotypie verarbeiten.}\end{entry}
\begin{entry}
\mainentry{aenbiki}
{亜鉛引き; 亜鉛引}
{ Verzinkung (f).}\end{entry}
\begin{entry}
\mainentry{aenbikino}
{亜鉛引きの; 亜鉛引の; 亜鉛びきの}
{verzinkt; mit Zink überzogen; galvanisiert.}\end{entry}
\begin{entry}
\mainentry{aenbikinoteppan}
{亜鉛引きの鉄板}
{verzinktes Blech (n); Zinkblech (n).}\end{entry}
\begin{entry}
\mainentry{aenheihan}
{亜鉛平版}
{ \textit{Druckw.} Zinkdruckplatte (f).}\end{entry}
\begin{entry}
\mainentry{aenmatsu}
{亜鉛末}
{Zinkstaub (m).}\end{entry}
\begin{entry}
\mainentry{aenmatsutoryou}
{亜鉛末塗料}
{Zinkstaubfarbe (f).}\end{entry}
\begin{entry}
\mainentry{aenmekki}
{亜鉛めっき; 亜鉛メッキ (亜鉛鍍金)}
{ Verzinkung (f).}\end{entry}
\begin{entry}
\mainentry{aenmekkisuru}
{亜鉛メッキする}
{verzinken.}\end{entry}
\begin{entry}
\mainentry{aenmekkitessen}
{亜鉛めっき鉄線}
{verzinkter Draht (m).}\end{entry}
\begin{entry}
\mainentry{aenmekkino}
{亜鉛めっきの}
{verzinkt.}\end{entry}
\begin{entry}
\mainentry{aenryokuban}
{亜鉛緑バン; 亜鉛緑ばん (亜鉛緑礬)}
{ \textit{Mineral.} Zinkmelanterit (m).}\end{entry}
\begin{entry}
\mainentry{aenryokubando}
{亜鉛緑バンド; 亜鉛緑ばんど (亜鉛緑礬土)}
{ \textit{Mineral.} Zinkmelanterit (m).}\end{entry}
\begin{entry}
\mainentry{aenwokaburaseru}
{亜鉛を被らせる}
{mit Zink überziehen; verzinken.}\end{entry}
\begin{entry}
\mainentry{aenwokiseru}
{亜鉛をきせる}
{verzinken.}\end{entry}
\begin{entry}
\mainentry{aenwoshoujiru}
{亜鉛を生じる}
{verzinken; mit Zink versetzen.}\end{entry}
\begin{entry}
\mainentry{aenwofukumu}
{亜鉛を含む}
{Zink enthalten.}\end{entry}
\begin{entry}
\mainentry{aenwohouwasaseru}
{亜鉛を飽和させる}
{mit Zink versetzen.}\end{entry}
\begin{entry}
\mainentry{ao}
{青}
{ [A] (als Nomen) [1] Blau (n) // Grün (n). [2] rote Ampel (f). [3] schwarzes Pferd (n); blauschwarzes Pferd (n). [4] Aotan (n) (japan. Kartenspiel; ⇒ ao·tan 青短; 青丹3852075). [5]  \textit{Literaturw.} Ao·hon}\end{entry}
\begin{entry}
\mainentry{aoao}
{青々; 青青; あおあお}
{ leuchtend grün; frisch; blass.}\end{entry}
\begin{entry}
\mainentry{aoaoshita}
{青々した; 青青した}
{leuchtend grün; frisch; blass.}\end{entry}
\begin{entry}
\mainentry{aoaoshiteiru}
{青々している}
{leuchtend grün sein; frisch sein; blass.}\end{entry}
\begin{entry}
\mainentry{aoaosuru}
{青々する; 青青する}
{grün werden; üppig grün werden.}\end{entry}
\begin{entry}
\mainentry{aoaza}
{青あざ (青痣)}
{ [1]  \textit{Med.} blauer Fleck (m). [2]  \textit{Anat.} Muttermal (n).}\end{entry}
\begin{entry}
\mainentry{aoazanodekita}
{青痣のできた}
{mit blauen Flecken übersät.}\end{entry}
\begin{entry}
\mainentry{aoashi}
{あおあし; アオアシ (青蘆; 青葦)}
{  \textit{Bot.} grün wucherndes sommerliches Schilf (Sommer).}\end{entry}
\begin{entry}
\mainentry{aoashishigi}
{あおあししぎ; アオアシシギ (あおあし鴫; 青足鷸)}
{  \textit{Vogelk.} Grünschenkel (m) (Tringa nebularia).}\end{entry}
\begin{entry}
\mainentry{aoatama}
{青頭}
{ rasierter Kopf (der wegen der durchscheinenden Haare blau erscheint).}\end{entry}
\begin{entry}
\mainentry{aoamigasa}
{青編み笠; 青編笠}
{ neuer Binsenkorbhut (m) aus frischer Binse.}\end{entry}
\begin{entry}
\mainentry{aoarashi}
{青あらし (青嵐 [1])}
{  \textit{Meteor.} (schriftspr.) Frühjahrssturm (m); Maisturm (m) (→ seiran 青嵐2188077; ⇒ kunpū 薫風6507808).}\end{entry}
\begin{entry}
\mainentry{aoan}
{青あん (青餡)}
{ blaues An (n) (aus weißem An, in das z. B. Blaualge oder gemahlener Tee gemischt ist).}\end{entry}
\begin{entry}
\mainentry{aoi }
{あおい [1]; アオイ (葵)}
{ [1]  \textit{Bot.} Malve (f); Stockmalve (f) (Malva). [2] changierende Farbe (f) mit Hellblau im Vordergrund und und Hellviolett im Hintergrund. [3] Malvenwappen (n) (ein japan. Familienwappen; ein Malvenwappenaoi).}\end{entry}
\begin{entry}
\mainentry{aoi }
{青い; あおい [2] (蒼い; 碧い)}
{ [1] unerfahren.}\end{entry}
\begin{entry}
\mainentry{aoi }
{青井}
{  \textit{Familienn.} AoiNAr.}\end{entry}
\begin{entry}
\mainentry{aoiomutsushoukougun}
{青いおむつ症候群}
{ \textit{Med.} Blaue-Windeln-Syndrom (n) (engl. blue diaper syndrome).}\end{entry}
\begin{entry}
\mainentry{aoika}
{葵科}
{ \textit{Bot.} Malvacaefpl.}\end{entry}
\begin{entry}
\mainentry{aoigai}
{アオイガイ; あおいがい (葵貝)}
{  \textit{Zool.} Papierboot (n); Papiernautilus (m) (eine Krakenart; Argonauta argo).}\end{entry}
\begin{entry}
\mainentry{aoikaoshiteiru}
{青い顔している}
{blass sein.}\end{entry}
\begin{entry}
\mainentry{aoikaowoshiteiru}
{青い顔をしている}
{blass sein.}\end{entry}
\begin{entry}
\mainentry{aoikajitsu}
{青い果実}
{unreife Frucht (f); grüne Frucht (f).}\end{entry}
\begin{entry}
\mainentry{aoiki}
{青息}
{ schweres Atmen (n); Seufzen (n).}\end{entry}
\begin{entry}
\mainentry{aoikitoiki}
{青息吐息}
{ verzweifelte Lage (f).}\end{entry}
\begin{entry}
\mainentry{aoikitoikida}
{青息吐息だ}
{in der Klemme sitzen; in einer verzweifelten Lage sein; aufgeschmissen sein.}\end{entry}
\begin{entry}
\mainentry{aoikitoikidearu}
{青息吐息である}
{in der Klemme sitzen; in einer verzweifelten Lage sein; aufgeschmissen sein.}\end{entry}
\begin{entry}
\mainentry{aoikitoikinokaradade}
{青息吐息の体で}
{in verzweifelter Lage.}\end{entry}
\begin{entry}
\mainentry{aoikiwohaiteiru}
{青息を吐いている}
{seufzen; schwer atmen.}\end{entry}
\begin{entry}
\mainentry{aoikiwohaku}
{青息を吐く}
{seufzen; einen tiefen Seufzer tun.}\end{entry}
\begin{entry}
\mainentry{aoigoke}
{アオイゴケ; あおいごけ (葵苔)}
{  \textit{Bot.} Silberwinde (Dichondra repens).}\end{entry}
\begin{entry}
\mainentry{aoishi}
{青石}
{ [1] blauer Stein (z. B. als Material für die Hausverkleidung verwendet). [2]  \textit{Gartenkunst} blauer Gartenstein (m).}\end{entry}
\begin{entry}
\mainentry{aoishitouba}
{青石塔婆}
{ steinerne Grabtafel (f) aus blauem Stein aus Chichibu.}\end{entry}
\begin{entry}
\mainentry{aoishiwata}
{青石綿}
{ \textit{Mineral.} Krokydolith (m); graublauer Riebeckit (m); Falkenauge (n).}\end{entry}
\begin{entry}
\mainentry{aoisora}
{青い空}
{blauer Himmel (m).}\end{entry}
\begin{entry}
\mainentry{aoita}
{青板}
{ grün gefärbtes und getrocknetes Seegras (n) (wird für viele Gerichte verwendet).}\end{entry}
\begin{entry}
\mainentry{aoitakonbu}
{青板昆布}
{grün gefärbtes und getrocknetes Seegras (n) (wird für viele Gerichte verwendet).}\end{entry}
\begin{entry}
\mainentry{aoiduki}
{あおい月 (葵月)}
{ sechster Monat (m) des Mondkalenders.}\end{entry}
\begin{entry}
\mainentry{aoitsuba}
{あおいつば (葵鍔; 葵鐔)}
{ Malven-Tsuba (n) (Stichblatt mit oben, unten und an den Seiten herausgezogenen Spitzen und in den Diagonalen eingezogenen Ecken). (Aoi·tsubaaoitsuba).}\end{entry}
\begin{entry}
\mainentry{aoideageru}
{あおいであげる; あおいで上げる}
{jmdm. zufächeln.}\end{entry}
\begin{entry}
\mainentry{aoidehaewoou}
{扇いではえを追う}
{Fliegen durch Fächeln vertreiben.}\end{entry}
\begin{entry}
\mainentry{aoidefujiwomiru}
{仰いで富士を見る}
{zum Fuji hinaufblicken.}\end{entry}
\begin{entry}
\mainentry{aoitotonbo}
{アオイトトンボ; あおいととんぼ (青糸蜻蛉)}
{  \textit{Insektenk.} Gemeine Binsenjungfer (f) (Lestes sponsa).}\end{entry}
\begin{entry}
\mainentry{aoitotonboka}
{アオイトトンボ科 (青糸蜻蛉科)}
{ \textit{Insektenk.} Lestidaefpl.}\end{entry}
\begin{entry}
\mainentry{aoidomoe}
{葵巴}
{Malvenwappen (n) (ein japan. Familienwappen mit kreisförmig angeordneten Malvenblättern; von der Familie Tokugawa verwendet; Malvenwappenaoi).}\end{entry}
\begin{entry}
\mainentry{aoitori}
{青い鳥}
{[1] Vogel (m) des Glücks // Glück (n). [2]  \textit{Theat.} NAr (Theaterstück von Maurice Maeterlinck; 1909).}\end{entry}
\begin{entry}
\mainentry{aoihaaiyoriideteaiyoriaoshi。}
{青いは藍よりいでて藍より青し。}
{ \textit{Bsp.} Der Schüler übertrifft den Lehrer (üblicher ist die Version: Ao wa ai yori idete ai yori aoshi. 青は藍よりいでて藍より青し。1127026).}\end{entry}
\begin{entry}
\mainentry{aoimatsuri}
{あおい祭り (葵祭り; 葵祭)}
{ Aoi·matsuri (n); Kamo·matsuri (n); Kamo-Fest (n) (Fest am 15. Mai am Kamo-Schrein in Kyōto).}\end{entry}
\begin{entry}
\mainentry{aoime}
{青い目}
{blaue Augennpl.}\end{entry}
\begin{entry}
\mainentry{aoimegane}
{青い眼鏡}
{grüne Brille (f).}\end{entry}
\begin{entry}
\mainentry{aoimono}
{青い物; 青いもの}
{Grünpflanzenfpl; grünes Gemüse (n); Grünzeug (n).}\end{entry}
\begin{entry}
\mainentry{aoimonononaikoujougai }
{青いもののない工場街}
{Industriegebiet (n) ohne jede Grünpflanze.}\end{entry}
\begin{entry}
\mainentry{aoimonononaikoujougai }
{青い物のない工場街}
{Industriegebiet (n) ohne Grünpflanzen.}\end{entry}
\begin{entry}
\mainentry{aoiringo}
{青いリンゴ}
{grüner Apfel (m).}\end{entry}
\begin{entry}
\mainentry{aoiro}
{青色}
{ [1] Blau (n); blaue Farbe (f). [2] Steuererklärung (f) für Einkommens‑ und Körperschaftssteuer (Abk. für aoiro·shinkoku).}\end{entry}
\begin{entry}
\mainentry{aoirojigyousenjuusha}
{青色事業専従者}
{jmd., der Steuern bezahlt und einer Vollzeit-Beschäftigung nachgeht.}\end{entry}
\begin{entry}
\mainentry{aoiroshifuto}
{青色シフト}
{ \textit{Astron.} Blauverschiebung (f) (Verkürzung der gemessenen Wellenlänge gegenüber der emittierten Strahlung; ⇔ sekihō·hen’i 赤方偏移1095270).}\end{entry}
\begin{entry}
\mainentry{aoiroshinkoku}
{青色申告}
{ Steuererklärung (f) für Einkommens‑ und Körperschaftssteuer (auf blauem Formblatt; ⇔   白色申告0159532).}\end{entry}
\begin{entry}
\mainentry{aoiroshinkokuyoushi}
{青色申告用紙}
{Formular (n) für die Steuererklärung.}\end{entry}
\begin{entry}
\mainentry{aoiroshinkokuwosuru}
{青色申告をする}
{die Steuererklärung abgeben.}\end{entry}
\begin{entry}
\mainentry{aoiroshinkokuwoteishutsusuru}
{青色申告を提出する}
{die Steuererklärung einreichen.}\end{entry}
\begin{entry}
\mainentry{aoinku}
{青インク}
{blaue Tinte (f).}\end{entry}
\begin{entry}
\mainentry{aou}
{亜欧}
{ (schriftspr.) AsiennNAr und EuropanNAr; EurasiennNAr (⇒ Ō·A 欧亜4150094).}\end{entry}
\begin{entry}
\mainentry{aouo}
{青魚 [1]; あおうお; アオウオ}
{  \textit{Fischk.} Schwarzkarpfen (m); Schwarzer Amur (m) (Mylopharyngodon piceus).}\end{entry}
\begin{entry}
\mainentry{aoukikusa}
{青浮き草; 青浮草; 青うきくさ; アオウキクサ (青萍; 青浮萍)}
{ \textit{Bot.} Wasserlinse (f) (Lemna paucicostata).}\end{entry}
\begin{entry}
\mainentry{aoushuu}
{亜欧州}
{AsiennNAr und EuropanNAr; EurasiennNAr.}\end{entry}
\begin{entry}
\mainentry{aounabara}
{青海原}
{(poet.) blaue See (f); weites Meer (n).}\end{entry}
\begin{entry}
\mainentry{aouma}
{青馬 (白馬 {ir.} [1])}
{ [1] blauschwarzes Pferd (n); Rappe (m). [2] Schimmel (m); Apfelschimmel (m).}\end{entry}
\begin{entry}
\mainentry{aoumanosechie}
{白馬の節会; 白馬節会}
{Schimmel-Bankett (n) (jährlich am 7. Januar am Kaiserhof stattfindendes Bankett, das der Kaiser nach Besichtigung von 21 Schimmeln abhält; eine chin. Legende besagt, dass Arglist vertrieben würde, wenn man an diesem Tag einen Schimmel sieht; seit der Heian-Zeit üblich).}\end{entry}
\begin{entry}
\mainentry{aoumi}
{青海 [1]}
{ blaue See (f); blaues Meer (n).}\end{entry}
\begin{entry}
\mainentry{aoumigame}
{青海亀; 青海がめ; あおうみがめ; アオウミガメ (青海龜)}
{  \textit{Zool.} grüne Suppenschildkröte (f) (Chelonia mydas).}\end{entry}
\begin{entry}
\mainentry{aoume}
{青梅}
{  \textit{Bot.} unreife Pflaume (f).}\end{entry}
\begin{entry}
\mainentry{aouri}
{アオウリ; あおうり (青瓜)}
{  \textit{Bot.} Gemüsemelone (f) (Cucumis melo var. conomon).}\end{entry}
\begin{entry}
\mainentry{aoe}
{青絵}
{ [1] kobaltblaue Bemalung (f) auf Porzellan. [2] Porzellan (n) mit kobaltblauer Bemalung.}\end{entry}
\begin{entry}
\mainentry{aoendou}
{青えんどう; あおえんどう; アオエンドウ (青豌豆)}
{  \textit{Bot.} grüne Erbsenfpl (⇒ gurin·pīsu グリンピース8394320).}\end{entry}
\begin{entry}
\mainentry{aoenpitsu}
{青鉛筆; 青えんぴつ; 青エンピツ}
{blauer Stift (m).}\end{entry}
\begin{entry}
\mainentry{aootoko}
{青男}
{ junger unreifer Mann (m) (⇔ ao·onna 青女2014797).}\end{entry}
\begin{entry}
\mainentry{aoon'na}
{青女}
{ [1] junge unreife Frau (f). [2] Frau (f) von niederem sozialen Status.}\end{entry}
\begin{entry}
\mainentry{aogai}
{青貝}
{ [1]  \textit{Muschelk.} Napfmuschel (f) (Notoacmea schrenckii). [2] Perlmutter (m); Perlmutt (n).}\end{entry}
\begin{entry}
\mainentry{aogaizaiku}
{青貝細工}
{Perlmuttarbeit (f).}\end{entry}
\begin{entry}
\mainentry{aogainuri}
{青貝塗り; 青貝塗}
{Lack (m) mit eingelegtem Perlmutt.}\end{entry}
\begin{entry}
\mainentry{aogaeru}
{青がえる; あおがえる; 青ガエル; アオガエル (青蛙)}
{  \textit{Zool.} [1] grüner Frosch (m) (z. B. Leoparden‑ oder Laubfrosch; Sommer). [2] Ruderfrosch (m); eigentlicher Ruderfrosch (m) (Rhacophorus).}\end{entry}
\begin{entry}
\mainentry{aogaeruka}
{青蛙科}
{ \textit{Zool.} Laubfröschefpl (Polipedalidae).}\end{entry}
\begin{entry}
\mainentry{aogaeruka}
{アオガエル科 (青蛙科)}
{ \textit{Zool.} Ruderfrosch (m) (Rhacophoridae).}\end{entry}
\begin{entry}
\mainentry{aogaki}
{青垣}
{  \textit{Ortsn.} AogakinNAr (Ortschaft in der Gemeinde Hikami im Zentrum der Präf. Hyōgo).}\end{entry}
\begin{entry}
\mainentry{aokakiyama}
{青垣山 [a]}
{ blaue Bergkette (f).}\end{entry}
\begin{entry}
\mainentry{aogakiyama}
{青垣山 [b]}
{ blaue Bergkette (f).}\end{entry}
\begin{entry}
\mainentry{aokakesu}
{青懸巣}
{ \textit{Vogelk.} blauer Eichelhäher (m).}\end{entry}
\begin{entry}
\mainentry{aogashima}
{青ヶ島}
{  \textit{Inseln.} AogashimanNAr (zur Präf. Tōkyō gehörende Inselgruppe).}\end{entry}
\begin{entry}
\mainentry{aogashiwa}
{青柏}
{ Eiche (f) mit jungen Blüttern // junge Eichenblätter.}\end{entry}
\begin{entry}
\mainentry{aogasu }
{あおがす (扇がす; 煽がす)}
{ jmdn. fächern lassen.}\end{entry}
\begin{entry}
\mainentry{aogasu }
{仰がす}
{ [1] jmdn. aufblicken lassen. [2] jmdn. verehren lassen. [3] jmdn. bitten lassen. [4] jmdn. trinken lassen.}\end{entry}
\begin{entry}
\mainentry{aokata}
{青方}
{  \textit{Familienn.} Aokata.}\end{entry}
\begin{entry}
\mainentry{aokabi}
{青かび; 青カビ; アオカビ; あおかび (青黴)}
{  \textit{Mikrobiol.} grüner (blauer) Schimmel (m); Brotschimmel (m); Pinselschimmel (m); Penicillium (n).}\end{entry}
\begin{entry}
\mainentry{aokabirui}
{アオカビ類}
{ \textit{Mikrobiol.} Penicillium (n).}\end{entry}
\begin{entry}
\mainentry{aogami}
{青紙}
{ blaues Papier (n); blau gefärbtes Papier (n) // (insbes.) mit Commeline blau gefärbtes Papier (n).}\end{entry}
\begin{entry}
\mainentry{aokamikirimushi}
{青髪切り虫; アオカミキリムシ; あおかみきりむし}
{  \textit{Insektenk.} Ao·kamikirimushi (Chelidonium quadricolle).}\end{entry}
\begin{entry}
\mainentry{aokarakami}
{青唐紙}
{ [1] Stofffarbe (f), die sich ergibt, wenn Stoff aus gelben Kettfäden und blauem Schuss gewebt wirbt. [2] Stoff mit gelber Vorderseite und grüner Rückseite.}\end{entry}
\begin{entry}
\mainentry{aogari}
{青刈り; 青刈}
{ Ernte (f) von etw., bevor es reif ist (z. B. als Futter).}\end{entry}
\begin{entry}
\mainentry{aogare}
{青枯れ; 青枯}
{  \textit{Bot.} (plötzliche) Welke (f).}\end{entry}
\begin{entry}
\mainentry{aogarebyou}
{青枯れ病; 青枯病}
{ \textit{Bot.} Welke (f) (durch Bakterien).}\end{entry}
\begin{entry}
\mainentry{aoki}
{あおき; アオキ; 青木}
{  \textit{Bot.} Aoki (m); japanischer Lorbeer (m) (Aucuba japonica).}\end{entry}
\begin{entry}
\mainentry{aokishoten}
{青木書店}
{  \textit{Verlagsn.} Aoki Shoten (Tōkyō; ISBN 4-250-).}\end{entry}
\begin{entry}
\mainentry{aogitateru}
{扇ぎ立てる; 扇ぎたてる (煽ぎ立てる)}
{ [1] kräftig fächeln; (Feuer) anfachen; anschüren. [2] schmeicheln; agitieren; aufhetzen; aufwiegeln.}\end{entry}
\begin{entry}
\mainentry{aogippu}
{青切符}
{ [1] blaue Fahrkarte (f) der zweiten Klasse (früher bei der japan. Bahn). [2] Passagier (m) der zweiten Klasse.}\end{entry}
\begin{entry}
\mainentry{aokinako}
{青きな粉 (青黄な粉; 青黄粉)}
{ hellgrünes Sojabohnenmehl (n).}\end{entry}
\begin{entry}
\mainentry{aoginako}
{青ぎな粉 (青黄な粉; 青黄粉)}
{ hellgrünes Sojabohnenmehl (n) (Zählwort der Verkaufsform in Tüten ist 袋 tai).}\end{entry}
\begin{entry}
\mainentry{aokinohana}
{青木の花}
{ \textit{Bot.} Aoki-Blüte (f) (Frühling).}\end{entry}
\begin{entry}
\mainentry{aokinomi}
{青木の実}
{ \textit{Bot.} Aoki-Frucht (f) (Winter).}\end{entry}
\begin{entry}
\mainentry{aokiba}
{青木葉}
{ [1] grünes Blatt (n). [2] Blatt (n) einer immergrünen Pflanze. [3]  \textit{Bot.} Japanische Aucube; Aukube; Japanische Goldorange (Aucuba japonica).}\end{entry}
\begin{entry}
\mainentry{aogimiru}
{仰ぎ見る; 仰ぎみる; あおぎ見る}
{ [1] hinaufblicken; aufblicken. [2] respektieren; verehren.}\end{entry}
\begin{entry}
\mainentry{aogimiruwagashi}
{仰ぎ見る我が師}
{mein verehrter Lehrer (m).}\end{entry}
\begin{entry}
\mainentry{aogiri}
{青ぎり; あおぎり; アオギリ (青桐; 梧桐 {ir.})}
{  \textit{Bot.} Phönixbaum (m) (Firmiana simplex; Sommer).}\end{entry}
\begin{entry}
\mainentry{aogirishobou}
{青桐書房}
{  \textit{Verlagsn.} Aogiri Shobō (Tōkyō).}\end{entry}
\begin{entry}
\mainentry{aogiwakeru}
{扇ぎ分ける}
{ worfeln; Körner von der Spreu trennen.}\end{entry}
\begin{entry}
\mainentry{aoku}
{青く}
{ blau; in blauer Farbe.}\end{entry}
\begin{entry}
\mainentry{aogu }
{あおぐ [1] (扇ぐ; 煽ぐ)}
{ [1] fächeln. [2] anfachen. [3] worfeln; Körner von der Spreu trennen.}\end{entry}
\begin{entry}
\mainentry{aogu }
{仰ぐ; あおぐ [2]}
{ [1] aufsehen; nach oben sehen; emporblicken; hinaufblicken. [2] achten; verehren; respektieren. [3] um Hilfe bitten; um Gnade bitten; bitten. [4] in einem Zug austrinken; trinken; (Gift) nehmen. [5] nach oben richten.}\end{entry}
\begin{entry}
\mainentry{aokukasundeiruyama}
{青くかすんでいる山}
{die blauen Bergempl.}\end{entry}
\begin{entry}
\mainentry{aokuge}
{青公家 (青公卿)}
{ [1] Hofadliger (m) von niedrigem Rang // kleiner Hofadliger (m) (Ausdruck, um einen Angehörigen der Kuge verächtlich zu machen). [2]  \textit{Kabuki} Ao·kuge (m); blau geschminkter Hofadliger (m) in der Schurkenrolle.}\end{entry}
\begin{entry}
\mainentry{aokusa}
{青草}
{ grüne Pflanzenfpl; Grün (n).}\end{entry}
\begin{entry}
\mainentry{aokusai}
{青臭い; 青くさい; あおくさい}
{ [1] nach Gras riechen; unreif riechen. [2] grün; unreif; unerfahren; nicht trocken hinter den Ohren.}\end{entry}
\begin{entry}
\mainentry{aokusaiiken}
{青臭い意見}
{unreife Meinung (f).}\end{entry}
\begin{entry}
\mainentry{aokusairono}
{青草色の}
{grasgrün.}\end{entry}
\begin{entry}
\mainentry{aokusakamemushi}
{青臭亀虫; あおくさかめむし; アオクサカメムシ (青草椿象)}
{  \textit{Insektenk.} Aokusa·kamemushi; grüne Baumwanze (Nezara antennata).}\end{entry}
\begin{entry}
\mainentry{aokusasa}
{青臭さ; 青くささ}
{[1] Grasgeruch (m); unreifer Geruch (m). [2] fehlende Reife (f).}\end{entry}
\begin{entry}
\mainentry{aokusami}
{青臭み}
{[1] Grasgeruch (m); unreifer Geruch (m). [2] fehlende Reife (f).}\end{entry}
\begin{entry}
\mainentry{aokusundamizu}
{青く澄んだ水}
{klares, blaues Wasser (n).}\end{entry}
\begin{entry}
\mainentry{aokusomeru}
{青く染める}
{blau färben; grün färben.}\end{entry}
\begin{entry}
\mainentry{aokunattariakakunattarisuru}
{青くなったり赤くなったりする}
{abwechselnd blass und rot werden.}\end{entry}
\begin{entry}
\mainentry{aokunaru}
{青くなる}
{[1] grün werden. [2] blass werden.}\end{entry}
\begin{entry}
\mainentry{aokunuru}
{青く塗る}
{blau färben.}\end{entry}
\begin{entry}
\mainentry{aokubi}
{あおくび; アオクビ (青頸; 青首)}
{  \textit{Vogelk.} Stockente (f); Wildente (f) (Anas platyrhynchos; ⇒ ma·gamo 真鴨6286253).}\end{entry}
\begin{entry}
\mainentry{aoguma}
{青ぐま (青隈)}
{  \textit{Kabuki} blaue Theaterschminke (f).}\end{entry}
\begin{entry}
\mainentry{aokumo}
{青雲 [1a]}
{  \textit{Meteor.} blaue Wolke (m) // blauer Himmel (m).}\end{entry}
\begin{entry}
\mainentry{aogumo}
{青雲 [1b]}
{ Meteor.blaue Wolke (m) // blauer Himmel (m).}\end{entry}
\begin{entry}
\mainentry{aoguroi}
{青黒い (黝い)}
{ blauschwarz; dunkelblau.}\end{entry}
\begin{entry}
\mainentry{aogurosa}
{青黒さ}
{Blauschwarz (n); Dunkelblau (n).}\end{entry}
\begin{entry}
\mainentry{aoguromi}
{青黒み}
{Blauschwarz (n); Dunkelblau (n).}\end{entry}
\begin{entry}
\mainentry{aoge}
{青毛}
{ blauschwarze Farbe (f) (des Fells eines Pferdes).}\end{entry}
\begin{entry}
\mainentry{aogenouma}
{青毛の馬}
{blauschwarzes Pferd (n).}\end{entry}
\begin{entry}
\mainentry{aogera}
{アオゲラ; あおげら (青啄木鳥; 緑啄木鳥)}
{  \textit{Vogelk.} japanischer Specht (m) (Picus awokera).}\end{entry}
\begin{entry}
\mainentry{aogeru }
{あおげる (扇げる; 煽げる)}
{ fächern dürfen.}\end{entry}
\begin{entry}
\mainentry{aogeru }
{仰げる}
{ [1] hinaufblicken können. [2] respektieren können; verehren können // als Führer anerkennen können.}\end{entry}
\begin{entry}
\mainentry{aogo}
{青ご (青仔 [b])}
{ Fischbrut (f).}\end{entry}
\begin{entry}
\mainentry{aoko }
{アオコ; あおこ; 青粉}
{ [1] pulverisierter grüner Seetang (m). [2]  \textit{Bot.} Blaualge (f).}\end{entry}
\begin{entry}
\mainentry{aoko }
{青こ (青仔 [a])}
{ Fischbrut (f).}\end{entry}
\begin{entry}
\mainentry{aoko-na-}
{青コーナー}
{  \textit{Boxen} blaue Ecke (f); Ecke (f) des Herausvorderers.}\end{entry}
\begin{entry}
\mainentry{aogoori}
{青氷}
{ bläuliches Eis (n).}\end{entry}
\begin{entry}
\mainentry{aogoke}
{青ごけ; 青ゴケ (青苔)}
{  \textit{Bot.} grünes Moos (n); grüne Algenfpl.}\end{entry}
\begin{entry}
\mainentry{aogokenohaetaishi}
{青苔の生えた石}
{moosbewachsener Stein (m).}\end{entry}
\begin{entry}
\mainentry{aogokenohaetaishidesubetta。}
{青苔の生えた石で滑った。}
{ \textit{Bsp.} Ich bin auf einem moosbewachsenen Stein ausgerutscht.}\end{entry}
\begin{entry}
\mainentry{aokochi}
{青こち (青東風)}
{ im Frühsommer durch die frischen grünen Blätter wehender Ostwind (m).}\end{entry}
\begin{entry}
\mainentry{aokoma}
{青駒}
{ blauschwarzes Pferd (n).}\end{entry}
\begin{entry}
\mainentry{aogoromonoshounen}
{青衣の少年}
{  \textit{Werktitel} NAr (Gemälde von Gainsborough; um 1770).}\end{entry}
\begin{entry}
\mainentry{aosa }
{アオサ; あおさ (石蓴)}
{  \textit{Bot.} Blaualge (f).}\end{entry}
\begin{entry}
\mainentry{aosa }
{青さ}
{Grün (n); Blau (n).}\end{entry}
\begin{entry}
\mainentry{aozai}
{アオザイ}
{  \textit{Kleidung} Ao Dai (n) (vietnamesische Damenbekleidung mit langem geschlitztem Oberteil und Hose).}\end{entry}
\begin{entry}
\mainentry{aozakana}
{青魚 [2]}
{ Fisch (m) mit blauem Rücken wie Makrele, Sardine oder Makrelenhecht.}\end{entry}
\begin{entry}
\mainentry{aosagi}
{あおさぎ; アオサギ (青鷺)}
{  \textit{Vogelk.} Fischreiher (m) (Ardea cinerea).}\end{entry}
\begin{entry}
\mainentry{aozashi}
{青差 (青緡; 青繦)}
{ blaugefärbte Schnur (f), um (mit Loch versehene) Geldstücke aufzufädeln.}\end{entry}
\begin{entry}
\mainentry{aosabi}
{青さび (青錆; 青銹)}
{ Grünspan (m); Patina (f) (⇒ rokushō 緑青6346062).}\end{entry}
\begin{entry}
\mainentry{aosabigatsuku}
{青銹が付く}
{Grünspan bekommen; grünspanig werden.}\end{entry}
\begin{entry}
\mainentry{aosabinotsuita}
{青銹のついた}
{grünspanig; patiniert.}\end{entry}
\begin{entry}
\mainentry{aozamurai}
{青侍}
{ [1] junger, unerfahrener Samurai (m). [2] Samurai (m) von niedrigem Rang.}\end{entry}
\begin{entry}
\mainentry{aozame}
{あおざめ; アオザメ (青鮫)}
{  \textit{Fischk.} Blauhai (m) (Isurus oxyrynchus).}\end{entry}
\begin{entry}
\mainentry{aozamekiru}
{青ざめ切る; 青ざめきる}
{ ganz blass werden; vollkommen erblassen.}\end{entry}
\begin{entry}
\mainentry{aozameta}
{青ざめた}
{blass; weiß; käsig.}\end{entry}
\begin{entry}
\mainentry{aozameteiru}
{青ざめている}
{blass sein; blass aussehen.}\end{entry}
\begin{entry}
\mainentry{aozameru}
{青ざめる; あおざめる (青褪める; 蒼褪める; 蒼ざめる)}
{ blass werden; erblassen.}\end{entry}
\begin{entry}
\mainentry{aoji }
{あおじ; アオジ (蒿雀; 青鵐)}
{  \textit{Vogelk.} Maskenammer (f) (Emberiza spodocephala; Sommer).}\end{entry}
\begin{entry}
\mainentry{aoji }
{青磁 [1] (青瓷 [1])}
{ blaue Keramik (f); grüne Keramik (f).}\end{entry}
\begin{entry}
\mainentry{aoji }
{青地 [1]}
{ blauer Grund (m); blauer Stoff (m).}\end{entry}
\begin{entry}
\mainentry{aoshio}
{青潮}
{ blaue Meeresströmung (f) (durch Blaualgen verursachte blaue Verfärbung des Meeres).}\end{entry}
\begin{entry}
\mainentry{aoshigi}
{あおしぎ; アオシギ (青鷸)}
{  \textit{Vogelk.} Einsiedlerschnepfe (f) (Gallinago solitaria).}\end{entry}
\begin{entry}
\mainentry{aojiso}
{青じそ; 青ジソ; アオジソ; あおじそ (青紫蘇)}
{  \textit{Bot.} Ao·jiso (f) (eine Schwarznesselart; Perilla frutescens).}\end{entry}
\begin{entry}
\mainentry{aoshiba}
{青柴}
{ frisch geschnittenes, noch grünes Reisig (n).}\end{entry}
\begin{entry}
\mainentry{aoshima }
{青縞}
{ blau gefärbter Baumwollstoff (m).}\end{entry}
\begin{entry}
\mainentry{aoshima }
{青島 [1]}
{  \textit{Inseln.} AoshimanNAr (zur Präf. Miyazaki gehörende Insel; berühmt für eine „Teufelswaschbrett“ genannte erodierte Strandterasse).}\end{entry}
\begin{entry}
\mainentry{aojashin}
{青写真}
{ [1] Blaupause (f); Zyanotypie (f); Lichtpause (f). [2] (übertr.) Zukunftskonzeption (f).}\end{entry}
\begin{entry}
\mainentry{aojashinzumen}
{青写真図面}
{Blaupause (f); lichtgepauster Plan (m).}\end{entry}
\begin{entry}
\mainentry{aojashin'nitoru}
{青写真にとる}
{[1] eine Blaupause anfertigen. [2] einen Plan machen; planen.}\end{entry}
\begin{entry}
\mainentry{aojashinwoegaku}
{青写真を描く}
{für die Zukunft planen.}\end{entry}
\begin{entry}
\mainentry{aojushou}
{青綬章}
{Orden (m) an blauer Schärpe.}\end{entry}
\begin{entry}
\mainentry{aoshosei}
{青書生}
{unreifer Student (m).}\end{entry}
\begin{entry}
\mainentry{aoshiru}
{青汁}
{ [1] grüner Gemüsesaft (m). [2]  \textit{Kochk.} Ao·shiru (n) (Dashi aus Spinat und weißem Miso).}\end{entry}
\begin{entry}
\mainentry{aojiroi}
{青白い; あおじろい (蒼白い)}
{ blass; fahl; aschgrau; kränklich.}\end{entry}
\begin{entry}
\mainentry{aojiroiinteri}
{青白いインテリ}
{blasser Intellektueller (m) (dem es an Tatkraft fehlt).}\end{entry}
\begin{entry}
\mainentry{aojiroikaono}
{青白い顔の}
{bleichgesichtig; mit bleichem Gesicht.}\end{entry}
\begin{entry}
\mainentry{aojiroikaowoshitaon'na}
{青白い顔をした女}
{blassgesichtige Frau (f).}\end{entry}
\begin{entry}
\mainentry{aojiroitsukinohikari}
{青白い月の光}
{blasses Mondlicht (n).}\end{entry}
\begin{entry}
\mainentry{aojirokiinteri}
{青白きインテリ}
{blasser Intellektueller (m) (dem es an Tatkraft fehlt).}\end{entry}
\begin{entry}
\mainentry{aojirosa}
{青白さ (蒼白さ)}
{Blässe (f); Bleicheit (f).}\end{entry}
\begin{entry}
\mainentry{aojiromu}
{青白む}
{ blass werden; erbleichen.}\end{entry}
\begin{entry}
\mainentry{aoshingou}
{青信号}
{ [1Gb] grüne Ampel (f); Grün (n). [2] grünes Licht (n); freie Bahn (f) (für ein Projekt o. Ä.). (⇔ aka·shingō 赤信号1146623).}\end{entry}
\begin{entry}
\mainentry{aoshingougatsuita。}
{青信号がついた。}
{ \textit{Bsp.} Die Ampel ist grün.}\end{entry}
\begin{entry}
\mainentry{aoshingoudemichiwowataru}
{青信号で道を渡る}
{bei Grün über die Straße gehen.}\end{entry}
\begin{entry}
\mainentry{aosu}
{青酢}
{  \textit{Kochk.} Spinatessig (m) (die abgeseihte Flüssigkeit aus gekochtem Spinat mit Essig u. a. gewürzt).}\end{entry}
\begin{entry}
\mainentry{aozu}
{青図}
{ schwarze und weiße Bändernpl (an Trauergeschenken).}\end{entry}
\begin{entry}
\mainentry{aosuji}
{青筋}
{ [1] blaue Ader (f); blaue Vene (f). [2] Zornader (f).}\end{entry}
\begin{entry}
\mainentry{aosujiageha}
{アオスジアゲハ; あおすじあげは; 青条揚げ羽; 青条揚羽 (青筋鳳蝶)}
{  \textit{Insektenk.} Aosuji·ageha (Graphium sarpedon).}\end{entry}
\begin{entry}
\mainentry{aosujigatatteiru}
{青筋が立っている}
{die Adern stehen hervor.}\end{entry}
\begin{entry}
\mainentry{aosujinomierute}
{青筋の見える手}
{adrige Hand (f).}\end{entry}
\begin{entry}
\mainentry{aosujiwotatetearasou}
{青筋を立てて争う}
{sich erbittert streiten; sich so streiten, dass die Zornesadern hervortreten.}\end{entry}
\begin{entry}
\mainentry{aosujiwotateteokoru}
{青筋を立てて怒る; 青筋をたてて怒る}
{sich grün und blau ärgern; vor Ärger treten einem die Adern hervor.}\end{entry}
\begin{entry}
\mainentry{aosujiwotateru}
{青筋を立てる}
{es schwellen einem die Zornesadern.}\end{entry}
\begin{entry}
\mainentry{aosudare}
{青すだれ (青簾)}
{ [1] Jalousie (f) aus grünem Bambus (Sommer). [2] Vorhang (m) aus grünen Fäden (für den Ochsenkarren).}\end{entry}
\begin{entry}
\mainentry{aozumi}
{青墨}
{ [1] Indigofarbe (f) in der Form eines Tuschsteines. [2] bläuliche Tusche (f).}\end{entry}
\begin{entry}
\mainentry{aozumu}
{青ずむ (蒼ずむ)}
{ leicht blau werden.}\end{entry}
\begin{entry}
\mainentry{aosekimen}
{青石綿}
{blauer Asbest (m); Krokidolit (n).}\end{entry}
\begin{entry}
\mainentry{aosen }
{青線}
{ [1] blaue Linie (f). [2] (obsol.) Vergnügungsviertel (n) zweiten Grades; Gebiet (n) mit unlizenzierten Bordellen im Anschluss an die lizenzierten Viertel. (Abk.).}\end{entry}
\begin{entry}
\mainentry{aosen }
{青銭}
{ (altertüml.) Bronzemünze (f).}\end{entry}
\begin{entry}
\mainentry{aosenkuiki}
{青線区域}
{(obsol.) Vergnügungsviertel (m) zweiten Grades; Gebiet (n) mit unlizenzierten Bordellen im Anschluss an die lizenzierten Viertel.}\end{entry}
\begin{entry}
\mainentry{aosenchitai}
{青線地帯}
{(obsol.) Vergnügungsviertel (n) zweiten Grades; Gebiet (n) mit unlizenzierten Bordellen im Anschluss an die lizenzierten Viertel.}\end{entry}
\begin{entry}
\mainentry{aoso}
{あおそ (青麻; 青苧)}
{ Ramie (f); Chinagras (n) (eine Bastfaser).}\end{entry}
\begin{entry}
\mainentry{aosokohi}
{青そこひ (青底翳)}
{  \textit{Med.} Glaukom (n); Glaucoma (f); grüner Star (m) (durch erhöhten Innendruck des Auges verursachte Augenkrankheit; ⇒ ryoku·naishō 緑内障5629329).}\end{entry}
\begin{entry}
\mainentry{aozora}
{青空}
{ blauer Himmel (m); Himmelsbläue (f).}\end{entry}
\begin{entry}
\mainentry{aozoraichi}
{青空市}
{Freiluftmarkt (m).}\end{entry}
\begin{entry}
\mainentry{aozoraichiba}
{青空市場}
{Freiluftmarkt (m); Markt (m) im unter freiem Himmel.}\end{entry}
\begin{entry}
\mainentry{aozoragamietekita。}
{青空が見えてきた。}
{ \textit{Bsp.} Der blaue Himmel zeigt sich.}\end{entry}
\begin{entry}
\mainentry{aozorakyoushitsu}
{青空教室}
{Unterricht (m) im Freien.}\end{entry}
\begin{entry}
\mainentry{aozorataikai}
{青空大会}
{Versammlung (f) unter freiem Himmel; Freiluftversammlung (f).}\end{entry}
\begin{entry}
\mainentry{aozorachuusha}
{青空駐車}
{Laternengarage (f); Parken (n) unter freiem Himmel.}\end{entry}
\begin{entry}
\mainentry{aozoranomotode}
{青空のもとで}
{unter freiem Himmel; im Freien.}\end{entry}
\begin{entry}
\mainentry{aota}
{青田}
{ grünes Reisfeld (n); unreifes Reisfeld (n).}\end{entry}
\begin{entry}
\mainentry{aodaishou}
{あおだいしょう; アオダイショウ; 青大将}
{  \textit{Zool.} Ao·daishō (f) (große blaugrüne Schlangenart; Elaphe climacophora).}\end{entry}
\begin{entry}
\mainentry{aotauri}
{青田売り; 青田売}
{Verkauf (m) von Reis bevor dieser geerntet ist.}\end{entry}
\begin{entry}
\mainentry{aotaka}
{あおたか; アオタカ (蒼鷹)}
{  \textit{Vogelk.} Hühnerhabicht (m).}\end{entry}
\begin{entry}
\mainentry{aotagai}
{青田買い; 青田買}
{[1] Reiskauf (m) bevor der Reis reif ist. [2] Anstellungsvertragsabschluss (m) mit Studenten bevor diese ihren Abschluss gemacht haben.}\end{entry}
\begin{entry}
\mainentry{aotagaiwosuru}
{青田買いをする}
{[1] Reis kaufen, bevor dieser reif ist. [2] mit Studenten einen Anstellungsvertrag abschließen, bevor diese ihren Abschluss gemacht haben.}\end{entry}
\begin{entry}
\mainentry{aotagari}
{青田刈り; 青田刈}
{[1] Reiskauf (m) bevor der Reis reif ist. [2] Anstellungsvertragsabschluss mit Studenten bevor diese ihren Abschluss gemacht haben.}\end{entry}
\begin{entry}
\mainentry{aotagariwosuru}
{青田刈りをする}
{[1] Reis kaufen, bevor dieser reif ist. [2] mit Studenten einen Anstellungsvertrag abschließen, bevor diese ihren Abschluss gemacht haben.}\end{entry}
\begin{entry}
\mainentry{aodake}
{青竹}
{ [1]  \textit{Bot.} grüner Bambus (n). [2] grüne Farbe (f) (zum Färben). [3] Flöte (f).}\end{entry}
\begin{entry}
\mainentry{aodakeiro}
{青竹色}
{Blaugrün (n); Malachitgrün (n).}\end{entry}
\begin{entry}
\mainentry{aodakeirono}
{青竹色の}
{bläulich grün.}\end{entry}
\begin{entry}
\mainentry{aodakewosupattowaru}
{青竹をすぱっと割る}
{grünen Bambus frisch spalten.}\end{entry}
\begin{entry}
\mainentry{aodatami}
{青畳}
{ frische Tatami (f); neue Tatami (f); nagelneue Tatami (f).}\end{entry}
\begin{entry}
\mainentry{aodachi}
{青立ち; 青立; 青だち}
{ Unreife (f) // etw. Reifendes (n) (bes. Reis); zarte Blüte (f).}\end{entry}
\begin{entry}
\mainentry{aotabaibai}
{青田売買}
{Handel (m) mit noch ungeerntetem Reis.}\end{entry}
\begin{entry}
\mainentry{aodamo}
{あおだも; アオダモ}
{  \textit{Bot.} Esche (f) (Fraxinus lanuginosa).}\end{entry}
\begin{entry}
\mainentry{aotan}
{青短; 青丹 [1]}
{ Aotan (n) (japan. Kartenspiel).}\end{entry}
\begin{entry}
\mainentry{aochi}
{青地 [2]}
{  \textit{Familienn.} AochiNAr.}\end{entry}
\begin{entry}
\mainentry{aochibinbou}
{あおち貧乏 (煽ち貧乏; 煽貧乏)}
{ Armut (f), der man nicht entkommen kann.}\end{entry}
\begin{entry}
\mainentry{aocha}
{青茶}
{ [1] bläuliches Braun (n). [2]  \textit{Tee} Aocha (m) (einfacher japan. Tee).}\end{entry}
\begin{entry}
\mainentry{aochirinsou}
{青地林宗}
{  \textit{Persönlichk.} Aochi Rinsō (Hollandwissenschaftler; 1775–1833).}\end{entry}
\begin{entry}
\mainentry{aokkiri}
{青っ切り; 青切り; 青っ切; 青切 (呷っ切り; 呷切り; 呷っ切; 呷切)}
{ [1] blaue Linie (f) um die Öffnung einer Teetasse // Teetasse (f) mit blauer Linie um die Öffnung. [2] Trinken aus einer Teetasse mit blauer Linie um die Öffnung. [3] Einschenken in eine Teetasse mit blauer Linie um die Öffnung.}\end{entry}
\begin{entry}
\mainentry{aoduke}
{青漬け; 青漬}
{  \textit{Kochk.} eingelegtes grünes Gemüse (n).}\end{entry}
\begin{entry}
\mainentry{aotsudurafuji}
{あおつづらふじ; アオツヅラフジ (青葛藤)}
{  \textit{Bot.} Dreilappiger-Kokkelstrauch (m) (Cocculus trilobus).}\end{entry}
\begin{entry}
\mainentry{aoppana}
{青っぱな (青っ洟; 青っ鼻汁)}
{  \textit{Med.} grünlicher Nasenschleim (m) (⇒ ao·bana 青洟2093734).}\end{entry}
\begin{entry}
\mainentry{aoppoi}
{青っぽい}
{ [1] bläulich. [2] unerfahren; grün; unreif.}\end{entry}
\begin{entry}
\mainentry{aoteru}
{青照}
{  \textit{Theat.} blaugrüne Beleuchtung (f) (z. B. wenn ein Geist erscheint).}\end{entry}
\begin{entry}
\mainentry{aodensha}
{青電車}
{ der vorletzte Zug (m) (am Abend).}\end{entry}
\begin{entry}
\mainentry{aotenjou}
{青天井}
{ [1] blaues Firmament (n); blauer Himmel (m). [2] große Höhe (f) (von Preisen etc.).}\end{entry}
\begin{entry}
\mainentry{aotenjouyosan}
{青天井予算}
{grenzenloses Budget (n).}\end{entry}
\begin{entry}
\mainentry{aodenwa}
{青電話}
{ öffentliches Telefon (n) (in grüner Farbe).}\end{entry}
\begin{entry}
\mainentry{aoto}
{青と (青砥)}
{ feiner Wetzstein (m).}\end{entry}
\begin{entry}
\mainentry{aodoushin}
{青道心}
{ [1] jmd.NArN, der gerade erst Priester geworden ist, und die buddh. Lehre noch nicht richtig kennt. [2] buddhistische Frömmigkeit (f), die aus einem eigenen Gefühl heraus entstanden ist.}\end{entry}
\begin{entry}
\mainentry{aotokage}
{青とかげ; 青トカゲ (青蜥蝪; 青石竜子)}
{  \textit{Zool.} junge Echse (f).}\end{entry}
\begin{entry}
\mainentry{aodosa}
{青土佐}
{ Tosa-Papier (n) (in Tosa hergestelltes dickes blaues Japanpapier; z. B. zum Bespannen von Wänden oder Kisten verwendet).}\end{entry}
\begin{entry}
\mainentry{aotodo}
{アオトド (青椴)}
{  \textit{Bot.} Dreilappiger-Kokkelstrauch (m) (Cocculus trilobus).}\end{entry}
\begin{entry}
\mainentry{aona}
{青菜}
{ Grünkraut (n); Küchenkraut (n); Raps (m); Rübe (f).}\end{entry}
\begin{entry}
\mainentry{aonairu}
{青ナイル}
{ \textit{Flussn.} blauer Nil (m) (einer der beiden Quellflüsse des Nil).}\end{entry}
\begin{entry}
\mainentry{aonairugawa}
{青ナイル川}
{  \textit{Flussn.} Blauer Nil (m); () Bahr el-Asrak (m) (einer der Quellflüsse des Nil).}\end{entry}
\begin{entry}
\mainentry{aonagadaikon}
{青長大根}
{  \textit{Bot.} Dreilappiger-Kokkelstrauch (m) (Cocculus trilobus; ao·tsuzurafuji 青葛藤9188936).}\end{entry}
\begin{entry}
\mainentry{aonashi}
{青梨; アオナシ; あおなし}
{  \textit{Bot.} japanische grüne Birne (f) (Pyrus ussuriensis var. hondoensis).}\end{entry}
\begin{entry}
\mainentry{aonanishio}
{青菜に塩}
{Niedergeschlagenheit (f) (Zustand wie Grünzeug, das welkt, weil es mit Salz bestreut wurde).}\end{entry}
\begin{entry}
\mainentry{aonanishiodearu}
{青菜に塩である}
{niedergeschlagen sein; kleinlaut sein.}\end{entry}
\begin{entry}
\mainentry{aonanishiowokaketayou}
{青菜に塩をかけたよう}
{kleinlaut; niedergeschlagen; wie ein begossener Pudel.}\end{entry}
\begin{entry}
\mainentry{aonaniyuwokaketayou。}
{青菜に湯をかけたよう。}
{ \textit{Bsp.} Er scheint zusammenzubrechen.}\end{entry}
\begin{entry}
\mainentry{aoni }
{青煮}
{  \textit{Kochk.} Aoni (n) (Kochmethode, bei der Gemüse seine grüne Farbe behält bzw. nach dieser Methode gekochtes Gemüse).}\end{entry}
\begin{entry}
\mainentry{aoni }
{青丹 [2]}
{ [1] grüne Erde (f). [2] grünes Pigment (n).}\end{entry}
\begin{entry}
\mainentry{aonisai}
{青二才 (青二歳)}
{ Grünschnabel (m); grüner Junge (m); Milchbart (m); junges Gemüse (n).}\end{entry}
\begin{entry}
\mainentry{aonisaijidai}
{青二才時代}
{unbedarfte Jugendzeit (f).}\end{entry}
\begin{entry}
\mainentry{aonisaino}
{青二才の}
{grün; unerfahren.}\end{entry}
\begin{entry}
\mainentry{aonisainokuseninanda。}
{青二才のくせになんだ。}
{ \textit{Bsp.} Du Grünschnabel! // Du bist ja noch grün hinter den Ohren!}\end{entry}
\begin{entry}
\mainentry{aonibi}
{青鈍}
{ [1] bläuliches Dunkelgrau (n). [2] Farbe (f) der Untergewänder in der Heian-Zeit.}\end{entry}
\begin{entry}
\mainentry{aonyuudou}
{青入道}
{ frisch geschorener Kopf (m) // jmd., dessen Kopf gerade vollkommen geschoren wurde.}\end{entry}
\begin{entry}
\mainentry{aonyoubou}
{青女房}
{ [1] Ao·nyōbō (m) (ein niedriger Hofdamenrang). [2] Ao·nyōbō (f) (junge Hofdame niederen Ranges). [3] Ehefrau (f) eines niedrigrangigen Samurai.}\end{entry}
\begin{entry}
\mainentry{aonekoza}
{青猫座}
{  \textit{Verlagsn.} Aonekoza (Matsukawa-mura; Nagano-ken; eigentlich der Name einer Schauspielertruppe).}\end{entry}
\begin{entry}
\mainentry{aono}
{青野}
{  \textit{Familienn.} Aono.}\end{entry}
\begin{entry}
\mainentry{aonoku }
{あおのく (仰く)}
{ aufsehen; hochblicken; emporblicken; den Kopf in den Nacken legen; nach oben drehen (⇒ aomuku 仰向く4910412).}\end{entry}
\begin{entry}
\mainentry{aonoku }
{仰のく}
{ emporblicken; aufblicken.}\end{entry}
\begin{entry}
\mainentry{aonoke}
{仰のけ; 仰け}
{ Nachobendrehen (n) des Gesichtes (⇒ ao·muke 仰向け7563405).}\end{entry}
\begin{entry}
\mainentry{aonokezama}
{仰のけ様; 仰様; 仰のけざま}
{ Auf-dem-Rücken-Liegen (n).}\end{entry}
\begin{entry}
\mainentry{aonokezamada}
{仰のけ様だ; 仰様だ}
{auf dem Rücken liegen.}\end{entry}
\begin{entry}
\mainentry{aonokeru}
{あおのける (仰ける; 仰のける)}
{ nach oben drehen (⇒ ao·mukeru 仰向ける0218151).}\end{entry}
\begin{entry}
\mainentry{aonotsugazakura}
{アオノツガザクラ (青の栂桜; 青ノ栂桜; 青栂桜)}
{  \textit{Bot.} Aleuten-Blauheide (f) (Phyllodoce aleutica).}\end{entry}
\begin{entry}
\mainentry{aonori}
{あおのり; アオノリ; 青のり; 青ノリ (青海苔)}
{  \textit{Bot.} grüner Seetang (m).}\end{entry}
\begin{entry}
\mainentry{aoba}
{あお歯 (白歯)}
{ ungeschwärzte Zähnempl; weiße Zähnempl (im alten Japan, als es für verheiratete Frauen üblich war, sich die Zähne zu schwärzen).}\end{entry}
\begin{entry}
\mainentry{aoba }
{あおば}
{  \textit{Eisenb.} Aoba-Shinkansen (m) (hält an allen Stationen der Tōhoku-Linie).}\end{entry}
\begin{entry}
\mainentry{aoba }
{青羽 (青翅)}
{ blaue Feder (f).}\end{entry}
\begin{entry}
\mainentry{aoba }
{青葉}
{ grünes Laub (n); grünes Blatt (n).}\end{entry}
\begin{entry}
\mainentry{aohaaiyoriideteaiyoriaoshi。}
{青は藍よりいでて藍より青し。}
{ \textit{Bsp.} Der Schüler übertrifft den Lehrer.}\end{entry}
\begin{entry}
\mainentry{aobaarigatahanekakushi}
{アオバアリガタハネカクシ; あおばありがたはねかくし (青翅蟻形隠翅虫; 青翅蟻形羽隠虫)}
{  \textit{Insektenk.} Ufer-Kurzflügler (Paederus fuscipes).}\end{entry}
\begin{entry}
\mainentry{aobae}
{青ばえ; あおばえ; アオバエ (青蝿 [a]; 青蠅 [a]; 蒼蝿; 蒼蠅)}
{  \textit{Insektenk.} (ugs.) Schmeißfliege (f).}\end{entry}
\begin{entry}
\mainentry{aobagaderu}
{青葉が出る}
{grüne Blätter bekommen.}\end{entry}
\begin{entry}
\mainentry{aobakama}
{あおばかま (襖袴)}
{  \textit{Kleidung} Unterhakama (m) für die Jagd.}\end{entry}
\begin{entry}
\mainentry{aohagi}
{青萩}
{  \textit{Bot.} blauer Süßklee (m).}\end{entry}
\begin{entry}
\mainentry{aobashigereruyamayama}
{青葉茂れる山々}
{mit üppigem Grün überwachsene Bergempl.}\end{entry}
\begin{entry}
\mainentry{aobashira}
{青柱}
{  \textit{Sumō} blauer Pfeiler (m) (in der Nordost-Ecke des Sumō-Ringes; heute durch eine blaue Quaste am Dach über dem Sumō-Ring repräsentiert).}\end{entry}
\begin{entry}
\mainentry{aobasu}
{青バス}
{vorletzter Bus (m).}\end{entry}
\begin{entry}
\mainentry{aobazuku}
{あおばずく; アオバズク (緑葉梟; 青葉梟; 青葉木兎; 青葉木菟; 緑葉木兎)}
{  \textit{Vogelk.} japanische Habichtseule (f) (Ninox scutulata japonica).}\end{entry}
\begin{entry}
\mainentry{aobaseseri}
{アオバセセリ; あおばせせり (青翅挵り; 青翅挵; 青翅せせり)}
{  \textit{Insektenk.} Aoba·seseri (Choaspes benjaminii; engl. Indian awlking).}\end{entry}
\begin{entry}
\mainentry{aohata}
{青旗}
{ blaue Fahne (f).}\end{entry}
\begin{entry}
\mainentry{aohada}
{青肌 (青膚)}
{ [1] bläuliche Haut (f) nach dem Rasieren. [2]  \textit{Bot.} Ao·hada (Ilex macropoda).}\end{entry}
\begin{entry}
\mainentry{aobato}
{アオバト; あおばと (青鳩; 緑鳩)}
{  \textit{Vogelk.} Aobata (f) (Sphenurus sieboldii; ⇒ yama·bato 山鳩7591816).}\end{entry}
\begin{entry}
\mainentry{aobana }
{青ばな (青洟; 青鼻汁)}
{  \textit{Med.} grünlicher Nasenschleim (m).}\end{entry}
\begin{entry}
\mainentry{aobana }
{青花}
{  \textit{Bot.} Commeline (f); Tagblume (f) (Commelina communis var. communis; ⇒ tsuyu·kusa 露草3970327).}\end{entry}
\begin{entry}
\mainentry{aobanagami}
{青花紙}
{ mit Commeline blau gefärbtes Papier (n).}\end{entry}
\begin{entry}
\mainentry{aobaninaru}
{青葉になる}
{belaubt sein.}\end{entry}
\begin{entry}
\mainentry{aobanokoro }
{青葉のころ}
{Zeit (f) des frischen Grüns.}\end{entry}
\begin{entry}
\mainentry{aobanokoro }
{青葉の頃}
{Jahreszeit (f) des frischen Grünes.}\end{entry}
\begin{entry}
\mainentry{aobamu}
{青ばむ}
{ ins Blaue gehen (eine Farbe).}\end{entry}
\begin{entry}
\mainentry{aobi}
{青火}
{ blassblauer Feuerball (m).}\end{entry}
\begin{entry}
\mainentry{aobikari}
{青光り; 青光}
{  \textit{Naturphän.} blaugrünes Leuchten (n) // Phosphoreszenz (f).}\end{entry}
\begin{entry}
\mainentry{aobikarigasuru}
{青光りがする}
{blaugrün leuchten; grünes Licht ausstrahlen.}\end{entry}
\begin{entry}
\mainentry{aobikarishiteiru}
{青光りしている}
{blaugrün leuchten.}\end{entry}
\begin{entry}
\mainentry{aobikarisuru}
{青光する; 青光りする}
{blaugrün leuchten.}\end{entry}
\begin{entry}
\mainentry{aobikaru}
{青光る}
{ blaugrün leuchten.}\end{entry}
\begin{entry}
\mainentry{aohige }
{青ひげ (青髯 [1]; 青髭 [1]; 青鬚)}
{ [1] blauer Bart (m); blau gefärbter Bart (m). [2]  \textit{Theat.} blaue Schminke (f). [3] rasierter, durch die Haut scheinender starker Bartwuchs (m); blauer Schimmer (m) der Bartstoppeln.}\end{entry}
\begin{entry}
\mainentry{aohige }
{青ひげ (青髯 [2]; 青髭 [2])}
{  \textit{Werktitel}  (Märchen von Charles Perrault).}\end{entry}
\begin{entry}
\mainentry{aohitogusa}
{青人草}
{ Volk (n); Bevölkerung (f); Bürger (m); Staatsvolk (n).}\end{entry}
\begin{entry}
\mainentry{aobiyu}
{アオビユ}
{  \textit{Bot.} Dreilappiger-Kokkelstrauch (m) (Cocculus trilobus).}\end{entry}
\begin{entry}
\mainentry{aohyou}
{青票 [a]}
{ blauer Stimmzettel (m) (als Gegenstimme; → seihyō 青票2505543).}\end{entry}
\begin{entry}
\mainentry{aobyoushi}
{青表紙}
{ [1] blauer Einband (m); blauer Buchdeckel (m). [2] Buch (n) mit blauem Umschlag; Buch (n) mit grünem Umschlag. [a] konfuzianischer Klassiker (m) // (übertr.) konfuzianischer Gelehrter (m). [b] Textrevision (f) des Genji-Monogatari von Fujiwara no Teika. [c] Übungsbuch (n) für Jōruri. [d] (populä}\end{entry}
\begin{entry}
\mainentry{aobyoutan}
{青びょうたん; あおびょうたん (青瓢箪)}
{ [1] wandelnde Leiche (f); eine lebende Leiche (f); leichenblasses Gesicht (n). [2] grüner Kürbis (m); Kalebasse (f).}\end{entry}
\begin{entry}
\mainentry{aofuku}
{青服}
{ [1] Blaumann (m); blaue Arbeitskleidung (f). [2] Arbeiter (m).}\end{entry}
\begin{entry}
\mainentry{aofukube}
{青ふくべ (青瓢)}
{  \textit{Bot.} grüner Flaschenkürbis (m); Kalebasse (f) (Herbst).}\end{entry}
\begin{entry}
\mainentry{aobukure}
{青膨れ; 青膨; 青ぶくれ (青脹れ; 青脹)}
{ blaugrüne Schwellung (f).}\end{entry}
\begin{entry}
\mainentry{aobukureshiteiru}
{青膨れしている (青脹れしている)}
{ein blau geschwollenes Gesicht haben.}\end{entry}
\begin{entry}
\mainentry{aobukuresuru}
{青膨れする; 青膨する; 青ぶくれする (青脹れする; 青脹する)}
{blau werden und anschwellen.}\end{entry}
\begin{entry}
\mainentry{aobukureda}
{青膨だ; 青膨れだ; 青ぶくれだ (青脹だ; 青脹れだ)}
{blau und geschwollen sein.}\end{entry}
\begin{entry}
\mainentry{aobukurenifukuretakao}
{青ぶくれにふくれた顔}
{bleiches geschwollenes Gesicht (n).}\end{entry}
\begin{entry}
\mainentry{aobukureno}
{青ぶくれの}
{bläulich geschwollen.}\end{entry}
\begin{entry}
\mainentry{aobusa}
{青房}
{  \textit{Sumō} blaue Quaste (f) an der Nordost-Ecke des Daches über dem Sumō-Ring (repräsentiert den entsprechenden blauen Pfeiler; die Pfeiler wurden abgeschafft, um dem Publikum bessere Sicht zu bieten).}\end{entry}
\begin{entry}
\mainentry{aofuda}
{青札}
{ [1] blaue Karte (f). [2] Tenshō-Spielkarten (f) (von den Portugiesen in Japan eingeführt und hier adaptiert).}\end{entry}
\begin{entry}
\mainentry{aofudou}
{青不動}
{  \textit{Buddh.} blaue Statue (f) des Feuergottes Acala.}\end{entry}
\begin{entry}
\mainentry{aobudou}
{青ブドウ; 青ぶどう (青葡萄)}
{blauer Wein (m) // blaue Weintraube (f).}\end{entry}
\begin{entry}
\mainentry{aohedo}
{青反吐}
{ grünes Erbrochenes (n).}\end{entry}
\begin{entry}
\mainentry{aobou}
{青帽}
{ [1] blauer Hut (m). [2] Träger (m); Dienstmann (m).}\end{entry}
\begin{entry}
\mainentry{aobouzu}
{青坊主}
{ [1] frisch rasierter Kopf (m). [2] Person (f) mit frisch rasiertem Kopf.}\end{entry}
\begin{entry}
\mainentry{aoboshi}
{青星}
{  \textit{Astron.} Sirius (m); Hundsstern (m) (der hellste Stern am Himmel im Sternbild Großer Hund).}\end{entry}
\begin{entry}
\mainentry{aohon}
{青本}
{  \textit{Literaturw.} [1] Ao·hon (n) (Genre illustrierter Geschichtenbücher mit Kabuki‑, Jōruri‑ und Kriegsgeschichten). [2] illustriertes Geschichtenbuch (n) (in der Edo-Zeit).}\end{entry}
\begin{entry}
\mainentry{aomai}
{青米}
{ Naturreis (m) mit blauem Schimmer.}\end{entry}
\begin{entry}
\mainentry{aomatsuba}
{青松葉}
{ grüne Kiefernadel (f).}\end{entry}
\begin{entry}
\mainentry{aomame}
{青豆; アオマメ; あおまめ}
{ grüne Bohnenfpl.}\end{entry}
\begin{entry}
\mainentry{aomameiro}
{青豆色}
{das Grün wie von Bohnen.}\end{entry}
\begin{entry}
\mainentry{aomi }
{青身}
{ blauer Teil (m) eines Fisches.}\end{entry}
\begin{entry}
\mainentry{aomi }
{青味; 青み; あおみ}
{ Bläue (f); grüne Färbung (f); blauer Schimmer (m); Anflug (m) von Blau.}\end{entry}
\begin{entry}
\mainentry{aomigakatta}
{青味がかった}
{bläulich; grünlich; mit einem Stich ins Blaue.}\end{entry}
\begin{entry}
\mainentry{aomigakattausumidori}
{青味がかった薄緑}
{bläuliches Hellgrün (n).}\end{entry}
\begin{entry}
\mainentry{aomigakattanezumiironoito}
{青味がかった鼠色の糸}
{bläulich grauer Faden (m).}\end{entry}
\begin{entry}
\mainentry{aomigakaru}
{青味がかる; 青みがかる}
{bläulich sein.}\end{entry}
\begin{entry}
\mainentry{aomizuhiki}
{青水引}
{ blau-weiße Knoten (m) (als Verzierung bei Trauerfeiern).}\end{entry}
\begin{entry}
\mainentry{aomidori}
{青緑}
{ [1] Blaugrün (n); Grünblau (n); Türquis (n). [2]  \textit{Bot.} (alter Name für) Teichgrün (n); Entengrütze (f) (Spyrogyra).}\end{entry}
\begin{entry}
\mainentry{aomidorino}
{青緑の}
{blaugrün; grünblau; türkis.}\end{entry}
\begin{entry}
\mainentry{aomidoro}
{あおみどろ; アオミドロ (水綿; 青味泥; 青緑)}
{  \textit{Bot.} Teichgrün (n); Entengrütze (f); Spyrogyra (f) (Schreibung 青緑 ist für diese Bed. veraltet).}\end{entry}
\begin{entry}
\mainentry{aomiwataru}
{青み渡る; 青みわたる (蒼み渡る; 蒼みわたる)}
{ ganz grün werden.}\end{entry}
\begin{entry}
\mainentry{aomiwoobiteiru }
{青みを帯びている}
{bläulich sein.}\end{entry}
\begin{entry}
\mainentry{aomiwoobiteiru }
{青味を帯びている}
{bläulich getönt sein.}\end{entry}
\begin{entry}
\mainentry{aomiwosoeru}
{青味を添える}
{einen grünen Touch geben.}\end{entry}
\begin{entry}
\mainentry{aomu}
{青む}
{ blau werden; grün werden // blass werden; bleich werden.}\end{entry}
\begin{entry}
\mainentry{aomuiteoyogu}
{仰向いて泳ぐ}
{rückenschwimmen.}\end{entry}
\begin{entry}
\mainentry{aomuiteneru}
{仰向いて寝る}
{auf dem Rücken liegend schlafen.}\end{entry}
\begin{entry}
\mainentry{aomuki}
{仰向き; 仰向; 仰むき; あお向}
{ Liegen (f) auf dem Rücken.}\end{entry}
\begin{entry}
\mainentry{aomukioyogi}
{仰向き泳ぎ}
{Rückenschwimmen (n).}\end{entry}
\begin{entry}
\mainentry{aomukinikorobu}
{仰向きに転ぶ}
{auf den Rücken fallen.}\end{entry}
\begin{entry}
\mainentry{aomukinineru}
{仰向きに寝る}
{auf dem Rücken liegend schlafen.}\end{entry}
\begin{entry}
\mainentry{aomukiho}
{仰向歩}
{  \textit{Turnen} Laufen (n) auf allen Vieren und auf dem Rücken.}\end{entry}
\begin{entry}
\mainentry{aomuku}
{仰向く; 仰むく; あお向く; あおむく}
{ aufsehen; hochblicken; emporblicken; den Kopf in den Nacken legen; nach oben drehen (⇔ utsumuku うつむく3090350).}\end{entry}
\begin{entry}
\mainentry{aomuke}
{あお向け; あおむけ (仰向け; 仰むけ)}
{ Rückenlage (f) (⇔ utsubuse うつぶせ1414068).}\end{entry}
\begin{entry}
\mainentry{aomukezama}
{仰向け様; 仰向様; 仰向けざま}
{ Rückenlage (f); Auf-dem-Rücken-Liegen (n).}\end{entry}
\begin{entry}
\mainentry{aomukezamada}
{仰向け様だ; 仰向けざまだ}
{auf dem Rücken liegen.}\end{entry}
\begin{entry}
\mainentry{aomukesamanitaosareru}
{仰むけ様に倒される}
{auf den Rücken geworfen werden.}\end{entry}
\begin{entry}
\mainentry{aomukesamanitaoreru}
{仰むけ様に倒れる}
{auf den Rücken fallen.}\end{entry}
\begin{entry}
\mainentry{aomukeni }
{仰むけに}
{auf dem Rücken liegend.}\end{entry}
\begin{entry}
\mainentry{aomukeni }
{仰向けに; あおむけに; あお向けに}
{rücklings; auf dem Rücken.}\end{entry}
\begin{entry}
\mainentry{aomukenioyogu}
{仰向けに泳ぐ}
{auf dem Rücken schwimmen.}\end{entry}
\begin{entry}
\mainentry{aomukenisuru }
{仰むけにする}
{auf den Rücken legen.}\end{entry}
\begin{entry}
\mainentry{aomukenisuru }
{仰向けにする}
{etw. auf den Rücken drehen.}\end{entry}
\begin{entry}
\mainentry{aomukenitaosareru}
{仰むけに倒される}
{auf den Rücken geworfen werden.}\end{entry}
\begin{entry}
\mainentry{aomukenitaoreta}
{仰向けに倒れた}
{auf den Rücken gefallen sein.}\end{entry}
\begin{entry}
\mainentry{aomukenitaoreru}
{仰むけに倒れる; あおむけに倒れる}
{auf den Rücken fallen.}\end{entry}
\begin{entry}
\mainentry{aomukeninatteoyogu}
{仰むけになって泳ぐ}
{auf dem Rücken schwimmen.}\end{entry}
\begin{entry}
\mainentry{aomukeninaru }
{仰むけになる}
{sich auf den Rücken legen.}\end{entry}
\begin{entry}
\mainentry{aomukeninaru }
{仰向けになる}
{sich auf den Rücken legen.}\end{entry}
\begin{entry}
\mainentry{aomukenineru }
{仰むけにねる; 仰向けに寝る}
{auf dem Rücken liegen.}\end{entry}
\begin{entry}
\mainentry{aomukenineru }
{仰むけに寝る}
{auf dem Rücken liegend schlafen.}\end{entry}
\begin{entry}
\mainentry{aomukeno}
{仰むけの}
{auf dem Rücken liegend.}\end{entry}
\begin{entry}
\mainentry{aomukeru}
{仰向ける; 仰むける; あお向ける; あおむける}
{ nach oben drehen; auf den Rücken legen.}\end{entry}
\begin{entry}
\mainentry{aomushi}
{青虫; あおむし; アオムシ}
{  \textit{Insektenk.} grüne Raupe (f).}\end{entry}
\begin{entry}
\mainentry{aomushikomayubachi}
{アオムシコマユバチ; あおむしこまゆばち (青虫小繭蜂)}
{  \textit{Insektenk.} Kohlweißlings-Raupenwespe (Apanteles glomeratus).}\end{entry}
\begin{entry}
\mainentry{aomurasaki}
{青紫 [1]}
{ [1] Blauviolett (n). [2] blaue und violette Schärpe (f). [3] Kleidungsfarbe (f) am Hof // Hofrang (m).}\end{entry}
\begin{entry}
\mainentry{aome }
{青芽}
{ frisch geschlagener entrindeter Weidenzweig (m).}\end{entry}
\begin{entry}
\mainentry{aome }
{青目; 青眼 [1]}
{ [1] blaue Augennpl. [2] (übertr.) Abendländer (m).}\end{entry}
\begin{entry}
\mainentry{aomono}
{青物}
{ [1] Grünzeug (n); Gemüse (n). [2]  \textit{Fischk.} Fisch (m) mit grüner Haut (wie Makrele od. Sardine).}\end{entry}
\begin{entry}
\mainentry{aomonoichi}
{青物市}
{Gemüsemarkt (m).}\end{entry}
\begin{entry}
\mainentry{aomonoichiba}
{青物市場}
{Gemüsemarkt (m).}\end{entry}
\begin{entry}
\mainentry{aomonoya}
{青物屋}
{Gemüsehändler (m).}\end{entry}
\begin{entry}
\mainentry{aomonowotsukuru}
{青物を作る}
{Gemüse anbauen.}\end{entry}
\begin{entry}
\mainentry{aomomiji}
{青紅葉}
{ [1] Ahorn (m), der sich noch nicht rot gefärbt hat. [2] Stoff (m), der auf der Vorderseite blau und auf der Rückseite orange ist.}\end{entry}
\begin{entry}
\mainentry{aomori}
{青森}
{  \textit{Stadtn.} AomorinNAr (Hptst. der Präf. Aomori in der Tōhoku-Region; 296.000 Ew.; Eisenbahnfähre und Seikan-Tunnel nach Hakodate auf Hokkaidō; Holzumschlag; Fischerei).}\end{entry}
\begin{entry}
\mainentry{aomoriken}
{青森県}
{ \textit{Gebietsn.} Präfektur (f) Aomori (im äußersten Norden der Tōhoku-Region).}\end{entry}
\begin{entry}
\mainentry{aomoritodomatsu}
{アオモリトドマツ; あおもりとどまつ (青森椴松)}
{  \textit{Bot.} Ō·shirabiso (f); Aomori-Weißtanne (f) (Abies mariesii).}\end{entry}
\begin{entry}
\mainentry{aoya}
{青谷}
{  \textit{Ortsn.} Aoya (Ortschaft in der Präf. Tottori).}\end{entry}
\begin{entry}
\mainentry{aoyaka}
{青やか}
{ leuchtend grün.}\end{entry}
\begin{entry}
\mainentry{aoyakada}
{青やかだ}
{leuchtend grün sein.}\end{entry}
\begin{entry}
\mainentry{aoyaki}
{青焼き; 青焼}
{ Blaupause (f); Zyanotypie (f); Lichtpause (f).}\end{entry}
\begin{entry}
\mainentry{aoyagi}
{青やぎ; あおやぎ; アオヤギ (青柳 [1])}
{ [1]  \textit{Bot.} grüne Weide (f). [2]  \textit{Muschelk.} Trogmuschel (f).}\end{entry}
\begin{entry}
\mainentry{aoyasai}
{青野菜}
{grünes Gemüse (n).}\end{entry}
\begin{entry}
\mainentry{aoyanagi}
{青柳 [2]}
{ [1]  \textit{Bot.} grüne Weide (f). [2] Farbe (f) einer grünen Weide.}\end{entry}
\begin{entry}
\mainentry{aoyama }
{青山 [1]}
{ [1] grün bewachsener Berg (m). [2]  \textit{Kochk.} Kamaboko (n) für Feierlichkeiten.}\end{entry}
\begin{entry}
\mainentry{aoyama }
{青山 [2]}
{  \textit{Ortsn.} Aoyama (Gebiet im Nordwesten des Hafens von Tōkyō).}\end{entry}
\begin{entry}
\mainentry{aoyamagakuindaigaku}
{青山学院大学}
{  \textit{Univ.-N.} Aoyama-Gakuin-Universität (christliche Privatuniv.; 1878 gegründet).}\end{entry}
\begin{entry}
\mainentry{aoyamashoin}
{青山書院}
{  \textit{Verlagsn.} Aoyama Shoin (Tōkyō).}\end{entry}
\begin{entry}
\mainentry{aoyu}
{アオユ; あおゆ (青柚)}
{ Frucht (f) der Yuzu; japanische Zitrone (f).}\end{entry}
\begin{entry}
\mainentry{aoyude}
{青茹で; 青茹}
{  \textit{Kochk.} [1] Ao·yude; kurzes Kochen (n) von Gemüse, bei dem die Farbe des Gemüse erhalten bleibt. [2] kurz gekochtes Gemüse (n).}\end{entry}
\begin{entry}
\mainentry{aorareru}
{煽られる}
{angefacht werden; im Wind flattern.}\end{entry}
\begin{entry}
\mainentry{aori }
{あおり [1] (障泥 [a]; 泥障)}
{ Schmutzfänger (m) an japanischen Sätteln (→ shōdei 障泥2132772).}\end{entry}
\begin{entry}
\mainentry{aori }
{あおり [2] (煽り)}
{ [1] Windstoß (m). [2] Schlag (m). [3] Mitleidenschaft (f). [4] dichtes Auffahren (n).}\end{entry}
\begin{entry}
\mainentry{aoriashi }
{あおり足 (煽り足 [1])}
{  \textit{Schwimmen} Scherenschlag (m).}\end{entry}
\begin{entry}
\mainentry{aoriashi }
{煽り足 [2]}
{ \textit{Schwimmen} Scherenschlag (m).}\end{entry}
\begin{entry}
\mainentry{aoriika}
{あおりいか; アオリイカ (障泥烏賊)}
{  \textit{Zool.} Aoriika (n) (eine Kalmarart; Sepioteuthis lessoniana).}\end{entry}
\begin{entry}
\mainentry{aoriita}
{あおり板 (障泥板; 泥障板)}
{ \textit{Archit.} Brett (n) am japanischen Vordach gegen das Eindringen von Regen.}\end{entry}
\begin{entry}
\mainentry{aorikoui}
{煽り行為}
{Anstiftung (f); Aufhetzung (f); Agitation (f).}\end{entry}
\begin{entry}
\mainentry{aorisugiru}
{あおりすぎる (煽り過ぎる; 煽りすぎる)}
{ [1] zu sehr fächeln. [2] zu sehr anfachen. [3] zu sehr antreiben.}\end{entry}
\begin{entry}
\mainentry{aorisuto}
{アオリスト}
{  \textit{Sprachw.} Aorist (m) (Zeitform, die eine momentane od. punktuelle Handlung ausdrückt).}\end{entry}
\begin{entry}
\mainentry{aoritateru}
{あおりたてる; あおり立てる (煽り立てる; 煽りたてる)}
{ [1] kräftig schwingen. [2] jmdn. antreiben.}\end{entry}
\begin{entry}
\mainentry{aoritsukeru }
{あおりつける [1]; あおり付ける [1] (煽り付ける [1]; 煽りつける [1])}
{ jmdn. anstiften; jmdn. antreiben.}\end{entry}
\begin{entry}
\mainentry{aoritsukeru }
{あおりつける [2]; あおり付ける [2] (煽り付ける [2]; 煽りつける [2]; 呷り付ける; 呷りつける)}
{ Alkohol in einem Zug herunterstürzen.}\end{entry}
\begin{entry}
\mainentry{aoritsudukeru}
{あおりつづける (煽り続ける; 煽りつづける)}
{ [1] etw. ständig schwanken lassen (der Wind). [2] weiter kräftig sein. [3] ständig zu etw. anstiften.}\end{entry}
\begin{entry}
\mainentry{aorido}
{あおり戸 (煽り戸; 煽戸)}
{ [1] vom Wind bewegte Tür (f). [2] nach außen oben öffnende Tür (f).}\end{entry}
\begin{entry}
\mainentry{aoridome}
{おり止め (煽り止め; 煽止め; 煽止)}
{ Türfeststellhaken (m); Türstopper (m).}\end{entry}
\begin{entry}
\mainentry{aorimado}
{あおり窓 (煽り窓; 煽窓)}
{ nach oben öffnendes Schiebefenster (n).}\end{entry}
\begin{entry}
\mainentry{aoriwokuu}
{あおりを食う (煽りを食う; あおりを喰う)}
{vom gleichen Übel erfasst werden; in einen Luftwirbel hineingezogen werden; einen Schlag einstecken.}\end{entry}
\begin{entry}
\mainentry{aoriwokuttetaoreta。}
{あおりを喰って倒れた。}
{ \textit{Bsp.} Er wurde vom Windstoß hingeworfen.}\end{entry}
\begin{entry}
\mainentry{aoru }
{あおる [1] (煽る [a])}
{ [1] anfachen. [2] antreiben. [3] aufreizen; aufhetzen. [4] dicht auffahren.}\end{entry}
\begin{entry}
\mainentry{aoru }
{あおる [2] (呷る; 呻る)}
{ (mit einem Schluck) herunterschlucken; in großen Schlucken trinken.}\end{entry}
\begin{entry}
\mainentry{aonsoku}
{亜音速}
{  \textit{Phys.} Unterschallgeschwindigkeit (f).}\end{entry}
\begin{entry}
\mainentry{aonsokuno}
{亜音速の}
{subsonisch; Unterschallgeschwindigkeits….}\end{entry}
\begin{entry}
\mainentry{aontai}
{亜温帯}
{ subgemäßigte Zone (f).}\end{entry}
\begin{entry}
\mainentry{aka }
{あか [1]; アカ (垢)}
{ [1] Schmutzablagerungenfpl in der Haut; Porenschmutz (m). [2] Ohrenschmalz (m). [3] Schmutz (m); Dreck (m); Abschaum (m); Bodensatz (m). [4] weltliche Angelegenheitenfpl. (⇒ kegare けがれ0611709).}\end{entry}
\begin{entry}
\mainentry{aka }
{あか [2] (銅 [a])}
{  \textit{Chem.} Kupfer (n) (⇒ aka·gane 銅5820799).}\end{entry}
\begin{entry}
\mainentry{aka }
{あか [3] (淦)}
{  \textit{Seef.} Bilgewasser (n); Schlagwasser (n); Grundsuppe (f).}\end{entry}
\begin{entry}
\mainentry{aka }
{あか [4] (閼伽)}
{ Ehrengabe (f) (für einen Ehrengast); Weihgabe (f) (vor einer Buddhafigur abgelegt) // Opferwasser (n) (an einem Grab oder vor einem Altar o. Ä.); Buddha dargereichtes Wasser (n) // Gefäß (n) für Opferwasser. (von sanskr. argha oder arghya).}\end{entry}
\begin{entry}
\mainentry{aka }
{亜科}
{  \textit{Biol.} Unterfamilie (f).}\end{entry}
\begin{entry}
\mainentry{aka }
{赤 (紅 [1]; 朱 [1])}
{ [A] (als N.) [1] Rot (n); rote Farbe (f) (eine der drei Grundfarben; Farbe des Blutes bzw. Farbton von rosa, orange, rötlich-braun bis braun; rot kann weiter Goldfarbe symbolisieren). [2] rote Ampel (f) (⇔ ao 青2391562). [3] (}\end{entry}
\begin{entry}
\mainentry{aka }
{緋 [1]}
{  \textit{Kanji} rot.}\end{entry}
\begin{entry}
\mainentry{akaaka }
{赤々; 赤赤; あかあか [1]}
{ Knallrot (n); Feuerrot (n).}\end{entry}
\begin{entry}
\mainentry{akaaka }
{明々 [1]; 明明 [1]; あかあか [2]}
{ Helligkeit (f); helles Strahlen (n) (⇒ kōkō 皓皓7353953).}\end{entry}
\begin{entry}
\mainentry{akaakato }
{赤々と; 赤赤と}
{knallrot; feuerrot.}\end{entry}
\begin{entry}
\mainentry{akaakato }
{明々と; 明明と}
{hell; brennend; lichterloh (⇔ kōkō こうこう2017091).}\end{entry}
\begin{entry}
\mainentry{akaakatohiniterasareteiru}
{明々と日に照らされている}
{von der Sonne voll beschienen werden.}\end{entry}
\begin{entry}
\mainentry{akaakatomoeru}
{赤々と燃える}
{lichterloh brennen.}\end{entry}
\begin{entry}
\mainentry{akaakatomoeruyounayuuyake}
{赤々と燃えるような夕焼け}
{feuriges Abendrot (n).}\end{entry}
\begin{entry}
\mainentry{akaaza}
{赤あざ; 赤アザ (赤痣)}
{  \textit{Med.} Hämangiom (n); Blutgefäßmal (n); Blutgefäßgeschwulst (n).}\end{entry}
\begin{entry}
\mainentry{akaashikatsuodori}
{アカアシカツオドリ (赤足鰹鳥; 赤脚鰹鳥)}
{  \textit{Vogelk.} Rotfußtölpel (m) (Sula sula).}\end{entry}
\begin{entry}
\mainentry{akaashishigi}
{アカアシシギ; あかあししぎ (赤足鷸; あかあし鷸; 赤足鴫)}
{  \textit{Vogelk.} Rotschenkel (m) (Tringa totanus).}\end{entry}
\begin{entry}
\mainentry{aga-te}
{アガーテ}
{  \textit{weibl. Vorname} AgathefNAr.}\end{entry}
\begin{entry}
\mainentry{akaari}
{赤あり (赤蟻)}
{  \textit{Insektenk.} rote Ameise (f).}\end{entry}
\begin{entry}
\mainentry{akai }
{亜界}
{  \textit{Biol.} Unterreich (n); Subregnum (n).}\end{entry}
\begin{entry}
\mainentry{akai }
{赤い; あかい (紅い [1]; 赭い)}
{ [1Gb] rot. [2] kommunistisch.}\end{entry}
\begin{entry}
\mainentry{akai }
{赤井}
{  \textit{Familienn.} Akai.}\end{entry}
\begin{entry}
\mainentry{akai }
{閼伽井}
{Brunnen (m) zum Schöpfen von Opferwasser.}\end{entry}
\begin{entry}
\mainentry{akaia}
{アカイア}
{  \textit{Gebietsn.} AchaianNAr (griech. Landschaft im Nordwesten des Peloponnes).}\end{entry}
\begin{entry}
\mainentry{akaiajin}
{アカイア人}
{  \textit{griech. Gesch.} Achäer (m).}\end{entry}
\begin{entry}
\mainentry{akaiinwotsukeru}
{赤い印を付ける}
{mit einem roten Punkt markieren.}\end{entry}
\begin{entry}
\mainentry{akaieka}
{赤家蚊; あかいえか; アカイエカ}
{  \textit{Insektenk.} rote Stechmücke (f) (Culex pipiens pallens).}\end{entry}
\begin{entry}
\mainentry{akaikao}
{赤い顔}
{rotes Gesicht (n).}\end{entry}
\begin{entry}
\mainentry{akaigawa}
{赤井川}
{  \textit{Ortsn.} Akaigawa (Ortschaft in Hokkaidō).}\end{entry}
\begin{entry}
\mainentry{akaikien}
{赤い気炎; 紅い気炎 (赤い気焔; 紅い気焔)}
{ gehobene Stimmung (f) einer Frau.}\end{entry}
\begin{entry}
\mainentry{akaikimono}
{赤い着物}
{[1] rote Kleidung (f). [2] Gefängnisuniform (f).}\end{entry}
\begin{entry}
\mainentry{akaikume-ru}
{赤いクメール}
{  \textit{Gesch.} Khmer Rougempl; Rote Khmermpl.}\end{entry}
\begin{entry}
\mainentry{akaike}
{赤池}
{  \textit{Ortsn.} Akaike (Ortschaft in der Präf. Fukuoka).}\end{entry}
\begin{entry}
\mainentry{akaishisanmyaku}
{赤石山脈}
{  \textit{Bergn.} Akaishi-Gebirge (n) (Teil der japanischen Alpen).}\end{entry}
\begin{entry}
\mainentry{akaishisou}
{赤い思想}
{kommunistische Ideenfpl; rote Gedankenfpl.}\end{entry}
\begin{entry}
\mainentry{akaijideinsatsusuru}
{赤い字で印刷する}
{in roten Lettern drucken.}\end{entry}
\begin{entry}
\mainentry{akaishin'nyo}
{赤い信女}
{ Witwe (f) (deren Name bereits zu ihren Lebzeiten auf dem Grabstein ihres Ehemannes in roter Schrift vermerkt ist).}\end{entry}
\begin{entry}
\mainentry{akaisshokuninuritsubusu}
{赤一色に塗りつぶす}
{ganz rot anmalen.}\end{entry}
\begin{entry}
\mainentry{akaito}
{赤糸}
{ [1] roter Faden (m). [2] rote Kordel (f) einer Akaito·odoshi-Rüstung.}\end{entry}
\begin{entry}
\mainentry{akaitoodoshi}
{赤糸おどし (赤糸威; 赤糸縅)}
{ Akaito·odoshi-Rüstung (f); Rüstung (f) mit roter eingeflochtener Schnur im Ringpanzer.}\end{entry}
\begin{entry}
\mainentry{akaitoge}
{赤糸毛}
{ mit roter Kordel verzierter Ochsenwagen (m) (Abk. für akaitoge·no·kuruma 赤糸毛の車8804738).}\end{entry}
\begin{entry}
\mainentry{akaitogenokuruma}
{赤糸毛の車}
{mit roter Kordel verzierter Ochsenwagen (m).}\end{entry}
\begin{entry}
\mainentry{akaitori}
{赤い鳥}
{ \textit{Werktitel}  (m) (Literaturzeitschrift für Kinder; gegründet von Suzuki Miekichi; 1918 bis 1936).}\end{entry}
\begin{entry}
\mainentry{akainikibi}
{赤いにきび}
{roter Pickel (m).}\end{entry}
\begin{entry}
\mainentry{akainu}
{赤犬}
{brauner Hut (m).}\end{entry}
\begin{entry}
\mainentry{akaino}
{赤いの}
{ein Rotes (n).}\end{entry}
\begin{entry}
\mainentry{akaihana}
{赤い花}
{ \textit{Werktitel} NAr (Novelle von Wsewolod Michailowitsch Garschin; 1883).}\end{entry}
\begin{entry}
\mainentry{akaihananaranandemoyoroshii。}
{赤い花なら何でもよろしい。}
{ \textit{Bsp.} Es geht jede Blume, solange sie rot ist.}\end{entry}
\begin{entry}
\mainentry{akaihane}
{赤い羽根}
{ rote Feder (f) (Zeichen dafür, dass man bei einer bestimmten Sammlung für wohltätige Zwecke gespendet hat).}\end{entry}
\begin{entry}
\mainentry{akaihanekyoudoubokin'undou}
{赤い羽根共同募金運動}
{Rote Feder (f) (Sammlung für wohltätige Zwecke; jedes Jahr im Oktober).}\end{entry}
\begin{entry}
\mainentry{akaihanebokin'undou}
{赤い羽根募金運動}
{Rote Feder (f) (Sammlung für wohltätige Zwecke; jedes Jahr im Oktober).}\end{entry}
\begin{entry}
\mainentry{akaihanewotsuketeiru}
{赤い羽根をつけている}
{die rote Feder tragen (als Zeichen, dass man bei einer Sammlung für wohltätige Zwecke gespendet hat).}\end{entry}
\begin{entry}
\mainentry{akaihoshi}
{赤い星}
{ Roter Stern (m); Krasnaja Zvjesda.}\end{entry}
\begin{entry}
\mainentry{akairibonwomejirushinitsukeru}
{赤いリボンを目印に付ける}
{mit einem roten Band markieren.}\end{entry}
\begin{entry}
\mainentry{akairyodan}
{赤い旅団}
{ \textit{Org.} Rote Brigadenfpl (ital. linksextremistische Terrororganisation).}\end{entry}
\begin{entry}
\mainentry{akairo}
{赤色 [1]}
{ Rot (n); rote Farbe (f).}\end{entry}
\begin{entry}
\mainentry{akaiwa}
{赤磐}
{  \textit{Ortsn.} Akaiwa (Ortschaft in der Präf. Okayama).}\end{entry}
\begin{entry}
\mainentry{akaiwashi}
{赤いわし (赤鰯)}
{ [1] in Reiskleie eingelegte Sardine (f). [2] verrostetes Schwert (n).}\end{entry}
\begin{entry}
\mainentry{akainkudekaku}
{赤インクで書く}
{mit roter Tinte schreiben.}\end{entry}
\begin{entry}
\mainentry{akauo}
{アカウオ; あかうお; 赤魚 [a]}
{  \textit{Fischk.} [1]  \textit{Fischk.} Rotbrasse (f) (Sebastes matsubarai). [2] Ugui (m) (Tribolodon hakonensis).}\end{entry}
\begin{entry}
\mainentry{akaukikusa}
{赤浮き草; 赤浮草; アカウキクサ; あかうきくさ}
{  \textit{Bot.} Aka·ukikusa (n); Azolla (f) (Azolla imbricata).}\end{entry}
\begin{entry}
\mainentry{akauso}
{赤うそ (赤嘘)}
{ grobe Lüge (f); fastdicke Lüge (f); glatte Lüge (f).}\end{entry}
\begin{entry}
\mainentry{akauni}
{赤ウニ; アカウニ; あかうに (赤海胆)}
{  \textit{Zool.} Aka·uni (ein roter Seeigel; Pseudocentrotus depressus).}\end{entry}
\begin{entry}
\mainentry{akauma}
{赤馬}
{ [1] rotes Pferd (n). [2] Feuer (n); Feuersbrunst (f) (in Edo). [3] (verschleirend) Menstruation (f); Regel (f).}\end{entry}
\begin{entry}
\mainentry{akaumigame}
{あかうみがめ; アカウミガメ (赤海亀)}
{  \textit{Zool.} Unechte Karettschildkröte (f) (Caretta caretta).}\end{entry}
\begin{entry}
\mainentry{akaume-ta-}
{アカウメーター}
{  \textit{Med.} Audiometer (n); Gehörempfindlichkeitsmessgerät (n) (von engl. acoumeter).}\end{entry}
\begin{entry}
\mainentry{akauntabiritexi-}
{アカウンタビリティー}
{ Verantwortlichkeit (f) (von engl. accountability).}\end{entry}
\begin{entry}
\mainentry{akauntexingu}
{アカウンティング}
{ Rechnungswesen (n) (von engl. accounting).}\end{entry}
\begin{entry}
\mainentry{akaunto}
{アカウント}
{ [1] Konto (n); Sparbuch (n). [2] Rechnung (f). [3]  \textit{EDV} Account (m); E‑Mail-Account (m). (von engl. account).}\end{entry}
\begin{entry}
\mainentry{akauntoeguzekutexibu}
{アカウント・エグゼクティブ; アカウントエグゼクティブ}
{Kontakter (m); Angestellter einer Werbeagentur, der den Kontakt zu den Auftraggebern hält (von engl. account executive).}\end{entry}
\begin{entry}
\mainentry{akauntokarento}
{アカウント・カレント; アカウントカレント}
{ \textit{Bankw.} Girokonto (n); Kontokorrentkonto (n); Kreditkonto (n).}\end{entry}
\begin{entry}
\mainentry{akauntobukku}
{アカウント・ブック; アカウントブック}
{Kontenbuch (n) (von engl. account book).}\end{entry}
\begin{entry}
\mainentry{akauntohenkou}
{アカウント変更}
{Änderung (f) des Accounts.}\end{entry}
\begin{entry}
\mainentry{akauntohoushiki}
{アカウント方式}
{Kontomethode (f) (eine Methode zur Berechnung der Mehrwertsteuer).}\end{entry}
\begin{entry}
\mainentry{akauntowotsukuru}
{アカウントを作る}
{ein Konto eröffnen.}\end{entry}
\begin{entry}
\mainentry{akae}
{赤絵}
{ Porzellan (n), bei dessen Bemalung (f) rot vorherrscht // Bemalung (f), bei der rot vorherrscht.}\end{entry}
\begin{entry}
\mainentry{akaei}
{あかえい; アカエイ; 赤えい; 赤エイ (赤鱏; 赤鱝; 赤海鷂魚)}
{  \textit{Fischk.} Rochen (m) (Dasyatis akajei; Sommer).}\end{entry}
\begin{entry}
\mainentry{akaezomatsu}
{アカエゾマツ; あかえぞまつ (赤蝦夷松)}
{  \textit{Bot.} Glehnsfichte (f) (Picea glehnii).}\end{entry}
\begin{entry}
\mainentry{akaeboshi}
{赤烏帽子}
{ [1] Akae·boshi (m); roter japanischer Hut (m). [2] Spleen (m); Marotte (f); Schrulle (f) // schrullige Person (f).}\end{entry}
\begin{entry}
\mainentry{akaeri}
{赤襟 (赤衿)}
{ [1] roter Ärmel (m). [2] junges Mädchen (n); (insbes.) junge Geisha (f).}\end{entry}
\begin{entry}
\mainentry{akaerikaitsuburi}
{あかえりかいつぶり; アカエリカイツブリ (赤襟鳰)}
{  \textit{Vogelk.} rotnackiger Lappentaucher (m) (Podiceps grisegena).}\end{entry}
\begin{entry}
\mainentry{akaerihireashishigi}
{アカエリヒレアシシギ; あかえりひれあししぎ (赤襟鰭足鷸; あかえりひれあし鷸; 赤襟鰭足鴫)}
{  \textit{Vogelk.} Odinshühnchen (n) (Phalaropus lobatus).}\end{entry}
\begin{entry}
\mainentry{akaenpitsu}
{赤鉛筆; 赤エンピツ; 赤えんぴつ}
{ Rotstift (m).}\end{entry}
\begin{entry}
\mainentry{akaenpitsudekaku}
{赤鉛筆で書く}
{mit Rotstift schreiben.}\end{entry}
\begin{entry}
\mainentry{akaookami}
{赤狼}
{ \textit{Zool.} Rotwolf (m); Rothund (m) (Cuon alpinus).}\end{entry}
\begin{entry}
\mainentry{akaookuchi}
{赤大口}
{  \textit{Kleidung} roter Hakama (m).}\end{entry}
\begin{entry}
\mainentry{akaoka}
{赤岡}
{  \textit{Ortsn.} Akaoka (Ortschaft im Osten der Präf. Kōchi).}\end{entry}
\begin{entry}
\mainentry{akaoke}
{閼伽桶}
{Bottich (m) für Opferwasser.}\end{entry}
\begin{entry}
\mainentry{agaotoakusento}
{揚音アクセント}
{ \textit{Sprachw.} Iktus (m); Betonung (f); Akzent (m).}\end{entry}
\begin{entry}
\mainentry{akaotoshi}
{垢落とし; 垢落し; 垢落}
{ [1] Abwaschen (n) des Hautschmutzes. [2] Gerätschaftenfpl zum Abwaschen des Hautschmutzes.}\end{entry}
\begin{entry}
\mainentry{akaodoshi}
{赤おどし (赤威; 赤緘)}
{ Aka·odoshi (f); Rüstung (f) mit roter eingeflochtener Schnur im Ringpanzer.}\end{entry}
\begin{entry}
\mainentry{akaoni}
{赤鬼}
{[1] roter Geist (m); roter Teufel (m); Kobold (m). [2] erbarmungsloser Geldeintreiber (m).}\end{entry}
\begin{entry}
\mainentry{agaka-n}
{アガ・カーン; アガカーン}
{  \textit{Islam} Aga Khan (m) (Oberhaupt einer islamischen Glaubensgemeinschaft).}\end{entry}
\begin{entry}
\mainentry{akagai}
{アカガイ; あかがい; 赤貝}
{ [1]  \textit{Muschelk.} Archenmuschel (f) (Scapharca broughtonii). [2] (vulg.) Vagina (f).}\end{entry}
\begin{entry}
\mainentry{akagaeru}
{赤ガエル; 赤がえる; アカガエル; あかがえる (赤蛙 [a])}
{  \textit{Zool.} brauner Frosch (m) (Ranidae).}\end{entry}
\begin{entry}
\mainentry{akakaeruka}
{赤蛙科}
{  \textit{Zool.} Fröschempl; Echte Fröschempl, Ranidae.}\end{entry}
\begin{entry}
\mainentry{akagaeruka}
{アカガエル科 (赤蛙科)}
{ \textit{Zool.} Echter Frosch (m); Familie (f) der Echten Frösche (Ranidae).}\end{entry}
\begin{entry}
\mainentry{akagaeruzoku}
{アカガエル属 (赤蛙属)}
{ \textit{Zool.} Echter Frosch (m); Gattung (f) der Echten Frösche (Rana).}\end{entry}
\begin{entry}
\mainentry{akagaochiru}
{あかが落ちる (垢が落ちる)}
{Schmutz geht ab.}\end{entry}
\begin{entry}
\mainentry{akagakatta}
{赤がかった}
{kommunistisch angehaucht; halbkommunistisch; pseudokommunistisch.}\end{entry}
\begin{entry}
\mainentry{akagakattaningen}
{赤がかった人間}
{kommunistisch angehauchte Person (f).}\end{entry}
\begin{entry}
\mainentry{akagakaru}
{赤がかる}
{rötlich werden; kommunistisch angehaucht werden.}\end{entry}
\begin{entry}
\mainentry{akakaki}
{垢掻き; 垢掻}
{ [1] Badefrau (f); Frau, die den Gästen eines Bades den Körper wusch und ihnen auch als Prostituierte zur Verfügung stand. [2] Prostituierte (f) in einem Badehaus. (in der Edo-Zeit).}\end{entry}
\begin{entry}
\mainentry{akagaki}
{赤垣}
{  \textit{Familienn.} Akagaki.}\end{entry}
\begin{entry}
\mainentry{akagakuru}
{淦が来る}
{Bilgewasser dringt ein.}\end{entry}
\begin{entry}
\mainentry{akakage}
{赤鹿毛}
{  \textit{Zool.} rötlich rehfarbenes Pferd (n).}\end{entry}
\begin{entry}
\mainentry{akagashi}
{あかがし; アカガシ (赤樫)}
{  \textit{Bot.} Spitzeiche (f) (Quercus acuta).}\end{entry}
\begin{entry}
\mainentry{akagashira}
{赤頭}
{ [1] rötliche Haare (n). // Person (f) mit roten Haaren. [3]  \textit{Theat.} rote Perücke (f) (für Tengu, Geister etc.). [4]  \textit{Vogelk.} Pfeifente (f).}\end{entry}
\begin{entry}
\mainentry{akagashiwa}
{赤柏 [1]}
{ [1]  \textit{Kochk.} Sekihan (n) (mit roten Bohnen gekochter Reis; ein Festessen). [2]  \textit{Bot.} Akamegashiwa (f) (eine Euphorbienart; Mallotus japonicus).}\end{entry}
\begin{entry}
\mainentry{agakasu}
{あがかす (足掻かす)}
{ [1] scharren lassen. [2] jmdn. zappeln lassen; jmdn. strampeln lassen. [3] jmdn. in Schwierigkeiten bringen.}\end{entry}
\begin{entry}
\mainentry{akagatakaru}
{垢がたかる}
{schmutzig werden.}\end{entry}
\begin{entry}
\mainentry{akagatsuku}
{垢がつく}
{schmutzig werden.}\end{entry}
\begin{entry}
\mainentry{akagappa}
{赤ガッパ; 赤合羽}
{  \textit{Kleidung} Regenjacke (f) aus Ölpapier.}\end{entry}
\begin{entry}
\mainentry{akagaderu}
{垢が出る}
{Schmutz geht ab.}\end{entry}
\begin{entry}
\mainentry{akagani}
{あかがに; アカガニ (赤蟹)}
{  \textit{Zool.} Benkei-Krabbe (f) (Sesarmops intermedia).}\end{entry}
\begin{entry}
\mainentry{akaganukeru}
{垢が抜ける}
{(übertr.) etw. aufpolieren.}\end{entry}
\begin{entry}
\mainentry{akagane}
{あかがね (銅 [b]; 赤金)}
{  \textit{Chem.} (schriftspr.) Kupfer (n) (⇒ kuro·gane 鉄4614168).}\end{entry}
\begin{entry}
\mainentry{akaganeiro}
{銅色 [a] (赤金色)}
{ Kupferfarbe (f).}\end{entry}
\begin{entry}
\mainentry{akaganeirono}
{銅色の}
{ kupferrot; kupferfarben.}\end{entry}
\begin{entry}
\mainentry{akaganeharokushougaagaru。}
{銅は緑青が上がる。}
{ \textit{Bsp.} Kupfer bekommt Grünspan.}\end{entry}
\begin{entry}
\mainentry{akagahairu}
{朱が入る}
{etw. korrigieren lassen.}\end{entry}
\begin{entry}
\mainentry{akakabibyou}
{赤かび病; アカカビ病 (赤黴病)}
{  \textit{Bot.} roter Mehltau (m) (eine Pflanzenkrankheit).}\end{entry}
\begin{entry}
\mainentry{akakabu}
{赤カブ; 赤かぶ (赤蕪)}
{  \textit{Bot.} Radieschen (n).}\end{entry}
\begin{entry}
\mainentry{akagami}
{赤紙}
{ [1] (ugs.) Einberufungsbefehl (m) // Pfändungsbefehl (m). [2] rotes Papier (n).}\end{entry}
\begin{entry}
\mainentry{akagari}
{赤狩り; 赤狩}
{ Kommunistenjagd (f); Verbannung von Kommunisten aus öffentlichen Ämtern und Unternehmen (in Japan 1949–1950; von engl. red purge; ⇒ reddo·pāji レッドパージ9581781).}\end{entry}
\begin{entry}
\mainentry{akakariginu}
{赤狩衣}
{  \textit{Kleidung} Aka·kariginu (rote Jagdkleidung bestimmter Beschäftigungsgruppen).}\end{entry}
\begin{entry}
\mainentry{akagarisuru}
{赤狩りする; 赤狩する}
{Kommunistenjagd betreiben.}\end{entry}
\begin{entry}
\mainentry{akagare}
{赤枯れ; 赤枯}
{ Verwelken (f) zu einer rötlich braunen Farbe.}\end{entry}
\begin{entry}
\mainentry{akagaresuru}
{赤枯れする; 赤枯する}
{zu einer rötlich braunen Farbe verwelken.}\end{entry}
\begin{entry}
\mainentry{akagareru}
{赤枯れる}
{ zu einer rötlich braunen Farbe verwelken.}\end{entry}
\begin{entry}
\mainentry{akagawa }
{赤革}
{ braunes Leder (n).}\end{entry}
\begin{entry}
\mainentry{akagawa }
{赤川}
{  \textit{Familienn.} Akagawa.}\end{entry}
\begin{entry}
\mainentry{akagawaodoshi}
{赤革おどし (赤革威; 赤革緘)}
{ Akagawa·odoshi (f); Rüstung (f) mit roter eingeflochtener Lederschnur im Ringpanzer.}\end{entry}
\begin{entry}
\mainentry{akagawara}
{赤がわら (赤瓦)}
{ roter Ziegel (m) (aus Lehm oder Beton).}\end{entry}
\begin{entry}
\mainentry{agaki}
{あがき (足掻き)}
{ [1] Scharren (n); Stampfen (n). [2] Zappeln (n); Strampeln (n). [3] Bewegungsfreiheit (f).}\end{entry}
\begin{entry}
\mainentry{akagi }
{赤城 [1]}
{  \textit{Familienn.} Akagi.}\end{entry}
\begin{entry}
\mainentry{akagi }
{赤城 [2]}
{  \textit{Ortsn.} AkaginNAr (Viertel von Shinjuku).}\end{entry}
\begin{entry}
\mainentry{akagi }
{赤木}
{ [1] geschältes rohes Holz (n). [2] rotes Holz (n). (⇔ kuro·ki 黒木9134984).}\end{entry}
\begin{entry}
\mainentry{akagi }
{赤来}
{  \textit{Ortsn.} Akagi (Ortschaft im Zentrum der Präf. Shimane).}\end{entry}
\begin{entry}
\mainentry{akagioroshi }
{赤城おろし [2] (赤城颪)}
{ starker Wind (m), der vom Berg Akagi blläst.}\end{entry}
\begin{entry}
\mainentry{akagioroshi }
{赤城おろし [1]}
{Wind (m) vom Berg Akagi.}\end{entry}
\begin{entry}
\mainentry{agakigatorenai}
{あがきがとれない; あがきが取れない (足掻が取れない; 足掻きがとれない)}
{in der Klemme stecken; in der Patsche sein; nicht vom Fleck kommen; im Schlamm stecken.}\end{entry}
\begin{entry}
\mainentry{agakigatorenaideiru}
{足掻きがとれないでいる}
{verzweifelt strampeln.}\end{entry}
\begin{entry}
\mainentry{akakikokoro}
{赤き心}
{ treues Herz (n).}\end{entry}
\begin{entry}
\mainentry{agakitsudukeru}
{あがき続ける (足掻き続ける; 足掻きつづける; 足掻つづける)}
{ [1] ständig scharren; ständig stampfen. [2] ständig zappeln; ständig strampeln. [3] ständig kämpfen.}\end{entry}
\begin{entry}
\mainentry{akagippu}
{赤切符}
{ Fahrkarte (f) dritter Klasse (diese Tickets waren früher rot).}\end{entry}
\begin{entry}
\mainentry{agakideru}
{足掻き出る}
{sich herauswinden.}\end{entry}
\begin{entry}
\mainentry{akaginu}
{赤ぎぬ (赤衣; 赤絹)}
{ [1] roter Seidenstoff (m). [2]  \textit{Kleidung} Hō (n) (Kleidung des fünften Ranges für den Dienst bei Hofe). [3]  \textit{Kleidung} Aka·kariginu (rote Jagdkleidung bestimmter Beschäftigungsgruppen).}\end{entry}
\begin{entry}
\mainentry{agakimi}
{あがきみ (我が君; 吾が君)}
{ [1] du; Sie (vertraulich Anrede für jmdn.). [2] Sie (höfliche Anrede für einen Herrn).}\end{entry}
\begin{entry}
\mainentry{akagire}
{あかぎれ (皸 [a]; 皹 [a])}
{ Schrunde (f); rissige Haut (f) (⇒ hibi ひび6027628).}\end{entry}
\begin{entry}
\mainentry{akagireninattate}
{あかぎれになった手}
{schrundige Händefpl; aufgerissene Händefpl.}\end{entry}
\begin{entry}
\mainentry{akagireninaru}
{あかぎれになる}
{schrundig werden; rissig werden.}\end{entry}
\begin{entry}
\mainentry{akagirenokireta}
{あかぎれの切れた; あかぎれのきれた}
{schrundig; rissig.}\end{entry}
\begin{entry}
\mainentry{akagirenokiretaashi}
{あかぎれの切れた足}
{schrundige Füßempl; rissig Füßempl.}\end{entry}
\begin{entry}
\mainentry{akagirenote}
{あかぎれの手}
{schrundige Händefpl; aufgerissene Händefpl.}\end{entry}
\begin{entry}
\mainentry{akaku}
{赤く}
{rot; in Rot.}\end{entry}
\begin{entry}
\mainentry{agaku}
{あがく (足掻く)}
{ [1] scharren; strampeln. [2] verzweifelt versuchen.}\end{entry}
\begin{entry}
\mainentry{akakukagayakukanojonohoho}
{赤く輝く彼女の頬}
{ihre rotglühenden Wangenfpl.}\end{entry}
\begin{entry}
\mainentry{akakusai}
{あか臭い (垢臭い)}
{ schmutzig; voll Körperschmutz.}\end{entry}
\begin{entry}
\mainentry{akagusare}
{赤腐れ; 赤腐}
{ [1] Braunfäule (f) z. B. an Kartoffeln und Kaktus. [2] Rotfäule (f) an Seelattich.}\end{entry}
\begin{entry}
\mainentry{akagusaresuru}
{赤腐れする; 赤腐する}
{die Braunfäule haben; die Rotfäule haben.}\end{entry}
\begin{entry}
\mainentry{akagusarebyou}
{赤腐れ病; 赤腐病}
{Rotfäule (f); Braunfäule (f).}\end{entry}
\begin{entry}
\mainentry{akakusomeru}
{赤く染める}
{rot färben.}\end{entry}
\begin{entry}
\mainentry{akakutani}
{赤九谷}
{ rotes Kutani-Porzellan (n).}\end{entry}
\begin{entry}
\mainentry{akagutsu}
{赤靴}
{ [1] braune Schuhempl; rotbraune Schuhempl. [2]  \textit{Fischk.} Rote Seefledermaus (f) (ein Armflosser; Halieutaea stellata).}\end{entry}
\begin{entry}
\mainentry{akagutsuyoukuri-mu}
{赤靴用クリーム}
{braune Schuhcreme (f).}\end{entry}
\begin{entry}
\mainentry{akakunatteokoru}
{赤くなって怒る}
{rot vor Ärger werden.}\end{entry}
\begin{entry}
\mainentry{akakunaru}
{赤くなる}
{[1] rot werden; erröten. [2] kommunistisch werden.}\end{entry}
\begin{entry}
\mainentry{akaguma}
{赤熊 [1]}
{  \textit{Zool.} Braunbär (m) (Ursus arctos).}\end{entry}
\begin{entry}
\mainentry{akakumi}
{淦汲み}
{Kelle (f) zum Ausschöpfen des Bilgewassers.}\end{entry}
\begin{entry}
\mainentry{akagumi}
{紅組; 赤組}
{ rotes Team (n); die Rotenmpl.}\end{entry}
\begin{entry}
\mainentry{akakura}
{赤倉}
{  \textit{Familienn.} Akakura.}\end{entry}
\begin{entry}
\mainentry{akakurage}
{アカクラゲ; あかくらげ (赤水母)}
{  \textit{Zool.} Aka·kurage (f) (eine rot-weiße Qualle; Dactylometra pacifica).}\end{entry}
\begin{entry}
\mainentry{akakurige}
{赤栗毛}
{  \textit{Zool.} Fuchs (m) (Pferd von rötlicher Farbe).}\end{entry}
\begin{entry}
\mainentry{akaguroi}
{赤黒い}
{ dunkelrot; rotschwarz.}\end{entry}
\begin{entry}
\mainentry{akakuro-ba-}
{赤クローバー; アカ•クローバー; アカクローバー}
{  \textit{Bot.} Wiesenklee (m); Rotklee (m) (Trifolium pratense).}\end{entry}
\begin{entry}
\mainentry{akakuro-ba-chuudoku}
{アカクローバー中毒}
{ \textit{Med.} Trifoliose (f); Kleekrankheit (f) (durch Verzehr von zu viel oder von schimmligem Klee verursachte Krankheit z. B. bei Pferden).}\end{entry}
\begin{entry}
\mainentry{akage}
{赤毛}
{ rotes Haar (n) (⇒ aoge 青毛4641416).}\end{entry}
\begin{entry}
\mainentry{akagezaru}
{アカゲザル; あかげざる; 赤毛猿}
{  \textit{Zool.} Rhesus-Affe (m) (Macaca mulatta).}\end{entry}
\begin{entry}
\mainentry{akagetto}
{赤ゲット; 赤げっと (赤毛布)}
{ [1] rote Decke (f). [2] Bauerntölpel (m). (getto steht als Abkürzung für engl. blanket).}\end{entry}
\begin{entry}
\mainentry{akagettodure}
{赤毛布連}
{Bauerntölpel (m); Pöbel (m); Plebsmf.}\end{entry}
\begin{entry}
\mainentry{akagettowoyaru}
{赤毛布をやる}
{sich wie ein Bauerntölpel verhalten.}\end{entry}
\begin{entry}
\mainentry{akageno}
{赤毛の}
{rothaarig.}\end{entry}
\begin{entry}
\mainentry{akagenoshounen}
{赤毛の少年}
{rothaariger Junge (m).}\end{entry}
\begin{entry}
\mainentry{akagenohito}
{赤毛の人}
{Rothaariger (m).}\end{entry}
\begin{entry}
\mainentry{akagera}
{アカゲラ; あかげら (赤啄木鳥)}
{  \textit{Vogelk.} Buntspecht (m) (Dendrocopos major hondoensis).}\end{entry}
\begin{entry}
\mainentry{akagewashu}
{あかげ和種 (褐毛和種)}
{  \textit{Zool.} braune japanische Kuh (f).}\end{entry}
\begin{entry}
\mainentry{akago}
{赤子 [a] (赤児)}
{ Säugling (m); Baby (n) (⇒ akan·bō 赤ん坊4621553).}\end{entry}
\begin{entry}
\mainentry{akakounou}
{赤こうのう (赤行嚢)}
{ (obsol.) roter Postsack (m) (für Einschreiben und Geldsendung; ⇒ aka·yūtai 赤郵袋1944526).}\end{entry}
\begin{entry}
\mainentry{akako-na-}
{赤コーナー}
{  \textit{Boxen} Ecke (f) des Champions; Ecke (f) des Titelverteidigers (⇒ ao·kōnā 青コーナー0142840).}\end{entry}
\begin{entry}
\mainentry{akagogaumareru}
{赤子が生まれる}
{ein Kind wird geboren.}\end{entry}
\begin{entry}
\mainentry{akakokko}
{アカコッコ; あかこっこ (赤鶫)}
{  \textit{Vogelk.} Aka·kokko (f); Kurodadrossel (f) (Turdus celaenop).}\end{entry}
\begin{entry}
\mainentry{akagonotewonejiruyouda。}
{赤子の手をねじるようだ。 (赤子の手を捩るようだ。)}
{ \textit{Bsp.} Das ist so einfach, wie einem Baby den Arm zu verdrehen! // Das ist nur ein Kinderspiel für mich!}\end{entry}
\begin{entry}
\mainentry{akagonotewonejiruyounamonoda。}
{赤子の手をねじるようなものだ。}
{ \textit{Bsp.} Das ist so einfach, wie einem Baby den Arm zu verdrehen! // Das ist nur ein Kinderspiel für mich!}\end{entry}
\begin{entry}
\mainentry{akagonotewohineruyouda。}
{赤子の手をひねるようだ。 (赤子の手を捻るようだ。)}
{ \textit{Bsp.} Das ist so einfach, wie einem Baby den Arm zu verdrehen! // Das ist nur ein Kinderspiel für mich!}\end{entry}
\begin{entry}
\mainentry{akagohan}
{赤御飯}
{  \textit{Kochk.} Sekihan (n) (mit roten Bohnen gekochter Reis; ein Festessen).}\end{entry}
\begin{entry}
\mainentry{akagoma}
{赤駒}
{ Pferd (n) mit rotem Fell.}\end{entry}
\begin{entry}
\mainentry{akagomu}
{赤ゴム}
{ Guttaperchamn.}\end{entry}
\begin{entry}
\mainentry{akagome}
{赤米}
{ [1] minderwertiger Reis (m) der sich rot verfärbt, wenn er alt wird. [2] ausländischer Reis (m) von minderer Qualität.}\end{entry}
\begin{entry}
\mainentry{akagowoumu}
{赤子を産む}
{gebären.}\end{entry}
\begin{entry}
\mainentry{akasa}
{赤さ}
{ rote Färbung (f); Röte (f) (⇒ aka·mi 赤み0296656).}\end{entry}
\begin{entry}
\mainentry{akaza }
{あかざ; アカザ (藜)}
{  \textit{Bot.} weißer Gänsefuß (m) (Chenopodium album var. centrorubrum).}\end{entry}
\begin{entry}
\mainentry{akaza }
{赤座}
{  \textit{Familienn.} Akaza.}\end{entry}
\begin{entry}
\mainentry{akasaka}
{赤坂}
{  \textit{Ortsn.} AkasakanNAr (Stadtteil in Tōkyō).}\end{entry}
\begin{entry}
\mainentry{akazaka}
{アカザ科 (藜科)}
{  \textit{Bot.} Gänsefußgewächsenpl (Chenopodiaceae).}\end{entry}
\begin{entry}
\mainentry{akasaki}
{赤碕}
{  \textit{Ortsn.} Akasaki (Ortschaft im Zentrum der Präf. Tottori).}\end{entry}
\begin{entry}
\mainentry{akazake}
{赤酒}
{ Akazake (m); Sake (m) aus nicht klebendem Reis.}\end{entry}
\begin{entry}
\mainentry{akazatou}
{赤砂糖}
{ Rohzucker (m); unraffinierter Zucker (m); brauner Zucker (m).}\end{entry}
\begin{entry}
\mainentry{akazanotsue}
{藜の杖}
{Gänsefuß-Stock (m).}\end{entry}
\begin{entry}
\mainentry{akasabi}
{赤さび; 赤サビ (赤錆び; 赤錆; 赤銹)}
{ Rost (m); roter Rost (m).}\end{entry}
\begin{entry}
\mainentry{akazara}
{アカザラ; あかざら; 赤皿}
{  \textit{Zool.} Azumanishiki-Muschel (f) (Chlamys farreri nipponensis; ⇒ azumanishiki·gai 東錦貝4584344).}\end{entry}
\begin{entry}
\mainentry{akazawa}
{赤沢}
{  \textit{Familienn.} Akazawa.}\end{entry}
\begin{entry}
\mainentry{agashi}
{アガシ}
{ Mädchen (n); junge Dame (f) (von korean. agassi).}\end{entry}
\begin{entry}
\mainentry{akashi }
{あかし [1] (証かし; 証し; 証 [1]; 證 [1]; 明かし; 明し)}
{ (schriftspr.) [1] Zeugnis (n). [2] Beweis (m); Beweisführung (f).}\end{entry}
\begin{entry}
\mainentry{akaji }
{赤字}
{ [1] rote Zahlenfpl; Defizit (n); Fehlbetrag (m); Verlust (m); Passivsaldo (n). [2] Korrekturlesen (n). (⇔ kuro·ji 黒字7871635).}\end{entry}
\begin{entry}
\mainentry{akashi }
{あかし [2] (灯 [1]; 燈 [1]; 灯火 [1])}
{ Licht (n); Lampe (f).}\end{entry}
\begin{entry}
\mainentry{akaji }
{赤地}
{ roter Grund (m); roter Stoff (m).}\end{entry}
\begin{entry}
\mainentry{akashi }
{明司}
{ [1]  \textit{männl. Name} Akashi. [2]  \textit{Familienn.} Akashi.}\end{entry}
\begin{entry}
\mainentry{akashi }
{明石}
{ [1]  \textit{Stadtn.} AkashinNAr (Burgstadt in der Präf. Hyôgo an der Seto-Inlandsee westlich von Kôbe; 277.000 Ew.; Fährhafen, Textilind., Motoren‑ und Maschinenbau; Ausgangspunkt der Akashi-Kaikyo-Brücke; der durch Akashi laufende 135. Meridian östlich von Greenwich bestimmt die japan. Normalzeit: die Akashizeit). [2] Kochk.</s}\end{entry}
\begin{entry}
\mainentry{akashia}
{アカシア}
{ [1]  \textit{Bot.} Akazie (f) (Cervus elaphus). [2]  \textit{Bot.} (ugs.) Weiße Robinie (f); Falsche Akazie (f); Scheinakazie (f) (Robinia pseudoacacia).}\end{entry}
\begin{entry}
\mainentry{agashi-}
{アガシー}
{  \textit{Persönlichk.} Jean Louis Rodolphe Agassiz (schweizerisch-amerikanischer Anatom, Zoologe und Paläontologe; 1807–1873).}\end{entry}
\begin{entry}
\mainentry{akashio}
{赤潮 (赤汐)}
{  \textit{Naturphän.} rote Meeresströmung (f) (Meersströmung mit durch Blutalgen verursachter roter Färbung).}\end{entry}
\begin{entry}
\mainentry{akashioohashi}
{明石大橋}
{  \textit{Ortsn.} Akashi-Brücke (f) (Abk. für Akashi-Kaikyō-Brücke).}\end{entry}
\begin{entry}
\mainentry{akashika}
{赤鹿; あかしか; アカシカ}
{  \textit{Zool.} Rothirsch (m); Edelhirsch (m) (Cervus elaphus).}\end{entry}
\begin{entry}
\mainentry{akashikaikyou}
{明石海峡}
{ \textit{Meeresn.} Meerenge (f) von Akaishi (zwischen Kōbe und der Insel Awaji-shima in der Seto-Inlandsee).}\end{entry}
\begin{entry}
\mainentry{akashikaikyouoohashi}
{明石海峡大橋}
{Akashi-Kaikyō-Brücke (f) (Brücke über die Meerenge von Akaishi zwischen Kōbe und der Insel Awaji-shima auf der Route nach Shikoku; Gesamtlänge 3 910 m, längste Strecke ohne Stützen 1 991 m; April 1998 eröffnet).}\end{entry}
\begin{entry}
\mainentry{akashigatatsu}
{証が立つ}
{von Anschuldigungen reingewaschen werden.}\end{entry}
\begin{entry}
\mainentry{akajikaranukedasu}
{赤字から抜け出す}
{aus den roten Zahlen herauskommen.}\end{entry}
\begin{entry}
\mainentry{akashikimouno}
{赤色盲の}
{ \textit{Med.} rotblind.}\end{entry}
\begin{entry}
\mainentry{akajikin'yuu}
{赤字金融}
{Defizitfinanzierung (f).}\end{entry}
\begin{entry}
\mainentry{akashikurasu}
{あかし暮らす (明かし暮らす; 明し暮す; 明かしくらす; 明かし暮す; 明し暮らす)}
{ Zeit verbringen.}\end{entry}
\begin{entry}
\mainentry{akajikeiei}
{赤字経営}
{defizitärer Betrieb (m).}\end{entry}
\begin{entry}
\mainentry{akajikeieiwosuru}
{赤字経営をする}
{mit Defizit betreiben.}\end{entry}
\begin{entry}
\mainentry{akashigenjin}
{明石原人}
{ \textit{Anthropol.} Akashi-Urmensch (m) (1931 in Akashi gefundene versteinerte menschliche Knochen).}\end{entry}
\begin{entry}
\mainentry{akajikousai}
{赤字公債}
{Ausgleichsfonds (m).}\end{entry}
\begin{entry}
\mainentry{akajikousainozouhatsu}
{赤字公債の増発}
{zusätzliche Ausgleichsfonds ausgeben.}\end{entry}
\begin{entry}
\mainentry{akajikoku}
{赤字国}
{Defizitland (n).}\end{entry}
\begin{entry}
\mainentry{akajikokusai}
{赤字国債}
{ \textit{Wirtsch.} Defizitfinanzierungsanleihe (f).}\end{entry}
\begin{entry}
\mainentry{akajisaikendantai}
{赤字再建団体}
{bankrotte Kommune (f); bankrotte selbstverwaltete Körperschaft (f).}\end{entry}
\begin{entry}
\mainentry{akajizaisei}
{赤字財政}
{Defizitfinanzierung (f); Finanzlage (f) mit Defizit; Defizit (n) im Staatshaushalt.}\end{entry}
\begin{entry}
\mainentry{akajizangaku}
{赤字残額}
{defizitäre Bilanz (f).}\end{entry}
\begin{entry}
\mainentry{akajishishutsu}
{赤字支出}
{Defizitfinanzierung (f).}\end{entry}
\begin{entry}
\mainentry{akashijimi}
{あかしじみ; アカシジミ (赤蜆; 赤小灰蝶)}
{  \textit{Insektenk.} Akashi-jimi (n) (Japonica lutea).}\end{entry}
\begin{entry}
\mainentry{akashishoten}
{明石書店}
{  \textit{Verlagsn.} Akashi Shoten (Tōkyō; ISBN 4-7503-).}\end{entry}
\begin{entry}
\mainentry{akajisen}
{赤字線}
{unrentable Eisenbahnstrecke (f); defizitäre Bahnstrecke (f).}\end{entry}
\begin{entry}
\mainentry{akajiso}
{赤ジソ (赤紫蘇)}
{  \textit{Bot.} rote Perilla (Perilla frutescens).}\end{entry}
\begin{entry}
\mainentry{akashichidimi}
{明石縮み; 明石縮}
{ Akashi-Krepp (m) (hochwertiger leichter Stoff für Sommerkimonos).}\end{entry}
\begin{entry}
\mainentry{akashide}
{赤四手; あかしで; アカシデ}
{  \textit{Bot.} Hainbuche (f); Hornbaum (m) (Carpinus laxiflora).}\end{entry}
\begin{entry}
\mainentry{akajidearu}
{赤字である}
{in den roten Zahlen sein; im Defizit sein.}\end{entry}
\begin{entry}
\mainentry{akashina}
{明科}
{  \textit{Ortsn.} AkashinanNAr (Ort in der Präf. Nagano).}\end{entry}
\begin{entry}
\mainentry{akajiniochiiru}
{赤字に陥る}
{ins Defizit abrutschen.}\end{entry}
\begin{entry}
\mainentry{akajinikanemoyou}
{赤地に金模様}
{goldenes Design (n) auf rotem Grund.}\end{entry}
\begin{entry}
\mainentry{akajinisuru}
{赤字にする}
{in die roten Zahlen führen.}\end{entry}
\begin{entry}
\mainentry{akajininatteiru}
{赤字になっている}
{in den roten Zahlen stecken.}\end{entry}
\begin{entry}
\mainentry{akajininaru}
{赤字になる}
{in die roten Zahlen geraten; ins Defizit geraten.}\end{entry}
\begin{entry}
\mainentry{akajino}
{赤字の}
{defizitär.}\end{entry}
\begin{entry}
\mainentry{akashibito}
{あかし人 (証人 [1]; 明かし人)}
{ Zeuge (m); Bürge (m).}\end{entry}
\begin{entry}
\mainentry{akajihoujinkazei}
{赤字法人課税}
{ Besteuerung (f) einer defizittären Firma.}\end{entry}
\begin{entry}
\mainentry{akajihoten}
{赤字補填}
{Defizitausgleich (m).}\end{entry}
\begin{entry}
\mainentry{akashima}
{阿嘉島}
{  \textit{Familienn.} AkashimaNAr.}\end{entry}
\begin{entry}
\mainentry{akajimita}
{あかじみた (垢染みた; 垢じみた)}
{speckig; dreckig.}\end{entry}
\begin{entry}
\mainentry{akajimitakao}
{垢じみた顔}
{schmutziges Gesicht (n).}\end{entry}
\begin{entry}
\mainentry{akajimiru}
{あかじみる; あか染みる (垢染みる)}
{ speckig sein; dreckig sein.}\end{entry}
\begin{entry}
\mainentry{akashiya }
{アカシヤ}
{  \textit{Bot.} Akazie (f).}\end{entry}
\begin{entry}
\mainentry{akashiya }
{明石家}
{  \textit{Familienn.} Akashiya.}\end{entry}
\begin{entry}
\mainentry{akashiyasushi}
{明石康}
{  \textit{Persönlichk.} Akashi Yasushi (ehem. UN-Bürokrat; 1931–).}\end{entry}
\begin{entry}
\mainentry{「akashatsu」toadanawotoru}
{「赤シャツ」と綽名を取る}
{den Spitznamen „Rothemd“ bekommen.}\end{entry}
\begin{entry}
\mainentry{akajiyuushi}
{赤字融資}
{Defizitfinanzierung (f).}\end{entry}
\begin{entry}
\mainentry{akashoubin}
{あかしょうびん; アカショウビン (赤翡翠)}
{  \textit{Vogelk.} großer, rötlich brauner asiatischer Eisvogel (m) (Halcyon coromanda).}\end{entry}
\begin{entry}
\mainentry{akashouma}
{アカショウマ; あかしょうま; 赤升麻}
{  \textit{Bot.} Akashōma (Astibe thunbergii).}\end{entry}
\begin{entry}
\mainentry{akajiyosan}
{赤字予算}
{unausgeglichenes Budget (n).}\end{entry}
\begin{entry}
\mainentry{akajiyosanwokumu}
{赤字予算を組む}
{ein defizitäres Budget erstellen.}\end{entry}
\begin{entry}
\mainentry{akashiraga}
{赤白髪}
{ rötlich-weiße Haarenpl.}\end{entry}
\begin{entry}
\mainentry{akajirushi}
{赤印}
{ rotes Siegel (n); rotes Abzeichen (n).}\end{entry}
\begin{entry}
\mainentry{akajiro-karusen}
{赤字ローカル線}
{defizitäre Lokalbahn (f).}\end{entry}
\begin{entry}
\mainentry{akajirosen}
{赤字路線}
{unrentable Line (f); defizitäre Bahnstrecke (f).}\end{entry}
\begin{entry}
\mainentry{akajiwoumeru}
{赤字を埋める}
{ein Defizit ausgleichen.}\end{entry}
\begin{entry}
\mainentry{akajiwodasu}
{赤字を出す}
{rote Zahlen schreiben; Verlust machen.}\end{entry}
\begin{entry}
\mainentry{akashiwotateru}
{あかしを立てる (証を立てる)}
{beweisen; Zeugnis ablegen.}\end{entry}
\begin{entry}
\mainentry{akajiwonakusu}
{赤字をなくす}
{ein Defizit ausgleichen.}\end{entry}
\begin{entry}
\mainentry{akashingetsusha}
{赤新月社}
{  \textit{Org.} Roter Halbmond (m).}\end{entry}
\begin{entry}
\mainentry{akashingou}
{赤信号}
{ [1] rotes Licht (n); rote Ampel (f); Warnsignal (n); Haltesignal (n). [2] (übertr.) warnendes Zeichen (n); böses Zeichen (n). (⇔ ao·shingō 青信号9515717).}\end{entry}
\begin{entry}
\mainentry{akashingougatsuku}
{赤信号が点く; 赤信号が付く}
{sich verschlechtern; die Alarmlampen aufleuchten lassen.}\end{entry}
\begin{entry}
\mainentry{akashingoudetomaru}
{赤信号で止まる}
{an der roten Ampel halten.}\end{entry}
\begin{entry}
\mainentry{akashingoumin'nadewatarebakowakunai。}
{赤信号みんなで渡れば怖くない。}
{ \textit{Sprichw.} Wenn alle bei Rot über die Kreuzung gehen, dann regt das schon keinen mehr auf.}\end{entry}
\begin{entry}
\mainentry{akashingouwomitekurumawotomeru}
{赤信号を見て車をとめる}
{an der roten Ampel halten.}\end{entry}
\begin{entry}
\mainentry{akashingouwomushishitetsuppashiru}
{赤信号を無視して突っ走る}
{bei einer roten Ampel durchfahren.}\end{entry}
\begin{entry}
\mainentry{akashingouwomushishitetooriwowataru}
{赤信号を無視して通りを渡る}
{bei Rot über die Straße gehen.}\end{entry}
\begin{entry}
\mainentry{akashinchuu}
{赤真ちゅう (赤真鍮)}
{  \textit{Metall.} Rotguss (m).}\end{entry}
\begin{entry}
\mainentry{akashinbun}
{赤新聞}
{ \textit{Zeitungsw.} (ugs.) Boulevardpresse (f); Sensationspresse (f); Yellowpress (f).}\end{entry}
\begin{entry}
\mainentry{akasu }
{あかす [1] (証す; 證す)}
{ (schriftspr.) beweisen (z. B. seine Unschuld).}\end{entry}
\begin{entry}
\mainentry{akazu }
{赤酢}
{  \textit{Kochk.} [1] Reisessig (m). [2] Pflaumenessig (m).}\end{entry}
\begin{entry}
\mainentry{akasu }
{飽かす; 飽す; あかす [2] (厭かす)}
{ [1] übersättigen. [2] jmdn. ermüden; jmdn. langweilen.}\end{entry}
\begin{entry}
\mainentry{akazu }
{飽かず}
{ (schriftspr.) nicht ermüden etw. zu tun.}\end{entry}
\begin{entry}
\mainentry{akasu }
{明かす; 明す; あかす [3] (証す)}
{ [1] (die Nacht) verbringen. [2] ein Geheimnis anvertrauen; gestehen; beichten.}\end{entry}
\begin{entry}
\mainentry{akasugi}
{赤杉; あかすぎ; アカスギ}
{  \textit{Bot.} rote Zeder (f) (Juniperus virginiana).}\end{entry}
\begin{entry}
\mainentry{akazukin}
{赤ずきん (赤頭巾)}
{ \textit{Buchtitel}  (n) (japan. Titel des Märchens der Brüder Grimm).}\end{entry}
\begin{entry}
\mainentry{akasuguri}
{アカスグリ; あか赤スグリ; 赤すぐり (赤酸塊)}
{  \textit{Bot.} Rote Johannisbeere (f) (Ribes rubrum); Johannisbeerstrauch (m).}\end{entry}
\begin{entry}
\mainentry{akasuji}
{赤筋 [1] (赤条)}
{ [1] rote Linie (f). [2] durch die Haut scheinende Ader (f).}\end{entry}
\begin{entry}
\mainentry{akazuna}
{赤砂}
{ Schmirgel (m).}\end{entry}
\begin{entry}
\mainentry{akazunagameru}
{飽かず眺める}
{unermüdlich betrachten; sich nicht satt sehen können.}\end{entry}
\begin{entry}
\mainentry{akazuni}
{飽かずに}
{unermüdlich; ohne daran satt zu werden.}\end{entry}
\begin{entry}
\mainentry{akazuno}
{あかずの; 開かずの; 明かずの}
{ [1] stets verschlossen; meist verschlossen. [2] zu öffnen verboten.}\end{entry}
\begin{entry}
\mainentry{akazunotoumikiri}
{あかずの踏み切り}
{Bahnübergang (m), der ständig verschlossen zu sein schein.}\end{entry}
\begin{entry}
\mainentry{akazunoma}
{明かずの間; 開かずの間; あかずの間 (不開の間; 不開間)}
{ [1] stets verschlossenes Zimmer (n); verbotene Kammer (f). [2] Kammer (f), die nie geöffnet wird, weil sie mysteriös ist.}\end{entry}
\begin{entry}
\mainentry{akazunomon}
{開かずの門 (不開の門)}
{ [1] meist verschlossenes Tor (n). [2] Tor (n), das nicht geöffnet wird, weil es als unheilvoll angesehen wird, es zu öffnen.}\end{entry}
\begin{entry}
\mainentry{akazumi}
{赤墨}
{ rote Tusche (f) (⇒ shu·zumi 朱墨5742276).}\end{entry}
\begin{entry}
\mainentry{akasuri}
{あかすり (垢擦り; 垢擦; 垢すり)}
{ [1] Waschlappen (m); Bimsstein (m); Luffaschwamm (m) (etwas um beim Waschen den Körperschmutz abzureiben). [2] Abreiben (n) des Körperschmutzes.}\end{entry}
\begin{entry}
\mainentry{akazuri}
{赤刷り; 赤刷}
{ Abdrucken (n) in roter Farbe; roter Buchstabe (m).}\end{entry}
\begin{entry}
\mainentry{akazurisuru}
{赤刷りする; 赤刷する}
{etw. in Rot drucken.}\end{entry}
\begin{entry}
\mainentry{akazurinisuru}
{赤刷りにする}
{etw. in Rot drucken.}\end{entry}
\begin{entry}
\mainentry{akasegawagenpei}
{赤瀬川原平}
{  \textit{Persönlichk.} Akasegawa Genpei (Maler und Schriftsteller; 1937–).}\end{entry}
\begin{entry}
\mainentry{akaseru }
{飽かせる}
{ [1] übersättigen. [2] jmdn. ermüden; jmdn. langweilen. (⇒ akasu 飽かす9167353).}\end{entry}
\begin{entry}
\mainentry{akaseru }
{明かせる; 明せる}
{ [1] jmds. Unschuld beweisen können. [2] die ganze Nacht aufbleiben können. [3] ein Geheimnis aufdecken können.}\end{entry}
\begin{entry}
\mainentry{akasen}
{赤線}
{ [1] rote Linie (f). [2] Rotlichtviertel (n) (weil dieses auf Karten z. B. der Polizei mit einer roten Linie markiert war).}\end{entry}
\begin{entry}
\mainentry{akasenkuiki}
{赤線区域}
{Rotlichtviertel (n).}\end{entry}
\begin{entry}
\mainentry{akasenchitai}
{赤線地帯}
{Rotlichtviertel (n).}\end{entry}
\begin{entry}
\mainentry{akaso}
{アカソ; あかそ (赤麻; 紫苧)}
{  \textit{Bot.} Akaso (m); roter Hanf (m) (Boehmeria sylvestrii).}\end{entry}
\begin{entry}
\mainentry{akazome}
{赤染}
{  \textit{Familienn.} Akazome.}\end{entry}
\begin{entry}
\mainentry{akada}
{阿伽陀; 阿掲陀; 閼伽陀}
{ Agadamn; heilige Medizin (f) gegen alle Krankheiten; Gegengift (n) gegen alle Gifte; Medizin (f) gegen Altern und Tod.}\end{entry}
\begin{entry}
\mainentry{agata}
{あがた (県 [1])}
{ [1]  \textit{japan. Gesch.} Provinz (f) unter Kontrolle der kaiserlichen Familie (f) (vor der Taika-Reform). [2]  \textit{japan. Gesch.} Provinz (f); Verwaltungsbezirk (m). [3] Provinz (f); Land (n); ländliches Gebiet (n).}\end{entry}
\begin{entry}
\mainentry{akadai}
{あかだい; アカダイ (赤鯛)}
{  \textit{Fischk.} rote Meerbrasse (f) (Lutjanus scbae).}\end{entry}
\begin{entry}
\mainentry{akadaikon}
{赤大根}
{ [1]  \textit{Bot.} Radieschen (n). [2] (übertr.) Person (f), die ihren Reden nach Kommunist ist.}\end{entry}
\begin{entry}
\mainentry{akadashi}
{あかだし; 赤だし; 赤出し; 赤出 (赤出汁)}
{  \textit{Kochk.} Akadashi (n); dunkelrote Miso-Suppe (f).}\end{entry}
\begin{entry}
\mainentry{akadasuki}
{赤だすき (赤襷)}
{  \textit{Kleidung} rote Schärpe (f).}\end{entry}
\begin{entry}
\mainentry{akatateha}
{あかたては; アカタテハ (赤立て羽蝶; 赤立羽蝶; 赤たて蝶)}
{  \textit{Insektenk.} Indischer Admiral (m) (Vanessa indica).}\end{entry}
\begin{entry}
\mainentry{akadana}
{閼伽棚}
{Bord (n) für Opferwasser oder ‑blumen.}\end{entry}
\begin{entry}
\mainentry{akadani}
{あかだに; アカダニ (赤壁蝨)}
{  \textit{Insektenk.} Kupferbrand (m); rote Spinne (f) (Tetranychus kanzawai).}\end{entry}
\begin{entry}
\mainentry{agatanushi}
{県主}
{  \textit{japan. Gesch.} Agatanushi (m); Präfekt (m) (in der Yamato-Zeit).}\end{entry}
\begin{entry}
\mainentry{akadama}
{赤玉}
{ [1] roter Ball (m). [2] braunes Ei (n). [3] rote Kugel (f) beim Billard. [4] Bernstein (m). [5] Jaspis (m). [6]  \textit{Med.} Akadama (n) (eine Edo-zeitliche Medizin gegen Magenbeschwerden).}\end{entry}
\begin{entry}
\mainentry{akadarakeno}
{垢だらけの}
{schmutzig; speckig.}\end{entry}
\begin{entry}
\mainentry{akadarakenote}
{垢だらけの手}
{schmutzige Händefpl.}\end{entry}
\begin{entry}
\mainentry{akatarou}
{赤太郎}
{  \textit{männl. Name} Akatarō.}\end{entry}
\begin{entry}
\mainentry{akatan}
{赤短; 赤丹}
{ eine rote Karte (f) bei den japanischen Spielkarten.}\end{entry}
\begin{entry}
\mainentry{akacha}
{赤茶}
{ Rotbraun (n).}\end{entry}
\begin{entry}
\mainentry{akachairo}
{赤茶色}
{Rotbraun (n).}\end{entry}
\begin{entry}
\mainentry{akachaketaiwa}
{赤茶けた岩}
{rötlich brauner Fels (m).}\end{entry}
\begin{entry}
\mainentry{akachaketayoufuku}
{赤茶けた洋服}
{rötlich braun verschossene Kleidung (f).}\end{entry}
\begin{entry}
\mainentry{akachakeru}
{赤茶ける}
{ rotbraun werden; rötlich braun anlaufen.}\end{entry}
\begin{entry}
\mainentry{akachan}
{赤ちゃん; あかちゃん}
{ Baby (n); Säugling (m) (⇒ akan·bō 赤ん坊4621553).}\end{entry}
\begin{entry}
\mainentry{akachangaumareta。}
{赤ちゃんが生れた。}
{ \textit{Bsp.} Sie hat ein Kind bekommen.}\end{entry}
\begin{entry}
\mainentry{akachangadekiru}
{赤ちゃんが出来る}
{ein Baby wird geboren.}\end{entry}
\begin{entry}
\mainentry{akachankonku-ru}
{赤ちゃんコンクール}
{Schönheitswettbewerb (m) für Babys.}\end{entry}
\begin{entry}
\mainentry{akachouchin}
{赤ちょうちん (赤提灯 [a]; 赤提燈 [a])}
{ (ugs.) billige Kneipe (f).}\end{entry}
\begin{entry}
\mainentry{akadyouchin}
{赤ぢょうちん (赤提灯 [b]; 赤提燈 [b])}
{ (ugs.) billige Kneipe (f) (→ aka·chōchin 赤提灯3370585).}\end{entry}
\begin{entry}
\mainentry{akachin}
{赤チン}
{  \textit{Pharm.} Jod (n); Mercurochrom (n) (Wz.).}\end{entry}
\begin{entry}
\mainentry{akatsuki }
{暁 [a]}
{ [1]  \textit{Naturphän.} (schriftspr.) Tagesanbruch (m); Morgendämmerung (f). [2] (in der Form akatsuki ni wa) im Falle von …; wenn … passiert.}\end{entry}
\begin{entry}
\mainentry{akatsuki }
{閼伽坏}
{Gefäß (n) für Opferwasser.}\end{entry}
\begin{entry}
\mainentry{akatsukigata}
{暁方}
{ Morgendämmerung (f).}\end{entry}
\begin{entry}
\mainentry{akatsukishoin}
{暁書院}
{  \textit{Verlagsn.} Akatsuki Shoin (Tōkyō).}\end{entry}
\begin{entry}
\mainentry{akatsukichikaku}
{暁近く}
{vor Morgengrauen.}\end{entry}
\begin{entry}
\mainentry{akatsukiduki}
{暁月}
{ Mond (m) am Morgen.}\end{entry}
\begin{entry}
\mainentry{akatsukidukuyo}
{暁月夜}
{ Morgendämmerung (f), bei der der Mond noch am Himmel steht // Mond (m) am Morgen.}\end{entry}
\begin{entry}
\mainentry{akatsukini}
{暁に}
{bei Tagesanbruch.}\end{entry}
\begin{entry}
\mainentry{akatsukinosora}
{暁の空}
{Himmel (m) bei Morgendämmerung.}\end{entry}
\begin{entry}
\mainentry{akatsukinosoranikinseigamieru。}
{暁の空に金星が見える。}
{ \textit{Bsp.} Die Venus steht am Morgenhimmel.}\end{entry}
\begin{entry}
\mainentry{akatsukiyami}
{暁やみ (暁闇 [1])}
{ mondloser Tagesanbruch (m).}\end{entry}
\begin{entry}
\mainentry{akatsukiwotsugeru}
{暁を告げる}
{den Morgen ankündigen.}\end{entry}
\begin{entry}
\mainentry{akatsukiwotsugerukane}
{暁を告げる鐘}
{Morgenglocke (f).}\end{entry}
\begin{entry}
\mainentry{akatsukiwotsugerukanenooto}
{暁を告げる鐘の音}
{Glockenklang (m), der die Morgendämmerung ankündigt.}\end{entry}
\begin{entry}
\mainentry{akaduku}
{あか付く (垢付く; 垢づく)}
{ schmutzig werden.}\end{entry}
\begin{entry}
\mainentry{agattari}
{上がったり; 上ったり; あがったり}
{ (ugs.) schlechter Geschäftsgang (m).}\end{entry}
\begin{entry}
\mainentry{agattarida}
{あがったりだ}
{nichts mehr einbringen; Pleite gehen; auf den Hund gekommen sein.}\end{entry}
\begin{entry}
\mainentry{agattarida}
{上がったりだ}
{jmd. ist auf den Hunde gekommen.}\end{entry}
\begin{entry}
\mainentry{agattaridearu}
{上がったりである}
{nichts mehr einbringen; Pleite gehen; auf den Hund gekommen sein.}\end{entry}
\begin{entry}
\mainentry{akattan}
{亜褐炭}
{  \textit{Bergbau} Lignit (n) (Braunkohle mit Holzstruktur).}\end{entry}
\begin{entry}
\mainentry{akatsuchi}
{赤土 (赭土 [1])}
{ rote Erde (f); roter Ton (m); roter Lehmboden (m).}\end{entry}
\begin{entry}
\mainentry{akatsuchisuyaki}
{赤土素焼}
{Terrakotta (f).}\end{entry}
\begin{entry}
\mainentry{akacchakeru}
{赤っ茶ける}
{ rötlichbraun werden.}\end{entry}
\begin{entry}
\mainentry{akappaji}
{赤っ恥; 赤恥 [a]; 赤っぱじ}
{ ungeheure Schande (f); ungeheure Schmach (f) (⇒ aka·haji 赤恥2884534).}\end{entry}
\begin{entry}
\mainentry{akappara}
{赤腹; 赤っ腹; あかっぱら}
{  \textit{Vogelk.} Braundrossel (f) (Turdus chrysloaus; ⇒ akahara 赤腹8444455).}\end{entry}
\begin{entry}
\mainentry{akappoi}
{赤っぽい}
{ rot.}\end{entry}
\begin{entry}
\mainentry{agatsuma}
{吾妻}
{  \textit{Ortsn.} AgatsumanNAr (Ort in der Präf. Gunma).}\end{entry}
\begin{entry}
\mainentry{akatsumekusa}
{赤詰草; アカツメクサ; あかつめくさ}
{  \textit{Bot.} Rotklee (m) (Trifolium pratense).}\end{entry}
\begin{entry}
\mainentry{akatsura}
{赤面 [1a] (赭面 [a])}
{ [1] rotes Gesicht (n). [2]  \textit{Kabuki} Schurke (m) mit rotem Gesicht.}\end{entry}
\begin{entry}
\mainentry{akadura}
{赤面 [1b] (赭面 [b])}
{ [1] rotes Gesicht (n). [2]  \textit{Kabuki} Schurke (m) mit rotem Gesicht.}\end{entry}
\begin{entry}
\mainentry{akadein'eiwotsukeru}
{赤で陰影をつける}
{rot schraffieren}\end{entry}
\begin{entry}
\mainentry{akademiamyu-jikku}
{アカデミア・ミュージック}
{  \textit{Verlagsn.} Academia Music; Akademia Myūjikku (Tōkyō; ISBN 4-87017-).}\end{entry}
\begin{entry}
\mainentry{akademi-}
{アカデミー}
{ Akademie (f).}\end{entry}
\begin{entry}
\mainentry{akademi-gakuha}
{アカデミー学派}
{ \textit{Philos.} Platonistenmpl.}\end{entry}
\begin{entry}
\mainentry{akademi-shou}
{アカデミー賞}
{ \textit{Film} „Oscar“ (m); (volkstümlich für) Academy Award (m) (amerik. Filmpreis; seit 1928 verliehen).}\end{entry}
\begin{entry}
\mainentry{akademi-shoujushouhaiyuu}
{アカデミー賞受賞俳優}
{ \textit{Film} Oskar-Schauspieler (m).}\end{entry}
\begin{entry}
\mainentry{akademi-furanse-zu}
{アカデミー・フランセーズ; アカデミーフランセーズ}
{Académie (f) française.}\end{entry}
\begin{entry}
\mainentry{akademishizumu}
{アカデミシズム}
{ Akademismus (m).}\end{entry}
\begin{entry}
\mainentry{akademishan}
{アカデミシャン}
{ [1] Mitglied (n) der Akademie. [2] Akademiker (m). (von engl. academician).}\end{entry}
\begin{entry}
\mainentry{akademisuto}
{アカデミスト}
{ Akademiker (m) (von engl. academist).}\end{entry}
\begin{entry}
\mainentry{akademizumu}
{アカデミズム}
{ Akademismus (m).}\end{entry}
\begin{entry}
\mainentry{akademikku}
{アカデミック}
{ [1] akademisch; gelehrt. [2] (übertr.) undurchführbar; unrealistisch. (von franz. académique).}\end{entry}
\begin{entry}
\mainentry{akademikkuda}
{アカデミックだ}
{[1] akademisch sein; gelehrt sein. [2] (übertr.) undurchführbar sein; unrealistisch sein.}\end{entry}
\begin{entry}
\mainentry{akademikkudyisukaunto}
{アカデミック・ディスカウント; アカデミックディスカウント}
{ \textit{Wirtsch.} Studentenrabatt (m); Rabatt (m) für den akademischen Bedarf (von engl. academic discount).}\end{entry}
\begin{entry}
\mainentry{akademikkuna}
{アカデミックな}
{[1] akademisch; gelehrt. [2] (übertr.) undurchführbar; unrealistisch.}\end{entry}
\begin{entry}
\mainentry{akademisshan}
{アカデミッシャン}
{ Akademiemitglied (n) // Wissenschaftler (m).}\end{entry}
\begin{entry}
\mainentry{akademeia}
{アカデメイア}
{ Akademie (f).}\end{entry}
\begin{entry}
\mainentry{akateru}
{赤照}
{  \textit{Theat.} rotes Licht (n) als Bühneneffekt.}\end{entry}
\begin{entry}
\mainentry{akaten}
{赤点}
{ ungenügende Zensur (f) (⇒ rakudai·ten 落第点9157774).}\end{entry}
\begin{entry}
\mainentry{akaden}
{赤電}
{ (ugs.) letzte Bahn (f); letzte Straßenbahn (f); Lumpensammler (m) (Abk. für aka·densha 赤電車7813004).}\end{entry}
\begin{entry}
\mainentry{akadensha}
{赤電車}
{ (ugs.) letzte Bahn (f); letzte Straßenbahn (f); Lumpensammler (m).}\end{entry}
\begin{entry}
\mainentry{akadenwa}
{赤電話}
{ öffentliches Telefon (n); öffentlicher Fernsprecher (m).}\end{entry}
\begin{entry}
\mainentry{akatoki}
{あかとき (暁 [b])}
{ Tagesanbruch (m); Morgendämmerung (f).}\end{entry}
\begin{entry}
\mainentry{akatokuro}
{赤と黒}
{ \textit{Werktitel} NAr (Roman von Stendhal; 1830).}\end{entry}
\begin{entry}
\mainentry{akatokuronokumiawase}
{赤と黒の組み合わせ}
{rot-schwarze Kombination (f).}\end{entry}
\begin{entry}
\mainentry{akatodo}
{アカトド; あかとど (赤椴)}
{  \textit{Bot.} Sachalin-Tanne (Abies sachalinensis).}\end{entry}
\begin{entry}
\mainentry{akatodomatsu}
{アカトドマツ; あかとどまつ (赤椴松)}
{  \textit{Bot.} Sachalin-Tanne (Abies sachalinensis).}\end{entry}
\begin{entry}
\mainentry{akadomari}
{赤泊}
{  \textit{Ortsn.} Akadomari (Ortschaft in der Präf. Niigata).}\end{entry}
\begin{entry}
\mainentry{akatomidorinosenwotagaichigainihiku}
{赤と緑の線を互い違いに引く}
{abwechselnd rote und grüne Linien ziehen.}\end{entry}
\begin{entry}
\mainentry{akadome}
{淦止め; 淦止; 淦留め; 淦留}
{Arbeitenfpl, um zu verhindern, dass Wasser in ein Boot eindringt.}\end{entry}
\begin{entry}
\mainentry{akadomesuru}
{淦留めする; 淦留する}
{Arbeiten durchführen, die verhindern, dass Wasser in ein Boot eindringt.}\end{entry}
\begin{entry}
\mainentry{akatori }
{垢取り [1]; 垢取 [1]}
{ [1] Striegel (m) für Hautschmutz. [2] Pferdestriegel (m).}\end{entry}
\begin{entry}
\mainentry{akatori }
{赤鳥; 垢取り [2]; 垢取 [2]}
{ Tuch (n), das Frauen des Kriegeradels über den Sattel ihres Reitpferdes legten, um ihre Kleidung sauber zu halten.}\end{entry}
\begin{entry}
\mainentry{akatori }
{淦取り; 淦取}
{Kelle (f) zum Ausschöpfen des Bilgewassers.}\end{entry}
\begin{entry}
\mainentry{akatonbo}
{赤とんぼ; 赤トンボ; アカトンボ; あかとんぼ (赤蜻蛉 {ir.}; 赤蜻蜓)}
{ [1]  \textit{Insektenk.} rote Libelle (f) (Herbst) // Heidelibelle (f) (Sympetrum) // Wanderlibelle (f) (Pantala flavescens) // Shōjō·tonbo (f) (Crocothemis servilia). [2]  \textit{Flugw.} Akatonbo (f) (ein Trainingsdoppeldecker im }\end{entry}
\begin{entry}
\mainentry{aganai }
{あがない (購い)}
{ [1] (schriftspr.) Kauf (m). [2] Umtausch (m).}\end{entry}
\begin{entry}
\mainentry{aganai }
{あがない (贖ない; 贖い; 贖)}
{ [1] Ersatz (m); Entschädigung (f); Wiedergutmachung (f); Kompensation (f). [2] Sühne (f); Buße (f).}\end{entry}
\begin{entry}
\mainentry{aganaikin}
{贖い金}
{Entschädigung (f); Lösegeld (n).}\end{entry}
\begin{entry}
\mainentry{aganainushi}
{贖い主}
{Erlöser (m); Heiland (m).}\end{entry}
\begin{entry}
\mainentry{aganainohi}
{贖いの日}
{ \textit{Rel.} Jom Kippur (m); Versöhnungstag (m) (höchster jüdischer Feiertag).}\end{entry}
\begin{entry}
\mainentry{aganau }
{あがなう [1] (購う)}
{ [1] (schriftspr.) kaufen. [2] eine Belohnung aussetzen. (⇒ kōnyū 購入9906449).}\end{entry}
\begin{entry}
\mainentry{aganau }
{あがなう [2] (贖う)}
{ büßen; sühnen; loskaufen; entschädigen; wiedergutmachen; versöhnen; ersetzen; entschädigen.}\end{entry}
\begin{entry}
\mainentry{akanasu}
{赤ナス; 赤なす (赤茄子)}
{  \textit{Bot.} Paradiesapfel (m); Tomate (f) (alte Bez.).}\end{entry}
\begin{entry}
\mainentry{akanabe}
{あかなべ (銅鍋)}
{ Kupferpfanne (f).}\end{entry}
\begin{entry}
\mainentry{akanawa}
{赤縄}
{ rotes Band (n) als Symbol für die Schicksalsverbindung eines Ehepaares // (übertr.) Beziehung (f) zwischen Eheleuten.}\end{entry}
\begin{entry}
\mainentry{akanishi}
{あかにし; アカニシ (赤螺)}
{ [1]  \textit{Muschelk.} Kinkhorn (n) (Rapana venosa). [2] Geizkragen (m); Knauser (m).}\end{entry}
\begin{entry}
\mainentry{akaninaru}
{赤になる}
{kommunistisch werden.}\end{entry}
\begin{entry}
\mainentry{akanimamireta}
{あかにまみれた}
{schmutzig; dreckig.}\end{entry}
\begin{entry}
\mainentry{akanu}
{飽かぬ}
{ ohne davon genug zu bekommen; ohne daran zu ermüden; ohne es satt zu bekommen. (n)}\end{entry}
\begin{entry}
\mainentry{akanuke}
{あか抜け; あかぬけ (垢抜け; 垢抜; 垢ぬけ)}
{ Verfeinerung (f); Läuterung (f); Verschönerung (f); Verbesserung (f).}\end{entry}
\begin{entry}
\mainentry{akanukeshita}
{垢抜けした; 垢ぬけした}
{schick; smart; elegant; schick; fesch; kultiviert.}\end{entry}
\begin{entry}
\mainentry{akanukeshitahitoda。}
{垢抜けした人だ。}
{ \textit{Bsp.} Er ist eine elegante Erscheinung.}\end{entry}
\begin{entry}
\mainentry{akanukeshitafuuda。}
{垢抜けした風だ。}
{ \textit{Bsp.} Er hat elegante Manieren.}\end{entry}
\begin{entry}
\mainentry{akanukesuru}
{あか抜けする; あかぬけする (垢抜けする; 垢抜する; 垢ぬけする)}
{schick sein; elegant sein; raffiniert sein.}\end{entry}
\begin{entry}
\mainentry{akanukenoshita}
{あかぬけのした (垢抜けのした)}
{schick; elegant; kultiviert; raffiniert; smart; wohlerzogen; hübsch; sauber.}\end{entry}
\begin{entry}
\mainentry{akanukenoshitae}
{垢抜けのした絵}
{geschmackvolles Gemälde (n).}\end{entry}
\begin{entry}
\mainentry{akanukenoshitajochuu}
{垢抜けのした女中}
{wohlerzogenes Dienstmädchen (n).}\end{entry}
\begin{entry}
\mainentry{akanukenoshinai}
{あかぬけのしない (垢抜けのしない)}
{derb; unbeholfen; unkultiviert; ungeschickt.}\end{entry}
\begin{entry}
\mainentry{akanukeru}
{あか抜ける; あかぬける (垢抜ける; 垢ぬける)}
{ schick sein; elegant sein; raffiniert sein.}\end{entry}
\begin{entry}
\mainentry{akanunagame}
{飽かぬ眺め}
{Anblick (m), dessen man nicht müde wird.}\end{entry}
\begin{entry}
\mainentry{akanuma}
{赤沼}
{  \textit{Familienn.} Akanuma.}\end{entry}
\begin{entry}
\mainentry{akanumafuuro}
{赤沼風露; あかぬまふうろ; アカヌマフウロ}
{  \textit{Bot.} Pelargonie (f); Storchenschnabel (m) (Geranium yesoense var. nipponicum).}\end{entry}
\begin{entry}
\mainentry{akanuri}
{赤塗り; 赤塗}
{ [1] etw. rot Angemaltes. [2]  \textit{Kabuki} Schurke (m) mit rot geschminktem Gesicht.}\end{entry}
\begin{entry}
\mainentry{akanurisuru}
{赤塗りする; 赤塗する}
{[1] etw. rot anmalen. [2]  \textit{Kabuki} die Rolle des Schurken mit dem rot geschminktem Gesicht geben.}\end{entry}
\begin{entry}
\mainentry{akane}
{あかね; アカネ (茜)}
{ [1]  \textit{Bot.} Krapp (m); Färberröte (f) (Rubia cordifolia). [2] Krapprot (n).}\end{entry}
\begin{entry}
\mainentry{akaneiro}
{あかね色 (茜色)}
{Krapprot (n).}\end{entry}
\begin{entry}
\mainentry{akaneka}
{茜科}
{ \textit{Bot.} Rötegewächsenpl; Krappgewächsenpl (Rubiaceae).}\end{entry}
\begin{entry}
\mainentry{akanegumo}
{あかね雲 (茜雲)}
{  \textit{Meteor.} krapprote Wolkenfpl.}\end{entry}
\begin{entry}
\mainentry{akanesasu}
{茜さす}
{rötlich leuchten; rot sein.}\end{entry}
\begin{entry}
\mainentry{akanesasusora}
{あかねさす空 (茜さす空; 茜差す空)}
{leuchtender Himmel (m).}\end{entry}
\begin{entry}
\mainentry{akanesasuhigashinosora}
{茜さす東の空}
{rötlich leuchtender Osthimmel (m).}\end{entry}
\begin{entry}
\mainentry{akanesasuhigashinozora}
{あかねさす東の空}
{leuchtender östlicher Himmel (m).}\end{entry}
\begin{entry}
\mainentry{akanesasuhinoonhata}
{茜さす日の御旗}
{die Fahne (f) mit der leuchtend roten Sonne.}\end{entry}
\begin{entry}
\mainentry{akanesasuhinohon'nokuni}
{茜さす日の本の国}
{das Land (n) der aufgehenden Sonne.}\end{entry}
\begin{entry}
\mainentry{akanesasuyuuhi}
{茜さす夕陽}
{rötlich leuchtende Sonne (f).}\end{entry}
\begin{entry}
\mainentry{akaneshikiso}
{茜色素}
{ \textit{Chem.} Alizarin (f).}\end{entry}
\begin{entry}
\mainentry{akaneshobou}
{あかね書房}
{  \textit{Verlagsn.} Akane Shobō (Tōkyō).}\end{entry}
\begin{entry}
\mainentry{akanezumi}
{アカネズミ; あかねずみ (赤鼠)}
{  \textit{Zool.} große japanische Feldmaus (f) (Apodemus speciosus).}\end{entry}
\begin{entry}
\mainentry{akanezome}
{あかね染め; 赤根染 (茜染め; 茜染)}
{ [1] Färben (n) mit Krapp. [2] mit Krapp gefärbter Stoff (m).}\end{entry}
\begin{entry}
\mainentry{akanezomeno}
{茜染めの}
{rotgefärbt.}\end{entry}
\begin{entry}
\mainentry{akanetonbo}
{アカネトンボ; あかねとんぼ (茜蜻蛉)}
{  \textit{Insektenk.} rote Libelle (f).}\end{entry}
\begin{entry}
\mainentry{akanemomen}
{茜木綿}
{rote Baumwolle (f); roter Baumwollstoff (m).}\end{entry}
\begin{entry}
\mainentry{akanendo}
{赤粘土}
{  \textit{Geol.} roter Tiefseeton (m) (toniges Meeressediment in Tiefen über 5.000 m).}\end{entry}
\begin{entry}
\mainentry{akanokattairo}
{赤の勝った色}
{warme Farbe (f).}\end{entry}
\begin{entry}
\mainentry{aganogawa}
{阿賀野川}
{  \textit{Flussn.} Agano-Fluss (m) (Fluss von der Präf. Fukushima bis in die Präf. Niigata; Länge 210 km).}\end{entry}
\begin{entry}
\mainentry{akanogohan}
{赤の御飯}
{ \textit{Kochk.} mit Azuki-Bohnen gekochter Reis (m).}\end{entry}
\begin{entry}
\mainentry{akanojououkasetsu}
{赤の女王仮説}
{ \textit{Evolutionsbiol.} Red-Queen-Hypothese (f); Rote-Königin-Hypothese (f) (Theorie über den Aufwand einer Art, um eine ökologische Nische zu halten).}\end{entry}
\begin{entry}
\mainentry{akanotanin}
{赤の他人}
{ Wildfremder (m).}\end{entry}
\begin{entry}
\mainentry{akanotaninjanai。}
{赤の他人じゃない。}
{ \textit{Bsp.} Er ist kein Wildfremder für mich.}\end{entry}
\begin{entry}
\mainentry{akanotaninda。}
{赤の他人だ。}
{ \textit{Bsp.} Ich kenne ihn überhaupt nicht.}\end{entry}
\begin{entry}
\mainentry{akanotsuitakimono}
{垢のついた着物}
{schmutzige Kleidung (f).}\end{entry}
\begin{entry}
\mainentry{akanotsuihou}
{赤の追放}
{kommunistische Säuberung (f).}\end{entry}
\begin{entry}
\mainentry{akanodouko}
{銅の銅壺}
{Kupferwasserkessel (m).}\end{entry}
\begin{entry}
\mainentry{akanohiroba}
{赤の広場}
{  \textit{Ortsn.} Roter Platz (m) (in Moskau).}\end{entry}
\begin{entry}
\mainentry{akanomanma}
{赤のまんま (赤の飯)}
{ [1]  \textit{Kochk.} Sekihan (n) (mit roten Bohnen gekochter Reis; ein Festessen). [2]  \textit{Bot.} Inutade (f) (eine Knöterichart; Polygonum longisetum).}\end{entry}
\begin{entry}
\mainentry{akanoren}
{赤のれん (赤暖簾)}
{ [1] roter Ladenvorhang (m). [2] billiges Restaurant (n).}\end{entry}
\begin{entry}
\mainentry{akaba}
{アカバ}
{  \textit{Stadtn.} AkabanNAr (Hafenstadt in Jordanien).}\end{entry}
\begin{entry}
\mainentry{agaha-n}
{アガ・ハーン; アガハーン}
{  \textit{Islam} Aga Khan (m) (Oberhaupt einer islamischen Glaubensgemeinschaft).}\end{entry}
\begin{entry}
\mainentry{akahage}
{赤はげ (赤禿げ; 赤禿)}
{ [1] Glatze (f) // glatzköpfige Person (f). [2] unbewachsener Berg (m). [3] Taihei·shi (n); Papier (n) von niedriger Qualität.}\end{entry}
\begin{entry}
\mainentry{akahaji}
{赤恥 [b]}
{ ungeheure Schande (f); ungeheure Schmach (f).}\end{entry}
\begin{entry}
\mainentry{akapajamaaopajamakipajamachapajama}
{赤パジャマ青パジャマ黄パジャマ茶パジャマ}
{ \textit{Bsp.} roter Pyjama, blauer Pyjama, gelber Pyjama, brauner Pyjama (japan. Zungenbrecher).}\end{entry}
\begin{entry}
\mainentry{akabashira}
{赤柱}
{  \textit{Sumō} roter Pfeiler (m) (in der Südost-Ecke des Sumō-Ringes; heute durch eine rote Quaste am Dach über dem Sumō-Ring repräsentiert).}\end{entry}
\begin{entry}
\mainentry{akahajiwokakasareru}
{赤恥をかかされる}
{öffentlich blamiert werden.}\end{entry}
\begin{entry}
\mainentry{akahajiwokakaseru}
{赤恥をかかせる}
{bis auf die Knochen blamieren; öffentlich lächerlich machen.}\end{entry}
\begin{entry}
\mainentry{akahajiwokaku}
{赤恥をかく (赤恥を掻く)}
{sich bis auf die Knochen blamieren; sich unsterblich blamieren; in Schmach und Schande geraten; vor Scham krebsrot werden.}\end{entry}
\begin{entry}
\mainentry{akabasu}
{赤バス}
{letzter Bus (m).}\end{entry}
\begin{entry}
\mainentry{akahata}
{赤旗}
{ [1] rote Fahne (f); rote Flagge (f) (der Kommunisten, der Gewerkschaften etc.). [2] Gefahrenfahne (f). [3] Fahne (f) der Gewerkschaften und Kommunisten. [4]  \textit{Zeitschriftenn.} Aka·hata (f); Rote Fahne (f) (Zentralorgan der Kommunistischen Partei Japans; 1928 gegründet).}\end{entry}
\begin{entry}
\mainentry{akahada}
{赤肌 (赤膚)}
{ [1] rote Haut (f); aufgekratzte Haut (f). [2] Nacktheit (f); Kahlheit (f).}\end{entry}
\begin{entry}
\mainentry{akahadaka}
{赤裸; 赤はだか}
{ vollkommene Nacktheit (f) (⇒ map·padaka 真っ裸7749134).}\end{entry}
\begin{entry}
\mainentry{akahadakadearukimawaru}
{赤裸で歩きまわる}
{vollkommen nackt herumlaufen.}\end{entry}
\begin{entry}
\mainentry{akahadakanisareru}
{赤裸にされる}
{splitternackt ausgezogen werden.}\end{entry}
\begin{entry}
\mainentry{akahadakanisuru}
{赤裸にする}
{jmdn. ganz ausziehen.}\end{entry}
\begin{entry}
\mainentry{akahadakaninaru}
{赤裸になる}
{sich ganz ausziehen.}\end{entry}
\begin{entry}
\mainentry{akahadakano}
{赤裸の}
{[1] splitternackt. [2] gerupft; geschoren.}\end{entry}
\begin{entry}
\mainentry{akahadanoyama}
{赤肌の山}
{kahler Berg (m).}\end{entry}
\begin{entry}
\mainentry{akahatawoutau}
{赤旗を歌う}
{ singen.}\end{entry}
\begin{entry}
\mainentry{akahatawooshitatetekoushinsuru}
{赤旗を押し立てて行進する}
{mit gehisster roter Fahne voranschreiten.}\end{entry}
\begin{entry}
\mainentry{akahana}
{赤鼻 [a]}
{  \textit{Med.} rote Nase (f); rötliche geschwollene Nase (f); Kupfernase (f); Rosazea (f) (Acne rosacea; → aka·bana 赤鼻1245845).}\end{entry}
\begin{entry}
\mainentry{akabana }
{赤花; あかばな; アカバナ}
{  \textit{Bot.} Weidenröschen (n) (Epilobium pyrricholophum).}\end{entry}
\begin{entry}
\mainentry{akabana }
{赤鼻 [b]}
{  \textit{Med.} rote Nase (f); rötliche geschwollene Nase (f); Kupfernase (f); Rosazea (f) (Acne rosacea).}\end{entry}
\begin{entry}
\mainentry{akahaniau}
{赤は似合う}
{jmdm. steht rot.}\end{entry}
\begin{entry}
\mainentry{akahane }
{赤羽 [1]}
{  \textit{Familienn.} Akahane.}\end{entry}
\begin{entry}
\mainentry{akabane }
{赤羽 [2]}
{ [1]  \textit{Gebietsn.} AkabanenNAr (Gebiet im Norden Tōkyōs). [2]  \textit{Familienn.} AkabaneNAr.}\end{entry}
\end{multicols}


\end{document}

